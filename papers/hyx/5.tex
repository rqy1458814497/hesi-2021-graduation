 
 Maryam Mirzakhani gives an integration formula over  $\mathscr{M}_{g,n}(L)$ with respect to the  Weil--Petersson volume form  in \cite{Mirzakhani:2006fta}. However, the integration formula in  \cite{Mirzakhani:2006fta} has wrong coefficients, she fixes the coefficients in \cite{growthofsimple}, and Petri summaries it in \cite{Teithe}. Although their definitions and notations are distinct slightly, their formula matches with each other. This section adopts Mirzakhani's notations, with which the coefficients are more concrete.

A multi-curve is of the form $\gamma=\sum_{i=1}^kc_i\gamma_i$ with $c_i\geq 0$ and $\gamma_i's$ are disjoint simple closed curves in different isotopy class, and none of them are homotopic to some boundary component. Then define $S_{g,n}(\gamma)$
to be the surface obtained by cutting $S_{g,n}$ along $\{\gamma_i\}_{i=1}^k$. Assume that $$
S_{g,n}(\gamma)=\cup_{i=1}^NS_{g_i,n_i}
$$
with $S_{g_i,n_i}$ all the connected components. Now consider the projection  $\pi\colon S_{g,n}(\gamma)\to S_{g,n}$, and  $\pi^{-1}(\gamma_i)=\{\gamma_i^1,\gamma_i^2\}$, then  $$
\partial S_{g,n}(\gamma)=\{\beta_1,\beta_2,\cdots,\beta_n,\gamma_1^1,\gamma_1^2,\cdots,\gamma_k^1,\gamma_k^2\}.
$$

If $\Gamma=(\gamma_1,\cdots,\gamma_k)$,  $\beta=(\beta_1,\cdots,\beta_n)$, $L=(L_1,\cdots,L_n)\in \mathbb{R}_+^n$, and $x=(x_1,\cdots,x_k)\in \mathbb{R}_+^k$, then denote the Teichm\"uller space of Riemann surface homeomorphic to $S_{g,n}(\gamma)$ with fixed boundary length $l_{\beta_i}=L_i$ and $l_{\gamma_i}=x_i$ by $$
\mathscr{T}(S_{g,n}(\gamma),l_\Gamma=x,l_\beta=L),
$$
denote the moduli space of Riemann surface homeomorphic to $S_{g,n}(\gamma)$, with fixed boundary length $l_{\beta_i}=L_i$ and $l_{\gamma_i}=x_i$ by $$
\mathscr{M}(S_{g,n}(\gamma),l_\Gamma=x,l_\beta=L),
$$
and  $$
 V_{g,n}(\Gamma,x,\beta,L)=\mathrm{Vol}(\mathscr{M}(S_{g,n}(\gamma),l_\Gamma=x,l_\beta=L))
 $$
 is the Weil--Petersson volume of this moduli space.
 
 Now take $$G(\Gamma)=\cap_{i=1}^k\operatorname{stab}(\gamma_i),$$
 it acts on $\mathscr{T}(S_{g,n}(\gamma),l_\Gamma=x,l_\beta=L)$, 
 and the quotient of $\mathscr{T}(S_{g,n}(\gamma),l_\Gamma=x,l_\beta=L)$ with respect to $G(\Gamma)$ is denoted by  $$
\mathscr{M}_{g,n}(\Gamma,x,\beta,L)= \mathscr{T}(S_{g,n}(\gamma),l_\Gamma=x,l_\beta=L)/G(\Gamma).
 $$
 
 \begin{remark}
 Notice that if $g\in G(\Gamma)$ doesn't preserve the orientation of $\gamma_i$, then $g$ will map $\gamma_i^1$ to $\gamma_i^2$ and map $\gamma_i^2$ to $\gamma_i^1$.
 
 Take $(X,f)\in \mathscr{T}_{g,n}(L)$, with $L_i=L_j$, and $h:X\to X$ exchanges $\beta_i$ to $\beta_j$, then $(X,h\circ f)\in \mathscr{T}_{g,n}(L)$. But this will bring inconvenience, and change the moduli space $\mathscr{M}_{g,n}(L)$  dramatically if $L$ is changed slightly. Such case could be ignored, since for almost every $L\in \mathbb{R}_+^{3g-3+n}$, $L_i$'s are  distinct. But for the case of $S_{g,n}(\gamma)$, $\gamma_i^1$ and $\gamma_i^2$ is always of the same length, so they are not ignored.  For the case that $x_i=x_j$ and $h$ exchanges $\gamma_i$  and $\gamma_j$, it also corresponds to a set of measure in $\mathbb{R}_+^k$, which can be ignored.
 
 The main formula in this section is theorem \ref{integrformu}. Suppose that $L_i\neq L_j$ for all $i\neq j$.Notice that the integral function on the right side has a factor  $\mathrm{Vol}(\mathscr{M}_{g,n}(\Gamma,x,\beta,L))$, for all $x\in \mathbb{R}_{+}^k$. So for all $x$, $l(\gamma_i^1)=l(\gamma_i^2)$. 
  But for almost every $x\in \mathbb{R}_{+}^k$, $(x_1,\cdots,x_k,L_1,\cdots,L_{3g-3+n})$ are pairwise distinct, so the exceptional case could be ignored since the integral over a subset of measure of zero is always zero.
 
 While for the exceptional case of $L_i=L_j$ for some $i\neq j$, it can be seen that both sides are influenced by the same factor to the  multiplicities, so the amendment can be done uniformly. 
 
 One main conclusion which  will be shown in  this article is that the $V_{g,n}(L)$ is a polynomial of $L_i$ of degree $6g-6+2n$. The above amendment avoids the  discontinuous points of $V_{g,n}(L)$ at $L$ with $L_i=L_j$.
\end{remark}

 
 
 
 Define $\stab_0(\gamma)\subset \stab(\gamma)\subset \Mod_{g,n}(L)$ to be the subset of  elements which preserve the orientation of $\gamma$. Then consider the mapping class group $\Mod(S_{g,n}(\gamma))$, $g\in \Mod(S_{g,n}(\gamma))$ can be seen as an element in $\cap_{i=1}^k \stab_0(\gamma_i)$ by gluing the diffeomorphisms along the boundary components, since any orientation preserving self-diffeomorphisms of the boundary circle is homotopical equivalent. While for element $g$ in  $\cap_{i=1}^k\stab_0(\gamma_i)$,  the restriction map of $g$ on each components of $S_{g,n}(\gamma)$ can be seen as a self-diffeomorphism of $S_{g,n}(\gamma)$. This gives the isomorphism $$
 \Mod (S_{g,n}(\gamma))\simeq \cap_{i=1}^k\stab_0(\gamma_i).
 $$
 The mapping class group acts discretely on the Teichm\"uller space, so $\mathscr{M}(S_{g,n}(\gamma),l_\Gamma=x,l_\beta=L)$ is  the covering space of $\mathscr{M}_{g,n}(\Gamma,x,\beta,L)$ of order $$
 N(\gamma)=|G(\Gamma)/\cap_{i=1}^k\stab_0(\gamma_i)|=|\cap_{i=1}^k\stab(\gamma_i)/\cap_{i=1}^k\stab_0(\gamma_i)|,
 $$
 
thus $$\mathrm{Vol}(\mathscr{M}_{g,n}(\Gamma,x,\beta,L))=\frac{1}{N(\gamma)}\mathrm{Vol}(\mathscr{M}(S_{g,n}(\gamma),l_\Gamma=x,l_\beta=L)).$$
 
\begin{remark}
In \cite{Mirzakhani:2006fta} the orientation of $\gamma_i$ is not taken into consideration. The corresponding of $\mathscr{M}_{g,n}(\Gamma,x,\beta,L)$ and $\mathscr{M}(S_{g,n}(\gamma),l_\Gamma=x,l_\beta=L)$ is not $1$ to $1$.

\end{remark}


 
 In the above definition, the coefficients $c_i$'s don't appear.
 
 For $\gamma=\sum_{i=1}^kc_i\gamma_i$, define the stabilizer of $\gamma$ to be $$
 \stab(\gamma)=\{g\in \Mod_{g,n}|g\gamma=\gamma\}, 
 $$
  and its symmetry group to be $$
  \sym(\gamma)=\stab(\gamma)/\cap_{i=1}^k \stab(\gamma_i).
  $$
 
 \begin{remark}
 If $\gamma=c_1\gamma_1$, then $|\sym(\gamma)|=1$. Generally,  $|\sym(\gamma)|\leq k!$, and if $|\sym(\gamma)|=k!$, then $\forall \sigma\in S(k)$, where $S(k)$ is the $k^{th}$ permutation group, there is an element $g$ such that $\gamma=\sum_{i=1}^kc_i\gamma_{i}=g\gamma=\sum_{i=1}^kc_{\sigma i}\gamma_{ i}$, thus $c_1=c_{\sigma 1}$ for all $\sigma\in S(k)$, so $c_1=c_2=\cdots=c_k.$
 \end{remark}
 
 
 
 
 

For a multi-curve $\gamma=\sum_{i=1}^k c_i\gamma_i$, define $$l_\gamma(X)=\sum_{i=1}^kc_il_{\gamma_i}(X),$$
for $X\in \mathscr{T}_{g,n}(L)$. This is defined on $\mathscr{T}_{g,n}(L)$, but is not well defined on $\mathscr{M}_{g,n}(L)$. It is not invariant under the mapping class group  actions. 

For any continuous function $f:\mathbb{R}_+\to \mathbb{R}$, define a related function $$
f_\gamma(X)=\sum_{[\alpha]\in \Mod\cdot[\gamma]}f(l_\alpha(X)).
$$
Since $\Mod_{g,n}[g\gamma]=\Mod_{g,n}[\gamma]$, and
 the summation is ergodic over the orbits of the  equivalent classes under the actions of mapping class group,
then $f_\gamma$ is a well defined function on $\mathscr{M}_{g,n}(L)$. 








 
 \begin{theorem}[Mirzakhani's integration formula]\label{integrformu}
 If  $\gamma=\sum_{i=1}^ka_i\gamma_i$ is a multi-curve  and  $(g,n)\neq (1,1),(0,3)$,   for the integrable function $f:\mathbb{R}_+\to \mathbb{R}$, the following integration formula holds:
 $$
 \int_{\mathscr{M}_{g,n}(L)}f_\gamma(X)dX=\frac{2^{-M(\gamma)}}{|\sym(\gamma)|}\int_{\mathbb{R}_+^k}f(|x|)\mathrm{Vol}(\mathscr{M}_{g,n}(\Gamma,x,\beta,L))x\cdot dx.
 $$
 
 Here  the notation $x\cdot dx$ represents $x_1\cdots x_k dx_1\cdots dx_k$, and  $|x|=\sum_{i=1}^k a_ix_i$,
 $M(\Gamma)$  is the number of $S_{1,1}$ cut by $\gamma$. $S_{1,1}$ admits a half twist. For the case that $\gamma$ cut $S_{2,0}$ into two pieces of $S_{1,1}$, take $M(\gamma)=1$ as an exception.
 \end{theorem}
 
 
\begin{lemma}\label{coverintegral}
 If $\pi:M\to N$ is a covering map, then for an volume form $\omega$ of $N$, it will be pulled back to a volume form $\pi^*\omega$ of $M$. 
 Then if $f\in L^1(M,\pi^*\omega)$, define the push forward $\pi_*f\in L^1(N,\omega)$ by 
 \begin{equation}
      (\pi_*f)(x)=\sum_{\pi(y)=x}f(y),
 \end{equation}
 then the equality holds:
 \begin{equation}
     \int_{N} (\pi_*f)d\omega=\int_{M}fd(\pi_*\omega).
 \end{equation}
\end{lemma}
 
 
 For  $\Gamma=(\gamma_1,\cdots,\gamma_k)$,  $h\in \Mod_{g,n}$ acts by
 $$
 h\cdot \Gamma=(h\gamma_1,\cdots,h\gamma_k),
 $$
 and $\mathscr{O}^\Gamma$ represents the set of  orbits of homotopy classes of  $\Mod_{g,n}\cdot \Gamma$. 
 Define the space $$
 \mathscr{M}_{g,n}(L)^\Gamma=\{(X,\eta)|X\in \mathscr{M}_{g,n}(L),\eta\in \mathscr{O}^\Gamma\},
 $$
 and the projection $$
\begin{aligned}
 \pi: \mathscr{M}_{g,n}(L)^\Gamma&\to  \mathscr{M}_{g,n}(L)\\
 (X,\eta)&\mapsto X,\\
\end{aligned}
 $$
 then $\pi\colon \mathscr{M}_{g,n}(L)^\Gamma\to  \mathscr{M}_{g,n}(L)$ is covering map with base space $\mathscr{M}_{g,n}(L)$. So the integral over $\mathscr{M}_{g,n}(L)$ can be derived by the integral over $\mathscr{M}_{g,n}(L)^\Gamma$.
 
 While for $\mathscr{T}_{g,n}(L)$, consider the map $$
 (X,f)\mapsto (X,[f(\gamma_1)],[f(\gamma_2)],\cdots,[f(\gamma_k)]),
 $$
 then it induces the equivalent $$
 \mathscr{M}_{g,n}(L)^\Gamma\simeq \mathscr{T}_{g,n}(L)/\cap_{i=1}^k \stab(\gamma_i).
 $$
 
 Now since the Weil--Petersson symplectic form is invariant under the mapping class group action,  $\mathscr{M}_{g,n}(L)^\Gamma$ admits the induced Weil--Petersson symplectic form $\pi^*\omega$, where $\omega$ is the Weil--Petersson symplectic form on the moduli space.
 
Consider the length vector with respect to  $\Gamma$, 
$$
\begin{aligned}
    l_\Gamma\colon\mathscr{T}_{g,n}(L)&\to \mathbb{R}_+^k\\
    X &\mapsto (l_{\gamma_1}(X),\cdots,l_{\gamma_k}(X)),\\
\end{aligned}
$$ 
and the level set $l_\Gamma^{-1}(a)\subset \mathscr{T}_{g,n}(L)$, 
is the invariant subspace of  the twist along $\gamma_i$, since $l_{\gamma_j}(T_{\gamma_i}^t(X))=l_{\gamma_j}(X)$.


Here the twist long $\gamma_i$ is defined by $$
T_{\gamma_i}^t(X)=tw_{\gamma_i}^{tl_{\gamma_i}(X)}(X),
$$
where $tw_{\alpha}^s$ is given by $$
(l_{\alpha_i}(X),\tau_{\alpha_i}(X))_{i=1}^{3g-3+n}\mapsto ((l_{\alpha_i}(X),\tau_{\alpha_i}(X)+s\delta_{1i}l_{\alpha_i}(X))_{i=1}^{3g-3+n},
$$
 if $P=\{\alpha_i\}_{i=1}^{3g-3+n}$ is a pants decomposition with $\alpha_1=\alpha$. For the case $s=l_{\alpha}(X)$, it corresponds with the Dehn  twist  $D(\alpha)$ along $\alpha$.
 
 The twists along  all $\gamma_i$  are commutable,  thus there is a $\mathbb{R}^k$ action on $l_\Gamma^{-1}(a)$. 
 
 The length function can degenerate to $\mathscr{M}_{g,n}(L)^\Gamma$ by 
 $L_{\Gamma}(X,\eta)=(l_{\eta_i}(X))_{i=1}^k$,
  and the level set of $a$ is  denoted by $\mathscr{M}_{g,n}(L)^\Gamma[a]$. Then the $\mathbb{R}^k$ action degenerates to a $T^k=S^1\times S^1\times \cdots\times S^1$ action on $\mathscr{M}_{g,n}(L)^\Gamma[a]$, since the Dehn twists belong to the mapping class group $\Mod_{g,n}$. The quotient space of $\mathscr{M}_{g,n}(L)^\Gamma[a]$ with respect to the $T^k$ action is denoted by $\mathscr{M}_{g,n}(L)^{\Gamma*}[a]$.
  
  $\mathscr{M}_{g,n}(L)^{\Gamma}[a]=L_{\Gamma}^{-1}(a)\subset \mathscr{M}_{g,n}(L)^{\Gamma}$ admits the induced volume form $\omega^*\in\Omega^{6g-6+2n-k}(\mathscr{M}_{g,n}(L)^{\Gamma})$,  which is
  $$
    dl_1\wedge d\tau_1\wedge \cdots \widehat{dl_{i_1}}\cdots \widehat{dl_{i_k}}\cdots\wedge dl_{3g-3+n}\wedge d\tau_{3g-3+n}
  $$
  under the Fenchel-Nielsen coordinates $(l_i,\tau_i)_{i=1}^{3g-3+n}$ with respect to the pants decomposition $(\alpha_i)_{i=1}^{3g-3+n}$, with $\alpha_{i_t}=\gamma_t$, since the length of $\alpha_{i_t}$ is fixed on the level set $\mathscr{M}_{g,n}(L)^{\Gamma}[a]$.
  
  While for the quotient space $\mathscr{M}_{g,n}(L)^{\Gamma*}[a]$, it admits the symplectic structure, $$\omega=\sum\limits_{j\neq i_t,\quad t=1,2,\cdots,k}dl_j\wedge d\tau_j.$$
 $d\tau_{i_t}$'s are  killed by the $T^k$ action of twists along $\alpha_{i_t}$.
 
 In most case, the parameters for $\mathscr{M}_{g,n}(L)^{\Gamma}[a]$ over $\mathscr{M}_{g,n}(L)^{\Gamma*}[a]$ are $\tau_{i_t}\in[0,l_{\gamma_{i_t}}(X))$. But if $\gamma_i$ cut off a $S_{1,1}$, then the half twist of it is not contained in the group generalized by Dehn twists along  closed geodesics in $\Gamma$, thus $\tau_{i_t}$ varies from $0$ to $\frac{1}{2}l_{\gamma_{i_t}}(X)$.
 
 Thus, under the projection $$\pi\colon\mathscr{M}_{g,n}(L)^{\Gamma}[a]\to \mathscr{M}_{g,n}(L)^{\Gamma*}[a],$$
 for open set $U\subset \mathscr{M}_{g,n}(L)^{\Gamma*}[a]$, 
 \begin{equation}\label{areaformula}
     \mathrm{Vol}(\pi^{-1}(U))=2^{-M(\gamma)}\mathrm{Vol}(U)\ a_1\cdots a_k.
 \end{equation}
 \begin{remark}
 The factor $2^{-M(\gamma)}$ comes from the half twist here, and for the case of $\gamma$ cut  $S_{2,0}$ into two pieces of $S_{1,1}$, take $M(\gamma)=1$. Due to the existence of $M(\gamma)$, when $\mathrm{Vol}(\mathscr{M}_{1,1}(L))$ is needed, it is $\frac{1}{24}(L^2+4\pi^2)$ instead of $\frac{1}{48}(L^2+4\pi^2)$. This is due to some historical reason. 
 \end{remark}
 $\mathscr{M}_{g,n}(L)^{\Gamma*}[a]$ can be  related  to the moduli space of $S_{g,n}(\gamma)$.
 
 For $X\in l_{\Gamma}^{-1}(a)$, cut it along $\gamma$, and get a surface $\cut_\gamma(X)$ in $\mathscr{T}(S_{g,n}(\gamma),l_\beta=L,l_\Gamma=a)$.
 $cut_{\gamma}$ is invariant under the $\mathscr{R}^k$ action on $l_{\Gamma}^{-1}(a)$, since the geodesics along with which the twists flow is cut. That is $$
 \cut_{\gamma}(X)=\cut_{\gamma}(T_{\gamma_1}^{s_1}\cdots T_{\gamma_k}^{s_k}(X)).
 $$
 
 Project the image to $\mathscr{T}(S_{g,n}(\gamma),l_{\Gamma}=a,L_\beta=L)/G(\Gamma)$, then for $[(X,\eta)]\in \mathscr{M}_{g,n}(L)^{\Gamma*}[a]$, $\cut_{\gamma}([(X,\eta)])\in\mathscr{T}(S_{g,n}(\gamma),l_{\Gamma}=a,L_\beta=L)/G(\Gamma)$ is well defined. $\cut_{\gamma}$ induces the map:
 $$
 \cut\colon\mathscr{M}_{g,n}(L)^{\Gamma*}[a]\to \mathscr{M}_{g,n}(\Gamma,a,\beta,L),
 $$
which keeps the symplectic structure by theorem \ref{wpmetricWolpert}, and is an isomorphism. The inverse map is obtained by gluing $X$ along the boundary components corresponding to the same $\gamma_i\in \Gamma$. 

This implies:
\begin{lemma}
\begin{equation}\label{areaidentity}
    \mathrm{Vol}(\mathscr{M}_{g,n}(L)^{\Gamma*}[a])=\mathrm{Vol}(\mathscr{M}_{g,n}(\Gamma,a,\beta,L)).
\end{equation}


\end{lemma}

Then combine (\ref{areaidentity}) and (\ref{areaformula}), the volume of the level set  is given by:
\begin{equation}
    \mathrm{Vol}(\mathscr{M}_{g,n}(L)^\Gamma[a])=2^{-M(\gamma)}\mathrm{Vol}(\mathscr{M}_{g,n}(\Gamma,x,\beta,L))\ a_1\cdots a_k.
\end{equation}

It helps to calculate the version of integration formula of geometric function independent of the coefficients of $\gamma_i$ in $\gamma$.

\begin{theorem}
For integrable function $F\colon\mathbb{R}^k\to \mathbb{R}_+$, and $L_\Gamma\colon\mathscr{M}_{g,n}(L)^\Gamma\to \mathbb{R}^k$ with  $L_\Gamma(X,\eta)=(L_{\eta_i}(X))$, then 
\begin{equation}\label{intver1}
    \int_{\mathscr{M}_{g,n}(L)^\Gamma}F(L_\Gamma(Y))dY=2^{-M(\Gamma)}\int_{\mathbb{R}_+^k}F(x)\mathrm{Vol}(\mathscr{M}_{g,n}(\Gamma,x,\beta,L))xdx.
\end{equation}
where $F(x)=F(x_1,\cdots,x_k)$ and $xdx=x_1\cdots x_kdx_1\cdots dx_k$.
\end{theorem}

\begin{proof}
$$
\int_{\mathscr{M}_{g,n}(L)^\Gamma}F(L_\Gamma(Y))dY
=\int_{x\in\mathbb{R}_+^k} dx \int_{\mathscr{M}_{g,n}(L)^\Gamma[x]}F(L_\Gamma(Z))dZ
$$
while on $\mathscr{M}_{g,n}(L)^\Gamma[x]$ $F(L_\Gamma(Z))=F(x)$,  thus $$
\int_{\mathscr{M}_{g,n}(L)^\Gamma[x]}F(L_\Gamma(Z))dZ
=F(X)2^{-M(\gamma)}\mathrm{Vol}(\mathscr{M}_{g,n}(\Gamma,x,\beta,L))x_1\cdots x_k,
$$
then (\ref{intver1}) holds.
\end{proof}

Then take the coefficient into consideration.
It will lead to the coefficient $\frac{1}{|\sym(\gamma)|}$ in Mirzakhani's integration formula.
\begin{proof}[Proof of theorem \ref{integrformu}]
 Apply the lemma \ref{coverintegral} to the function $F\circ L_\Gamma$ and the covering map
$$
\pi\colon \mathscr{M}_{g,n}(L)^\Gamma\to  \mathscr{M}_{g,n}(L),
$$
the identity follows $$
\int_{\mathscr{M}_{g,n}(L)^\Gamma}F(L_\Gamma(Y))dY=\int_{\mathscr{M}_{g,n}(L)}(\pi_*(F\circ L_{\Gamma}))(Y)dY.
$$
Take $F(x_1,\cdots,x_k)=f(\sum_{i=1}^kc_ix_i)$, then $F(x)=f(|x|)$, where $|x|=\sum_{i=1}^ka_ix_i$.
So the above equation is equal to  $$
2^{-M(\Gamma)}\int_{\mathbb{R}_+^k}f(|x|)\mathrm{Vol}(\mathscr{M}_{g,n}(\Gamma,x,\beta,L))xdx.
$$
While $f_\gamma(X)=\sum_{[\alpha]\in \Mod\cdot [\gamma]}f(l_\alpha(X))$, and 
$$
\begin{aligned}
&\pi_*(F\circ L_\Gamma)(X)\\
=&\sum\limits_{(\eta_1,\cdots,\eta_k)\in \Mod\cdot (\gamma_1,\cdots,\gamma_k)} F\circ L_\eta (X)\\
=&\sum_{g\in \Mod_{g,n}/\cap_{i=1}^k\stab(\gamma_i)}f( L_{g\gamma}(X))\\
=&\sum\limits_{g\in \Mod_{g,n}/\stab(\gamma)} \sum\limits_{h\in \sym(\gamma)} f(l_{gh\gamma}(X)) \\
=&|\sym(\gamma)|\sum\limits_{g\in \Mod_{g,n}/\stab(\gamma)}f(l_{g\gamma})(X))\\
=&|\sym(\gamma)|f_{\gamma}(X).\\
\end{aligned}
$$
then it follows 
$$
|\sym(\gamma)|\int_{\mathscr{M}_{g,n}(L)}f_\gamma(X)dV_{wp}(X)=2^{-M(\gamma)}\int_{\mathbb{R}_+^k}f(|x|)\mathrm{Vol}(\mathscr{M}_{g,n}(\Gamma,x,\beta,L))x\cdot dx.
$$
\end{proof}

  
 
