This paper is about Maryam Mirzakhani’s work in \cite{Mirzakhani:2006fta} and \cite{growthofsimple}.

The moduli space of Riemann surface $\mathscr{M}_{g,n}(L)$ of  genus $g$ and fixed $n$ boundary lengths $L=(L_1,\cdots,L_n)$ admits a Weil--Petersson volume form  $\mu=\frac{1}{(3g-3+n)!}\wedge^{3g-3+n} \omega$, with $\omega$ the Weil--Petersson symplectic  form. Mirzakhani gave a new method to show the volume of $\mathscr{M}_{g,n}(L)$  is a polynomial of degree $6g-6+2n$ with respect to $L_1,\cdots,L_n$.

Mirzakhani generalized  the McShane identity\cite{McShane1998SimpleGA} for Riemann surfaces with cusps to Riemann surfaces with geodesic boundary components, and integrate it over the moduli space using  the Mirzakhani's integration formula. This degenerates to a recursion formula with respect to the pair $(g,n)$.

In \cite{growthofsimple} Mirzakhani proved that the number of simple closed geodesics on $X\in \mathscr{M}_{g,n}(L)$ shorter than $L$  has the asymptotic behavior of $c(X)L^{6g-6+2n}$ as $L\to \infty$.

Generally, the number of  rational  multi-curves of type $\gamma=\sum_ic_i\gamma_i$ shorter then $L$  has the polynomial asymptotic behavior $c(\gamma)L^{6g-6+2n}$.

$c(\gamma)$ is related to the coefficients of the volume polynomial of $\mathscr{M}_{g,n}(L)$, for particular $(g,n)$'s it has explicit
expression, while for general $(g,n)$  the concrete coefficients  are unclear.

For details of the Teichm\"uller space and the muduli space one may refer to \cite{Imayoshi1992An}. For basic formula of hyperbolic geometry one may refer to \cite{Buser}.
The tool of geodesic lamination is included, it can be referred in  \cite{traintracks} and \cite{laminations}.








 
 
 

 
  
  
  








