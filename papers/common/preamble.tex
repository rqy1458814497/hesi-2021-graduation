\usepackage{geometry}
\geometry{
    paperwidth = 155mm,
    paperheight = 235mm,
    outer = 20mm,
    inner = 20mm,
    top = 25mm,
    bottom = 20mm
}

% fonts & unicode
\usepackage[PunctStyle=kaiming]{xeCJK}
\usepackage{amsmath}
\usepackage{unicode-math}

\setCJKmainfont{NotoSerifCJKsc-Regular.otf}[
    Path            = ../../fonts/,
    BoldFont        = NotoSansCJKsc-Medium.otf,
    ItalicFont      = fzktk.ttf,
    Scale           = .97,
    ItalicFeatures  = {Scale = 1}
]

\setCJKsansfont{NotoSansCJKsc-DemiLight.otf}[
    Path            = ../../fonts/,
    BoldFont        = NotoSansCJKsc-Bold.otf,
    Scale           = .97
]

\setCJKmonofont{NotoSansCJKsc-DemiLight.otf}[
    Path            = ../../fonts/,
    BoldFont        = NotoSansCJKsc-Bold.otf,
    Scale           = .9
]

\newCJKfontfamily{\KaiTi}{fzktk.ttf}[
    Path            = ../../fonts/,
    BoldFont        = NotoSansCJKsc-Medium.otf,
    BoldFeatures    = {Scale = .97},
    ItalicFont      = NotoSerifCJKsc-Regular.otf,
    ItalicFeatures  = {Scale = .97}
]

\setmainfont{XITS}[
    Path            = ../../fonts/,
    Extension       = .otf,
    UprightFont     = *-Regular,
    BoldFont        = *-Bold,
    ItalicFont      = *-Italic,
    BoldItalicFont  = *-BoldItalic
]

\setsansfont{Lato}[
    Path            = ../../fonts/,
    Scale           = MatchUppercase,
    Extension       = .ttf,
    UprightFont     = *-Regular,
    BoldFont        = *-Bold,
    ItalicFont      = *-Italic,
    BoldItalicFont  = *-BoldItalic
]

\setmonofont{FiraMono}[
    Path            = ../../fonts/,
    Scale           = .9,
    Extension       = .otf,
    UprightFont     = *-Regular,
    BoldFont        = *-Bold
]

\setmathfont{XITSMath-Regular.otf}[
    Path            = ../../fonts/,
    BoldFont        = XITSMath-Bold.otf
]

\setmathfont{latinmodern-math.otf}[
    Path            = ../../fonts/,
    range           = {frak, bffrak},
    BoldFont        = latinmodern-math.otf
]

\setmathfont{LatoMath.otf}[
    Path            = ../../fonts/,
    Scale           = .95,
    BoldFont        = LatoMath.otf,
    version         = sf
]

\setmathfont{LatoMath.otf}[
    Path            = ../../fonts/,
    Scale           = .95,
    BoldFont        = LatoMath.otf,
    range           = {bb, sfup -> up, sfit -> it, bfsfup -> bfup, bfsfit -> bfit}
]

\setmathfont{STIX2Math.otf}[
    Path            = ../../fonts/,
    BoldFont        = STIX2Math-Bold.otf,
    range           = {\int, \sum, \prod, \coprod, \bigoplus, \bigotimes, \bigcup, \bigcap, \bigvee, \bigwedge}
]

\Umathcode`/  =  "0 "0 "2215    % / -> U+2215 division slash

% patch 'text math' math alphabets in bold math
\setmathfontface\mathrm{XITS-Bold.otf}[
    Path            = ../../fonts/,
    version         = bold
]

\setmathfontface\mathit{XITS-BoldItalic.otf}[
    Path            = ../../fonts/,
    version         = bold
]

\setmathfontface\mathbf{XITS-Bold.otf}[
    Path            = ../../fonts/,
]

\setmathfontface\mathtt{FiraMono-Bold.otf}[
    Path            = ../../fonts/,
    Scale           = .9,
    version         = bold
]

\setmathfontface\mathrm{Lato-Regular.ttf}[
    Path            = ../../fonts/,
    Scale           = MatchUppercase,
    version         = sf
]

\setmathfontface\mathit{Lato-Italic.ttf}[
    Path            = ../../fonts/,
    Scale           = MatchUppercase,
    version         = sf
]

\setmathfontface\mathbf{Lato-Bold.ttf}[
    Path            = ../../fonts/,
    Scale           = MatchUppercase,
    version         = sf
]

\setmathfontface\mathtt{FiraMono-Regular.otf}[
    Path            = ../../fonts/,
    Scale           = .9,
    version         = sf
]

% bold math in bold text
\usepackage{amsthm}
\makeatletter
    \g@addto@macro\bfseries{\boldmath}
    \def\thmhead@plain#1#2#3{%
        \thmname{#1}\thmnumber{\@ifnotempty{#1}{ }\@upn{#2}}%
        \thmnote{ {\the\thm@notefont{\unboldmath(#3)}}}}
    \let\thmhead\thmhead@plain
\makeatother

% title & abstract
\def\title#1\author#2{%
    \headertitle{#1}
    \vspace*{0mm}
    \begin{center}
        {\sf\LARGE#1\par}
        \vspace{10mm}
        {\large#2}
    \end{center}
    \vspace{10mm}
}
\def\headertitle#1{
    \def\theheadertitle{#1}
}

\renewcommand{\abstractname}{ABSTRACT}
\makeatletter
    \let\endabstract@orig\endabstract
    \def\endabstract{\endabstract@orig\vspace{5mm}}
\makeatother

% section titles
\usepackage{titlesec}
\titleformat*{\section}{\Large\sffamily\mathversion{sf}}
\titleformat*{\subsection}{\large\sffamily\mathversion{sf}}
\titleformat*{\paragraph}{\normalsize\sffamily\mathversion{sf}}

% headers and footers
\usepackage{fancyhdr}
\fancyhf{}
\fancyhead[CE]{\sf\mathversion{sf}\theheadertitle}
\fancyhead[CO]{\sf\mathversion{sf}\nouppercase{\leftmark}}
\fancyhead[LE,RO]{\textbf{\textsf{\thepage}}}
\headsep=8mm
\headheight=6mm

\AtBeginDocument{
    \pagestyle{fancy}\thispagestyle{empty}
}

% spacing
\AtBeginDocument{
    \hfuzz=2pt
    \emergencystretch 2em
    \setlength{\belowdisplayshortskip}{\belowdisplayskip}
}

% environments
\newtheorem{theorem}{Theorem}[section]
\newtheorem{lemma}[theorem]{Lemma}
\newtheorem{corollary}[theorem]{Corollary}
\newtheorem{proposition}[theorem]{Proposition}

\theoremstyle{definition}
\newtheorem{definition}[theorem]{Definition}
\newtheorem{example}[theorem]{Example}
\newtheorem{remark}[theorem]{Remark}
\theoremstyle{plain}

\def\qedsymbol{$◻$}
\def\thmqedhere{\pushQED{\qed}\qedhere\popQED}

\numberwithin{equation}{theorem}

% renew theorem: https://tex.stackexchange.com/q/103013/
\makeatletter
\def\renewtheorem#1{%
    \expandafter\let\csname#1\endcsname\relax
    \expandafter\let\csname c@#1\endcsname\relax
    \gdef\renewtheorem@envname{#1}
    \renewtheorem@secpar
}
\def\renewtheorem@secpar{\@ifnextchar[{\renewtheorem@numberedlike}{\renewtheorem@nonumberedlike}}
\def\renewtheorem@numberedlike[#1]#2{\newtheorem{\renewtheorem@envname}[#1]{#2}}
\def\renewtheorem@nonumberedlike#1{  
\def\renewtheorem@caption{#1}
\edef\renewtheorem@nowithin{\noexpand\newtheorem{\renewtheorem@envname}{\renewtheorem@caption}}
\renewtheorem@thirdpar
}
\def\renewtheorem@thirdpar{\@ifnextchar[{\renewtheorem@within}{\renewtheorem@nowithin}}
\def\renewtheorem@within[#1]{\renewtheorem@nowithin[#1]}
\makeatother

% ref & biblatex
\usepackage[colorlinks,allcolors=black,bookmarksnumbered,linktoc=all]{hyperref}

\def\thesection{\arabic{section}\texorpdfstring{}{.}} % pdf bookmark numbering
\setcounter{secnumdepth}{1} % suppress subsection numbering

\usepackage[style=alphabetic,sorting=anyvt,useprefix=true]{biblatex}
\usepackage{xpatch}
\renewcommand*{\bibfont}{\small}
\DeclareFieldFormat[article]{volume}{\mkbibbold{#1}}
\DeclareFieldFormat[book,inbook]{number}{\mkbibbold{#1}}
\DeclareFieldFormat[article]{number}{(#1)}
\DeclareFieldFormat*{year}{(#1)}
\DeclareFieldFormat{pages}{#1}
\renewbibmacro{in:}{}
\renewbibmacro*{volume+number+eid}{%
    \printfield{volume}%
    \setunit*{\addnbspace}% originally: \setunit*{\adddot}
    \printfield{number}%
    \setunit{\addcomma\space}%
    \printfield{eid}%
}
\xapptobibmacro{author/editor+others/translator+others}{%
    \setunit{\space}%
    \printfield{year}%
    \clearfield{year}%
}{}{}
\xapptobibmacro{author/translator+others}{%
    \setunit{\space}%
    \printfield{year}%
    \clearfield{year}%
}{}{}
    \renewbibmacro*{issue+date}{%
    \printfield{issue}%
    \newunit%
}
\AtBeginBibliography{
    \DeclareFieldFormat{labelalpha}{#1}
    \DeclareFieldFormat{extraalpha}{\mknumalph{#1}}
}
\AtEveryBibitem{
    \ifentrytype{online}{%
        \clearfield{year}%
    }{}
}

% tikz
\usepackage{tikz}
\usepackage{tikz-cd}
\tikzset{
    > = latex
}
\tikzcdset{
    arrow style = tikz,
    arrows = {
        /tikz/line width = .5pt
    },
    diagrams = {
        > = {Straight Barb[scale = 0.8]}
    },
    nodes = {
        inner xsep = 3pt,
        inner ysep = 3pt
    }
}

% Make total pages even
\usepackage[strict]{changepage}

\AtEndDocument{%
  \checkoddpage\ifoddpage\newpage\mbox{}\thispagestyle{empty}\fi
}
