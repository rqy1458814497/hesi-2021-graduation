\subsection{Zeta integrals of automorphic forms}
In this subsection, we denote $G/Z(G)$ by $\overline G$. 
For $H$ is a unimodular closed subgroup of a unimodular locally compact group $G$, there is a unique $G$-invariant measure on the quotient space $H\backslash G$ and Fubini theorem holds. 
Moreover, there is a nontrivial fact that $\bar G(\mathbb Q)$ is a discrete subgroup of $\bar G(\mathbb A)$. Thus we have 
\begin{equation}\label{im2}
\int\limits_{\bar G(\mathbb A)} f(g) d^*g=\int\limits_{\bar G(\mathbb Q)\backslash \bar G(\mathbb A)} \sum\limits_{\xi \in \bar G(\mathbb Q)} f(\xi g) d^*g.
\end{equation}
Let $G^1$ denote the group
\[
\left \{ g\in G(\mathbb A); \left  | \det(g)\right | =1 \right \}.
\]
One can check that 
\[
\bar G(\mathbb Q)\backslash \bar {G^1}\to \bar G(\mathbb Q)\backslash \bar G(\mathbb A)
\]
is in fact an isomorphism. Thus we have 
\begin{equation}\label{im}
\int\limits_{\bar G(\mathbb Q)\backslash \bar G(\mathbb A)} f(g) d^* g=
\int\limits_{\bar G(\mathbb Q)\backslash \bar {G^1}} f(g) d^* g.
\end{equation}
For the detail, we recommend chapter 7 of \bl{\cite{A-C}}.


\begin{definition}[\textbf{Zeta integrals of automorphic form}]
Let $f$ be a cuspidal form. Then we define
\[
\zeta(s,\Phi,f)=\int\limits_{G(\mathbb A)} \Phi(g) f(g) \left |\det (g) \right |^s d^*g.
\]
\end{definition}



For $\Re(s)>4$, we have
\begin{equation*}
\begin{split}
\zeta(s,\Phi,f)&\= \int\limits_{ \overline {G}(\mathbb A)}\int\limits_{\mathbb I} \Phi(ag) \omega(a) f(g) \left |\det (ag) \right |^sd^*ad^*g\\
&\=\limits^{\bl{\ref{im2}}}  \int\limits_{\overline G(\mathbb Q) \backslash \overline G(\mathbb A)}\left ( \sum\limits_{\xi\in \mathbb Q^*\backslash G(\mathbb Q)}  \int\limits_{\mathbb I} \Phi(a\xi g) \omega(a) f(\xi g)\left |\det (a\xi g) \right |^s d^*a \right ) d^*g\\
&\=  \int\limits_{\overline G(\mathbb Q)\backslash \overline G(\mathbb A)}\left (   \int\limits_{\mathbb Q^*\backslash \mathbb I} \sum\limits_{\xi\in G(\mathbb Q)}\Phi(a\xi g) \omega(a) \left | a \right |^{2s}d^*a \right ) f(g)\left |\det ( g) \right |^sd^*g\\
&\=\limits^{\bl{\ref{im}}} \int\limits_{\overline G(\mathbb Q)\backslash \overline {G^1}}f(g)\left (\theta^0_2 (s,{^g \Phi},\omega) +\theta^1_2 (s,{^g \Phi},\omega)\right )d^*g
\end{split}
\end{equation*}
Since functions $g\mapsto f(g) \theta^i_j(s,{^g \Phi} ,\omega)$ and $g\mapsto f(g)\theta^i_j(s,{ \Phi^{g}} ,\omega^{-1})$ are left $G(\mathbb Q)Z(G^1)$-invariant and continuous, By $\bl{\ref{eq1}}$ we have
\begin{equation}\label{eq2}
\begin{split}
\zeta(s,\Phi,f)=&\int\limits_{\overline G(\mathbb Q)\backslash \overline G^1}f(g)\theta^0_2 (2s,{^g \Phi},\omega)d^*g\\
&+\int\limits_{\overline G(\mathbb Q)\backslash \overline {G^1}}f(g)\theta^0_2 (4-2s,{ \widehat{\Phi}^{g}},\omega^{-1})d^*g\\
&+\int\limits_{\overline G(\mathbb Q)\backslash \overline {G^1}}f(g)\theta^0_1 (4-2s,{\widehat{\Phi}^{g}},\omega^{-1})d^*g\\
&-\int\limits_{\overline G(\mathbb Q)\backslash \overline {G^1}}f(g)\theta^1_1 (2s,{^g \Phi},\omega)d^*g\\
&+\widehat \Phi(0)\lambda(2s-4,\omega)\int\limits_{\overline G(\mathbb Q)\backslash \overline G^1}f(g)d^*g\\
&-\Phi(0)\lambda(2s,\omega)\int\limits_{\overline G(\mathbb Q)\backslash \overline {G^1}} f(g)d^*g.
\end{split}
\end{equation}
Now we prove the last four terms on the right side of $\bl{\ref{eq2}}$ vanish.

\begin{lemma}\label{58} The following two statements hold:
\begin{enumerate}
  \item If $\omega$ is a principle character, i.e. $\omega$ is trivial on $\mathbb I^1=\left \{ a\in \mathbb I, \left |a \right |=1\right \}$. Then
\[ \int\limits_{\bar G(\mathbb Q)\backslash \bar {G^1}} f(g)d^*g=0. \]
\item If $\omega$ is not principle, then $\lambda(s,\omega)=0$ for any $s$.
\end{enumerate}
\end{lemma}

\begin{proof}
We give a sketch of the first statement. Let $L^2(\bar G(\mathbb Q)\backslash \bar {G}(\mathbb A), \omega)$ be the space of all functions $\phi$ on $G(\mathbb Q)\backslash G(\mathbb A)$ such that
\begin{enumerate}
\item $\phi(ag)=\omega(a) \phi(g)$;
\item) $\int\limits_{\bar G(\mathbb Q)\backslash \bar G(\mathbb A)} \left |\phi(g)\right |^2d^*g \le +\infty$.
\end{enumerate}
This space is a unitary representation with left transformation.  And cuspidal functions span a closed and right $G(\mathbb A)$-invariant subspace $L_0^2(\bar G(\mathbb Q)\backslash \bar {G}(\mathbb A), \omega)$.  If $\omega$ is principal, then there is a principal character $\omega_0$ of $\mathbb Q^*\backslash \mathbb I$ such that $\omega_0^2=\omega$. We can see that $\omega_0\circ \det \in L^2(\bar G(\mathbb Q)\backslash \bar {G}(\mathbb A), \omega)$ and $\mathbb C\omega_0\circ \det$ is right $G(\mathbb A)$-invariant and thus $\left \langle \phi,\omega_0\circ \det \right \rangle=0$ for any cuspidal forms $\phi$. Then we obtain
\begin{equation*}
\begin{split}
0=\int\limits_{\bar G(\mathbb Q)\backslash \overline {G}(\mathbb A)} \phi(g) \overline { \omega_0(\det (g))} d^*g\=\limits^{\bl{\ref{im}}} \int\limits_{\bar G(\mathbb Q)\backslash \bar {G^1}} \phi(g)d^*g.
\end{split}
\end{equation*}

If $\omega$ is not principal. Then $\omega$ is nontrivial on the compact subgroup $\mathbb Q^*\backslash \mathbb I^1$. Thus
\[
\int\limits_{\mathbb Q^*\backslash \mathbb I^1} \omega(a) d^*a=0
\]
which implies $\lambda (s,\omega)=0$.
\end{proof}

\begin{proposition}\label{59}
\[
\int\limits_{\bar G(\mathbb Q)\backslash \bar {G^1}}f(g)\theta^i_1 (s,{^g \Phi},\omega)d^*g=0.
\]
\end{proposition}

\begin{proof}
Let $\gamma\in M(\mathbb Q)$ with rank $1$ and $\mathrm {St}(\gamma)$ be the group $\left \{ \xi\in G(\mathbb Q); \gamma\xi=\gamma \right \}$. Then the function
\[
F_\gamma \colon g\mapsto f(g)\int\limits_{\mathbb Q^*\backslash \mathbb I} \omega(a)a^{2s} F_i(\left |a \right |) d^*a\sum \limits_{\xi\in\mathrm {St} (\gamma)\backslash G(\mathbb Q) }\Phi(\gamma\xi ga)
\]
is left $G(\mathbb Q)Z(G^1)$-invariant. It is sufficient to show
\[
\int\limits_{\bar G(\mathbb Q)\backslash \bar {G^1}} F_\gamma(g) dg =0.
\]
Define the groups
\[ P=\left ( \begin{array} {cc}
1 & *\\
0 & *
\end{array} \right ), \
U=\left ( \begin{array} {cc}
1 & *\\
0 & 1
\end{array} \right )
\]
Suppose
\[
\gamma=\gamma_1\omega \gamma_2, \ \omega=\left ( \begin{array} {cc}
0 & 1\\
0 & 0
\end{array} \right )
\]
We can see $\mathrm {St}(\gamma)=\gamma_2^{-1} P(\mathbb Q) \gamma_2$. Thus we obtain
\begin{equation*}
\begin{split}
&\int\limits_{\bar G(\mathbb Q)\backslash \bar {G^1}} F_\gamma(g) dg\\
&\=\int\limits_{\bar G(\mathbb Q)\backslash \bar {G^1}}f(g)\int\limits_{\mathbb Q^*\backslash \mathbb I} \omega(a)\left |a \right |^{2s} F_i(\left |a \right |) d^*a\sum \limits_{\xi\in P(\mathbb Q)\backslash G(\mathbb Q) }\Phi(\gamma_1\omega\xi ga)d^*g\\
&\= \int\limits_{\mathbb Q^*\backslash \mathbb I} \int\limits_{G(\mathbb Q)\backslash \bar {G^1}}f(g)\omega(a)\left |a \right |^{2s} F_i(\left |a \right |) d^*a\sum \limits_{\xi\in P(\mathbb Q)\backslash G(\mathbb Q) }\Phi(\gamma_1\omega\xi ga)d^*g\\
\end{split}
\end{equation*}
Consider the inner integral:
\begin{equation*}
\begin{split}
&\int\limits_{G(\mathbb Q)Z(G^1)\backslash G^1}f(g)\sum \limits_{\xi\in P(\mathbb Q)\backslash G(\mathbb Q) }\Phi(\gamma_1\omega\xi ga)d^*g\\
&\= \int\limits_{P(\mathbb Q)\backslash \bar {G^1}}f(g) \sum\limits_{\xi\in Z(G(\mathbb Q))}\Phi(\gamma_1\omega\xi ga)d^*g\\
&\=  \int\limits_{P(\mathbb Q)\backslash \bar {G^1}}f(g)\sum\limits_{\xi\in Z(G(\mathbb Q))}\Phi(\gamma_1\omega\xi ga)d^*g\\
&\=  \int\limits_{U(\mathbb A)P(\mathbb Q)\backslash \bar {G^1}}\int\limits_{U(\mathbb Q)\backslash U(\mathbb A) }f(ug) \sum\limits_{\xi\in Z(G(\mathbb Q))}\Phi(\gamma_1\omega\xi uga)dud^*g\\
&\=\limits^{\omega u= \omega}  \int\limits_{U(\mathbb A)P(\mathbb Q)\backslash \bar {G^1}}\left (\int\limits_{U(\mathbb Q)\backslash U(\mathbb A) }f(ug)du \right )\sum\limits_{\xi\in Z(G(\mathbb Q))}\Phi(\gamma_1\omega\xi ga)d^*g\\
&\=\limits^{\bl{\ref{cuspidal}}} 0
\end{split}
\end{equation*}
Then the statement holds.
\end{proof}

\begin{theorem}
$\zeta(s,\Phi,f)$ can be continued analytically in the whole plane as an entire function. And it satisfies the functional equation:
\begin{equation}\label{eq4}
\zeta(s,\Phi,f)=\zeta(2-s,\widehat \Phi, \check f),
\end{equation}
where $\check f(g)=f(g^{-1})$.
\end{theorem}

\begin{proof}
Combine $\bl{\ref{eq2}}$, $\bl{\ref{58}}$, $\bl{\ref{59}}$. We obtain
\begin{equation}
\begin{split}
\zeta(s,\Phi,f)=&\int\limits_{\bar G(\mathbb Q)\backslash \bar {G^1}}f(g)\theta^0_2 (2s,{^g \Phi},\omega)d^*g\\
&+\int\limits_{\bar G(\mathbb Q)\backslash \bar {G^1}}f(g)\theta^0_2 (4-2s,{ \widehat{\Phi}^{g}},\omega^{-1})d^*g.
\end{split}
\end{equation}
By $\bl{\ref{62}}$, $\zeta(s,\Phi,f)$ is an entire function.
Similarly we have
\begin{equation*}
\begin{split}
\zeta(2-s,\widehat \Phi,\check f)&=\int\limits_{ \bar {G^1}/\bar G(\mathbb Q)}f(g^{-1})\theta^0_2 (4-2s,{\widehat \Phi}^{g^{-1}},\omega^{-1})d^*g\\
& \ +\int\limits_{\bar {G^1}/\bar G(\mathbb Q)}f(g^{-1})\theta^0_2 (2s,{^{g^{-1}}}{ {\Phi}},\omega)d^*g\\
&\=\limits^{g\mapsto g^{-1}} \int\limits_{\bar G(\mathbb Q)\backslash \bar {G^1}}f(g)\theta^0_2 (4-2s,{\widehat \Phi}^g,\omega^{-1})d^*g\\
& \ +\int\limits_{\bar G(\mathbb Q)\backslash \bar {G^1}}f(g)\theta^0_2 (2s,{^g} \Phi,\omega)d^*g\\
&= \zeta(s,\Phi,f).
\end{split}
\end{equation*}
Then the functional equation holds.
\end{proof}
