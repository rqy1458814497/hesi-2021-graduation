\subsection{\texorpdfstring{$L$}{L}-factors of unitary unramified representations}

In the previous subsection, we have shown that 
\[
\frac {\zeta (s+\frac 1 2,\Phi, \beta )} {L(s,\chi_1) L(s,\chi_2) }\in \mathbb C[q^s,q^{-s}]. 
\]
where $\beta$ is the matrix coefficient of the infinite-dimensional irreducible subrepresentation of $\mathcal B(\chi_1,\chi_2)$. However, we do not show the fact that there is a suitable $\Phi$ such that 
\[
{\zeta (s+\frac 1 2,\Phi, \beta )} ={L(s,\chi_1) L(s,\chi_2) }.
\]
In this subsection, we prove this fact for $V$ is unramified and unitary.

\begin{definition}
An irreducible admissible representation $V$ of $G$ is said unramified if $V^K\ne 0$. 
\end{definition} 

\begin{lemma}
An infinite-dimensional irreducible admissible representation $V$ of $G(F)$ is unramified if and only if
\[
V\cong \mathcal B(\chi_1,\chi_2),
\]
where $\mathcal B(\chi_1,\chi_2)$ is irreducible and $\chi_i=\left | \ \right |^{t_i}$ are unramified. And if $V$ is unramified, we have $\operatorname {dim} V^K=1$.
\end{lemma}

\begin{proof}
See theorem 3.3 of \bl{\cite{B-G}}.
\end{proof}

Moreover, if $V$ is unitary, we can see 
\[
V\cong \mathcal B(\left | \ \right |^t, \left |  \ \right |^{-t})
\]
for some complex number $t\ne \pm 1/2$.

\begin{proposition}\label{6.3.1}
The $L$-factor of $V$ is given by 
\begin{equation*}
	\begin{split}
	L(s,\pi)=L(s,\left| \ \right |^t)L(s,\left | \ \right |^{-t})=\frac 1 {(1-q^{-t-s}  )(1-q^{t-s})}.
\end{split}
\end{equation*}
Moreover, let $\Phi$ be the characteristic function of $M(\fr o)$ and $f\in V^K$ such that $\left\langle  f, f \right \rangle =1$. Then \begin{equation*}
\zeta (s+\frac 1 2 ,\Phi_v,\beta_{f,f})=L(s,\pi).
\end{equation*}
\end{proposition}

\begin{proof}
Let $\Phi_v$ be the characteristic function of $M(\fr o_v)$. Recall $\bl{(\ref{1.3.1})}$ we have
\begin{equation*}
\zeta (s+\frac 1 2,\Phi_v,\beta_{f,f})=\int\limits _{F^*}\int\limits_{F^*} f_{\Phi_v}(a_1,a_2) |a_1|^{t+s}|a_2|^{s-t} d^*a_1d^*a_2,
\end{equation*}
where 
\[
f_\Phi(a_1,a_2) =\int \limits _ {K}\int\limits _{K} \int\limits _{F}\Phi\left ( k_2^{-1}\left( \begin{array}{cc}
a_1 & x\\
0 & a_2
\end{array}
\right ) k_1 \right )f(k_1)f(k_2)d^*k_1d^*k_2dx.
\]
Since $f\in V^K$ and $\left \langle f,f \right \rangle =1$, we have $f(k)=\pm\frac 1 {\mathrm{V} (K, d^*k)}$. Thus
\begin{equation*}
	\begin{split}
	f_\Phi(a_1,a_2) &=\frac 1 {\mathrm {V}(K, d^*k)^2}\int \limits _ {K}\int\limits _{K} \int\limits _{F}\Phi\left ( k_2^{-1}\left( \begin{array}{cc}
a_1 & x\\
0 & a_2
\end{array}
\right ) k_1 \right )d^*k_1d^*k_2dx.\\
&\=\limits^{^{k_1} \Phi ^{k_2} =\Phi} \Phi \left( \begin{array}{cc}
a_1 & x\\
0 & a_2
\end{array}
\right )dx\\
&=\Phi_0(a_1)\Phi_0(a_2).
\end{split}
\end{equation*}
where $\Phi_0$ is the characteristic function of $\fr o_v$. Apply to $\zeta (s,\Phi,\beta )$ we get
\[
\zeta (s,\Phi ,\beta )=\zeta(s,\Phi_0,\left| \ \right |^t)\zeta(s,\Phi_0,\left | \ \right |^{-t})=L(s,\left| \ \right |^t)L(s,\left | \ \right |^{-t}).
\]
Thus the statement holds.
\end{proof}
