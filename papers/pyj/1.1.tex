\subsection{Irreducible admissible representations}
Let $G=\mathrm {GL}_2(F)$ where $F$ is a $p$-adic field. And let $M$ denote $\mathrm {Mat } _{2\times 2} (F)$. Let $S(M)$ be the space of the locally constant functions with compact support. 
Here we give a brief introduction to the irreducible admissible representation of $G$.

\begin{definition}[\textbf{Smooth representations}]
A representation $(\pi,V)$ of $G$ is called smooth if the map $g\mapsto \pi(g)v$ is locally constant for any $v\in V$.
\end{definition}


Suppose $H$ is a subgroup of $G$, let $V^H$ be the space of those vectors fixed by $H$.

\begin{definition}[\textbf{Admissible representations}]
A smooth representation $V$ of $G$ is called admissible if $V^H$ is finite-dimensional for any open compact subgroup $H$ of $G$.
\end{definition}

\begin{definition}[\textbf{Cuspidal representations}]
An irreducible admissible representation $(\pi,V)$ of $G$ is said cuspidal if for any sufficiently large $n$,
\[
\int_{\fr p^{-n}} \pi\left (
\begin{array}{cc}
	1 & u \\
	0 & 1
\end{array} \right ) vdu=0.
\]
The integral is well defined since $\fr p^{-n}$ is open compact.
\end{definition}

\begin{definition}[\textbf{Principal series representations}]
Let $\chi_1,\chi_2$ be the quasi-characters of $F^*$. Define $\mathcal B(\chi_1,\chi_2)$ to be the space of the smooth function $f$ on $G$ such that 
\begin{equation}\label{eq1.1.1}
f\left ( \left ( \begin{array} {cc}
a & b\\
0 & d
\end{array} \right ) g\right )
= \left | \frac a d \right | ^{\frac 1 2} \chi_1(a) \chi_2(d) f(g).
\end{equation}
$\mathcal B(\chi_1,\chi_2)$ is a representation of $G$ by the left transformation.
\end{definition} 

\begin{proposition}
$\mathcal B(\chi_1,\chi_2)$ is an irreducible admissible representation. 
\end{proposition}

\begin{definition}[\textbf{Irreducible principal series representations}]
An irreducible representation which can be realized as $\mathcal B(\chi_1,\chi_2)$ is called irreducible principle representation.
\end{definition}

\begin{definition}[\textbf{Special representations}]
An infinite-dimensional irreducible representation that can be realized as a proper subrepresentation of $\mathcal B(\chi_1,\chi_2)$ is called spacial representation.
\end{definition} 

\begin{theorem}[\textbf{Classification of irreducible admissible representations}]
An irreducible admissible representation with infinite dimension is exactly one of the following:\\
\begin{enumerate}
  \item cuspidal representations;
  \item Irreducible principle representations;
  \item Special representations.
\end{enumerate}
\end{theorem}

\begin{proof}
See 6.16.1 of \bl{\cite{G-H}} or 9.11 of \bl{\cite{B-H}} for a general case.
\end{proof}
