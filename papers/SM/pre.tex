\subsection{Basic definitions}

Firstly, we want to learn the properties of the group of
isometries $I(M)$ of a manifold $M$. The goal is to reach the
identification between $M$ and $I(M)/I(M)_p\cong
I_0(M)/I_0(M)_p$, where the notations can be found in the
appendix or the following context.

We begin with basic definitions on Lie groups and Lie algebras.
If the reader is familiar with Lie theory, this section can be
skipped. Those who are not can simply read this section to get to
the advanced part.
\begin{definition}
	A Lie group is a group $G$ which is also a differential
	manifold such that the map $G\times G\to G,\
	(\sigma, \tau)\mapsto \sigma\tau^{-1}$ is smooth. A
	homomorphism of Lie groups is a homomorphism of groups which
	is smooth.
\end{definition}

\begin{definition}
	A Lie algebra $\mathfrak{g}$ over a field $k$ is a vector
	space over $k$  with a bilinear map
	$\mathfrak{g}\times\mathfrak{g}\to \mathfrak{g},\
	(X, Y)\mapsto [X, Y]$  such that
	
  \begin{enumerate}
    \item $[X,X]=0,\ \forall X\in \mathfrak{g} $,
    
    \item $[X, [Y, Z]]+[Y, [Z, X]]+[Z, [X, Y]]=0,\ \forall X, Y, Z\in \mathfrak{g}$.
  \end{enumerate}
	
	This bilinear map is also called the Lie bracket.
\end{definition}

\begin{theorem}
	Let $G$ be a Lie group. Then the tangent space of $G$ at $e$
	forms a Lie algebra over $\bbR$, usually called the Lie algebra
	of $G$, denote by Lie $G$, with the Lie bracket defined as
	$[X,Y]\\coloneq [\tilde X, \tilde Y]$, where $\tilde X, \tilde Y$ are
	the induced left-invariant vector fields by $X, Y$, and the
	Lie bracket is the standard one for vector fields.
\end{theorem}

\begin{definition}
	Let $\ssg$ be a Lie algebra over field $k$. A Lie subalgebra
	of it $\ssl$ is a vector space  with $[\ssl, \ssl]\in \ssl$,
	where $[\ssm, \ssn]$ is the vector space generated by all the
	elements $[X, Y]$, $X\in \ssm, Y\in \ssn$.
\end{definition}

\begin{definition}
	A homomorphism of Lie algebras from $\ssg$ to $\ssm$ is a
	linear map $f\:\ssg\to \ssm$ such that $[fX, fY]=f[X, Y]$.
\end{definition}
\begin{definition}
	Let $G$ be a Lie group and $ \nabla$ be an affine connection
	on $G$. $\nabla$ is called left invariant on $G$ is for any
	vector fields $X,Y$ on $G$ and $g\in G$, there is an equality
	$\nabla_{dL_gX}(dL_gY)=dL_g(\nabla_XY)$.
\end{definition}
\begin{lemma}
	Every left-invariant vector field on a Lie group is complete.
\end{lemma}
\bproof
See \cite{Lee} Cor 9.17.
\eproof
\subsection{The exponential map}

Next we define the exponential map on a Lie group.

\begin{lemma}\label{8}
	There is a one-to-one correspondence between the set of
	left-invariant affine connections $\nabla$ on $G$ and the set
	of bilinear functions $\alpha$ on $\ssg\times\ssg$ with
	values in $\ssg$ given by
	\[ \alpha(X, Y)=(\nabla_{\tilde X}\tilde Y)_e \]
	where $\tilde X, \tilde Y$ are the induced left-invariant vector fields by
	$X, Y$. Moreover, for a given bilinear function $\alpha$(and
	left invariant affine connection $\nabla$) and $X\in \ssg$,
	the following are equivalent:
	
  \begin{enumerate}
	  \item $\alpha(X, X)=0$;
	
	  \item The geodesic $t\to \gamma_X(t)$ is a smooth homomorphism
	of $\bbR$ into $G$.($\gamma_X(t)$ indicates the geodesic
	$\gamma(t)$ with $\gamma(0)=e, \gamma'(0)=X$.)
  \end{enumerate}
	\bproof
	Given a bilinear mapping $\alpha$, we defined the affine
	connection by
	\[
	\nabla_{\tilde{E_i}}\tilde{E_j}=\tilde{\alpha(E_i,E_j)},
	1\le i,j\le n
	\]
	where $\{E_i\}_{i=1}^n$ is an arbitrary basis of the tangent
	space of $G$ at $e$. Then the correspondence follows.
	
	Next it comes to the equivalence. If $\alpha(X,X)=0$, then
	$(\nabla_{\tilde X}\tilde{X})_e=0$, and $\nabla_{\tilde
	X}\tilde{X}=0$ by the left invariance of $\nabla$.
	
	Let $\gamma\:\bbR\to G$ be the integral curve of $\tilde{X}$
	starting at $e$. Then $\gamma $ is a geodesic. For any $s\in
	R$, the curves $t\to \gamma(s+t)$ and $t\to
	\gamma(s)\gamma(t)$ are both geodesics since $\nabla$ is left
	invariant. These two curves have tangent vectors $\gamma'(s)$
	and $dL_{\gamma(s)}X$ at $t=0$. So by uniqueness,
	$\gamma(s+t)=\gamma(s)\gamma(t)$. On the other hand, suppose
	$\gamma_X\:\bbR\to G$ is a smooth homomorphism, then by
	$\gamma_X(s+t)=\gamma_X(s)\gamma_X(t)$, we have
	$\gamma_X'(s)=\tilde X_{\gamma_X(s)}$ for all $s\in \bbR$. So
	$\nabla_{\tilde X}\tilde{X}=\nabla_{\gamma_X'}\gamma_X'=0$ on
	$\gamma_X$ and $\alpha(X,X)=(\nabla_{\tilde
	X}\tilde{X})_e=0$.
	\eproof
\end{lemma}
\begin{corollary}\label{9}
	Let $G$ be a Lie group and $\ssg$ be its Lie algebra. For any
	$X\in \ssg$, there exists a unique smooth homomorphism
	$\theta\:\bbR\to G$ such that $\theta'(0)=X$.
\end{corollary}
\bproof
Let $\alpha$ be a bilinear form on $ssg$ such that
$\alpha(X,X)=0$ and $\nabla$ is the left invariant affine
connection correspondingly. Let $\gamma_X\:\bbR\to G$ be the
geodesic as in lemma $\ref{8}$. Then, for any smooth homomorphism
$\theta$ with $\theta'(0)=X$, similarly as the proof of lemma
$\ref{8}$, we have $\theta(s+t)=\theta(s)\theta(t)$. Then
$\theta'(t)=dL_{\theta(t)}\theta'(0)=dL_{\theta(t)}X$ and
$\nabla_{\theta'}\theta'=\nabla_{\tilde X}\tilde
X=dL_{\theta(t)}(\nabla_{\tilde X}\tilde X)_e=0$. Hence $\theta$
is a geodesic with $\theta'(0)=X$, so by uniqueness of geodesic,
$\theta=\gamma_X$.
\eproof
\begin{definition}
	Let $G$ be a Lie group and $\ssg$ be its Lie algebra. The
	exponential map $\exp\:\ssg\to G$ is defined to be $\exp X =
	\theta_X(1)$, where $\theta_X$ is the unique smooth
	homomorphism from $\bbR$ to $G$ such that $\theta_X'(0) = X$.
\end{definition}

By now we have known the definition of an exponential map on a
Lie group. The following lemma is a basic and useful property
related.

\begin{lemma}
	Let $H$ and $K$ be Lie groups with Lie algebras $\ssh$ and
	$\ssk$ respectively. Let $\phi$ be a smooth homomorphism from
	$H$ to $K$. Then $d\phi_e$ is a homomorphism from $\ssh$ to
	$\ssk$ and
	\[\phi(\exp X)=\exp d\phi_e(X), \forall X\in \ssh.\]
\end{lemma}
\bproof
Let $X \in \ssh$. The mapping $t \to \phi(\exp tX)$ is a smooth
homomorphism of $\bbR$ into $K$. If we put $X' = d\phi_e(X)$,
corollary $\ref{9}$ implies that $\phi(\exp tX) = \exp tX'$ for
all $t \in R$. Since $\phi $ is a homomorphism, we have
$\phi(\sigma\tau) = \phi(\sigma)\phi(\tau)$. Hence $\phi\circ
L_{\sigma}=L_{\phi(\sigma)}\circ\phi$ for $\sigma \in H$. It
follows that $d\phi_{\sigma}\circ
dL_{\sigma}(X)=dL_{\phi(\sigma)}(X')$. Then
$d\phi_{\sigma}(\tilde X_{\sigma})=\tilde X_{\phi(\sigma)}'$.
This means that the left invariant vector fields $\tilde{X}$ and
$\tilde X'$ are $\phi $ related. So $[\tilde X', \tilde
Y'](\phi(\sigma))=d\phi_{\sigma}([\tilde X, \tilde Y](\sigma))$
and $[X', Y']=[d\phi_eX, d\phi_eY]=d\phi_e[X, Y]$. Hence
$d\phi_e$ is a homomorphism between Lie algebras and the lemma is
shown.
\eproof
\subsection{Lie subgroups and subalgebras}

Then it comes to the part of Lie subgroups and Lie subalgebras.
The following are also basic knowledges but are essential for
building the whole theory.

\begin{definition}
	Let $G$ be a Lie group. $H$ is called a Lie subgroup if it is
	an immersed submanifold $H$ such that
	
  \begin{enumerate}
    \item $H$ is a subgroup of group $G$;
    
    \item $H$ is a topological subgroup.
  \end{enumerate}
\end{definition}

 We see that a Lie subgroup is itself a Lie group by definition.

\begin{theorem}
	Let $G$ be a  Lie group and $H$ be its arbitrary Lie
	subgroup. Suppose $\ssg$ and $\ssh$ are their Lie algebras
	respectively,
	then
	 $\mathfrak{h}$ is a subalgebra of $\mathfrak{g} .$ Moreover,
	 each Lie subalgebra of $\ssg$ is the Lie algebra of a unique
	 connected Lie subgroup of $G$.
\end{theorem}
\bproof See \cite{Hel} Theorem 2.1 Chapter II.
\eproof
\begin{theorem}
	Let $G$ be a Lie group,  Lie $G=\ssg$ , and $H$ be a
	subgroup. Suppose $H$ is a closed subset of $G$, then there
	exists a unique smooth
	structure of ${H}$ such that it is a topological subgroup of
	$G$.
\end{theorem}
\bproof
See \cite{Hel} Theorem $2.3$ Chapter II.
\eproof

\begin{definition}
	A topological group $G$ is a group endowed with a topology
	such that the mapping $(\sigma, \tau) \mapsto \sigma
	\tau^{-1}$ from $G \times G$ to $G$ is continuous. A
	topological subgroup $H$ of $G$ is a subgroup endowed with
	the subspace topology such that it is a topological group.
\end{definition}

Next we define Lie transformation groups.
\begin{definition}
Let $M$ be a Hausdorff space. A topological transformation group
$G$ is a topological group such that each $g \in G$ is associated
with a homeomorphism $p \rightarrow g \cdot p$ of $M$ onto itself
such that

\begin{enumerate}
  \item $\left(g_{1} g_{2}\right) \cdot p=g_{1} \cdot \left( g_{2}
\cdot p \right)$,

  \item the mapping $(g, p) \mapsto g \cdot p$ is a continuous
mapping of the product space $G \times M$ to $M$.
\end{enumerate}

If for any $p, q \in M$, there exists $g \in G$ such that $g
\cdot p=q$, then we say $G$ is transitive on $M$.	
\end{definition}
\subsection{quotient manifolds}

The next two theorems shows how we see a coset space $G / H$ as a
topological manifold.

\begin{theorem}
	Let $G$ be a topological group and $H$ a closed topological
	subgroup of $G$. Then the set $G / H$ can be endowed
	with the quotient topology to be a Hausdorff space, and also
	the natural projection $\pi\: G \rightarrow G / H$ is open
	and continuous. $G$ acts naturally on $G / H$ by $\mathrm{g}
	\cdot {\sigma H}={g} \sigma {H}$ and $\mathrm{G}$ is a
	topological transformation group on ${G} /{H}$.
\end{theorem}
\bproof
See \cite{Hel} p120 for example. 
\eproof

\begin{theorem}
	Let $G$ be a locally compact group which is a topological
	manifold. Suppose $G$ is a transitive topological
	transformation group of a locally compact Hausdorff space
	$M$. Let $p$ be any point in $M$ and $H$ the subgroup of $G$
	which leaves $p$ fixed. Then ${H}$ is closed and the mapping
	\[
	g H \rightarrow g \cdot p
	\]
	is a homeomorphism of $G / H$ onto $M$.
\end{theorem}
\bproof
See \cite{Hel} Theorem $3.2$ Chapter II.
\eproof
\begin{definition}
	Let $G$ be a Lie group and $M$ a differential manifold.
	Suppose $G$ is a topological transformation group of $M .\bbG$ is
	called a Lie transformation of $M$ if the mapping $(g, p)
	\mapsto g \cdot p$ is a differentiable mapping from $G \times
	M$ to $M$.
\end{definition}

From the definition we see that a Lie transformation group adds a
differentiable condition from a topological one. The next theorem
shows the equivalence between $G / H$ and a smooth manifold under
this new condition.

\begin{theorem}
Let $G$ be a Lie group and $H$ a closed subgroup.  The space $G / H$
of left cosets $g H$ with the natural topology. Then $G / H$ ha
s a unique smooth structure with the property that $G$ is a Lie
transformation group of $G / H$.	
\end{theorem}
\bproof
See \cite{Hel} Theorem 4.\ 2, Chapter II.
\eproof

\begin{proposition}
	Let $G$ be a transitive Lie transformation group of a
	$C^{\infty}$ manifold $M$. Let $p_{0}$ be a point in $M$ and
	let $G_{p_{0}}$ denote the isotropy group of $G$ that leaves
	$p_{0}$ fixed. Then $G_{p_0}$ is closed. Let $\alpha$ be the
	mapping
	\[
	\alpha\:g G_{p_{0}} \rightarrow g \cdot p_{0} \text { of \
	}\  G / G_{p_{0}} \text { onto } M.
	\]
	If $\alpha$ is a homeomorphism, then it is a diffeomorphism (
	$G / G_{p_{0}}$ having the analytic structure defined above).
\end{proposition}
\bproof
See \cite{Hel} Proposition 4.\ 3 Chapter II.
\eproof
\subsection{The group of isometries}

Finally, we start to learn the properties of $I(M)$ as a Lie
group. To begin with, we need to know its topology. 

\begin{definition}
	Let $M$ be a Riemannian manifold and $I(M)$ be the group of
	all isometries of
	M. The compact open topology on $I(M)$ is defined as the
	smallest topology on $I(M)$ for which all the sets
	\[
	W(C, U)\\coloneq \{ g \in I(M) \mid g (C) \subset U \}
	\]
	where $C \subset M$ is a compact subset and $U \subset M$ is
	an open subset.
\end{definition}
\begin{lemma}
	The space $I(M) $ is second countable.
\end{lemma}
\bproof
See \cite{Hel} Lemma 2.\ 1 Chapter IV.
\eproof
\begin{lemma}
	Assume that a sequence $\left\{ f_{n}
	\right\}_{n=1}^{\infty}$ in $I(M)$ converges pointwisely on a
	set $A \subset M$. Then $\left\{ f_{n}
	\right\}_{n=1}^{\infty}$ also converges pointwisely on
	$\bar{A}$.
\end{lemma}
\bproof
See \cite{Hel} Lemma $2.3$ Chapter IV.
\eproof
\begin{lemma}
	Let $M$ be a Riemannian manifold and $\left\{ f_{n} \right\}$
	be a sequence in $I(M) .$ Suppose there exists a point $p \in
	M$ such that $\left\{ f_{n} (p) \right\}$ converges in $M$.
	Then there exists a subsequence $\left\{f_{n_{k}}\right\}$ of
	$\left\{f_{n}\right\}$ converging pointwisely.
\end{lemma}
\bproof
See \cite{Hel} Theorem $2.2$ Chapter IV.
\eproof
\begin{theorem}
	Let $\left\{ f_{n} \right\}$ be a sequence in $I(M)$, then
	$\left\{ f_{n} \right\}$ converges pointwisely on $M$ if and
	only if $\left\{ f_{n} \right\}$ converges to some $f \in I(M)$
	in the compact open topology.
\end{theorem}
\bproof
See \cite{Hel} Lemma 2.\ 4 Chapter IV.
\eproof
\begin{theorem}
	Let $M$ be a Riemannian manifold. Then $I(M)$ is a locally
	compact topological
	transformation group of $M$ with respect to the compact open
	topology on $I(M)$. Moreover, the stabilizer  $I(M)_p$  is
	compact for all $p \in M$.
\end{theorem}
\bproof
See \cite{Hel} Theorem 2.\ 5 Chapter IV.
\eproof
