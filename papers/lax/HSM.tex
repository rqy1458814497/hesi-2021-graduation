Now let's see the topic of Hermitian symmetric manifold.
\subsection{Basic definitions}
	\bdefinition
Let $M$ be a connected complex manifold with a Hermitian
structure. $M$ is called a Hermitian symmetric manifold if  each
point $p\in M$ is an isolated fixed point of an involutive
holomorphic isometry $s_p$ of $M$.
\edefinition
\btheorem
A Hermitian symmetric manifold is \ka. In particular, irreducible Hermitian symmetric manifolds of semisimple types are simply-connected.
\etheorem
\bproof
Suppose it has the almost complex structure $J$. It suffices to
show $\nabla J=0$. Actually, for any vector fields $X, Y$, we
have
\[
\left(\nabla_{X} J\right)(Y)=\nabla_{X}(J(Y))-J\left(\nabla_{X}
Y\right)
\]
Let $\left\{T_{t}\right\}$ be the transvection of $X$ at $p$.
Then
\[
\begin{aligned}
	\nabla_{X}(J(Y)) &=\left.\frac{d}{d t}\right|_{t=0}\left(d
	T_{-t} J(Y)\right)-J\left(\left.\frac{d}{d
	t}\right|_{t=0}\left(d T_{-t} Y\right)\right) \\
	&=\left.\frac{d}{d t}\right|_{t=0}\left( d T_{-t}
	J(Y)-J\left( d T_{-t} Y \right) \right)=0
\end{aligned}
\]
since $T_{-t}$ is holomorphic for any $t$, hence $d T_{-t}$
commutes with $J$. For the second assertion, if $M$ is of compact
type, then it is a compact Kähler manifold of positive Ricci
curvature, hence is simply-connected. If $M$ is of non-compact
type, it is simply-connected.
\eproof

The following proposition shows when a Riemannian symmetric
manifold can be a Hermitian one.

\begin{proposition}\label{10}
	Let $(M, g)=G / K$ be a Riemannian symmetric manifold. Then
	$M$ admits a hermitian symmetric manifold structure if and
	only if there exists a linear map $ J\: \mathfrak{p}
	\rightarrow \mathfrak{p}$ which satisfies
	
	(a) $J^{2}=-i d$ and $g(J X, J Y)=g(X, Y)$,
	
	(b) $J \circ \operatorname{Ad}(h)=\operatorname{Ad}(h) \circ
	J$ for all $h \in K$.
\end{proposition}
\bproof
Define $J$ to be $J_{g p}=\left( L_{g} \right)_{*}(J) = d\left(
L_{g} \right)_{p} \circ J \circ d\left( L_{g^{-1}} \right)_{g p}
.$ Since $J$ commutes with $\operatorname{Ad} K$, we have $J$ is
well-defined. We claim that $\left( s_{p} \right)_{*} J = J$,
hence $d\left( s_{p} \right) \circ J \circ d\left(s _{p}^{-1}
\right) = J .$ Actually, recall that $s_{g p} = L_{g} \circ s_{p}
\circ L_{g}^{-1}$, we have
$\left( s_{p} \right)_{*} J$ is another almost complex structure
which is G-invariant. Note that
$\left( s_{p} \right)_{*} J=J$ at $p$ since $d s_{p} = -i d$ at
$p$, so $\left( s_{p} \right)_{*} J = J$ globally.

Then for Nijenhuis tensor
\[
N(X, Y)=[J X, J Y]-[X, Y]-J[J X, Y]-J[X, J Y]
\]
we have
\[
\begin{aligned}
	\left(s_{p}\right)_{*} N(X, Y) &=d
	s_{p}\left(N\left(d\left(s_{p}^{-1}\right) X,
	d\left(s_{p}^{-1}\right) Y\right)\right) \\
	&=d\left(s_{p}\right)\left(\left[J d\left(s_{p}^{-1}\right)
	X, J d\left(s_{p}^{-1}\right)
	Y\right]-\left[d\left(s_{p}^{-1}\right) X,
	d\left(s_{p}^{-1}\right) Y\right]-\right.\\
	&\left.-J\left[J d\left(s_{p}^{-1}\right) X,
	Y\right]-J\left[d\left(s_{p}^{-1}\right) X, J
	d\left(s_{p}^{-1}\right) Y\right]\right) \\
	&=[J X, J Y]-[X, Y]-J[J X, Y]-J[X, J Y]=N(X, Y),
\end{aligned}
\]
hence $\left(s_{p}\right)_{*} N=N$. Then for any vector field
$Z$, we have
\[
g(N(X, Y), Z)(p)=g\left(\left(s_{p}\right)_{*} N(X, Y),
Z\right)(p)=g\left(\left(d s_{p}\right) N\left(d s_{p}^{-1} X, d
s_{p}^{-1} Y\right), Z\right)(p)
\]
hence $N(p)=0$. Then $N_{g p}=\left(d s_{p}\right)_{p} \circ
N_{p} \circ\left(d s_{p}^{-1}\right)_{g p}=0$, hence $N \equiv 0$
and so $J$ is integrable.
\eproof
\subsection{de Rham decomposition}

The following shows that the canonical decomposition of
Riemannian situation still holds for Hermitian situation.

\begin{theorem}
Let $(M, g)$ be a simply-connected hermitian symmetric manifold.
Suppose $M=M_{1} \times \cdots \times M_{k}$ be the decomposition
of $M$ into irreducible Riemannian symmetric manifolds. Then each
factor itself is also an hermitian symmetric manifold.	
\end{theorem}
\bproof
Let $T_{p} M=V_{1} \oplus \cdots \oplus V_{k}$ be the
decomposition into irreducible spaces with respect to the action
of $K$. By proposition $\ref{10}$ , we have
$\operatorname{Ad}(h)$ commutes with $J$ for any $h \in K$, then
the subspace $V:=\left\{v+J v \mid v \in V_{i}\right\}$ is
invariant under
K. Note that $J V_{1}, \cdots, J V_{k}$ is a permutation of
$V_{1}, \cdots, V_{k}$ and $V_{1}, \cdots, V_{k}$ are orthogonal
to each other, then $V$ is orthogonal to $\left\{V_{1}, \cdots,
V_{k}\right\}-\left\{V_{i}, J V_{i}\right\}$.

Since any invariant space is the direct sum of irreducible
subspaces, we have that either $V=V_{i}$ or $V=J V_{i}$ by
$\operatorname{dim} V=\operatorname{dim} V_{i}$, then $J
V_{i}=V_{i}$, hence the integral manifold of $M_{i}$ satisfies
the conditions of proposition $\ref{10}$ , hence $M_{i}$ is
an Hermitian symmetric manifold.
\eproof

\begin{corollary}
Let $(M, g)$ be a simply connected Riemannian symmetric manifold.
Then $M$ admits an hermitian symmetric structure if and only if
$M^{*}$ admits.	
\end{corollary}
\bproof
It suffices to show that when $M$ is simply-connected, we have $J
\circ \operatorname{Ad}(h)=$ $\operatorname{Ad}(h) \circ J$ if
and only if $J \circ
\operatorname{ad}(\mathfrak{k})=\operatorname{ad}(\mathfrak{k})
\circ J .$ By taking derivatives,
one direction is clear. For the other direction, suppose J
commutes with
$\mathrm{ad} \mathfrak{l}$, we prove that $J$ commutes with
$\mathrm{Ad} K$.


Since $M$ is simply-connected, we have $K$ is connected and then
$K$ is generated by a neighborhood of the identity element. Then
it suffices to show that $J$ commutes with
$\operatorname{Ad}(\exp v)$, for any $v \in \mathfrak{k}$. Then
since $\operatorname{Ad}(\exp v)=e^{\mathrm{ad} v}=\sum_{i=0}^{\infty} \frac{1}{k !}(\mathrm{ad}
v)^{k}$, we have $\operatorname{Ad}(\exp v)$ commutes with $J .$
\eproof

The following statement shows another way to make a Riemannian
symmetric manifold into a Hermitian one.
\begin{theorem}\label{13}
	Let $(M, g)$ be an irreducible Riemannian symmetric manifold of semisimple type with $G=I_{0}(M)$ and $K=G_{p}$. Then $M$ admits an integrable almost complex structure with respect to which $M$ is a hermitian symmetric manifold
	if and only if $K$ is not semisimple.
\end{theorem}
\bproof

By duality, we can assume that $M$ is of compact type.  If $M=G /
K$ is of non-compact type, then it is simply-connected, then it
suffices to show
the dual manifold is hermitian symmetric if and only if $K^{*}$
is semisimple since $K^{*}$ and $K$ have the same Lie algebras.

If $M$ is Hermitian symmetric, we have $M$ is Kähler and $H_{D
R}^{2}(M) \neq 0$. Since $G$ is connected, then by the long exact
sequence
\[
\cdots \rightarrow \pi_{1}(G) \longrightarrow \pi_{1}(G / K)
\longrightarrow \pi_{0}(K) \longrightarrow \pi_{0}(G)
\longrightarrow \pi_{0}(G / K) \rightarrow 1
\]
and that the fundamental group of a complete manifold is finitely
generated, we have $\pi_{1}(G / K)$ is finitely generated, hence
$\pi_{0}(K)$ is finitely generated and $K$ has finitely many
components, hence $\pi_{0}(K)$ is a finite group.

Note that $\pi_{1}(G)$ is finite since $G$ is compact and
semisimple hence has trivial center, thus $\pi_{1}(M)$ is finite
since the quotient of $\pi_{1}(M)$ by a finite group is a finite
group.

If $\tilde{M}$ is the (finite) universal cover, it is well known
that the DeRham cohomology of $M$ is the DeRham cohomology of
$\tilde{M}$ invariant under the deck group. Thus $H_{D
R}^{2}(\tilde{M}) \neq 0$ as well. By applying Hurewicz theorem,
we have $\mathbb{Z} \subset \pi_{2}(\tilde{M})=H_{2}(\tilde{M},
\mathbb{Z})$ since $\tilde{M}$ is 0,1 -connected, hence
$\mathbb{Z} \subset \pi_{2}(M) .$ Now
we use the fact that $\pi_{2}(G)=0$ for every compact Lie groups
$G$. Using the long homotopy sequence again
\[
\{1\}=\pi_{2}(G) \longrightarrow \pi_{2}(M) \longrightarrow
\pi_{1}(K) \longrightarrow \cdots
\]
we see that $\mathbb{Z} \subset \pi_{1}(K)$ which means that $K$
cannot be semisimple.


Conversely, if $K$ is not semisimple, by Theorem $\ref{3}$,
either $G$ is simple or $\mathfrak{g} \simeq \mathfrak{k} \oplus
\mathfrak{k}$ with $\mathfrak{k}$ simple. So $G$ must be simple.
We will use the fact that in this case $Z(K) \simeq S^{1}$. We
are going to describe how the integrable almost complex structure
arises from $Z(K)$. Since $Z(K) \simeq S^{1}$, there exists an
element $j \in Z(K)$ of order $4 .$ Write $s=j^{2}$. Then
$s^{2}=e$. Consider the automorphism $\sigma$ on $G$ defined by
$\sigma(g)=s g s^{-1}$. Clearly $\sigma^{2}(g)=g$ and the fixed
point set of $\sigma$ is the centraliser $\mathrm{Z}(\mathrm{s})$
of $\mathrm{s}$ in $G$ with Lie algebra $\xi(s) \subset
\mathfrak{g} .$ As $s \in Z(K)$ we have $K \subset Z(s) .$ Since
$G$ is centerless $Z(s) \neq G .$ As $\ssk$ is a maximal proper
subalgebra we have $\mathfrak{k}=\xi(s)$. Consider
$\theta^{\prime}:=\mathrm{d} \sigma=\mathrm{Ad}(\mathrm{s})\:
\mathfrak{g} \rightarrow \mathfrak{g} \cdot \theta^{\prime}$ is
an involution with fixed point set
$\ssk$.


 Suppose $(\operatorname{Ad}(s))(v)=v$, then $\left.\frac{d}{d
 t}\right|_{t=0} s \exp (t v) s^{-1}=v .$ Since $t \rightarrow$
$s \exp (t x) s^{-1}$ is also a one-parameter subgroup, we have
it coincides with $\exp (t v)$, hence $s$ commutes with $\exp (t
v)$, hence $\exp (t v) \in Z(s)$ and $v \in$
$\xi(s)=\mathfrak{k}$
Let $\ssg=\ssk\oplus\ssp^{\prime}$ be the canonical decomposition
with respect to $\theta^{\prime}$, so that
$\left.\theta^{\prime}\right|_{\ssp^{\prime}}=-i d$. Then
$\ssp^{\prime}$ agrees with the orthogonal complement of in $g$
with
respect to the Killing form $B_{g}$ so that in fact we must have
$\theta=\theta^{\prime}$ and $\mathfrak{p}=\mathfrak{p}^{\prime}$
Define $\eta: G \rightarrow G$ by $\eta(g)=j g j^{-1}$. Define
$J\: \mathfrak{p} \rightarrow \mathfrak{p}$ as
$J:=\left.\mathrm{d}
\eta\right|_{\ssp}=\left.\operatorname{Ad}(\mathrm{j})\right|_{\ssp}$.
Then, from $\eta^{2}=\sigma$ we have $J^{2}=-\mathrm{id}$. From
the irreducibility of $M$ we have $g(v, w)=c B_{\ssg}(v, w)$ for
some non-zero constant $c$ and any $v, w \in \mathfrak{p} .$

Then
\[
\begin{array}{c}
	g(\mathrm{Jv}, \mathrm{J} w)=g(\mathrm{Ad}(\mathrm{j})
	\mathrm{v}, \operatorname{Ad}(\mathrm{j})
	\mathrm{w})=\mathrm{cB}_{\ssg}(\mathrm{Ad}(\mathrm{j})
	\mathrm{v}, \mathrm{Ad}(\mathrm{j}) \mathrm{w}) \\
	=\mathrm{cB}_{\ssg}(\mathrm{v},
	\mathrm{w})=\mathrm{g}(\mathrm{v}, \mathrm{w}) .
\end{array}
\]
Since $j \in Z(K)$, we have $J$ commutes with
$\operatorname{Ad}(K)$, hence by Theorem $\ref{10}$ , we have $J$
is integrable and $M$ is a Hermitian symmetric manifold.
\eproof
\subsection{Borel embedding theorem}

Next we give the Borel embedding theorem for realizing Hermitian
manifolds
of non-compact type as an open subset of its compact dual. Let
$M=G / K$ be an irreducible Hermitian symmetric manifold of
noncompact type with
$\mathfrak{g}=\mathfrak{k}\oplus\mathfrak{p}$ and $\theta:
\mathfrak{g} \rightarrow \mathfrak{g}$ the canonical involution.
We've proved that $(\cdot, \cdot):=-B_{\ssg}(\theta(\cdot),
\cdot)$ is positive definite. By Theorem $\ref{13}$,
$\mathfrak{k}$ is not semisimple.



Let $\mathfrak{g}^{\mathbb{C}}=\mathfrak{k}^{\mathbb{C}}+
\mathfrak{p}^{\mathbb{C}}$ be the complexification of $\mathfrak{g}$. Let
$\mathfrak{g}_{c}=\mathfrak{k}+\sqrt{-1} \mathfrak{p}$ be the Lie
algebra of the compact dual of $M$. Let $G^{\mathbb{C}}$ be the
simply-connected complex Lie group associated to
$\mathfrak{g}^{\mathbb{C}}$. Let $G_{c}$ be the connected real
Lie subgroups of
$G^{\mathbb{C}}$ corresponding to the real Lie subalgebras
$\mathfrak{g}_{c}$. Extend $\theta: \mathfrak{g} \rightarrow
\mathfrak{g}$ by complex linearity to an involution
$\theta^{\mathbb{C}}: \mathfrak{g}^{\mathbb{C}} \rightarrow
\mathfrak{g}^{\mathbb{C}}$ and
restrict $\theta^{\mathbb{C}}$ to $\mathfrak{g}_{c}$ then we get
an involution $\theta_{c}: \mathfrak{g}_{c} \rightarrow
\mathfrak{g}_{c} .$ Let $\sigma: G^{\mathbb{C}} \rightarrow
G^{\mathbb{C}}$
be the involution induced by $\theta^{\mathbb{C}}:
\mathfrak{g}^{\mathbb{C}} \rightarrow \mathfrak{g}^{\mathbb{C}}$.
Since $G$ and $G_{c}$ are connected Lie subgroups corresponding
to $\mathfrak{g}$ and $\mathfrak{g}_{c}$ respectively and
$\theta^{\mathbb{C}}$ preserves $\mathfrak{g}_{c}$ and
$\mathfrak{g}$, then $\sigma$ preserves $G_{c}$ and $G$. Since
$\ssk$ is not semisimple, by Theorem $\ref{13}$,
$\operatorname{Ad}(j): \mathfrak{p} \rightarrow \mathfrak{p}$ and
$\mathrm{Ad} j: \sqrt{-1} \mathfrak{p} \rightarrow \sqrt{-1}
\mathfrak{p}$
are the unique integrable almost structure of $M=G / K$ and the
dual manifold $G_{c} / K$ respectively, where $j$ is an element
of order 4 in $Z(K)=S^{1}$.


Denote the complex linear extension of $\operatorname{Ad}(j):
\mathfrak{p} \rightarrow \mathfrak{p}$ to
$\mathfrak{p}^{\mathbb{C}} \rightarrow \mathfrak{p}^{\mathbb{C}}$
by $J$.
\begin{lemma}
	Let $\mathfrak{p}^{+}$ be the $\sqrt{-1}$ eigenspace in
	$\mathfrak{p}^{\mathbb{C}}$ of $\mathrm{J}$ and
	$\mathfrak{p}^{-}$ be the $-\sqrt{-1}$ eigenspace. Then
	$\mathfrak{p}^{+}$ and $\mathfrak{p}^{-}$ are both abelian
	subalgebras of $\mathfrak{g}^{\mathbb{C}}$. Moreover, the
	complex vector subspace
	$\mathfrak{q}:=\mathfrak{k}^{\mathbb{C}}+\mathfrak{p}^{-}
	\subset \mathfrak{g}^{\mathbb{C}}$ is a complex Lie
	subalgebra of $\mathfrak{g}^{\mathbb{C}}$.
\end{lemma}
\bproof

For any $v_{1}, v_{2} \in \mathfrak{p}^{+}$, we have
$\left[v_{1}, v_{2}\right] \in \ssk^{\mathbb{C}}$ and. Since
$\operatorname{Ad}(j)$ fixes $\mathfrak{k}^{\mathbb{C}}$, we have
\[
\left[\sqrt{-1} v_{1}, \sqrt{-1}
v_{2}\right]=\left[\operatorname{Ad}(j) v_{1},
\operatorname{Ad}(j)
v_{2}\right]=\operatorname{Ad}(j)\left[v_{1},
v_{2}\right]=\left[v_{1}, v_{2}\right]
\]
hence $\left[v_{1}, v_{2}\right]=0 .$ Similarly, we have
$\mathfrak{p}^{-}$ is also abelian. Then $[\mathfrak q, \mathfrak
q] \subset$ $\left[\mathfrak{k}^{\mathbb{C}},
\mathfrak{k}^{\mathbb{C}}\right]+\left[\mathfrak{k}^{\mathbb{C}},
\mathfrak{p}^{-}\right] \subset
\mathfrak{k}^{\mathbb{C}}+\mathfrak{p}^{-}$ since
$\operatorname{Ad}(j)$ acts on $\left[\mathfrak{k}^{\mathbb{C}},
\mathfrak{p}^{-}\right]$ by $-\sqrt{-1}$.
\eproof



Let $Q$ be the complex Lie subgroup of $G^{\mathbb{C}}$
corresponding to $\mathfrak{q}$. Now we state  the Borel
embedding theorem:

\begin{theorem}
	Let $M=G / K$ be an irreducible Hermitian symmetric manifold
	of non-compact type and $G_{c}, G^{\mathbb{C}}, Q$ are
	defined as above. Then the emdedding $G_{c} \hookrightarrow
	G^{\mathbb{C}}$ induces a biholomorphism $G_{c} / K \simeq
	G^{\mathbb{C}} / Q .$ Moreover, the embedding $G
	\hookrightarrow G^{\mathbb{C}}$ induces an open embedding $M
	\hookrightarrow G^{\mathbb{C}} / Q .$ In particular, the
	embedding realizes $M$ as an open subset of its compact dual.
\end{theorem}
\bproof
Since $Q \subset G^{\mathbb{C}}$ is a complex Lie subgroup, then
$G^{\mathbb{C}} / Q$ is naturally a complex manifold. Since
$\mathfrak{k} \subset \mathfrak{q}$, we have $K \subset Q$ and
the embedding $G_{c} \hookrightarrow G^{\mathbb{C}}$ induces a
smooth mapping $\varphi: G_{c} / K \hookrightarrow G^{\mathbb{C}}
/ Q$. Then we have the commutative
diagram
\[
\begin{matrix}
	G_c&\stackrel{i}{\longrightarrow}&G^{\C}\\
	\ \ \downarrow \pi_1& &\ \ \downarrow \pi_2\\
	G_c/K&\stackrel{\phi}{\longrightarrow}&G^{\C}/Q
\end{matrix}
\]
Since $T_{o}\left(G_{c} / K\right)$ is identified with
$\mathfrak{g}_{c} / \mathfrak{k}=\sqrt{-1} \mathfrak{p}$ and
$T_{o}\left(G^{\mathbb{C}} / Q\right)$ identified with
$\mathfrak{g}^{\mathbb{C}} /
\mathfrak{q}=\mathfrak{p}^{+}\left(\mathfrak{g}^{\mathbb{C}}
=\mathfrak{k}^{\mathbb{C}}+\mathfrak{p}^{-}+\mathfrak{p}^{+}
=\mathfrak{q}+\mathfrak{p}^{+}\right)$ via projection maps $\pi_{1},\pi_{2}$
We have that $d \varphi(o): T_{o}\left(G_{c} / K\right)
\rightarrow T_{o}\left(G^{\mathbb{C}} / Q\right)$ is given by the
map $\sqrt{-1} \mathfrak{p} \rightarrow$
$\mathfrak{p}^{+}$ induced by the inclusion
$\mathfrak{k}+\sqrt{-1} \mathfrak{p} \hookrightarrow
\mathfrak{k}^{\mathbb{C}}+\mathfrak{p}^{\mathbb{C}}=
\mathfrak{k}^{\mathbb{C}}+\mathfrak{p}^{-}+\mathfrak{p}^{+}$. Then
$\operatorname{ker} d \varphi(o)=\sqrt{-1} \mathfrak{p} \cap
\mathfrak{p}^{-}=0$. Actually, if $\sqrt{-1} v \in
\mathfrak{p}^{-}$ with $v \in \mathfrak{p}$, we have
$\operatorname{Ad} j(\sqrt{-1} v)=-\sqrt{-1} \sqrt{-1} v=v$,
hence $\operatorname{Ad} j(v)=-\sqrt{-1} v$, hence $v$ must
be zero. By $\operatorname{dim}_{\mathbb{R}}(\sqrt{-1}
\mathfrak{p})=\operatorname{dim}_{\mathbb{R}}\left(\mathfrak{p}^{+}
\right)$, we have $d \varphi(o)$ is an isomorphism.

Since $\varphi$ is $G_{c}$ equivariant, and $G_{c}$ is transitive
on $G_{c} / K$, we have $\varphi$ is of
constant rank, so $\varphi$ is a local diffeomorphism. 

Since $X_{c}$ is compact, we have $\varphi$ is a smooth covering.
Since $G^{\mathbb{C}} / Q$ is simply-connected, we have $\varphi$
is a diffeomorphism.

We claim that $\varphi$ is holomorphic, hence $d \varphi$
commutes with $J$. In fact, by the description of the $d\varphi(o)$, 
we have that for any $v \in \sqrt{-1}\ssp$, it holds that $d \varphi(o)(v)$
is the $\sqrt{-1}$ eigenvector part of $v$, hence $d
\varphi(o)(v)=\frac{1}{2}(v-$ $\sqrt{-1} J v)$.
Then we have
\[
\begin{aligned}
	d \varphi(o)(J v) &=\frac{1}{2}\left(J v-\sqrt{-1} J^{2}
	v\right)=\frac{1}{2}(J v+\sqrt{-1} v) \\
	&=\frac{\sqrt{-1}}{2}(v-\sqrt{-1} J v)=\sqrt{-1} d
	\varphi(o)(v)=J(d \varphi(o)(v))
\end{aligned}
\]
hence $J$ commutes with $d \varphi(o)$, so $\varphi$ is
holomorphic and then $\varphi$ is biholomorphic since the
determinant of the real Jacobi matrix of a holomorphic
map is the square of the determinant of the complex Jacobi
matrix, which is
nonzero since $\varphi$ is diffeomorphism.

For the embedding of $M$, note that the inclusion map is also a
local diffeomorphism since the differential map is given by
$\mathfrak{p} \rightarrow \mathfrak{p}^{+}$ which is induced by
the inclusion $\mathfrak{k}+\mathfrak{p} \rightarrow
\mathfrak{k}^{\mathbb{C}}+\mathfrak{p}^{+}+\mathfrak{p}^{-}$,
hence $\operatorname{ker} d \varphi(o)=\mathfrak{p} \cap
\mathfrak{p}^{-}=0 .$ Similarly, $\varphi$ is also holomorphic
and then by the injectivity of $\varphi$, $\varphi$ is an open
embedding and is a biholomorphism onto its image.
\eproof

