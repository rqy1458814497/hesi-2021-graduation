\subsection{Example 1}
	\[
	M=Sl(n,\mathbb{R})/SO(n)
	\]
	In the present case where $G \subset G l(n, \mathbb{R})$, the
	exponential map is the usual matrix exponential map:
	For $X \in \mathfrak{gl}(n, \mathbb{R})$
	\[
	\exp X=e^{X}\coloneq {Id}+X+\frac{1}{2} X^{2}+\cdots.
	\]
	For $X, Y \in \mathfrak{gl}(n, \mathbb{R})$, 
	\[
	({ad} X) Y=[X, Y]=X Y-Y X.
	\]
	For a Lie algebra $\mathfrak{g}$, the Killing form is
	\[
	B(X, Y)=\operatorname{tr} \mathrm{ad} X \mathrm{ad} Y \quad
	\text { for } X, Y \in \mathfrak{g}.
	\]
	$\mathfrak{g}$ and $G$ are called semisimple if $B$ is
	nondegenerate.

	On $\mathfrak{gl}(n, \mathbb{R})$, we have 
	\[
	B(X,Y)=\text{tr}
	\mathrm{ad}X\mathrm{ad}Y=2n\text{tr}XY-2\text{tr}X\text{tr}
	\]
	where the trace on the right hand side is the usual matrix
	trace. 	

	This can be computed as follows: 
	
	 A basis of $\mathfrak{gl}(n,\mathbb{R})$ is $\{E_{ij}\}$
	 where the only nonzero entry is (i,j)-th with value $1$.
	\[\mathrm{ad}X
	\mathrm{ad}Y(E_{ij})=[X,[Y,E_{ij}]]=XYE_{ij}+E_{ij}YX-XE_{ij
	}Y-YE_{ij}X\]
	\[(\mathrm{ad}X
	\mathrm{ad}Y(E_{ij}))_{ij}=(XY)_{ii}+(YX)_{jj}-X_{ii}Y_{jj}-
	Y_{ii}X_{jj}.\]
	So
	\[\text{tr}\mathrm{ad}X\mathrm{ad}Y=\sum_{i,j}(\mathrm{ad}X
	\mathrm{ad}Y(E_{ij}))_{ij}=2n\text{tr}XY-2\text{tr}X\text{tr
	}Y.\]

	For $X=\alpha {Id}$, we have
	\[
	\mathrm{ad} X=0
	\]
	so $B(\ ,\ )$ is degenerate and therefore $\mathfrak{g l}(n,
	\mathbb{R})$ is not semisimple. $\mathfrak{sl}(n,
	\mathbb{R})$, however, is semisimple, and we have for $X, Y
	\in \mathfrak{sl}(n, \mathbb{R})$
	\[
	B(X, Y)=2 n \operatorname{tr} X Y
	\]
	in particular 
	\[
	B\left(X, X^{t}\right)>0 \quad \text{for}\ \  X \neq 0
	\]
	showing nondegeneracy.	

	
	Now return to $M=S l(n, \mathbb{R}) / S O(n)$.  Put $G=S l(n,
	\mathbb{R})$,
	$K=S O(n)$, and $\mathfrak{g}=\mathfrak{sl}(n, \mathbb{R}),
	\mathfrak{k}=\mathfrak{s o}(n)$. 
	
	Let
	\begin{itemize}
		\item $P$\coloneq \{$A \in S l(n, \mathbb{R})$: $A^{t}=A$, $A$
		{ positive definite }\}
		\item $\mathfrak p$\coloneq \{$X\in
		\mathfrak{sl}(n,\mathbb{R})$: $X^t=X$\}
	\end{itemize}
	Then  $\mathfrak{g}=\mathfrak{k} \oplus \mathfrak{p}$ is the
	orthogonal decomposition w.r.t. $B$, i.e. 
	for $\forall X\in \mathfrak k, \forall Y\in \mathfrak p$, 
	\[
	B(X,Y)=2n\text{tr}
	XY=2n\text{tr}(Y^tX^t)=2n\text{tr}(-YX)=-B(X,Y)
	\]
	so
	\[B(X,Y)=0.\]
	We see that:
	\begin{itemize}
		\item 	Any $A \in Gl(n, \mathbb{R})$ can be uniquely
		decomposed ( polar decomposition theorem) as
		\[
		A=V R
		\]
		with a symmetric positive definite matrix $V$ and an
		orthogonal matrix $R$.
	 
		\item $V \in P$ in the above decomposition.
		
		\item In fact, one deduces from the polar decomposition
		theorem that $M=S l(n, \mathbb{R}) / S O(n)$ is
		homeomorphic to $P$ by the obvious map:
		\[
		M\to P,\ A\cdot S O(n)\mapsto V
		\]
		and since $P$ is naturally a differentiable manifold, so
		then is $M$.
	\end{itemize}
	
	

	$G$ operates transitively on $G / K$ by diffeomorphisms, here
	$G=S l(n, \mathbb{R}), K=S O(n)$, but this holds in general,
	like all the structural results that we shall describe:
	\[
	\begin{aligned}
		G \times G / K & \rightarrow G / K \\
		(h, g K) & \rightarrow h g K .
	\end{aligned}
	\]
	The isotropy group of $Id \cdot K$ is $K$, while the isotropy group of $g K$ is $g K g^{-1}$.
	
	
	In order to describe the metric, we put
	\[
	\langle X, Y\rangle_{\mathfrak{g}}\coloneq \left\{\begin{array}{cl}
		B(X, Y) & \text { for } X, Y \in \mathfrak{p} \\
		-B(X, Y) & \text { for } X, Y \in \mathfrak{k} \\
		0 & \text { otherwise. }
	\end{array}\right.
	\]
	It follows from the result $B(X,Y)=2n\text{tr} XY$ that
	$\langle\cdot, \cdot\rangle_{\mathfrak g}$ is positive
	definite. 

	We put $e\coloneq \mathrm{Id} \in G$, and we identify $T_{e} G$ with
	$\mathfrak{g} .\ \ \langle\cdot, \cdot\rangle_{\mathfrak{g}}$
	then defines a metric on $T_{e} G$, and we then obtain a
	metric on $T_{g} G$ for arbitrary $g$ by requiring that the
	left translation
	\[
	\begin{aligned}
		L_{g}\: G & \rightarrow G \\
		h & \rightarrow g h
	\end{aligned}
	\]
	induces an isometry between $T_{e} G$ and $T_{g} G$. 
	
	\
	
	Likewise, we get an induced metric on $G / K:$ restricting
	$\left\langle\cdot,  \cdot\rangle_{g}\right.$ to $\mathfrak
	p$ yields a metric on $T_{e K} G / K \simeq \mathfrak{p}$,
	and we require again
	that
	\[
	\begin{aligned}
		L_{g}\: G / K & \rightarrow G / K \\
		h K & \rightarrow g h K
	\end{aligned}
	\]
	induces an isometry between $T_{e K} G / K$ and $T_{g K} G /
	K$.

	One checks that this metric is well defined, and $G$ then
	operates isometrically on $G / K$.
	
	In order to show that $G / K$ is a symmetric space in the
	usual differential geometric sense, we need to describe an
	appropriate involution. We start with
	\[
	\begin{aligned}
		\tau_{e}\: G & \rightarrow G \\
		h & \rightarrow\left(h^{-1}\right)^{t}, \text { in
		particular } \tau_{e }|_K=\text { id }
	\end{aligned}
	\]
	with derivative
	\[
	\begin{aligned}
		d\tau_{e}\: \mathfrak g & \rightarrow \mathfrak g \\
		X & \rightarrow-X^{t}
	\end{aligned}
	\]
	Therefore,
	\[
	d \tau_{e}|_{\mathfrak k}=\mathrm{id}, \quad d \tau_{e
	}|_{\mathfrak{p}}=-\mathrm{id}
	\]
	

	Next, for $g \in G$
	\[
	\begin{aligned}
		\tau_{g}\: G & \rightarrow G \\ h & \rightarrow g
		g^{t}\left(h^{-1}\right)^{t} .
	\end{aligned}
	\]
	Then
	\[
	\tau_{g}^{2}=\mathrm{Id}, \text { and } \tau_{g}(g)=g .
	\]
	Consequently, we get corresponding involutions
	\[
	\tau_{g K}\: G / K \rightarrow G / K
	\]
	with
	\[
	\tau_{g K}(g K)=g K, \quad d \tau_{g K}=-\mathrm{id}, \quad
	\tau_{g K}^{2}=\mathrm{id}
	\]
	and $G / K$ is a symmetric space, indeed.

	Moreover, the matrix exponential map 
	\[
	\exp X=e^{X}\coloneq {Id}+X+\frac{1}{2} X^{2}+\cdots
	\]
	when restricted to $\mathfrak{p}$ becomes the Riemannian
	exponential map for $G / K$ at $T_{e K} G / K \simeq
	\mathfrak{p}$. 
	
	Since $G$ acts by isometries on $G / K$, each $X \in
	\mathfrak{g} \simeq T_{e} G$ becomes a Killing vector field
	on $G / K$. Since the stabilizer of $e K \in G / K$ is in
	$K$, the elements of $\mathfrak{k}$ represent those Killing
	fields that vanish at $e K$. 

	By Proposition $\ref{geo}$,
		with the identification $T_{p} G / K \simeq
		\mathfrak{p}$, the curvature tensor of $G / K$ satisfies
		\[
		R(X, Y) Z=-[[X, Y], Z] \quad \text { for } X, Y, Z \in
		\mathfrak{p}
		\]

	
	In particular, the sectional curvature of a plane in $T_{p}
	M$ spanned by orthonormal vectors $Y_{1}, Y_{2}$ is given by
	\[
	\begin{aligned}
		sec\left(Y_{1}, Y_{2}\right)
		&=-\left\langle\left[\left[Y_{1}, Y_{2}\right],
		Y_{2}\right], Y_{1}\right\rangle \\
		&=-B\left(\left[\left[Y_{1}, Y_{2}\right], Y_{2}\right],
		Y_{1}\right) \\
		&=-B\left(\left[Y_{2},\left[Y_{2}, Y_{1}\right]\right],
		Y_{1}\right)\\
		&=B\left(\left[Y_{2}, Y_{1}\right],\left[Y_{2},
		Y_{1}\right]\right).
	\end{aligned}
	\]
	We have the general relations
	$
	[\mathfrak{k}, \mathfrak{k}] \subset \mathfrak{k}
	$,
	$[\mathfrak{p}, \mathfrak{p}] \subset \mathfrak{k}$ (in
	particular, $\ssp$ is {not a Lie algebra) },
	$
	[\mathfrak{p}, \mathfrak{k}] \subset \mathfrak{p}
	$.
	
	Therefore, for $Y_{1}, Y_{2} \in \mathfrak{p},\left[Y_{1},
	Y_{2}\right] \in \mathfrak{k}$, and since the Killing form is
	negative definite on $\mathfrak{k}$ (since $X^{t}=-X$ for $X
	\in \mathfrak{k}$), we obtain from above:
	
		The symmetric space ${Sl}(n, \mathbb{R}) / S O(n)$ has
		nonpositive sectional curvature and it is of non-compact
		type.

\subsection{Example 2}
	
	\begin{center}
			Compact Lie group $G$ (with bi-invariant metric)
	\end{center}	
	
	It can be realized as $$G\times G/G^{\triangle},\ G^{\triangle}=\{(g,g):g\in G\}.$$
	
	$G\times G$ is another Lie group, with Lie algebra $\ssg\oplus\ssg$. $G^{\triangle} $ is also a Lie group with Lie algebra $\ssk=\ssg^{\triangle}$. In this case, $\ssp=\{(X,-X):X\in \ssg\}$, then $\ssg\oplus\ssg=\ssk\oplus\ssp$. 
	
	The action of $G\times G$ on itself is given by $(a,b)\cdot (g,h)=(ag,bh)$, the homeomorphism $G\times G/G^{\triangle}\cong G$ is by $(a,b)G^{\triangle}=(ab^{-1},e)G^{\triangle}\mapsto ab^{-1}$, and therefore this action on $G$ is $(a,b)\cdot g=agb^{-1}$.

	The involution automorphism on $G\times G$ is $\sigma\:G\times G\to G\times G,\ (g,h)\mapsto (h,g)$, then we see $\sigma|_{G^{\triangle}}=id$, and hence we get the induced involution isometry on $G\times G/G^{\triangle}\cong G$ as $g\mapsto g^{-1}$.
	
	We see that $(d\sigma)_e\:
	(X,Y)\mapsto (Y,X)$, and that $(d\sigma)_e|_{\ssk}=id$, $(d\sigma)_e|_{\ssp}=-id$, and $\ssp\cong T_eG$ as desired.

	When $\ssg$ has no center, $g=-B$ defines a biinvariant
	metric. Then from the general result $\ref{geo2}$,
	$\mathrm{Ric}=-\frac{1}{2}B=\frac{1}{2}g$, the curvatures are
	nonnegative, and this is a case of compact type when $G$
	irreducible.

\subsection{Example 3}
\begin{center}
	A noncompact type
\end{center}
	
	We can also get a noncompact type (characterized by nonpositive curvature) using the same Lie algebra  $\ssg$. 
	
	Consider $\ssg\otimes\mathbb C$ with involution $\sigma\:
	X\mapsto \bar X$, the complex conjugation. Then $\ssk=\ssg$,
	$\ssp=i\ssg$. The inner product on $\ssk$ is just $-B$ on
	$\ssg$, while on $\ssp$ the metric is given by
	$g(iX,iY)=-B(X,Y)=B(iX,iY)$. This gives
	$\mathrm{Ric}(iX,iY)=-\frac{1}{2}B(iX,iY)=-\frac{1}{2}g(iX,i
	Y)$. Hence the manifold $\exp(\ssg\otimes\mathbb
	C)/\exp(\ssk)$ is a symmetric space of noncompact type.

\subsection{Example 4}
\begin{center}
	Grassmann manifolds
\end{center}			
	
	Let $G_{k}\left(R^{n}\right)$ be the Grassmanian of
	unoriented $k$ -planes in $\mathbb{R}^{n}$ and let
	$G_{k}^{0}\left(R^{n}\right)$ be the  oriented one. We know
	that the oriented one is a two-cover of the unoriented. Hence
	their Cartan decompositions are the same. We see that
	$G=\mathrm{SO}(n)$  acts transitively on $k$ planes, with Lie
	algebra $\mathfrak{so}(n)$. 
	
	Suppose point $p_{0}\in G_{k}\left(R^{n}\right)$ is spanned
	by vectors $e_{1}, \cdots, e_{k}$, then its stabilizer is
	\[
	G_{p_{0}}=S(\mathrm{O}(k)
	\mathrm{O}(n-k))=\left\{\left(\begin{array}{cc}
		A & 0 \\
		0 & B
	\end{array}\right) \middle| A \in \mathrm{O}(k), B \in
	\mathrm{O}(n-k),|A||B|=1\right\}.
	\]
	For the oriented $G_{k}^{0}\left(R^{n}\right)$, we have
	$G_{p_{0}}=\mathrm{SO}(k) \mathrm{SO}(n-k)$.
	
	Obviously, the diagonal matrix with diagon entries 1 and -1
	is an involution on $G$ as long as it is not the identity
	matrix. Consider such matrix as $I_{p,q}$ with $p$ numbers of
	value $1$ on the left top and other $q$ numbers of $-1$ on the right bottom.
	
	We denote the matrix with $p$ entries of $1$ and $q$
	$\sigma(A)=I_{k, n-k} A I_{k, n-k}$ is a automorphism of $G$.
	Obiously $\sigma|_{G_{p_0}}=id$ for both Grassmanians.

	And we see that $d\sigma_e$:
	\[
	\begin{pmatrix}
		X_{11}&X_{12}\\
		X_{21}&X_{22}
	\end{pmatrix}\mapsto
	\begin{pmatrix}
		X_{11}&-X_{12}\\
		-X_{21}&X_{22}
	\end{pmatrix}.
	\]
	
	
	So the $-1$ eigenspace $\ssp$ of $d \sigma$ 
	is :
	\[
	\mathfrak{p}=\left\{\left(\begin{array}{cc}
		0 & X \\
		-X^{T} & 0
	\end{array}\right) \middle| X \in M(p, q, \mathbb{R})\right\}
	\subset\mathfrak{so}(n)\ 
	\]
	and the $\ad$ representation is $\mathrm{Ad}((A, B)) X=A X B^{T}$, 
	where $(A, B) \in S(\mathrm{O}(k) \mathrm{O}(n-k))$.

	We may let the metric on $\ssp$ be $\langle
	X,Y\rangle=\mathrm{tr}(X^tY)=-\mathrm{tr}(XY)=-\frac{1}{k+l-
	2}B(X,Y)$, then $\mathrm{Ric}=\frac{k+l-2}{2}g$, the
	curvature is nonnegative and these are compact types for
	irreducible ones.
	
	Note that $B(X,X)=(n-2)\mathrm{tr}X^2$ on
	$\mathfrak{so}(n,\mathbb R)$. This can be similarly computed
	by taking basis $\{E_{ij}-E_{ji}\}$.
	
	\subsection{Example 5}
	\begin{center}
		Trivial cases
	\end{center}
	
	With only the definition of a symmetric Riemannian manifold,
	one can see that $\bbR^n$ with standard flat matric is one.
	$s_p$ are obviously the total reflection at a vector, i.e.,
	$s_p(p+v)=p-v$ for any $p\in \bbR^n$. 
	
	Another trivial example is the unit sphere, with
	$s_p(v)=2\langle v,p\rangle-v$. We see that indeed it act on
	the tangent space at $p$ by $-id$. 
	
	For a compact Lie group, if endow it with bi-invariant matric
	as before, $s_e\colon g\mapsto g$ is clearly an involution
	isometry. Since for any $X\in\ssg$, there is $s_p(\exp
	tX)=\exp(-tX)$, so $d(s_e)_e=-id$. This is consistent with
	what we analyzed previouly.
