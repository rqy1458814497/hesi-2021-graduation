\hspace*{1em}
The thesis is to introduce the concept of (globally) symmetric manifolds. This is an important topic since many rigidity theorems are based on it. The advantage of this specific manifold is that the computation usually comes clear and easy, unlike some abstract manifold. Hence we turn to this topic.

Many people have already done quite a few work on it, including the authors of my reference books. They have classified different symmetric manifolds and have computed clearly. Since my learning is limited, I will only introduce some of the most important and basic knowledge and classification.

Next I would summarize the structure of the thesis. There are mainly four parts:
\begin{itemize}
    \item In chapter 2, I introduce some basic ideas of Lie groups and Lie algebras, which form a firm foundation of the advanced theories.
    \item In chapter 3, I formally introduce the concept of Riemannian symmetric manifold(space), and then discuss the properties and classification. There are several different statement of the classification of compact type, non-compact type and Euclidean type, and I will show the equivalence between them.
    \item In chapter 4, I introduce the concept of Hermitian symmetric manifold, which is not too much different from the Riemannian one. The de Rham decomposition theorem is still useful in the complex situation. And there is a borel embedding theorem worth to be known since it is useful.
    \item In chapter 5, I give some classical examples to show intuitively what a symmetric manifold is like.
\end{itemize}
The main reference of chapter 2 is \cite{Hel}, that of chapter 3 is \cite{Hel},\cite{Ziller},\cite{Besse}, and for chapter 4. it is \cite{Mok} and \cite{Ziller}. Examples in chapter 5 comes from several books including \cite{Peter},\cite{Ziller} and \cite{MR1451625}.











	 
	