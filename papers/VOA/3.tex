\subsection{Regularized integrals}

Following \cite{regularized},
we describe a way to integrate meromorphic forms
on Riemann surfaces, called the \emph{regularized integral}.

\begin{notation}
    Let $\Sigma$ be a compact Riemman surface,
    and let $D = \{ p_1, \dotsc, p_n \} \subset \Sigma$ be a discrete subset.
    \begin{itemize}
        \item
            Let
            \[
                \mathscr{A}^{p,q} (\Sigma, \star D)
            \]
            denote the space of smooth $(p, q)$-forms on $\Sigma \setminus D$,
            with possible holomorphic poles at $p_1, \dotsc, p_n$.
        \item
            Let
            \[
                \mathscr{A}^{p,q} (\Sigma, \log D)
            \]
            denote the space of smooth $(p, q)$-forms on $\Sigma \setminus D$,
            with possible logarithmic poles at $p_1, \dotsc, p_n$.
            In other words, in local holomorphic coordinates around $p_i$,
            the form can be written as the product of $d z / z$ and a smooth form,
            where the local coordinate $z$ satisfies $z (p_i) = 0$.
            \varqed
    \end{itemize}
\end{notation}

The integral of a form with logarithmic poles
is absolutely convergent,
so there is a well-defined integral operator
\[
    \int \colon
    \mathscr{A}^{1,1} (\Sigma, \log D)
    \to \bbC .
\]
The regularized integral will be defined as an operator
\[
    \intbar \colon
    \mathscr{A}^{1,1} (\Sigma, \star D)
    \to \bbC ,
\]
which extends the usual integral operator
on $\mathscr{A}^{1,1} (\Sigma, \log D)$.

\begin{definition}
    Let $\omega \in \mathscr{A}^{1,1} (\Sigma, \star D)$.
    If there exist
    $\alpha \in \mathscr{A}^{1,1} (\Sigma, \log D)$
    and $\beta \in \mathscr{A}^{0,1} (\Sigma, \star D)$,
    such that
    \[
        \omega = \alpha + \partial \beta ,
    \]
    then we define the \emph{regularized integral} of $\omega$ to be
    \[
        \intbar_\Sigma \omega = \int_\Sigma \alpha.
    \]
    In fact, such forms $\alpha$ and $\beta$ always exist,
    and the regularized integral does not depend on
    the choice of $\alpha$ and $\beta$.
    \varqed
\end{definition}

See \cite[Definition~2.5]{regularized}.

In fact, the regularized integral can be described
as an intrinsic notion of the Cauchy principal value
on Riemann surfaces, as follows.

\begin{theorem}
    \label{thm-regularized}
    Let $\omega \in \mathscr{A}^{1,1} (\Sigma, \star D)$,
    where $D = \{ p_1, \dotsc, p_n \}$.
    Let $B_{\epsilon} (p_i)$
    be the $\epsilon$-ball centred at $p_i$,
    with respect to some chosen local holomorphic coordinates.
    Then
    \[
        \intbar_{\Sigma} \omega =
        \lim_{\epsilon \to 0}
        \int_{\Sigma \setminus \bigcup_{i = 1}^n B_{\epsilon} (p_i)} \omega,
    \]
    where the limit always exists
    and does not depend on the choice of local holomorphic coordinates.
\end{theorem}

See \cite[Theorem~2.8]{regularized}.

We also have a version of the residue/Stokes theorem.

\begin{definition}
    Let $\omega \in \mathscr{A}^1 (\Sigma, \star D)$.
    The \emph{residue} of $\omega$ at a point $p \in \Sigma$
    is defined to be the limit
    \[
        \Res_{z = p} \omega =
        \frac{1}{2 \uppi \upi} \oint_p \ \omega =
        \frac{1}{2 \uppi \upi} \lim_{\epsilon \to 0}
        \oint_{\partial B_{\epsilon} (p)} \omega,
    \]
    where $B_{\epsilon} (p)$ is the $\epsilon$-ball
    with respect to some local holomorphic coordinate
    around $p$.
    
    In fact, this limit always exists and does not depend on
    the choice of the local coordinate.
    \varqed
\end{definition}

See \cite[Lemma/Definition~2.10]{regularized}.

\begin{theorem}
    \label{thm-residue-stokes}
    Let $\alpha \in \mathscr{A}^1 (\Sigma, \star D)$,
    where $D = \{ p_1, \dotsc, p_n \}$, with $p_i$ all distinct. Then
    \[
        \intbar_\Sigma d \alpha =
        -2 \uppi \upi
        \sum_{i=1}^n \Res_{p_i} \alpha.
    \]
\end{theorem}

See \cite[Theorem~2.13]{regularized}.
Note that the right hand side is not always $0$,
as $\alpha$ is not necessarily meromorphic.

\begin{proof}
    Let $B_{\epsilon} (p_i)$
    be a small coordinate ball centred at $p_i$.
    By Theorem~\ref{thm-regularized}, we have
    \begin{align*}
        \intbar_\Sigma d \alpha
        & = \lim_{\epsilon \to 0}
        \int_{\Sigma \setminus \bigcup_{i = 1}^n B_{\epsilon} (p_i)} d \alpha \\
        & = -{\lim_{\epsilon \to 0} \,
        \sum_{i = 1}^n \,
        \oint_{\partial B_{\epsilon} (p_i)} \alpha} \\
        & = -2 \uppi \upi 
        \sum_{i=1}^n \Res_{p_i} \alpha.
        \qedhere
    \end{align*}
\end{proof}

This provides a technique to compute regularized integrals
on elliptic curves, which we will use later.

On the elliptic curve $E = \bbC / (\bbZ 1 \oplus \bbZ \tau)$,
where $\im \tau > 0$,
we have the volume form
\[
    \frac {d^2 z} {\im \tau} = \frac {\upi} {2 \im \tau} \, d z \wedge d \bar{z},
\]
so that $\int_E d^2 z / {\im \tau} = 1$.

Recall that for the elliptic curve $E$,
we have the \emph{Weierstrass zeta function} $\zeta (z)$,
which is a meromorphic function defined on $\bbC$,
satisfying the quasi-periodicity conditions
\[
    \zeta (z + 1) = \zeta (z) + 2 \eta_1, \quad
    \zeta (z + \tau) = \zeta (z) + 2 \eta_2,
\]
where $\eta_1 = \zeta (1/2)$ and $\eta_2 = \zeta (\tau/2)$.
The function $\zeta (z)$ has a pole at $0$,
satisfying $\zeta (z) = 1/z + o(1)$ as $z \to 0$,
and it has no poles other than in the lattice $\bbZ 1 \oplus \bbZ \tau$.
For details, see for example \cite{elliptic}.

We modify the Weierstrass zeta function a little bit,
by defining
\[
    \xi (z) = \zeta (z) - 2 \eta_1 z.
\]
Then $\xi$ is a meromorphic function on $\bbC$,
satisfying the quasi-periodicity conditions
\[
    \xi (z + 1) = \xi (z), \quad
    \xi (z + \tau) = \xi (z) - 2 \uppi \upi,
\]
where we have used the identity
$\eta_1 \tau - \eta_2 = \uppi \upi$.
Again, see  \cite{elliptic} for this fact.
Note that we have the asymptotic behaviour
$\xi (z) = 1/z + o(1)$ as $z \to 0$.

\begin{lemma}
    \label{lem-zeta-technique}
    Let $E = \bbC / (\bbZ 1 \oplus \bbZ \tau)$
    be an elliptic curve, where $\im \tau > 0$,
    and let $f \in \mathscr{A}^0 (E, \star D)$ be a meromorphic function,
    where $D = \{ p_1, \dotsc, p_n \} \subset E$.
    Suppose that the points $0, p_1, \dotsc, p_n \in E$
    are all distinct.
    Then
    \[
        \intbar_E f \, \frac {d^2 z} {\im \tau} =
        2 \uppi \upi
        \sum_{i=0}^n \Res_{z = p_i} \biggl[
            \biggl( \frac {\im z} {\im \tau} +
            \frac {1} {2 \uppi \upi} \, \xi (z) \biggr) f
        \biggr] ,
    \]
    where $p_0 = 0$,
    and $\xi$ denotes the function defined above.
\end{lemma}

\begin{proof}
    By Theorem~\ref{thm-residue-stokes}, we have
    \begin{align*}
        \intbar_E f \, \frac {d^2 z} {\im \tau}
        & = -\intbar_E \dbar \biggl[
            \biggl( \frac {\im z} {\im \tau} +
            \frac {1} {2 \uppi \upi} \, \xi (z) \biggr)
            f \, d z
        \biggr] \\
        & = 2 \uppi \upi
        \sum_{i=0}^n \Res_{z = p_i} \biggl[
            \biggl( \frac {\im z} {\im \tau} +
            \frac {1} {2 \uppi \upi} \, \xi (z) \biggr)
            f \, d z
        \biggr] . \qedhere
    \end{align*}
\end{proof}


\subsection{Regularized integrals and residues}

The goal of this subsection is to prove the following lemma,
which shows that the operation of taking the residue of a function
commutes with the operation of taking its regularized integral.

\begin{lemma}
    \label{lem-commutator}
    Let $\Sigma$ be a compact Riemann surface,
    with a distinguished point $0 \in \Sigma$,
    and let $\mathrm{vol}$ be a volume form on $\Sigma$.
    Let
    \[
        D =
        \{ z = z_1, \dotsc, z_l \} \cup
        \{ w = w_1, \dotsc, w_m \} \cup
        \{ z = w \}
        \subset \Sigma^2
    \]
    be a divisor of $\Sigma^2$,
    where $z_1, \dotsc, z_l, w_1, \dotsc, w_m \in \Sigma$.
    Let
    \[
        F \colon \Sigma^2 \setminus D \to \bbC
    \]
    be a real analytic function,
    with possible holomorphic poles along $D$.
    Then
    \begin{enumerate} [label={\textup{(\roman*)}}]
        \item
            Taking the residue commutes with
            taking the regularized integral:
            \[
                \Biggl[ \ 
                    \oint_0 \ d z \ , \ 
                    \intbar_\Sigma \mathrm{vol}_w
                \Biggr]
                F (z, w)
                = 0,
            \]
        \item 
            Taking the holomorphic partial derivative commutes with
            taking the regularized integral:
            \[
                \Biggl[ 
                    \frac {\partial} {\partial z} \ , \ 
                    \intbar_\Sigma \mathrm{vol}_w
                \Biggr]
                F (z, w)
                = 0,
            \]
            whenever $z \neq z_1, \dotsc, z_l$.
    \end{enumerate}
\end{lemma}

\begin{proof} [Proof of \textup{(i)}]
    \allowdisplaybreaks
    Choose $\delta > \epsilon > 0$ such that
    $\{ z_i, w_i \} \cap ( \bar{B}_{\delta} (0) \setminus \{ 0 \} ) = \varnothing$,
    where $\bar{B}_{\delta} (0)$ denotes the closed ball of radius $\delta$
    centred at $0 \in \Sigma$.
    We use the notation $O_{\delta} (\epsilon^2)$
    to denote a function that is $O (\epsilon^2)$ for any fixed $\delta$.
    Thus,
    \begin{align*}
        & \phantom{{} = {}}
        \intbar_\Sigma \mathrm{vol}_w \ 
        \oint_0 \ d z \ 
        F (z, w) \\
        & =
        \intbar_{\Sigma \setminus B_{\delta} (0)} \mathrm{vol}_w \ 
        \oint_{|z| = \epsilon} d z \ 
        F (z, w)
        +
        \intbar_{B_{\delta} (0)} \mathrm{vol}_w \ 
        \oint_{0} \ d z \ 
        F (z, w)
        + O_{\delta} (\epsilon^2) \\
        & =
        \oint_{|z| = \epsilon} d z \ 
        \intbar_{\Sigma \setminus B_{\delta} (0)} \mathrm{vol}_w \ 
        F (z, w)
        +
        O (\delta^2)
        +
        O_{\delta} (\epsilon^2),
    \end{align*}
    where we were able to swap the two integrals in the first term,
    because the regularized integral
    can be written as a limit:
    \[
        \intbar_{\Sigma \setminus B_{\delta} (0)} \mathrm{vol}_w \ 
        F (z, w)
        =
        \lim_{\epsilon' \to 0}
        \int_{
            \Sigma \setminus \left(
                B_{\delta} (0) \ 
                \cup \ 
                \bigcup_i B_{\epsilon'} (w_i)
            \right)
        } \mathrm{vol}_w \ 
        F (z, w),
    \]
    and the limit on the right hand side
    converges uniformly with respect to all $z$ such that $|z| = \epsilon$.
    It follows that
    \[
        \Biggl[ \ 
            \oint_0 \ d z \ , \ 
            \intbar_\Sigma \mathrm{vol}_w
        \Biggr]
        F (z, w)
        =
        \oint_{|z| = \epsilon} d z \ 
        \intbar_{B_{\delta} (0)} \mathrm{vol}_w \ 
        F (z, w)
        + O (\delta^2) + O_{\delta} (\epsilon^2).
    \]
    For $(z, w) \in B_{\delta} (0) ^2$,
    we may expand $F$ as
    \[
        F (z, w) \, \mathrm{vol}_w
        =
        \Biggl(
            \sum _{k = 1} ^{K}
            \frac {a_k (z)} {w^k}
            +
            \sum _{k = 1} ^{K}
            \frac {b_k (z)} {(w - z)^k}
            +
            \sum _{k = 1} ^{K}
            \frac {c_k (w)} {z^k}
            +
            F_0 (z, w)
        \Biggr) \,
        d^2 w,
    \]
    where $F_0$ is non-singular in $B_{\delta} (0) ^2$,
    the functions $c_k$ are non-singular in $B_{\delta} (0)$,
    and the functions $a_k, b_k$ have a possible holomorphic pole at $z = 0$
    and no other poles.
    Indeed, the first two terms come from expanding $F (z, w)$
    in terms of its poles in $w$ for fixed $z \neq 0$,
    and the regular part becomes the last two terms.
    Since
    \[
        \intbar_{B_{\delta} (0)}
        \frac {d^2 w} {w^k}
        = 0 \quad
        \text{and} \quad
        \intbar_{B_{\delta} (0)}
        d^2 w \, c_k (w)
        = O (\delta^2) \quad
        \text{for all }
        k \geq 1,
    \]
    we have
    \[
        \Biggl[ \ 
            \oint_0 \ d z \ , \ 
            \intbar_\Sigma \mathrm{vol}_w
        \Biggr]
        F (z, w)
        =
        \sum _{k = 1} ^{K} \ 
        \oint_{|z| = \epsilon}
        b_k (z) \ d z \ 
        \intbar_{B_{\delta} (0)} 
        \frac {d^2 w} {(w - z)^k}
        + O (\delta^2) + O_{\delta} (\epsilon^2).
    \]
    Notice that if $r > |z|$, and if we write $\theta = \arg w$, then
    \begin{align*}
        & \phantom{{} = {}}
        \oint _{|w| = r}
        \frac {d \theta} {(w - z)^k} \\
        & =
        - \upi r \ 
        \oint _{|w| = r}
        \frac {d w} {w (w - z)^k} \\
        & =
        2 \uppi r \biggl(
            \Res_{w = 0} \frac {1} {w (w - z)^k}
            +
            \Res_{w = z} \frac {1} {w (w - z)^k}
        \biggr) \\
        & =
        2 \uppi r \biggl(
            \frac {(-1)^k} {z^k}
            +
            \frac {1} {(k - 1)!}
            \biggl( \frac {d} {d w} \biggr) ^{k-1}
            \frac {1} {w}
            \bigg|_{w = z}
        \biggr) \\
        & = 0,
    \end{align*}
    so that for any $0 < |z| < \delta$, we have
    \[
        \intbar_{B_{\delta} (0)} 
        \frac {d^2 w} {(w - z)^k}
        =
        \intbar_{B_{|z|} (0)} 
        \frac {d^2 w} {(w - z)^k}.
    \]
    This means that if we replace $z$ by $c z$,
    with $c \in \bbC$ such that $0 < |c z| < \delta$,
    then this integral becomes
    \[
        \frac {|c|^2} {c^k}
        =
        \frac {\bar{c}} {c^{k-1}}
    \]
    times the original value.
    Therefore, this integral must have the form
    \[
        \intbar_{B_{\delta} (0)} 
        \frac {d^2 w} {(w - z)^k}
        =
        C_k
        \frac {\bar{z}} {z^{k-1}}
    \]
    for some constant $C_k \in \bbC$,
    for all $0 < |z| < \delta$.
    If we expand $b_k$ as
    \[
        b_k (z) =
        \sum _{ p = -P } ^{\infty}
        \sum _{ q = 0 } ^{\infty}
        b_{k, p, q} \, z^p \, \bar{z}^q ,
    \]
    then
    \begin{align*}
        & \phantom{{} = {}}
        \oint_{|z| = \epsilon}
        b_k (z) \ d z \ 
        \intbar_{B_{\delta} (0)} 
        \frac {d^2 w} {(w - z)^k} \\
        & =
        C_k \ 
        \sum _{q = 0} ^{\infty}
        b_{k, k - 1 + q, q} \ 
        \oint_{|z| = \epsilon}
        z^q \, \bar{z}^{q + 1} \, d z \\
        & =
        2 \uppi \upi \, C_k \ 
        \sum _{q = 0} ^{\infty}
        b_{k, k - 1 + q, q} \, \epsilon^{2q + 2} \\
        & =
        O (\epsilon^2),
    \end{align*}
    by the analyticity of $b_k$.
    Therefore,
    \[
        \Biggl[ \ 
            \oint_0 \ d z \ , \ 
            \intbar_\Sigma \mathrm{vol}_w
        \Biggr]
        F (z, w)
        =
        O (\delta^2) + O_{\delta} (\epsilon^2).
    \]
    It follows that this commutator must be zero.
\end{proof}


\begin{proof} [Proof of \textup{(ii)}]
    We have
    \begin{align*}
        & \phantom{{} = {}}
        \Biggl[
            \frac {\partial} {\partial z} \ , \ 
            \intbar_\Sigma \mathrm{vol}_w
        \Biggr]
        F (z, w) \\
        & =
        \frac {1} {2 \uppi \upi}
        \Biggl[ \ 
            \oint_z \ d \zeta \ , \ 
            \intbar_\Sigma \mathrm{vol}_w
        \Biggr]
        \frac {F (\zeta, w)} {(\zeta - z)^2} \\
        & = 0,
    \end{align*}
    where the last equality almost follows from (i),
    except that the function $F (\zeta, w) / (\zeta - z)^2$
    is only defined when $\zeta$ is in a neighbourhood of $z$.
    However, the proof of (i)
    works even if $F$ is not analytic,
    as long as it is smooth on $\Sigma^2 \setminus D$
    and analytic near $\{ 0 \} \times \Sigma$.
    Therefore, we can extend the function $1 / (\zeta - z)^2$
    smoothly to $\Sigma \setminus \{ z \}$,
    multiply it with $F (\zeta, w)$,
    and then apply (i) with $0 = z$ to complete the proof.
\end{proof}

\begin{remark}
    In general, the anti-holomorphic differential
    does \emph{not} commute with the regularized integral.
    That is, the expression
    \[
        \Biggl[
            \frac {\partial} {\partial \bar{z}} \ , \ 
            \intbar_\Sigma \mathrm{vol}_w
        \Biggr]
        F (z, w)
    \]
    is in general not zero.
    In fact, even if $F$ is meromorphic in $z$,
    so that $\partial F / \partial \bar{z} = 0$,
    the regularized integral
    \[
        \intbar_\Sigma \mathrm{vol}_w \ F (z, w)
    \]
    may not be meromorphic in $z$.
    For example, on the elliptic curve,
    using Lemma~\ref{lem-zeta-technique},
    one computes that
    \[
        \intbar_E \frac {d^2 w} {\im \tau} \ 
        \bigl( \zeta (z) - \zeta (w) + \zeta (w - z) \bigr)
        = 2 \uppi \upi \, \frac {\im z} {\im \tau} + \xi (z),
    \]
    which is not meromorphic in $z$. \varqed
\end{remark}


\subsection{Twisted conformal blocks}

Let $\scrV$ be a vertex algebra,
and let $I \in \scrV$ be a vertex operator such that $I_{(0)} I = 0$.
Let $\langle \cdots \rangle$ be an element of the conformal block of $\scrV$
on the elliptic curve $E = \bbC / (\bbZ 1 \oplus \bbZ \tau)$,
which we fix throughout this section.

Recall that we have the BRST reduction $H (\scrV, I_{(0)})$.
In this section, we attempt to construct elements
of the conformal block of $H (\scrV, I_{(0)})$ on $E$,
based on the existing element $\langle \cdots \rangle$ of the conformal block of $\scrV$.

For any formal power series
\[
    F = \sum_{m=0}^\infty a_m x^m \in \bbC [[x]],
\]
we write
\begin{align*}
    & \phantom {{} = {}}
    \biggl\langle
        \scrO_1 (z_1) \cdots \scrO_n (z_n) \ 
        F \biggl( \frac{1}{\hbar} \intbar I \biggr)
    \biggr\rangle \\
    & =
    \sum_{m=0}^\infty \frac{a_m}{\hbar^m}
    \intbar_E \frac {d^2 w_1} {\im \tau} \cdots
    \intbar_E \frac {d^2 w_m} {\im \tau} \ 
    \langle
        \scrO_1 (z_1) \cdots \scrO_n (z_n) \ 
        I (w_1) \cdots I (w_m)
    \rangle
\end{align*}
as a shorthand,
where $\hbar$ is a formal variable.
One particularly interesting example is
\[
    \langle
        \scrO_1 (z_1) \cdots \scrO_n (z_n) \ 
        \upe^{\frac{1}{\hbar} \intbar I}
    \rangle,
\]
which should give the correlation functions of the new theory
obtained via BRST reduction.

\begin{lemma}
    \label{lem-new-conf-block-well-defined}
    Suppose that $\scrO_1, \dotsc, \scrO_n \in \scrV$
    are $I_{(0)}$-closed, and let $F \in \bbC[[x]]$. Then
    \[
        \biggl\langle
            \scrO_1 (z_1) \cdots \scrO_n (z_n) \ 
            F \biggl( \frac{1}{\hbar} \intbar I \biggr)
        \biggr\rangle
    \]
    does not depend on the choice of representatives
    of the $I_{(0)}$-cohomology classes of the elements $\scrO_i$.
\end{lemma}

\begin{proof}
    Suppose that $\scrO_1 = I_{(0)} \scrO'_1$ for some $\scrO'_1$.
    We need to show that
    \[
        \langle
            I_{(0)} \scrO'_1 (z_1) \ 
            \scrO_2 (z_2) \cdots \scrO_n (z_n) \ 
            I (w_1) \cdots I (w_m)
        \rangle
        = 0
    \]
    for any $m \geq 0$.
    But by the global residue theorem, we have
    \begin{align*}
        0
        & =
        (
            \Res_{w = z_1} + \cdots + \Res_{w = z_n}
            + \Res_{w = w_1} + \cdots + \Res_{w = w_m}
        ) \\
        & \hspace{4em}
        \langle
            \scrO'_1 (z_1) \ 
            \scrO_2 (z_2) \cdots \scrO_n (z_n) \ 
            I (w) \ I (w_1) \cdots I (w_m)
        \rangle \\
        & =
        \langle
            I_{(0)} \scrO'_1 (z_1) \ 
            \scrO_2 (z_2) \cdots \scrO_n (z_n) \ 
            I (w_1) \cdots I (w_m)
        \rangle ,
    \end{align*}
    since by assumption, $I_{(0)} \scrO_i = 0$ and $I_{(0)} I = 0$.
    It follows that
    \[
        \biggl\langle
            I_{(0)} \scrO'_1 (z_1) \ 
            \scrO_2 (z_2) \cdots \scrO_n (z_n) \ 
            F \biggl( \frac{1}{\hbar} \intbar I \biggr)
        \biggr\rangle
        = 0.
        \qedhere
    \]
\end{proof}


\begin{lemma}
    \label{lem-meromorphic}
    Let $\scrO_1, \dotsc, \scrO_n, I_1, \dotsc, I_m \in \scrV$
    be elements such that $I_{j, (0)} I_i = 0$
    and $I_{j, (0)} \scrO_i = 0$ for all $i, j$.
    Then the function
    \[
        \intbar_E \frac {d^2 w_1} {\im \tau} \cdots
        \intbar_E \frac {d^2 w_m} {\im \tau} \ 
        \langle
            \scrO_1 (z_1) \cdots \scrO_n (z_n) \ 
            I_1 (w_1) \cdots I_m (w_m)
        \rangle
    \]
    is meromorphic in $z_1, \dotsc, z_n$.
\end{lemma}

\begin{proof}
    \allowdisplaybreaks
    For fixed $z_1, \dotsc, z_n, w_1, \dotsc, w_{m-1} \in E$,
    such that these points are distinct and are different from $0$,
    we denote $z_{n+j} = w_j$ ($1 \leq j \leq m - 1$).
    Using Lemma~\ref{lem-zeta-technique}, we have
    \begin{align*}
        & \phantom{{} = {}}
        \intbar_E \frac {d^2 w_m} {\im \tau} \ 
        \langle
            \scrO_1 (z_1) \cdots \scrO_n (z_n) \ 
            I_1 (w_1) \cdots I_m (w_m)
        \rangle \\
        & =
        2 \uppi \upi
        \left(
            \sum _{i=1} ^{n+m-1} \Res_{w_m = z_i} +
            \Res _{w_m = 0}
        \right) \\
        && \hspace{-24em}
        \left[
            \Bigl(
                \frac{\im w_m}{\im \tau} +
                \frac {1} {2 \uppi \upi} \, \xi (w_m)
            \Bigr)
            \langle
                \scrO_1 (z_1) \cdots \scrO_n (z_n) \ 
                I_1 (w_1) \cdots I_m (w_m)
            \rangle
        \right] \\
        & =
        2 \uppi \upi
        \sum _{i=1} ^{n+m-1} \Res_{w_m = z_i}
        \Biggl[
            \Bigl(
                \frac{\im (w_m - z_i)}{\im \tau} +
                \frac{\im z_i}{\im \tau} +
                \frac {1} {2 \uppi \upi} \, \xi (w_m)
            \Bigr) \cdot {} \\
        && \hspace{-24em}
            \langle
                \scrO_1 (z_1) \cdots \scrO_n (z_n) \ 
                I_1 (w_1) \cdots I_m (w_m)
            \rangle
        \Biggr]
        + \text{(meromorphic term)} \\
        & =
        2 \uppi \upi
        \sum _{i=1} ^{n+m-1}
        \frac{\im z_i}{\im \tau}
        \Res_{w_m = z_i}
        \langle
            \scrO_1 (z_1) \cdots \scrO_n (z_n) \ 
            I_1 (w_1) \cdots I_m (w_m)
        \rangle \\
        && \hspace{-24em}
        {} + \text{(meromorphic term)} \\
        & =
        0 + \text{(meromorphic term)},
    \end{align*}
    by the assumption that
    $I_{m, (0)} \scrO_i = I_{m, (0)} I_{j} = 0$
    for all $i, j$.
    
    Similarly, if $m > 1$, then we have
    \begin{align*}
        & \phantom{{} = {}}
        \intbar_E \frac {d^2 w_{m-1}} {\im \tau}
        \intbar_E \frac {d^2 w_m} {\im \tau} \ 
        \langle
            \scrO_1 (z_1) \cdots \scrO_n (z_n) \ 
            I_1 (w_1) \cdots I_m (w_m)
        \rangle \\
        & =
        2 \uppi \upi
        \sum _{i=1} ^{n+m-2}
        \frac{\im z_i}{\im \tau}
        \Res_{w_{m-1} = z_i}
        \intbar_E \frac {d^2 w_m} {\im \tau} \\
        && \hspace{-24em}
        \langle
            \scrO_1 (z_1) \cdots \scrO_n (z_n) \ 
            I_1 (w_1) \cdots I_m (w_m)
        \rangle
        + \text{(meromorphic term)} \\
        & =
        2 \uppi \upi
        \sum _{i=1} ^{n+m-2}
        \frac{\im z_i}{\im \tau}
        \intbar_E \frac {d^2 w_m} {\im \tau} \ 
        \Res_{w_{m-1} = z_i} \\
        && \hspace{-24em}
        \langle
            \scrO_1 (z_1) \cdots \scrO_n (z_n) \ 
            I_1 (w_1) \cdots I_m (w_m)
        \rangle
        + \text{(meromorphic term)} \\
        & =
        0 + \text{(meromorphic term)},
    \end{align*}
    where we have applied Lemma~\ref{lem-commutator}
    to swap the regularized integral
    and the residue operator.
    
    Continuing in this way,
    we see that the $n$-fold regularized integral
    is meromorphic in $z_1, \dotsc, z_n$.
\end{proof}


\begin{theorem}
    \label{thm-main}
    Let $\scrV$ be a vertex algebra, and let $\langle \cdots \rangle$
    be an element of the conformal block for $\scrV$ on $E$.
    Let $I \in \scrV$ be a vertex operator such that $I_{(0)} I = 0$.
    Then for any $F \in \bbC [[x]]$, the expression
    \[
        \biggl\langle
            \scrO_1 (z_1) \cdots \scrO_n (z_n) \ 
            F \biggl( \frac{1}{\hbar} \intbar I \biggr)
        \biggr\rangle
    \]
    gives an element of the conformal block for the BRST reduction
    $H (\scrV, I_{(0)})$ on $E$.
\end{theorem}

\begin{proof}
    \allowdisplaybreaks
    The expression
    $\bigl\langle \scrO_1 (z_1) \cdots \scrO_n (z_n) \ 
    F \bigl( \frac{1}{\hbar} \intbar I \bigr) \bigr\rangle$
    is well-defined on the BRST reduction by
    Lemma~\ref{lem-new-conf-block-well-defined},
    and is meromorphic in $z_1, \dotsc, z_n$
    by Lemma~\ref{lem-meromorphic}.
    
    Write $F (x) = \sum_m a_m x^m$.
    To show that
    $\bigl\langle \cdots F \bigl( \frac{1}{\hbar} \intbar I \bigr) \bigr\rangle$
    indeed has the singularities specified by the OPE in
    $H (\scrV, I_{(0)})$, we notice that
    \begin{align*}
        & \phantom{{} = {}}
        \oint_{z_1} \ 
        (z - z_1)^k \ d z \ 
        \biggl\langle
            \scrO (z) \ 
            \scrO_1 (z_1) \cdots \scrO_n (z_n) \ 
            F \biggl( \frac{1}{\hbar} \intbar I \biggr)
        \biggr\rangle \\
        & =
        \sum _{m = 0} ^{\infty}
        \frac {a_m} {\hbar^m} \ 
        \oint_{z_1} \ 
        (z - z_1)^k \ d z
        \intbar_E \frac {d^2 w_1} {\im \tau}
        \cdots
        \intbar_E \frac {d^2 w_m} {\im \tau} \\
        & \hspace{8em}
        \langle
            \scrO (z) \ 
            \scrO_1 (z_1) \cdots \scrO_n (z_n) \ 
            I (w_1) \cdots I (w_m)
        \rangle \\
        & =
        \sum _{m = 0} ^{\infty}
        \frac {a_m} {\hbar^m} 
        \intbar_E \frac {d^2 w_1} {\im \tau}
        \cdots
        \intbar_E \frac {d^2 w_m} {\im \tau}
        \oint_{z_1} \ 
        (z - z_1)^k \ d z \\
        & \hspace{8em}
        \langle
            \scrO (z) \ 
            \scrO_1 (z_1) \cdots \scrO_n (z_n) \ 
            I (w_1) \cdots I (w_m)
        \rangle \\
        & =
        \sum _{m = 0} ^{\infty}
        \frac {a_m} {\hbar^m} 
        \intbar_E \frac {d^2 w_1} {\im \tau}
        \cdots
        \intbar_E \frac {d^2 w_m} {\im \tau} \\
        & \hspace{8em}
        \langle
            \scrO_{(k)} \scrO_1 (z_1) \ 
            \scrO_2 (z_2) \cdots \scrO_n (z_n) \ 
            I (w_1) \cdots I (w_m)
        \rangle \\
        & =
        \biggl\langle
            \scrO_{(k)} \scrO_1 (z_1) \ 
            \scrO_2 (z_2) \cdots \scrO_n (z_n) \ 
            F \biggl( \frac{1}{\hbar} \intbar I \biggr)
        \biggr\rangle ,
    \end{align*}
    where we have applied Lemma~\ref{lem-commutator} (i)
    to swap the residue operator and the regularized integrals.
    
    To show that
    $\bigl\langle \cdots F \bigl( \frac{1}{\hbar} \intbar I \bigr) \bigr\rangle$
    is compatible with the translation operator of
    $H (\scrV, I_{(0)})$, we have
    \begin{align*}
        & \phantom{{} = {}}
        \frac {\partial} {\partial z_1}
        \biggl\langle
            \scrO_1 (z_1) \cdots \scrO_n (z_n) \ 
            F \biggl( \frac{1}{\hbar} \intbar I \biggr)
        \biggr\rangle \\
        & =
        \sum _{m = 0} ^{\infty}
        \frac {a_m} {\hbar^m} \,
        \frac {\partial} {\partial z_1}
        \intbar_E \frac {d^2 w_1} {\im \tau}
        \cdots
        \intbar_E \frac {d^2 w_m} {\im \tau} \\
        & \hspace{8em}
        \langle
            \scrO_1 (z_1) \cdots \scrO_n (z_n) \ 
            I (w_1) \cdots I (w_m)
        \rangle \\
        & =
        \sum _{m = 0} ^{\infty}
        \frac {a_m} {\hbar^m} 
        \intbar_E \frac {d^2 w_1} {\im \tau}
        \cdots
        \intbar_E \frac {d^2 w_m} {\im \tau} \,
        \frac {\partial} {\partial z_1} \\
        & \hspace{8em}
        \langle
            \scrO_1 (z_1) \cdots \scrO_n (z_n) \ 
            I (w_1) \cdots I (w_m)
        \rangle \\
        & =
        \sum _{m = 0} ^{\infty}
        \frac {a_m} {\hbar^m} 
        \intbar_E \frac {d^2 w_1} {\im \tau}
        \cdots
        \intbar_E \frac {d^2 w_m} {\im \tau} \\
        & \hspace{8em}
        \langle
            T \scrO_1 (z_1) \ 
            \scrO_2 (z_2) \cdots \scrO_n (z_n) \ 
            I (w_1) \cdots I (w_m)
        \rangle \\
        & =
        \biggl\langle
            T \scrO_1 (z_1) \ 
            \scrO_2 (z_2) \cdots \scrO_n (z_n) \ 
            F \biggl( \frac{1}{\hbar} \intbar I \biggr)
        \biggr\rangle ,
    \end{align*}
    where we have applied Lemma~\ref{lem-commutator} (ii)
    to swap the partial derivative and the regularized integrals.
\end{proof}

