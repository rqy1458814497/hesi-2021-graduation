% \section{Calabi--Yau Completion}
%=========================================================
\label{sec 3}

%=========================================================
\subsection{Differential double graded categories}
%=========================================================
Let $\Bk$ be a commutative ring.
We introduce differential double graded algebras
which are differential graded algebras graded by $\ZZ \oplus G$ where $G$ is a group.

\begin{definition}
  A \textit{differential double graded algebra} (ddg algebra for short)
  $A$ is a $\ZZ \oplus G$-graded $\Bk$-algebra
  \[ A \coloneq \bigoplus_{m \in \ZZ, g \in G} A^{(m, g)} \]
  with the differential $d \colon A^{(m, g)} \to A^{(m+1, g)}$
  of degree 1 such that $d^2 = 0$ and the graded Leibniz rule
  \[ d(ab) = (da)b + (-1)^pa(db) \]
  holds for $a \in A^{(p, g)}, b \in A.$
\end{definition}

\begin{definition}
  A \textit{(right) ddg $A$-module} $M$ is a $\ZZ \oplus G$ graded $A$-module
  \[ M \coloneq \bigoplus_{m \in \ZZ, g \in G} M^{(m, g)} \]
  endowed with the differential $d \colon M^{(m,g)} \to M^{(m+1,g)}$ of degree 1
  such that $d^2 = 0$ and the graded Leibniz rule
  \[ d(ma) = (dm)a + (-1)^pm(da) \]
  holds for $m \in M^{(p, g)}, a \in A$.
  The \textit{shift} of ddg $\Bk$-module $M' = M[(k, h)]$ is defined in component
  \[ (M')^{(m,g)} \coloneq M^{(m+k,g \cdot h)}, \]
  with the new differential $d' \coloneq (-1)^kd$.
\end{definition}

Let
\[
  M^{-, g} = 
  \begin{tikzcd}[cramped, column sep=small]
    \ldots \ar[r] & M^{(i-1, g)} \ar[r, "d^{i-1}"] & M^{(i,g)} \ar[r, "d^i"]
                  & M^{(i+1, g)} \ar[r] & \ldots
  \end{tikzcd}, g \in G
\]
be a dg $A$-module.
We have the standard truncation functors $\tau^g_{\leq 0}$ and $\tau^g_{> 0}$ as:
\begin{align*}
  {}_{\tau^g_{\leq 0}}M^{-,g} &=
  \begin{tikzcd}[cramped, column sep=small]
    \ldots \ar[r] & M^{(-2,g)} \ar[r, "d^{-2}"] & M^{(-1,g)} \ar[r, "d^{-1}"]
      & \ker d^0 \ar[r] & 0\ar[r] & 0\ar[r] & 0 \ldots,
  \end{tikzcd} \\
  {}_{\tau^g_{> 0}}M^{-,g} &=
  \begin{tikzcd}[cramped, column sep=small]
  \ldots \ar[r] & 0\ar[r] & 0\ar[r] & M^{(0,g)} \quot \ker d^0 \ar[r, "d^0"]
      & M^{(1,g)} \ar[r, "d^{1}"] & M^{(2,g)} \ar[r, "d^{2}"] & M^{(3,g)} \ldots
  \end{tikzcd}
\end{align*}

Let $M$ and $N$ be two ddg $A$-modules.
A \textit{morphism $f \colon M \to N$ of ddg $A$-modules}
is a $\ZZ \oplus G$ graded $\Bk$-linear map $\hom_A(M, N)$.
The $(i, g)$-th component $\hom^{(i, g)}_A(M, N)$ is
\[
  \prod_{j \in \ZZ, h \in G} \Hom_{\Bk}(M^{(j,h)}, N^{(i+j,h\cdot g)}),
\]
consisting of morphisms $f$ such that
\[ f(ma) = f(m)a \]
holds for all $m \in M$ and $a \in A$.
Moreover, the bi-graded map $\hom_A(M, N)$
is also equipped with a differential $d$ such that
\[ d(f) = f \circ d_M - (-1)^{|f|} d_N \circ f \]
holds for a homogeneous morphism $f$ of degree $|f|$.

\begin{definition}
  A \textit{double differential graded $\Bk$-category} (ddg $\Bk$-category for short)
  is a $\Bk$-category $\mathcal{A}$ with the following data:
  \begin{itemize}
    \item a class of objects, denoted by $\Obj{\mathcal{A}}$,
    \item morphism spaces $\hom_{\mathcal{A}}(X, Y)$ as complexes of $\Bk$-modules,
      where $X, Y \in \Obj{\mathcal{A}}$,
    \item The composition maps
      \begin{align*}
        \hom_{\mathcal{A}}(Y, Z) \otimes_{\Bk} \hom_{\mathcal{A}}(X, Y)
        & \to \hom_{\mathcal{A}}(X, Z)\\
        g \otimes f &\mapsto g \circ f
      \end{align*}
      are chain maps for bi-complexes,
      namely that it satisfies the graded Leibniz rule
      \[ d(g \circ f) = d(g)\circ f + (-1)^{|f|}g \circ d(f) \]
      for homogeneous morphisms $f$ and $g$.
  \end{itemize}
\end{definition}

Let $\mathcal{A}$ and $\mathcal{B}$ be two ddg $\Bk$-categories.
A \textit{ddg functor} from $\mathcal{A}$ to $\mathcal{B}$ is a $\Bk$-linear functor
$F \colon \mathcal{A} \to \mathcal{B}$ such that all maps
\[ F(X,Y) \colon \hom_{\mathcal{A}}(X, Y) \to \hom_{\mathcal{B}}(F(X), F(Y)) \]
are chain maps for bi-complexes.
Note that the category $\dgcat$ formed by all small ddg categories
is also a ddg category, 
with morphism spaces consisting of dg functors \cite{K7}.
\begin{remark}
  Similar to the dg case \cite{K7},
  if a ddg category $\mathcal{A}$ has only one object $*$,
  it can be viewed as a ddg algebra,
  namely the endomorphism algebra $\hom_{\mathcal{A}}(*, *)$ of $*$.
\end{remark}

%=========================================================
\subsection{The derived category}
%=========================================================
Let $\Ce(A)$ denote the category of ddg $A$-modules,
where the objects are the ddg $A$-modules
and the morphisms are the morphisms of ddg $A$-modules.
Moreover, this is an abelian $\Bk$-category.
A morphism $f \colon M \to N$ of $\Ce(A)$ is called \textit{null-homotopic}
if there is a homogeneous morphism $s \colon M \to N$ of degree $-1$
such that $f = ds+sd$, where we omit the degree for simplicity.
We call a morphism $f$ in $\Ce(A)$ \textit{quasi-isomorphism}
if $H^p(f)$ is invertible for all $p \in \ZZ$.

The \textit{homotopy category} $\mathcal{H}(A)$ has the same objects as $\Ce(A)$.
Its morphisms are residue classes of morphisms of $\Ce(A)$
modulo null-homotopic morphisms.
Moreover, the homotopy category is a triangulated category.

The \textit{derived category of ddg $A$-modules},
denoted by $\D(A)$, is the localization of $\mathcal{H}(A)$
with respect to the quasi-isomorphisms.
The \textit{perfect derived category} $\per A \subset \D(A)$
is the smallest full triangulated subcategory of $\D(A)$ containing $A$
which is closed under taking shifts, extensions and direct summands.
The \textit{finite dimensional derived category} $\D_{fd}(A)$
is the full subcategory of $\D(A)$ consisting of all ddg $A$-modules
whose cohomology are of finite total dimension, i.e.,
\[ \D_{fd}(A) = \Bigl\{ M \in \D(A) \Bigm| \sum_p \dim H^p(M) < \infty \Bigr\}. \]

A ddg $A$-module $P$ is \textit{cofibrant} if,
for every quasi-isomorphism $L \to M$ which is surjective in each component,
every morphism $P \to M$ factors through $L$.
A ddg $A$-module $I$ is \textit{fibrant} if,
for every quasi-isomorphism $L \to M$ which is injective in each component,
every morphism $L \to I$ factors through $M$.
The following two diagrams illustrate the definition:

\[
  \begin{tikzcd}
    & P \ar[dl, dashed] \ar[d] \\ L \ar[r] & M,
  \end{tikzcd}
  \quad \quad
  \begin{tikzcd}
    L \ar[r] \ar[d] & M. \ar[dl, dashed] \\ I
  \end{tikzcd}
\]

\begin{proposition}\cite{K1}
  \begin{enumerate}
    \item Let $M$ be a ddg module.
      There are quasi-isomorphisms
      \[ \Bp M \to M, \quad M \to \Bi M\]
      where $\Bp M$ is cofibrant and $\Bi M$ is fibrant.
    \item The projection functor $\mathcal{H}(A) \to \D(A)$
      has a fully faithful left adjoint as $M \mapsto \Bp M$
      and a fully faithful right adjoint as $M \mapsto \Bi M$,
      which is shown in the following diagram,
      \[
        \begin{tikzcd}
          \mathcal{H}(A) \ar[d] \\
          \D(A) \ar[u, shift left=2, "\Bp"] \ar[u, shift right=2, "\Bi"']
        \end{tikzcd}
      \]
  \end{enumerate}
\end{proposition}
We call $\Bp M \to M$ a \textit{cofibrant resolution}
and $M \to \Bi M$ a \textit{fibrant resolution} of $M$.
If $A$ is a usual $\Bk$-algebra and $M$ is a right $A$-module
considered as a complex concentrated in degree $0$,
then $\Bp M \to M$ becomes a projective resolution
and $M \to \Bi M$ becomes an injective resolution of $M$.

%=========================================================
\subsection{The derived Hom and tensor functors}
%=========================================================
Let $A$ and $B$ be two ddg $\Bk$-algebras.
A ddg $B$-$A$-\textit{bimodule} $X$ is a right ddg module
over the ddg algebra $B^{\rm op} \otimes_{\Bk} A$ such that
\[ x(b \otimes a) = (-1)^{|b| \cdot |x|}bxa \]
holds for all $b \in B, x \in X, a \in A$.

Let $X$ be a ddg $B$-$A$-bimodule. We have an adjoint pair of ddg functors:
\begin{align*}
  - \otimes_B X&: \Ce(B) \to \Ce(A), \text{and}\\
  \Hom_A(X, -)&: \Ce(A) \to \Ce(B).
\end{align*}
Taking $0$-degree cohomology, we obtain an adjoint pair of triangle functors:
\[
  \begin{tikzcd}[column sep=4em, every arrow/.append style=shift left]
    \mathcal{H}(B) \ar[r, "- \otimes_B X"]
    & \mathcal{H}(A). \ar[l, "{\Hom_A(X, -)}"]
  \end{tikzcd}
\]
Generally, we have the following diagram of adjoint pairs of triangle functors:

\[
  \begin{tikzcd}[column sep=4em, every arrow/.append style=shift left]
    \D(B) \ar[r, "\Bp_B"]
    & \mathcal{H}(B) \ar[r, "- \otimes_B X"] \ar[l, "\pi_B"]
    & \mathcal{H}(A) \ar[l, "{\Hom_A(X, -)}"] \ar[r, "\pi_A"]
    & \D(A). \ar[l, "\Bi_A"]
  \end{tikzcd}
\]

The left derived functor $- \Ltimes_B X \colon \D(B) \to \D(A)$
of $- \otimes_B X$ is defined as the compositions:
\[ - \Ltimes_B X = \pi_A \circ (- \otimes_B X) \circ \Bp_B \]
which sends a ddg module $L$ to $(\Bp L) \otimes_B X$.
The right derived functor $\RHom_A(X, -) \colon \D(A) \to \D(B)$
of $\Hom_A(X, -)$ is defined as the compositions:
\[ \RHom_A(X, -) = \pi_B \circ \Hom_A(X, -) \circ \Bi_A \]
which sends a ddg module $M$ to $\Hom_A(X, \Bi M)$.
\begin{proposition}\cite{K1}
  $(- \Ltimes_B X, \RHom_A(X, -))$ forms an adjoint pair of triangle functors,
  i.e. there is a canonical isomorphism
  \[ \Hom_{\D(A)}(L \Ltimes_B X, M) \cong \Hom_{\D(B)}(L, \RHom_A(X, M)). \]
\end{proposition}

%=========================================================
\subsection{The inverse dualizing complex}
%=========================================================
Let $\Bk$ be a field and $A$ be a ddg $\Bk$-algebra.
The \textit{enveloping algebra} of $A$ is defined as $A^e \coloneq A \otimes A^{\rm op}$.
The ddg-algebra $A$ can be regarded
as an $A^e$-module through the $A$-bimodule action.

A ddg algebra $A$ is called \textit{homologically smooth} if $A$,
regarded as an $A$-bimodule,
admits a finite resolution by finite generated projective bimodules,
i.e. $A \in \per A^e$.
Let $A$ be a homologically smooth ddg algebra.
The \textit{inverse dualizing complex} $\Theta_A$
is given by the cofibrant resolution of $\RHom_{A^e}(A, A^e)$.
\begin{remark}
  If $A$ is homologically smooth,
  \cite[Lemma 4.1]{K1} implies that the finite-dimensional category
  $\D_{fd}(A)$ is a subcategory of $\per A$.
  Hence, it motivates us to define $\per A \quot \D_{fd}(A)$ by Verdier quotient.
\end{remark}

\begin{lemma}\cite[Lemma 4.1]{K1}
  Let $A$ be a homologically smooth ddg $\Bk$-algebra.
  For any ddg module $L \in \D(A)$ and $M \in \D_{fd}(A)$,
  there exists a canonical isomorphism
  \[
    \DHom_{\D(A)}(M,L)
    \xrightarrow{\sim} \Hom_{\D(A)}\bigl(L \Ltimes_A \Theta_A, M\bigr)
  \]
  where $D = \Hom_{\Bk}(-, \Bk)$.
\end{lemma}

\begin{definition}
  Let $\mathcal{N} \in \ZZ \oplus G$ where $G$ is a group.
  We call a ddg algebra $A$ a \textit{Calabi--Yau-$\mathcal{N}$}
  algebra if $A$ is homologically smooth and it satisfies
  \[ \Theta_A \xrightarrow{\sim} A[-\mathcal{N}] \]
  in $\per A$.
  Similarly, a category $\mathcal{C}$ is called
  a \textit{Calabi--Yau-$\mathcal{N}$} category if
  \[ \Hom(X, Y) \simeq \DHom(Y, X[\mathcal{N}]) \]
  holds for any $X, Y \in \mathcal{C}$.
\end{definition}

By definition, we observe that the shift functor $[\mathcal{N}]$
is a Serre functor in a Calabi--Yau-$\mathcal{N}$ category.
In particular, if $A$ is a Calabi--Yau-$\mathcal{N}$ algebra,
then $\D_{fd}(A)$ is a Calabi--Yau-$\mathcal{N}$ triangulated category.

%=========================================================
\subsection{Calabi--Yau-\texorpdfstring{$g$}{g} completions for ddg algebras}
%=========================================================
Let $G$ be a group and $g$ be an element in $G$.
We regeneralize the definition of Calabi--Yau completions
for dg algebras in \cite{K2}.

\begin{definition}\label{CYcomp}
  Let $A$ be a homologically smooth ddg $\Bk$-algebra
  and $\Theta_A$ be its inverse dualizing complex.
  The \textit{Calabi--Yau-$g$ completion} of $A$ is defined as the tensor algebra
  \[
    \Pi_g(A) \coloneq T_A(\theta)
    = A \oplus \theta \oplus (\theta \otimes_A \theta) \oplus \cdots
  \]
  where $\theta \coloneq \Theta_A[g-1]$.
\end{definition}

Moreover, Keller's results \cite{K2} still holds in double grading case.

\begin{theorem}\label{DfdCY}\cite[Theorem 4.8]{K2}
  Let $A$ be a homologically smooth ddg $\Bk$-algebra.
  The Calabi--Yau-$g$ completion $\Pi_gA$ of $A$ is homologically smooth
  and is a Calabi--Yau-$g$ algebra.
  Particularly, $\D_{fd}(\Pi_gA)$ is a Calabi--Yau-$g$ category.
\end{theorem}

\begin{remark}
  If we consider the case when $G$ is a free abelian group $\ZZ\XX$
  of rank one generated by $\XX$, we deduce Theorem 2.4 in \cite{Q12}.
\end{remark}

%=========================================================
\subsection{Ginzburg ddg algebra}
%=========================================================
\begin{definition}
  Let $(Q, W)$ be a $\ZZ$-graded quiver with potential $W$
  where $Q = (Q_0,Q_1)$ with vertices set $Q_0 = \{1, 2, \ldots, n\}$
  and arrows set $Q_1$.
  We define the \textit{Ginzburg ddg algebra}
  $\Gamma_g(Q, W) \coloneq (\Bk \overline{Q}, d)$ as follows.
  The $\ZZ \oplus G$-graded quiver $\overline{Q}$ has the vertices set
  $\overline{Q}_0$ the same as the original vertices set $Q_0$.
  The arrows set of $\overline{Q}$ consists of the following data:
  \begin{itemize}
    \item Original arrows $a \colon i \to j \in Q_1$ with degree $d$;
    \item Opposite arrows $a^* \colon j \to i$ for the original arrow
      $a \colon i \to j \in Q_1$ endowed with degree $2 - g - d$;
    \item A loop $t_i$ for each vertex $i \in Q_0$ endowed with degree $1 - g$.
  \end{itemize}
  Consider the ddg path algebra $\Bk \overline{Q}$ which is graded by $\ZZ \oplus G$.
  We define a differential $d \colon \Bk \overline{Q} \to \Bk \overline{Q}$
  of degree 1 on it by
  \begin{itemize}
    \item $da = 0$ for $a \in Q_1$;
    \item $da^* = \partial_aW$ for $a \in Q_1$, where $\partial_aW$
      represents the cyclic derivative of $W$ with respect to $a$;
    \item $dt_i = e_i\bigl(\sum_{a \in Q_1}(aa^* - a^*a)\bigr) e_i$
      for $i \in Q_0$, where $e_i$ is the idempotent at $i$.
  \end{itemize}
  Thus we obtain the ddg algebra $\Gamma_g(Q,W) = (\Bk \overline{Q}, d)$.
\end{definition}
Particularly if $W = 0$, it degenerates into the acyclic quiver case,
which implies $da = da^* = 0$ for $a \in Q_1$.
We denote $\Gamma_gQ$ the Ginzburg ddg algebra when $W = 0$,
and $\D_{fd}(\Gamma_gQ)$ the finite-dimensional derived category of $\Gamma_gQ$.
Analogous to \cite[Corollary 2.8]{Q12},
$\Gamma_gQ$ is the Calabi--Yau-$g$ completion of $\Bk Q$.

\begin{theorem}\cite{Q12}\label{3.10}
  With the notation above,
  the Calabi--Yau-$g$ completion $\Pi_g(\Bk Q)$ of the path algebra $\Bk Q$
  is quasi-isomorphic to the Ginzburg Calabi--Yau-$g$ algebra $\Gamma_gQ$.
  Particularly, $\D_{fd}(\Gamma_gQ)$ is a Calabi--Yau-$g$ triangulated category.
\end{theorem}
%=========================================================
