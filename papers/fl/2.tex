% \section{Quiver Representations}
%=========================================================
\subsection{Quiver and path algebra}
%=========================================================
\begin{definition}
  A \textit{quiver} $Q =(Q_0,Q_1,s,t)$ consists of the data below:
  \begin{itemize}
    \item $Q_0$ is a set of vertices.
    \item $Q_1$ is a set of arrows.
    \item A map $s \colon Q_1\to Q_0$ sends an arrow to its starting point.
    \item A map $t \colon Q_1\to Q_0$ sends an arrow to its terminal point.
  \end{itemize}
  Moreover, if the set $Q_1$ of arrows is equipped with a grading,
  i.e., a map
  \[ Q_1 \to \ZZ,\quad a \mapsto |a|, \]
  we obtain a \textit{graded quiver} $Q$.
\end{definition}

\begin{example} \label{ex}
  Let $Q$ be the following quiver
  \[
    \begin{tikzcd}
    & 2 \ar[dr, "\beta"] & \\
      1 \ar[ur, "\alpha"] \arrow[rr, "\gamma"'] & & 3.
    \end{tikzcd}
  \]
  If we add grading to each arrow by $|\alpha| = |\beta| = 0$ and $|\gamma| = -1$,
  we get a graded quiver.
\end{example}

Let $Q = (Q_0, Q_1)$ be a quiver and $i, j \in Q_0$.
A \textit{path} $c$ from $i$ to $j$ of length $l$ in $Q$
is a composition of arrows, i.e., a sequence
\[ c = \alpha_1 \alpha_2 \ldots \alpha_l \]
with $s(\alpha_1) = i, s(\alpha_h) = t(\alpha_{h-1})$
for $h =2, 3, \ldots, l,$ and $t(\alpha_l) = j$.
The grading of $c$ is defined to be
\[ |c| = \sum_{i=1}^l |\alpha_i|.  \]
In addition, at each vertex $i \in Q_0$,
there is a trivial path $e_i$ of length zero with $s(e_i) = t(e_i) = i$.

\begin{definition}
  Let $Q$ be a graded quiver.
  The \textit{path algebra} $\Bk Q$ of $Q$ is the algebra
  whose basis set consists of all paths in $Q$.
  The multiplication of two basis elements $c=\alpha_1 \alpha_2 \ldots\alpha_k$
  and $c'= \beta_1 \beta_2\ldots\beta_l$ is defined by
  \begin{equation*}
    cc' =
    \begin{cases}
      \alpha_1 \alpha_2 \ldots \alpha_k \beta_1 \beta_2 \ldots \beta_l
        & \text{if} \ t(\alpha_k)= s(\beta_1),\\
      0 & \text{otherwise.}
    \end{cases}
  \end{equation*}

  Thus we can define the product of any two elements
  $\sum_{c} \lambda_c c$ and $\sum_{c'} \lambda'_{c'} c'$ by linear combination
  as $\sum_{c,c'} \lambda_c \lambda'_{c'} cc'$.
  We observe that the unit in $\Bk Q$ is given by the sum of all trivial paths as
  \[ 1 = \sum_{i\in Q_0} e_i. \]
  %The grading of $cc'$ is $|cc'| = |c|+|c'|.$
\end{definition}

\begin{definition}
  The \textit{path category} $\mathcal{P}Q$ of quiver $Q$
  has the objects the same as the vertices in $Q_0$,
  and its morphisms from $i$ to $j$ are all formal linear combinations
  of paths $i$ to $j$ of length greater than zero, where $i,j \in Q_0$.
  In particular, the trivial path $e_i$ of length 0 induces identity.
\end{definition}
In Example \ref{ex}, we have
\begin{align*}
  \Hom_{\mathcal{P}Q}(i, i) &= \Bk e_i, \forall i \in \{1,2,3\},\\
  \Hom_{\mathcal{P}Q}(1, 2) &= \Bk\alpha, \Hom_{\mathcal{P}Q}(2, 3)= \Bk \beta, \\
  \Hom_{\mathcal{P}Q}(1,3) &= \Bk \gamma \oplus \Bk \alpha \beta,
\end{align*}
where $|\gamma| = -1$ and $|\alpha\beta| = 0$.

Denote $Q_l$ all paths of length $l$ in $Q$.
Note that the path algebra $\Bk Q$ is graded by path length, i.e.,
\[ \Bk Q = \bigoplus_{l \in \mathbb{N}} \Bk Q_l, \]
where for each $l \geqslant 0$, $\Bk Q_{l}$ is
the subspace of $\Bk Q$ generated by $Q_{l}$.
It is clear that 
\[ (\Bk Q_{n}) \cdot (\Bk Q_{m}) \subseteq (\Bk Q_{n+m}) \]
for all $n, m \geqslant 0$.
If the graded algebra $\Bk Q$ is equipped with a differential
satisfying the graded Leibniz rule,
we have the \textit{differential grading (dg for short) path algebra}.

In Example \ref{ex}, $\mathcal{P}Q$ has a unique differential of dg category
such that $d(\gamma) = \alpha \beta$.
The associated dg path algebra is the matrix algebra
\[ A = \oplus_{i,j \in Q_0} \Hom_{\mathcal{P}Q}(i,j). \]
More details of differential grading (dg for short) category
will be introduced in subsection 3.

Let $Q$ be a quiver. A \textit{representation}
$M = (M_i, \varphi_{\alpha})_{i \in Q_0,\alpha \in Q_1}$ of $Q$
consists of a collection of $\Bk$-vector spaces $\{M_i\}_{i \in Q_0}$
and a collection of $k$-linear maps
\[
  \bigl\{\varphi_{\alpha} : M_{s(\alpha)} \to M_{t(\alpha)}\bigr\}_{\alpha \in Q_1}.
\]
Let $M = (M_i, \varphi_{\alpha}), M' = (M'_i, \varphi'_{\alpha})$
be two representations of $Q$.
A \textit{morphism of representations} $f \colon M \to M'$
is the datum $(f_i)_{i \in Q_0}$ of linear maps
\[ f_i: M_i \to M'_i \]
that are compatible with the structure maps $\varphi_{\alpha}$,
that is, for each arrow $i \stackrel{\alpha}{\to} j$ in $Q_1$, the diagram
\[
  \begin{tikzcd}
    M_i \ar[d, "f_i"'] \ar[r, "\phi_\alpha"] & M_j \ar[d, "f_j"] \\
    M_i' \ar[r, "\phi'_\alpha"'] & M_j'
  \end{tikzcd}
\]
commutes.

In Example \ref{ex}, a representation of $Q$ is a diagram
\[
  \begin{tikzcd}
    & M_2 \ar[dr, "\phi_\beta"] & \\
    M_1 \ar[ur, "\phi_\alpha"] \ar[rr, "\phi_\gamma"'] & & M_3.
  \end{tikzcd}
\]
which may not be commutative.

Let $\rep_{\Bk}(Q$) be the category formed by
finite dimensional $\Bk$-linear representations of $Q$.
Suppose $Q$ is a finite, connected and acyclic quiver.
A classical result says that there exists an equivalence of Abelian categories
\begin{equation}\label{rep}
  \mod \Bk Q \cong \rep_{\Bk}(Q).
\end{equation}

A \textit{relation} $R$ in a quiver $Q$ is a set generated by
linear combinations as the form $\sum \lambda_c c$
of paths in $Q$ with length greater than one.
We introduce a method of adding some differential to ``kill'' the relations.

\begin{construction}\cite[Construction 2.2]{Op}
  Let $A = \Bk Q^{(0)} \quot R$ be a dg path algebra with relation $R$.
  First we take a minimal set of relations,
  and let $Q^{(1)}$ be the graded quiver by adding to $Q^{(0)}$
  new arrows corresponding to the relation $R$
  with degrees of the corresponding relation minus 1,
  and a differential of degree 1, such that $H^0(kQ^{(1)}) = A$.
  Then we add arrows in degree of the corresponding new relations minus 2
  to $Q^{(1)}$ to kill the generators in $H^1(kQ^{(1)})$.
  The new quiver with differential we obtain is denoted by $Q^{(2)}$.
  Hence, we have $H^1(kQ^{(2)}) = 0$ and $H^0(kQ^{(2)}) = A$.
  Considering $Q^{(2)}$ and iterating the process above,
  we obtain a quiver $Q$ such that $\Bk Q$ is quasi-isomorphic to $A$.\label{Opp}
\end{construction}
For example, let $A = \Bk Q \quot R$ where
\[
  Q = 
  \begin{tikzcd}
    1 \ar[r, "\alpha"] & 2 \ar[r, "\beta"] & 3
  \end{tikzcd}
\]
with $|\alpha| = d_1$ and $|\beta| = d_2$ and $R$ is generated by $\alpha\beta$.
Then $A$ is quasi-isomorphic to $\Bk Q'$ where
\[
  Q' =
  \begin{tikzcd}
    1 \ar[r, "\alpha"] \arrow[rr, bend right, "\gamma"'] & 2 \ar[r, "\beta"] & 3
  \end{tikzcd},
\]
where $|\gamma| = d_1+d_2-1$ and $d\gamma = \alpha\beta$.
%=========================================================
\subsection{Auslander--Reiten theory}
%=========================================================
Let $\Bk$ be a commutative ring and $A$ be a finite dimensional $\Bk$-algebra,
we construct Auslander--Reiten (AR for short) sequence in $\mod A$.
The $A$-dual functor is defined as the left exact contravariant functor:
\[ (-)^{t} \coloneq \Hom_A(-, A) \colon \mod A \to \mod A^{\rm op}. \]
It induces a duality between the category $\Proj A$ and $\Proj A^{\rm op}$,
where $\Proj A$ (resp. $\Proj A^{op}$) represents
the category of projective right (resp. left) $A$-modules.
Let $M$ be a right $A$-module,
and we take the minimal projective presentation of $M$, i.e., the exact sequence
\begin{equation}\label{projp}
  \begin{tikzcd}
    P_1 \rar{p_1} & P_0 \rar{p_0} & M \rar & 0,
  \end{tikzcd}
\end{equation}
where both $p_0: P_0 \to M$ and $p_1: P_1 \to \ker p_0$ are projective covers. Applying $(-)^{t}$ to (\ref{projp}), we get an exact sequence of left $A$-modules
\begin{equation}
  \begin{tikzcd}
    0 \ar[r] & M^t \ar[r, "p_0^t"] & P_0^t \ar[r, "p_1^t"]
             & P_1^t \ar[r] & \coker p_1^t \ar[r] & 0.
  \end{tikzcd}
\end{equation}
With the assumption above,
we set $\Tr M \coloneq \coker p_1^t$ and call it the \textit{transpose} of $M$.

Note that $M$ is a projective $A$-module if and only if $\Tr M = 0$.
So the functor $\Tr$ kills the projectives in $\mod A$.
Let $M$ and $N$ be two $A$-modules.
We denote $\mathcal{P}(M, N)$ the subset of $\Hom_A(M, N)$
formed by all homomorphisms which factor through some projective $A$-modules.
We then consider the \textit{projective stable category} $\modd A$,
whose objects are same as the objects in $\mod A$.
The morphism space $\Homm_A(M,N)$ from $M$ to $N$ in $\modd A$
is defined as the quotient
\[ \Homm_A(M,N) \coloneq \Hom_A(M, N) \quot \mathcal{P}(M, N). \]

\begin{propdef}\cite[Proposition 2.2]{ASS}
  The functor $\Tr$
  \[ \Tr \colon \modd A \to \modd A^{op}, \]
  which is called \textit{transposition} induces
  a $\Bk$-linear duality between categories $\modd A$ and $\modd A^{op}$.
  The \textit{Auslander--Reiten translation} $\tau$
  is defined as the compositions of standard duality
  $D = \Hom_{\Bk}(-,\Bk)$ with $\Tr$, i.e.,
  \[ \tau = D \circ \Tr. \]
\end{propdef}
Dually, we have the definition of \textit{injective stable category}
$\overline{\mod} A$ and the inverse of the Auslander--Reiten
translation $\tau^{-1} = \Tr \circ D$.
Then $\tau$ and $\tau^{-1}$ induce an equivalences between
$\modd A$ and $\overline{\mod} A$ as \cite[Corollary 2.11]{ASS}:
\[
  \begin{tikzcd}
    \modd A \ar[r, shift left, "\tau"]
    & \overline{\mod} A \ar[l, shift left, "\tau^{-1}"].
  \end{tikzcd}
\]

\begin{theorem}[the Auslander--Reiten formulas]\cite[Theorem 2.13]{ASS}
  Let $A$ be a $\Bk$-algebra and $M$, $N$ be two $A$-modules.
  Then there exist natural isomorphisms
  \begin{equation}\label{AR}
    D \Homm_A(\tau^{-1}N, M) \cong \Ext_A^1(M, N)
    \cong D\overline{\Hom}_A(N, \tau M).
  \end{equation}
\end{theorem}

Now we consider the triangulated setting.
An important thing is that the Auslander--Reiten translation $\tau$
can be realized as an auto-equivalence on the bounded derived category
$\D^b(A)$ of $\mod A$, which we will further discuss in Section~\ref{sec 2.4}.
There are many equivalent definitions of Auslander--Reiten triangle
and we list one of them.

\begin{definition}
  Let $\mathcal{T}$ be a triangulated category. A triangle
  $\begin{tikzcd}[cramped, sep=small, scale cd=0.9]
    X \ar[r, "u"] & Y \ar[r, "v"] & Z \ar[r, "w"] & X[1]
  \end{tikzcd}$
  in $\mathcal{T}$ is called an \textit{Auslander--Reiten triangle, AR-triangle}
  for short, if it satisfies the following conditions:
  \begin{itemize}
    \item The objects $X, Z$ are indecomposable.
    \item The morphism $w$ is not zero.
    \item If $f \colon W \to Z$ is not a retraction,
      then there exists $f' \colon W \to Y$ such that $f'v = f$.
  \end{itemize}
\end{definition}

We say that a triangulated category $\mathcal{T}$
has Auslander--Reiten triangles if for every indecomposable object $X$,
there are Auslander--Reiten triangles as follows:
\[ X \to Y \to Z \to X[1]\quad\text{and}\quad V \to W \to X \to V[1]. \]


%=========================================================
\subsection{Example: Auslander--Reiten quiver}
%=========================================================
\begin{definition}
  The \textit{Auslander--Reiten quiver} of a category $\mathcal{C}$
  is a directed graph $\Gamma_{\mathcal{C}}$ defined as follows:
  \begin{itemize}
    \item The vertices of $\Gamma_{\mathcal{C}}$
      are the isomorphism classes $[M]$ of indecomposable objects in $\mathcal{C}$.
    \item For two vertices $[M]$, $[N]$ in $\Gamma_{\mathcal{C}}$,
      the arrows $[M] \to [N]$ are irreducible morphisms
      which does not factor nontrivially through another morphism.
    \item The dash arrows represent the Auslander--Reiten translation.
  \end{itemize}
  Particularly, if $\mathcal{C}$ is an abelian category,
  the vertices of $\Gamma_{\mathcal{C}}$
  are finitely generated indecomposable modules in $\mathcal{C}$.
  Let $Q$ be a finite, connected and acyclic quiver. By the isomorphism
  \[ \rep_{\Bk}(Q) \cong \mod \Bk Q \]
  in \eqref{rep}, we define that the Auslander--Reiten quiver of $Q$
  is $\Gamma_{\mod \Bk Q}$ and we always denote it by $\Gamma_Q$.
\end{definition}

\begin{example}
  Let $Q$ be the quiver $1 \to 2 \to 3$.
  Its Auslander--Reiten quiver is shown in Figure~\ref{fig:label1},
  \begin{figure}[htbp]
    \centering
    \begin{tikzcd}
                  & & P_1 \cong I_3 \ar[dr] & & \\
                  & P_2 \ar[ur] \ar[dr] & & I_2 \ar[ll, dashed] \ar[dr] & \\
      P_3 \ar[ur] & & S_2 \ar[ll, dashed] \ar[ur] & & I_1 \ar[ll, dashed]
    \end{tikzcd}
    \caption{The Auslander--Reiten quiver of $Q = 1 \to 2 \to 3$}
    \label{fig:label1}
  \end{figure}
  where the dashed arrows are the Auslander--Reiten translations $\tau$.
\end{example}

\begin{example}
  Let $Q$ be the AR-quiver of $1 \to 2 \to 3$.
  Then the category $\mod \Bk Q$ is hereditary.
  The quiver of $\D^b(\mod \Bk Q)$ is of the form in Figure~\ref{fig:label2},
  \begin{figure}[htbp]
    \centering
    \begin{tikzcd}[column sep=1em, row sep=.6em]
        & \circ \ar[dr] & & \circ \ar[dr] & & \bullet
      \makebox[0pt]{\qquad\quad (0, 3)} \ar[dr] &
        & \circ \ar[dr] & & \circ \ar[dr] & & \circ \ar[dr] & & \bullet \\
      \cdots & & \circ \ar[ur] \ar[dr] & & \bullet
      \makebox[0pt]{\qquad\quad (0, 2)}\ar[ur] \ar[dr]
        & & \bullet \ar[ur] \ar[dr] & & \circ \ar[ur] \ar[dr]
        & & \circ \ar[ur] \ar[dr] & & \bullet \ar[ur] \ar[dr] & & \cdots \\
        & \circ \ar[ur] & & \bullet \makebox[0pt]{\qquad\quad (0, 1)} \ar[ur]
        & & \bullet \makebox[0pt]{\qquad\quad (1, 1)} \ar[ur] & & \bullet \ar[ur]
        & & \circ \ar[ur] & & \bullet \ar[ur] & & \bullet 
    \end{tikzcd}
    \caption{The Auslander--Reiten quiver of $\D^b(\mod \Bk Q)$}
    \label{fig:label2}
  \end{figure}
  which is isomorphic to the infinite translation quiver $\ZZ Q$
  whose vertices are $\ZZ \times Q$.
  Moreover, the Auslander--Reiten translation $\tau$
  is given by $(g, h) \mapsto (g - 1, h)$.
  %The indecomposable objects in $\mod \Bk Q$
  %are indecomposable under the translation functor.
\end{example}

%=========================================================
\subsection{Cluster categories as orbit category}
%=========================================================
We first recall the definition of cluster categories as orbit categories.
Let $\Bk$ be a field and $A$ be a finite-dimensional $\Bk$-algebra.
We denote $\D^b(A)$ the bounded derived category of $\mod A$. \label{sec 2.4}
\begin{definition}
  We say the category $\mathcal{C}$ admits a \textit{Serre functor} $\BBS$
  if there is an isomorphism
  \begin{equation}\label{2.2}
    \Hom(X, Y) \cong \DHom(Y, \BBS X)
  \end{equation}
  for any $X, Y \in \mathcal{C}$.
\end{definition}

\begin{theorem}\cite[Theorem 3.1]{K1}\label{3equiv}
  With the assumption above, the following are equivalent.
  \begin{enumerate}
    \item $\D^b(A)$ admits a Serre functor $\BBS$.
    \item $\D^b(A)$ admits Auslander--Reiten triangles.
    \item The global dimension of $A$ is finite.
  \end{enumerate}
\end{theorem}
Suppose that $A$ is a finite-dimensional associative $\Bk$-algebra
of finite global dimension with unit.
The bounded derived category $\D^b(A)$ is $\Hom$-finite
and is a Krull-Schmidt category.
Moreover, $\D^b(A)$ admits a Serre functor $\BBS$,
which is given by the left derived functor \cite{K1}
\[ \BBS = - \Ltimes_A DA, \]
where $D = \Hom_{\Bk}(-, \Bk)$.

As in Theorem \ref{3equiv},
the existence of Auslander--Reiten translation functor $\tau$
is equivalent to the existence of Serre functor $\BBS$.
Furthermore, the Auslander--Reiten formula \eqref{AR}
can be written in the triangulated version
\begin{equation}\label{2.3}
  \Hom(X, Y) %\cong \Ext^1(X, Y[-1])\cong \DHom(Y[-1],\tau X)
  \cong \DHom(Y,\tau X[1]),
\end{equation}
where $[1]$ represents the shift functor, for any $X, Y \in \D^b(A)$.
By comparison with the Serre duality in (\ref{2.2}),
the relation between the Serre functor
and the Auslander--Reiten translation is given by
\begin{equation}\label{rea}
  \BBS = \tau \circ [1].
\end{equation}
Hence, the Auslander--Reiten translation
\begin{equation}\label{rela}
  \tau = \BBS \circ [-1]
\end{equation}
is an autoequivalence of $\D^b(A)$.

\begin{definition}
  Let $\mathcal{C}$ be an additive category
  and $F$ be an autoequivalence of $\mathcal{C}$.
  The \textit{orbit category} $\mathcal{C} \quot F$
  is the additive category defined as follows.
  \begin{itemize}
    \item The objects are the same as objects in $\mathcal{C}$.
    \item The morphism space is
      \[
        \Hom_{\mathcal{C} \quot F} \coloneq
        \oplus_{p \in \ZZ} \Hom_{\mathcal{C}}(X, F^pY)
      \]
      for any objects $X, Y\in \mathcal{C}$.
  \end{itemize}
\end{definition}

\begin{remark}
  The orbit category $\mathcal{T} \quot F$ of a triangulated category $\mathcal{T}$,
  where $F$ is an autoequivalence on $\mathcal{T}$,
  is not trivially triangulated \cite{K3}.
  Not all triangles in $\mathcal{T} \quot F$
  are induced from the triangles in $\mathcal{T}$.
  With extra assumptions in \cite[Theorem 1]{K3},
  Keller pointed out a way to endow the orbit category $\mathcal{T} \quot F$
  with a triangulated structure naturally.
  What's more, the projection functor $\mathcal{T} \to \mathcal{T} \quot F$
  is a strict triangulated functor \cite{K3}.
\end{remark}

\begin{definition}
  Let $A$ be a finite-dimensional associative $\Bk$-algebra
  of finite global dimension with unit,
  and $N$ be an integer which is greater than 2.
  The \textit{$N$-cluster category} of $A$ is the $\Bk$-linear orbit category
  \[ \Ce_N(A) = \D^b(A) \quot (\tau \circ [1-N]), \]
  defined by the autoequivalence $\tau \circ [1-N]$.
\end{definition}

By definition, the objects of $\Ce_N(A)$ are the same as the objects of $\D^b(A)$
and its morphisms are of the form
\[
  \Hom_{\Ce_N(A)}(X, Y)
  = \bigoplus_{p \in \ZZ} \Hom_{\D(A)}(X, (\tau \circ [1-N])^pY)
\]
where $X, Y$ are objects in $C_N(A)$.
From the construction, the Serre functor of $\D^b(A)$
naturally induces a Serre functor of $\mathcal{C}_NA$
and we still denote it by $\BBS$.
The equation~\eqref{rela} implies that there is an isomorphism
\[ \BBS \simeq [N] \]
in $\mathcal{C}_NA$.
The category is called a \textit{Calabi--Yau-$N$ category},
if it admits a Serre functor $\BBS$
which is isomorphic to the composition of shift functor.
In subsection \ref{sec 3}, we will introduce the Calabi--Yau property in details.
\begin{definition}
  Let $\mathcal{C}_N(A)$ be a $N$-cluster category of $A$ defined as above.
  An \textit{$N$-cluster-tilting set} $\{T_j\}_{j=1}^n$ in $\mathcal{C}_N(A)$
  is a maximal set of non-isomorphic indecomposables such that
  \[ \Ext^k_{\mathcal{C}_N(A)}(T_i, T_j) = 0 \]
  for all $1 \leq i, j \leq$ and $1 \leq k \leq N$.
  An \textit{$N$-cluster-tilting object} $\textbf{T} = \oplus_{i=1}^n T_i$
  is the sum of them.
\end{definition}
\begin{remark} \hfill
  \begin{enumerate}
    \item We have two actions on a given $N$-cluster tilting set $\{T_j\}_{j=1}^n$.
      One is the \textit{forward mutation} $\mu_i$ which sends $T_i$ to
      \[
        T_i^\sharp
        = \Cone\Bigl(T_i \to \bigoplus_{j\neq i} \Irr(T_i,T_j)^* \otimes T_j\Bigr),
      \]
      and the other is the \textit{backward mutation} $\mu_i^{-1}$ which sends $T_i$ to
      \[
        T_i^\flat
        = \Cone\Bigl(\bigoplus_{j\neq i} \Irr(T_j,T_i) \otimes T_j \to T_i\Bigr)[-1].
      \]
      More details can be found in \cite[definition 4.1]{KQ}.
      The mutations of the cluster-tilting set in cluster category
      correspond to the mutations of the seeds in cluster algebra \cite{K5},
      which is one aspect in the categorification of cluster algebra.
    \item Assume that $N = \infty$,
      then the $\infty$-cluster category is $\D^b(A)$ itself.
      In this case, the cluster-tilting set (resp. object)
      becomes the silting set (resp. object, see \cite{IY2}) in $\D^b(A)$.
  \end{enumerate}
\end{remark}

Let $\pi \colon \D^b(A) \to \Ce_N(A)$ be the natural projection functor.
We have the following theorem.
\begin{theorem}\cite[Theorem 3.2]{K1}
  Suppose the extra hypothesis that $A$ is a dg $\Bk$-algebra of global dimension 1.
  Then the $N$-cluster category is $\Hom$-finite,
  and has a natural triangulated structure
  which makes the projection $\pi$ become a triangle functor.
  Moreover, the $N$-cluster category is a Calabi--Yau-$N$ triangulated category.
\end{theorem}
%=========================================================
