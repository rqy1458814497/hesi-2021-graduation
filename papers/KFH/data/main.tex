% !TeX root = ../KFH.tex

\section{Introduction}

In \cite{L}, Lusztig introduced a twisted bialgebra $\f$, which is 
the quantized enveloping algebra associated to a maximal nilpotent 
\
subalgebra of a finite dimensional complex semisimple Lie algebra.
In simply laced types, Lusztig gave a categorification of $\f$ in terms of certain
categories of sheaves on some quiver variety, which transforms
the multiplication on $\f$ to the Lusztig's convolution product.

In \cite{KL1}, \cite{KL2} and \cite{R}, Khovanov, Lauda and Rouquier
introduced a new type of family of algebras associated to a quiver,
known as the \term{quiver Hecke algebras}. They proved that under the assumption
that the quiver has no edge loops, the quiver Hecke algebras
categorify the Lusztig's twisted bialgebra $\f$.

Later, in \cite{VV}, Varagnolo and Vasserot showed that 
in the case that the underlying field is of characteristic zero,
quiver Hecke algebras are naturally isomorphic to the equivariant 
extension algebras of certain equivariant complexes of sheaves
on the quiver variety, which links the algebraic categorification
and the geometric categorification of $\f$.

From this geometric interpretation of quiver Hecke algebras,
Kato introduced certain infinite-dimensional modules of quiver Hecke algebras,
called the \term{geometric standard modules}, in \cite{K} and \cite{K2},
where the definition and certain homological properties are introduced.
These modules satisfy properties familiar from the theory of
highest weight categories. 
These modules are also studied later by McNamara in \cite{Mc2}.

On the other hand, in \cite{BKM}, Brundan, Kleshchev and McNamara constructed
certain modules that satisfy similar properties to that of 
geometric standard modules, in a purely algebraic way.
These modules are called \term{algebraic standard modules}.

In this paper, we will show that the algebraic standard modules
and the geometric standard modules coincide. We will also
recall the definition of \term{polynomial highest weight categories},
which is a modification of highest weight categories, and survey on that
the categories of finitely generated graded modules of quiver Hecke algebras
are polynomial highest weight categories, by using the 
homological properties of standard modules. Our main theorems 
are Theorem \ref{alg-geo} and Theorem \ref{p-h-w-c}.

The paper is mainly divided into four parts. In Chapter 2 we will
recall the definition of quiver Hecke algebras, and their basic properties;
we will also recall the Lusztig's twisted bialgebra $\f$ and give a
sketch of proof of the categorification theorem, Theorem
\ref{categorification}. 
In Chapter 3 we will recall 
the geometric realization theorem, Theorem \ref{geo-real};
we will give the motivation and definition of geometric standard modules,
and list some basic properties of geometric standard modules.
In Chapter 4 we will give the algebraic 
construction of standard modules. 
In Chapter 5 we will give the two main theorems of the paper;
We will show that the algebraic standard modules
and the geometric standard modules coincide;
we will recall the definition of polynomial highest weight categories,
and show that the categories of finitely generated graded modules of quiver Hecke algebras
are polynomial highest weight categories.

\begin{convention}
    Fix once for all a ground field $\K$. All algebras, if not specified, are over $\K$.
    By a module over a $\Z$-graded algebra $H$, we always mean a graded left
    $H$-module. Likewise all submodules, quotient modules are also graded.
    For a module $V$, $\rad(V)$ is the intersection of all maximal submodules,
    where $\soc(V)$ is the sum of all irreducible submodules of $V$.
    We write $q$ for the upward degree shifting functor: if $V = \bigoplus V_n$
    then $(qV)_n = V_{n-1}$. Given a formal Laurent series $f(q) = \sum f_nq^n$,
    defined $f(q)V\coloneq  \bigoplus (q^nV)^{\oplus f_n}$. For modules $U,V$,
    we write $\hom(U,V)$ for all homogeneous $H$-module morphisms, and write
    $\Hom(U,V)$ for the graded vector space $\bigoplus\hom(U,V)_n$,
    where $\Hom(U,V)_n\coloneq  \hom(q^nU,V) = \hom(U,q^{-n}V)$. $\ext^k(U,V)$
    and $\Ext^n(U,V)$ are defined similarly. If $V$ is locally finite dimensional
    and bounded below, its graded dimension is the formal Laurent series
    $\Dim V = \sum(\dim V_n)q^n$.

    Let $\Gamma$ be a quiver with finite vertex set $I$
    and arrow set $E$, such that there exists no arrow
    that join a vertex to itself. For any arrow $e\in E$,
    we define $s(e)$ to be the source of $e$, and 
    $t(e)$ to be the target of $e$. For any $i,j\in I$, we define
    $m_{i,j}$ to be the number of directed edges $i\to j$. 
    The corresponding Cartan matrix $C = (c_{i,j})$ 
    is defined as $c_{i,i} = 2$ and $c_{i,j} = -m_{i,j}-m_{j,i}$ for $i\ne j$.
    Thus $(I,C)$ is a Cartan datum, which gives a root lattice.
    We will denote the simple roots by $\alpha_i,\ i\in I$, and the set of positive roots by
    $R^+$, and define $Q^+ \coloneq  \bigoplus_{i\in I}\mathbb{N}\alpha_i$. For 
    an element $\alpha = \sum_{i\in I}c_i\alpha_i\in Q^+$, we define its \term{height}
    to be $\operatorname{ht}(\alpha) \coloneq  \sum_{i\in I}c_i$. Also let $W$ be the Weyl group, with generators
    $\{s_i\mid i\in I\}$. Let $\mathfrak{g}$ be the corresponding Kac--Moody
    algebra.

    The path algebra of $\Gamma$ is denoted by $\K\Gamma$.
    Giving a $\K\Gamma$-module, is equivalent to giving an $I$-graded vector space
    $V = \bigoplus_i V_i$, and for each $e\in E$, a map of vector spaces $x_e\:V_{s(e)}\to V_{t(e)}$.

    For any $n\in\Z$, let $[n]$ by the quantum
    integer $(q^n-q^{-n})/(q-q^{-1})$. If $n\ge 0$, let $[n]! \coloneq  [n][n-1]\cdots[1]$
    be the quantum factorial. 
\end{convention}

\section{Quiver Hecke Algebras and the Lusztig's Algebra \texorpdfstring{$\f$}{f}}

We begin by collecting some basic facts to quiver Hecke algebras
and the Lusztig's algebra $\f$.

\subsection{Quiver Hecke Algebras}

In this subsection, we collect the definition and some properties of quiver Hecke algebras.
These results are originally given in \cite{KL1}, \cite{KL2} and \cite{R}.

Recall that $\Gamma$ is a quiver with finite vertex set $I$ and arrow set $E$.
Let $\<I\>$ be the free monoid generated by $I$. 
For $\alpha\in Q^+$ with height $n$,
define $\<I\>^\alpha \subseteq \<I\>$ to be the subset of words
$\boldsymbol{i} = i_1\cdots i_n$ such that $\abs{\boldsymbol{i}} = \alpha$,
where here $\abs{\boldsymbol{i}} \coloneq   \alpha_{i_1}+ \cdots+ \alpha_{i_n}$.
The group $S_n$ acts on $\<I\>^\alpha$ by permuting the letters
in the obvious way. For any $w\in S_n$, we define 

\begin{equation}
    \label{deg-w-i}\deg(w;\boldsymbol{i}) \coloneq  -\sum_{1\le j<k\le n,\ w(j)>w(k)}
    \alpha_{i_j}\cdot\alpha_{i_k}.
\end{equation}

Let $Q_{i,j}(u,v)\in \K[u,v]$
denote $0$ if $i = j$ or $(v-u)^{m_{i,j}}(u-v)^{m_{j,i}}$ if $i\ne j$.

\begin{definition}[{\cite[Section 2.1]{KL1}}]\label{def-of-KLR-alg}
    For any $\alpha\in Q^+$ with height $n$, the \term{quiver Hecke algebra}
    $H(\alpha)$ is the associative $\K$-algebra on generators
    \[
        \{\bi_{\ii}\mid\ii\in\<I\>^\alpha\}\cup\{x_1,\cdots,x_n\}
        \cup\{\tau_1,\cdots,\tau_{n-1}\},
    \]
    subject to the following relations:
    \begin{itemize}
        \item The $\bi_{\ii}$'s are orthogonal idempotents summing to the identity,
        which we will call $\bi_\alpha\in H(\alpha)$;
        \item $\bi_{\ii} x_k = x_k \bi_{\ii}$ and $\bi_{\ii} \tau_k = \tau_k\bi_{s_k(\ii)}$;
        \item The elements $x_1,\cdots,x_n$ commute;
        \item $(\tau_k x_l-x_{s_k(l)}\tau_k)\bi_{\ii}
         = \delta_{i_k,i_{k+1}}(\delta_{k+1,l}-\delta_{k,l})\bi_{\ii}$;
        \item $\tau_k^2\bi_{\ii} = Q_{i_k,i_{k+1}}(x_k,x_{k+1})\bi_{\ii}$;
        \item $\tau_k\tau_l = \tau_l\tau_k$ whenever $\abs{k-l}>1$;
        \item $(\tau_k\tau_{k+1}\tau_k-\tau_{k+1}\tau_k\tau_{k+1})\bi_{\ii}
         = \delta_{i_k,i_{k+2}}\dfrac{Q_{i_k,i_{k+1}}(x_k,x_{k+1})
        -Q_{i_k,i_{k+1}}(x_{k+2},x_{k+1})}{x_k-x_{k+2}}\bi_{\ii}$.
    \end{itemize}
\end{definition}

The algebra $H(\alpha)$ is $\Z$-graded,
with $\bi_{\ii}$ in degree $0$, $x_k\bi_{\ii}$ in degree $2$, and $\tau_k\bi_{\ii}$
in degree $-\alpha_{i_k}\cdot\alpha_{i_{k+1}}$. There exists an algebra
isomorphism $T\colon H(\alpha)\to H(\alpha)^{\mathrm{op}}$ that fixes all generators.

For each $w\in S_n$, we fix once and for all a reduced expression $w = s_{i_1}\cdots s_{i_l}$,
and define $\tau_w\coloneq  \tau_{i_1}\cdots\tau_{i_l}\in H(\alpha)$. Note that $\tau_w\bi_{\ii}$ is of degree
$\deg(w;\ii)$, where $\deg(w;\ii)$ is defined in (\ref{deg-w-i}).

\begin{proposition}[{\cite[Theorem 2.5]{KL1}}]\label{basis-of-H}
    The monomials 
    \[
        \{x_1^{k_1}\cdots x_n^{k_n}\tau_w\bi_{\ii}\mid
        \ii\in\<I\>^\alpha,\ w\in S_n,\ k_1,\cdots,k_n\ge 0\}
    \] gives a basis
    for $H(\alpha)$. In particular for any $\ii,\ii'\in H(\alpha)$,
    \[
        \Dim \bi_{\ii'}H(\alpha)\bi_{\ii} = \dfrac1{(1-q^2)^n}
        \sum_{w\in S_n,\ w(\ii) = \ii'}q^{\deg(w;\ii)}.
    \]
\end{proposition}

\begin{example}
    If $\Gamma$ is the quiver with one vertex and no edges,
    then there exists only one root $\alpha$.
    In this case the quiver Hecke algebra $H(n\alpha)$
    is the so-called ``nil Hecke algebra'', denoted by $\NH(n)$.
    By the results in Appendix A, $P_n \coloneq  q^{n(n-1)/2}\NH(n)e_n$ is the only
    indecomposable projective $H(\alpha)$-module up to isomorphism and degree shift,
    where $e_n \coloneq  x_2x_3^2\cdots x_n^{n-1}\tau_{w_0}$ 
    is the primitive idempotent of $\NH(n)$, $w_0$ is the longest element in $S_n$. Its head $L_n$
    has graded dimension $[n]!$, and $\NH_n\cong[n]!P_n$ as
    graded $\NH(n)$-module. Moreover, 
    \[
        \NH(n)\cong\operatorname{Mat}
        _{[n]!}(\K[x_1,\cdots,x_n]^{S_n}).
    \]   
\end{example}

In particular, if $\alpha = n\alpha_i$ is a multiple of a positive root,
then $H(\alpha)$ is isomorphic to the nil Hecke algebra $\NH(n)$.

Now, we define $\mod(H(\alpha))$, $\fmod(H(\alpha))$ and $\pmod(H(\alpha))$ to be the category
of finitely generated $H(\alpha)$-modules, finite dimensional $H(\alpha)$-modules and finitely generated
projective $H(\alpha)$-modules. We set 
\[
    \fmod(H) \coloneq  \bigoplus_{\alpha\in Q^+}
    \fmod(H(\alpha)),\hspace{3em}\pmod(H) \coloneq   \bigoplus_{\alpha\in Q^+}
    \pmod(H(\alpha)).
\]
Here morphisms in both categories are module homomorphisms
that are homogeneous of degree $0$. The category $\fmod(H)$ is abelian,
and the category $\pmod(H)$ is additive. Define $[\fmod(H)]$
to be the Grothendieck group of the category $\fmod(H)$, and define
$[\pmod(H)]$ to be the split Grothendieck group of the category $\pmod(H)$. They are
naturally $\Z[q,q^{-1}]$-modules, where the action $q$ is given by the
upward degree shifting functor.

For any $V\in\fmod(H)$, the module $V$ decomposes
as $\bigoplus_{\ii\in\<I\>}\bi_{\ii}V$. We define the \term{character} of
$V$ to be 
\[
    \Ch(V) \coloneq   \sum_{\ii\in\<I\>} (\Dim\bi_{\ii}V)\ii\in\Z[q,q^{-1}]\<I\>.
\]

\begin{proposition}[{\cite[Theorem 3.17]{KL1}}]
    The map $\Ch\:[\fmod(H)]\to\Z[q,q^{-1}]\<I\>$ is an injective
    $\Z[q,q^{-1}]$-module homomorphism.
\end{proposition}

There exists dualities $\*$ and $\#$ on $\fmod(H)$ and $\pmod(H)$,
respectively, given by $V^\* \coloneq  \Hom_{\K}(V,\K)$ and $P^\#  \coloneq   \Hom_H(P,H)$,
where the left $H$-action arises from the automorphism $T\: H\to H^{\mathrm{op}}$.
There also exists a nondegenerate bilinear pairing $[\pmod(H)]\times[\fmod(H)]
\to\Z[q,q^{-1}]$ given by 
\[
    ([P],[V]) \coloneq  \Dim\Hom(P^\#,V) = 
    \overline{\Dim\Hom(P,V^\*)},
\]
where the bar indicates the
automorphism on $\Z[q,q^{-1}]$ which sends $q$ to $q^{-1}$.
By \cite[Lemma 3.5]{Br}, any simple $H$-module $L$ can be shifted uniquely
in degree such that it becomes $\*$-self-dual, in which case the projective cover
of $L$ is $\#$-self-dual. Since the category $\pmod(H)$ is Krull--Schmidt,
and objects in the category $\fmod(H)$ are of finite length,
\[
    \begin{aligned}
        \mathbf{B}& \coloneq  \{[P]\mid P\text{ is an }\#\text{-self-dual indecomposable
        projective module}\},\\
        \mathbf{B}^*& \coloneq  \{[L]\mid L\text{ is an }\*\text{-self-dual simple module}\}
    \end{aligned}
\]    
give bases of $[\pmod(H)]$ of $[\fmod(H)]$ as free $\Z[q,q^{-1}]$-modules,
respectively.

For $\beta,\gamma\in Q^+$, there exists a natural non-unital algebra embedding
\[
    H(\beta)\ox H(\gamma)\hookrightarrow H(\beta+\gamma).
\]     
The image of the 
identity $\bi_\beta\ox\bi_\gamma$ will be denoted by $\bi_{\beta,\gamma}$.
Now for any $H(\beta)\ox H(\gamma)$-module $V$ and any $H(\beta+\gamma)$-module 
$U$, we define 
\[
    \Ind_{\beta,\gamma}(V) \coloneq  H(\beta+\gamma)\bi_{\beta,\gamma}
    \ox_{H(\beta)\ox H(\gamma)}V;\hspace{3em}\Res_{\beta,\gamma}(U)
     \coloneq  \bi_{\beta,\gamma}U.
\] 
$\Ind$ and $\Res$ are exact functors; the exactness
of $\Ind$ follows from Proposition \ref{basis-of-H}.
%Therefore, they induce well-defined operators on the Grothendieck groups
%$[\fmod(H)]$ and $[\pmod(H)]$
For any $H(\beta)$-module $X$ and $H(\gamma)$-module $Y$, we define
\[
    X\circ Y \coloneq  \Ind_{\beta,\gamma}(X\ex Y).
\]   
The following Mackey-style
result from \cite[Proposition 2.18]{KL1} will play an important role:

\begin{proposition}\label{mackey}
    Suppose $\beta,\gamma,\beta',\gamma'\in Q^+$ are of height $m,n,m',n'$,
    respectively, such that $\beta+\gamma = \beta'+\gamma'$. Set
    $k = \min(m,n,m',n')$. Let $\{1 = w_0<\cdots<w_k\}$ be the set of minimal
    length representatives in 
    $(S_{m'}\times S_{n'})\backslash S_{m+n}/(S_m\times S_n)$ ordered by
    the Bruhat order. Then for any $H(\beta)\ox H(\gamma)$-module $V$,
    there exists a filtration $$0 = V_{-1}\subseteq V_0\subseteq \cdots
    \subseteq V_k = \Res_{\beta',\gamma'}\Ind_{\beta,\gamma},$$ given by
    $$V_j \coloneq  \sum_{i = 0}^j\sum_{w\in(S_{m'}\times S_{n'})w_i(S_m\times S_n)}
    \bi_{\beta',\gamma'}\tau_w\bi_{\beta,\gamma}\ox V.$$ Moreover there exists
    a natural isomorphism of $H(\beta)\ox H(\gamma)$-modules
    \[
        \begin{aligned}
            V_j/V_{j-1}&\cong\bigoplus_{\beta_1,\beta_2,\gamma_1,\gamma_2}
            q^{-\beta_2\cdot\gamma_1}(\Ind_{\beta_1,\gamma_1}\ox\Ind_{\beta_2,\gamma_2})
            \\&\hspace{3em}\iota^*(\Res_{\beta_1,\beta_2}\ox\Res_{\gamma_1,\gamma_2})(V),\\
            \bi_{\beta',\gamma'}\tau_{w_j}\bi_{\beta,\gamma}\ox v+V_{j-1}&
        \mapsto\sum_{\beta_1,\beta_2,\gamma_1,\gamma_2}
        \bi_{\beta_1,\gamma_1,\beta_2,\gamma_2}\ox
        \bi_{\beta_1,\beta_2,\gamma_1,\gamma_2}v,
        \end{aligned}
    \]
    where $$\iota\:H(\beta_1)\ox H(\gamma_1)\ox H(\beta_2)\ox H(\gamma_2)
    \to H(\beta_1)\ox H(\beta_2)\ox H(\gamma_1)\ox H(\gamma_2)$$ is the
    natural isomorphism, and the sum is taken over all 
    $\beta_1,\beta_2,\gamma_1,\gamma_2\in Q^+$ such that $\beta_1+\beta_2 = \beta$,
    $\gamma_1+\gamma_2 = \gamma$, $\beta_1+\gamma_1 = \beta'$,
    $\beta_2+\gamma_2 = \gamma'$ and $\min(\operatorname{ht}(\beta_2),
    \operatorname{ht}(\gamma_1)) = j$.
\end{proposition}

By definition, there exist natural isomorphisms $$\Ext^i(\Ind(V),U)\cong
\Ext^i(V,\Res(U))$$ for any $H(\beta)\ox H(\gamma)$-module $V$ and 
any $H(\beta+\gamma)$-module $U$, known as the ``Frobenius reciprocity''.

If $V$ is an 
$H(\beta)\ox H(\gamma)$-module, we may construct the coinduction
\[
    \operatorname{CoInd}(V) \coloneq  \Hom_{H(\beta)\ox H(\gamma)}
    (\bi_{\beta,\gamma}H(\alpha+\beta),V).
\] 
From Proposition \ref{basis-of-H},
the coinduction is exact. There exist natural isomorphisms
\[
    \Ext^i(U,\operatorname{CoInd}(V))\cong
    \Ext^i(\Res(U),V)
\] 
for any $H(\beta)\ox H(\gamma)$-module $V$ and 
any $H(\beta+\gamma)$-module $U$. 

By \cite[Theorem 2.2]{LV},
for any $X\in\fmod(H(\beta))$ and $Y\in\fmod(H(\gamma))$,
there exists a natural isomorphism 
\[
    q^{-\beta\cdot\gamma}\Ind_{\beta,\gamma}(X\ex Y)
    \cong\operatorname{CoInd}_{\gamma,\beta}(Y\ex X).
\]
Using this isomorphism, one can show the following result, which is from
\cite[Lemma 2.3]{BKM}:
\begin{proposition}\label{dual-of-induction}
    For any $X\in\fmod(H(\beta))$ and $Y\in\fmod(H(\gamma))$,
    there exists a natural isomorphism $$(U\circ V)^\*
    \cong q^{\beta\cdot\gamma}V^\*\circ U^\*.$$
\end{proposition}

Since the induction is exact, the operator $\circ$
induces a tensor product operation on $\fmod(H)$ and $\pmod(H)$,
making $[\fmod(H)]$ and $[\pmod(H)]$ become $\Z[q,q^{-1}]$-algebras.
Also, we introduce the \term{shuffle product} on $\Z[q,q^{-1}]\<I\>$,
which is given by $$\ii\circ\ii' \coloneq  \sum_{w}q^{\deg(w;\ii\ii')}w(\ii\ii')$$
for $\ii,\ii'\in\<I\>$ with length $m,n$, respectively, and the sum is taken
over all elements $w\in S_{m+n}$ such that $w(1)<\cdots<w(m)$
and $w(m+1)<\cdots<w(m+n)$. The shuffle product makes $\Z[q,q^{-1}]\<I\>$
a $\Z[q,q^{-1}]$-algebra. Now Proposition \ref{mackey} has the following consequence:

\begin{proposition}
    For $X,Y\in\fmod(H)$ we have $$\Ch(X\circ Y) = (\Ch X)\circ(\Ch Y).$$
    Therefore $\Ch\:[\fmod(H)]\to\Z[q,q^{-1}]\<I\>$ is an injective
    $\Z[q,q^{-1}]$-algebra homomorphism.
\end{proposition}

On the other hand, since the restriction is exact, the functor $\Res$ induces $\Z[q,q^{-1}]$-module 
homomorphisms $[\fmod(H)]\to[\fmod(H)]\ox_{\Z[q,q^{-1}]}[\fmod(H)]$
and $[\pmod(H)]\to[\pmod(H)]\ox_{\Z[q,q^{-1}]}[\pmod(H)]$,
making $[\fmod(H)]$ and $[\pmod(H)]$ coalgebras.
Proposition \ref{mackey} also has the following consequence:

\begin{proposition}
    $[\fmod(H)]$ is a twisted bialgebra with respect to the 
    multiplication on $[\fmod(H)]\ox_{\Z[q,q^{-1}]}[\fmod(H)]$
    given by $$(a\ox b)(c\ox d) \coloneq  q^{-\beta\cdot\gamma}(ac\ox bd)$$
    for $b\in[\fmod(H(\beta))],\ c\in[\fmod(H(\gamma))]$.
    Similar result holds for $[\pmod(H)]$.
\end{proposition}

We note that if we define the pairing between 
$[\pmod(H)]\ox_{\Z[q,q^{-1}]}[\pmod(H)]$ and 
$[\fmod(H)]\ox_{\Z[q,q^{-1}]}[\fmod(H)]$ by
$(a\ox b,c\ox d) \coloneq  (a,c)(b,d)$, then by Frobenius reciprocity,
the mulplication on $[\pmod(H)]$ is dual to the comultiplication on $[\pmod(H)]$,
and vice versa.

We finally record the following theorem, which is from \cite[Corollary 2.9]{K}:

\begin{theorem}
    The algebra $H(\alpha)$ has finite global dimension for any $\alpha\in Q^+$.
\end{theorem}

\subsection{The Lusztig's Algebra and the Algebraic Categorification}

In this subsection, we collect the definition and some basic properties of the Lusztig's algebra,
and we will recall the algebraic categorification theorem. The results for 
$\f$ are taken from \cite{L}, and the algebraic categorification theorem
is taken from \cite{KL1}.

\begin{definition}
    The Lusztig's algebra $\f$ is the free associative $\Q(q)$-algebra
    on generators $\{\theta_{i}\mid i\in I\}$ subject to the quantum Serre relations
    \[
        \sum_{r+s = 1-\alpha_i\cdot\alpha_j}(-1)^r\theta_i^{(r)}\theta_j\theta_i^{(s)} = 0,
    \]
    for all $i,j\in I$, $i\ne j$, where $\theta_i^{(r)}$ is the divided power
    $\theta_i^r/[r]!$, and $[r]!$ is the quantum factorial.
\end{definition}

The algebra $\f$ is $Q^+$-graded such that $\theta_i$ is in degree $\alpha_i$.
Define the multiplication on $\f\ox\f$ by $$(a\ox b)(c\ox d)
 = q^{-(\deg b)\cdot(\deg c)}(ac\ox bd),$$ then there exists a unique algebra
homomorphism $$r\:\f\to\f\ox\f,\hspace{3em}\theta_i\mapsto\theta_i\ox 1
+1\ox \theta_i,$$ making $\f$ a twisted bialgebra.

The algebra $\f$ processes a unique nondegenerate symmetric bilinear form satisfying
\[
    (\theta_i,\theta_j) = \dfrac{\delta_{ij}}{1-q^2},\hspace{3em}
    (ab,c) = (a\ox b,r(c)),
\] 
where the bilinear form on $\f\ox \f$ is given by 
$(a\ox b,c\ox d) = (a,c)(b,d)$.

We define $\f_{\Z[q,q^{-1}]}$ to be the $\Z[q,q^{-1}]$-subalgebra of $\f$
generated by all divided powers $\theta_i^{(r)}$, and let 
\[
    \f_{\Z[q,q^{-1}]}^* \coloneq  \{y\in\f\mid (x,y)\in\Z[q,q^{-1}],\ 
    \forall x\in\f_{\Z[q,q^{-1}]}\}.
\]
$\f_{\Z[q,q^{-1}]}^*$ is again a subalgebra
of $\f$. Both $\f_{\Z[q,q^{-1}]}$ and $\f_{\Z[q,q^{-1}]}^*$ are free
$\Z[q,q^{-1}]$-modules, and $$\f\cong\f_{\Z[q,q^{-1}]}\ox_{\Z[q,q^{-1}]}\Q(q)
\cong\f_{\Z[q,q^{-1}]}^*\ox_{\Z[q,q^{-1}]}\Q(q).$$

Let $\ttb\:\f\to\f$ be the antilinear algebra automorphism that fixes all the
$\theta_i$'s, and let $\ttb^*\:\f\to\f$ be the map given by 
\[
    (x,\ttb^*(y))= \overline{(\ttb(x),y)}.
\] 
The map $\ttb^*$ satisfies that 
\[
    \ttb^*(xy)= q^{\deg(x)\cdot\deg(y)}\ttb^*(y)\ttb^*(x).
\] 
The maps $\ttb$ and $\ttb^*$ preserve
$\f_{\Z[q,q^{-1}]}$ and $\f_{\Z[q,q^{-1}]}^*$, respectively.

For any $\ii\in\<I\>$, we define $\theta_{\ii} \coloneq  \theta_{i_1}\cdots\theta_{i_n}$.
The monomials $\theta_{\ii},\ \ii\in\<I\>^\alpha$ spans $\f_\alpha$. There exists
an injective $\Z[q,q^{-1}]$-algebra homomorphism 
\[
    \Ch\: \f_{\Z[q,q^{-1}]}^*
    \to\Z[q,q^{-1}]\<I\>,\hspace{3em} x\mapsto\sum_{\ii\in\<I\>}(\theta_{\ii},x)\ii.
\]

We are now ready to give the statement to the algebraic categorification theorem.

\begin{theorem}[{\cite[Theorem 3.18]{KL1}}]\label{categorification}
    There exists an isomorphism of twisted bialgebras
    \begin{equation}\label{categorification-of-f}
        \gamma\:\f_{\Z[q,q^{-1}]}\to[\pmod(H)],
    \end{equation} such that
    $\theta_i\mapsto[H(\alpha_i)]$ for all $i\in I$,
    and it intertwines the involution $\ttb$ with $\#$.
    The dual of $\gamma$ induces an isomorphism 
    \begin{equation}\label{categorification-of-f*}
        \gamma^*\:[\fmod(H)]\to\f_{\Z[q,q^{-1}]}^*,
    \end{equation}
    and it intertwines the involution $\*$ with $\ttb^*$.
\end{theorem}

\begin{proof}[Sketch of proof]
    For any $\ii\in\<I\>^\alpha$, we have 
    \[
        [H(\alpha_{i_1})\circ\cdots\circ
        H(\alpha_{i_n})] = [H(\alpha)\bi_{\ii}].
    \] 
    Thus for any $i\in I$ and $r\ge 0$,
    we have $[H(\alpha_i)]^r = [H(r\alpha_i)]$, hence $[H(\alpha_i)]^{(r)} = [P(r\alpha_i)]$.
    By \cite[Lemma 3.10]{Br}, in $[\pmod(H)]$ one has $$\sum_{r+s = 1-\alpha_i
    \cdot\alpha_j}(-1)^r[P(r\alpha_i)\circ P(\alpha_j)\circ P(s\alpha_j)] = 0.$$
    Thus there exists a well-defined algebra homomorphism 
    \[
        \gamma\:\f_{\Z[q,q^{-1}]}\to[\pmod(H)],
    \]    
    such that 
    $\theta_i\mapsto[H(\alpha_i)]$, and $\theta_i^{(r)}\mapsto[P(r\alpha_i)]$.
    It is straightforward that $\gamma$ intertwines the comultiplications,
    thus $\gamma$ is a homomorphism of twisted bialgebras. Taking the dual map
    with respect to the nondegenerate pairings gives a homomorphism 
    \[
        \gamma^*\:[\fmod(H)]\to\f_{\Z[q,q^{-1}]}^*
    \] 
    of twisted bialgebras.

    Since $\ttb$ fixes $\theta_{\ii}$ and $\#$ fixes $H\bi_{\ii}$ for all $\ii$,
    and $\{\theta_{\ii}\}$ is a basis for $\f$, the map $\gamma$ intertwines 
    the involution $\ttb$ with $\#$. Since $\ttb^*$ is dual to $\ttb$ and
    $\*$ is dual to $\#$, the map $\gamma^*$ intertwines the involution $\*$ with $\ttb^*$.

    For any $V\in\fmod(H)$, we have 
    \[
        (\theta_{\ii},\gamma^*[V])
         = (\gamma(\theta_{\ii}),[V]) = ([H\bi_{\ii}],[V]) = \Dim\Hom(H\bi_{\ii},V)
         = \Dim\bi_{\ii}V,
    \] 
    by definition of $\Ch$, the following diagram commutes:
    \[
      \begin{tikzcd}
        {[\fmod(H)]}\ar[rr,"\Ch"]\ar[rd, "\gamma^*"']
        &&\Z[q,q^{-1}]\<I\>.\\
        &\f_{\Z[q,q^{-1}]}^*\ar[ru, "\Ch"']&
      \end{tikzcd}
    \]
    Since the map $\Ch\:[\fmod(H)]\to\Z[q,q^{-1}]\<I\>$ is injective,
    the map $\gamma^*$ must be injective, hence $\gamma$ is surjective.

    We define a pairing on $\Q(q)\ox_{\Z[q,q^{-1}]}[\pmod(H)]$ by 
    \[
        ([P],[Q]) \coloneq  \Dim\Hom(P^\#,Q).
    \] 
    Since $\{[H\bi_{\ii}]\}$ spans 
    $\Q(q)\ox_{\Z[q,q^{-1}]}[\pmod(H)]$ and 
    $([H\bi_{\ii'}],[H\bi_{\ii}]) = \Dim \bi_{\ii'}H\bi_{\ii}$ which lies in $\Q(q)$, the pairing
    is well-defined over $\Q(q)$. From the definition of the pairing on $\f$,
    we have 
    \[
        (\theta_{\ii'},\theta_{\ii}) = \dfrac{1}{(1-q^2)^n}
        \sum_{w\in S_n,\ w(\ii) = \ii'}q^{\deg(w;\ii)}
    \] 
    for all $\ii,\ii'\in\<I\>^\alpha$
    with height of $\alpha$ being $n$. which is equal to
    $([H\bi_{\ii'}],[H\bi_{\ii}])$ by Proposition \ref{basis-of-H}.
    Therefore the map $\hat\gamma\:\f\to\Q(q)\ox_{\Z[q,q^{-1}]}[\pmod(H)]$
    preserves the pairing. Now if $\hat\gamma(x) = 0$ for some $x\in\f$, then
    for all $y\in\f$ we have $(x,y) = (\hat\gamma(x),\hat\gamma(y)) = 0$,
    thus $x = 0$ since the bilinear form on $\f$ is nondegenerate.
    Therefore $\hat\gamma$ is injective, which shows that
    $\gamma$ is injective, hence $\gamma^*$ is surjective.
\end{proof}

\begin{corollary}[{\cite[Corollary 3.19]{KL1}}]
    Every simple $H$-module is absolutely irreducible, i.e. its 
    endemorphism ring is $\K$.
\end{corollary}

\begin{proof}
    Suppose $L$ is a simple $H(\alpha)$-module, $P$ is its projective cover.
    We may assume that $L$ is $\*$-self-dual, in which case $P$ is $\#$-self-dual.
    Since the map $\Hom(P,L)\to\Hom(L,L)$ is surjective, it suffices to show
    that $\dim\Hom(P,L) = 1$. By Theorem \ref{categorification},
    the dual of $[P]$ with respect to the basis $\mathbf{B}$ lies in
    $[\fmod(H)]$. We call the dual $[P]^*$. Since for any $[P']\in\mathbf{B}$,
    $[P']\ne[P]$, we have $([P'],[L]) = 0$, $[L]$ must be a multiple of $[P]^*$.
    But $L$ is simple, thus $[L] = q^m[P]^*$ for some $m$. By the $\#$-self-duality
    of $P$, we have $\dim\Hom(P,L) = 1$, completing the proof.
\end{proof}

\section{The Geometric Realization Theorem and the Geometric Approach to Standard Modules}

In this section, we will recall the geometric interpretation to 
quiver Hecke algebras. We will also survey on the relation
between the algebraic categorification theorem and the 
original geometric categorification theorem introduced by Lusztig,
and the motivation and definition of geometric standard modules. 

The base field $\K$ is assumed to be of characteristic zero in this section.

\subsection{The Geometric Realization Theorem}

Let $\alpha\in Q^+$ be an element in the root lattice, with height $n$.
Then $\alpha$ can be canonically viewed as an element in $\mathbb{N}I$.
We suppose that $\alpha = \sum_{i\in I}c_i\alpha_i$.
We define $$X(\alpha) \coloneq  \prod_{e\in E}\Hom(\K^{\oplus c_{s(e)}},\K^{\oplus c_{t(e)}}).$$
It is an affine space. Equivalently, $X(\alpha)$ is the space
of all $\K\Gamma$-module structures on the $I$-graded vector space
$V(\alpha) \coloneq  \bigoplus_{i\in I}\K^{\oplus c_i}$.
For any element $x\in X(\alpha)$, we write $x_e$ for the corresponding component of $x$
in $\Hom(\K^{\oplus c_{s(e)}},\K^{\oplus c_{t(e)}})$. The algebraic group
$G(\alpha)$ acts on $X(\alpha)$ by $((g_i),(x_e))\mapsto
(g_{t(e)}x_eg_{s(e)}^{-1})$, where $G(\alpha)$ is the 
algebraic group $\prod_{i}\mathrm{GL}(c_i)$.

For any $\ii\in\<I\>^\alpha$, a flag of type $\ii$ in $V(\alpha)$ 
is a sequence 
\[
    \phi\:0 = V_0\subseteq V_1\subseteq\cdots
    \subseteq V_n = V(\alpha)
\] 
of $I$-graded subspaces such that
$V_l/V_{l-1}$ has dimension vector $i_l$ for all $l$.
Let $F_{\ii}$ be the variety of all flags of type $\ii$ in
$V(\alpha)$. The group $G(\alpha)$ naturally acts on $F_{\ii}$.

For any $\ii\in\<I\>^\alpha$
we define 
\[
    Y_{\ii} \coloneq  \{(x\in X(\alpha),\phi = (V_l)_{0\le l\le n}\in F_{\ii})\mid
    x(V_l)\subseteq V_{l-1}\text{ for all }l\}.
\] 
The group $G(\alpha)$ 
acts naturally on $Y_{\ii}$. We set 
\[
    d_{\ii} \coloneq  \dim Y_{\ii}.
\]
The projection $\pi_{\ii}\:Y_{\ii}
\to X(\alpha)$ is $G(\alpha)$-equivariant and proper.

We furthermore take 
\[
    Y(\alpha) \coloneq  \coprod_{\ii\in\<I\>^\alpha}Y_{\ii},
    \hspace{3em}F(\alpha) \coloneq  \coprod_{\ii\in\<I\>^\alpha}F_{\ii}.
\]    

We now take the complex 
\[
    \LL_{\ii} \coloneq  (\pi_{\ii})_*\underline{\K}[d_{\ii}]
\]    
on $X(\alpha)$, where $(\pi_{\ii})_*$ is the derived pushforward 
functor. This complex lies in $\D_{G(\alpha)}^b(X(\alpha))$.
For any $\ii,\ii'\in\<I\>^\alpha$, we define $$R_{\ii,\ii'}
 \coloneq  \Ext^*_{G(\alpha)}(\LL_{\ii},\LL_{\ii'}).$$ Now for any 
$\ii,\ii',\ii''\in\<I\>^\alpha$, the Yoneda
product $R_{\ii,\ii'}\times R_{\ii',\ii''}\to R_{\ii,\ii''}$,
is homogeneous of degree $0$. Thus the space
\[
    R(\alpha) \coloneq  \bigoplus_{\ii,\ii'\in\<I\>^\alpha}R_{\ii,\ii'}
\]    
is an associated graded $\K$-algebra. Also note that 
the identity morphism $\LL_{\ii}\to\LL_{\ii}$ yields 
an element of $R_{\ii,\ii}$. We denote this element by $\bi_{\ii}'$.
It is clear that $\{\bi_{\ii}'\}$ gives a complete set of
orthogonal idempotents in $R(\alpha)$, and $R_{\ii,\ii'}
 = \bi_{\ii}'R(\alpha)\bi_{\ii'}'$.

The following theorem shows the relation between the 
algebras $R(\alpha)$'s and the quiver Hecke algebras:

\begin{theorem}[Geometric Realization Theorem, {\cite[Theorem 3.6]{VV}}]\label{geo-real}
    There exists an isomorphism of graded $\K$-algebras
    $H(\alpha)\to R(\alpha)$. Moreover, under this isomorphism,
    the idempotents $\bi_{\ii}$ are mapped to $\bi_{\ii}'$.
\end{theorem}

Because of this theorem, from now on we will simply identify
the algebras $H(\alpha)$ and $R(\alpha)$, and identify the 
idempotents $\bi_{\ii}$ and $\bi_{\ii}'$.

\subsection{The Geometric Approach to Standard Modules}

In this subsection, we assume that the quiver $\Gamma$ is of finite type.
In this case the set $R^+$ of positive roots is finite. 
We fix a total order on the set of positive roots, and index them by $\lambda_1,\cdots,\lambda_N$.

\begin{definition}
    For any $\alpha\in Q^+$, a \term{Kostant partition} of $\alpha$ is a $N$-tuple
    $m = (m_1,\cdots,m_N)$ such that $\sum m_i\lambda_i = \alpha$.
    The set of Kostant partitions of $\alpha$ is denoted by 
    $\KP(\alpha)$.
\end{definition}

We recall the Gabriel's Theorem \cite{G}:

\begin{theorem}\label{gabriel}
    If the quiver $\Gamma$ is of finite type, then for any 
    positive root $\lambda$, there exists a unique indecomposable
    $\K\Gamma$-module $M(\lambda)$ with dimension vector $\lambda$.
    Moreover all indecomposable $\K\Gamma$-modules arise in this way.
\end{theorem}

Now, for any $\alpha\in Q^+$, the $G(\alpha)$-orbits in $X(\alpha)$ are
the isomorphism classes of $\K\Gamma$-modules with dimension vector
$\alpha$. Since $\K\Gamma$ is finite dimensional, by the
Krull--Schmidt theorem, every $\K\Gamma$-module is a direct
sum of $M(\lambda)$'s, and the multiplicity
of each indecomposable factor $M(\lambda)$ is uniquely determined. 
Thus there are only finitely many $G(\alpha)$-orbits on $X(\alpha)$, each corresponds to
a Kostant partition of $\alpha$. 

\textbf{Notation.} For a Kostant partition $m\in\KP(\alpha)$,
we define $\OO_m$ to be the corresponding $G(\alpha)$-orbit
in $X(\alpha)$, and we define $M(m)$ to be the corresponding
$\K\Gamma$-module.

We now let $\LL(\alpha) = \bigoplus_{\ii\in\<I\>^\alpha}
\LL_{\ii}$. Since each $Y_{\ii}$ is smooth, and each
$\pi_{\ii}$ is proper, by the decomposition theorem, 
each $\LL_{\ii}$ is a semisimple complex, so is 
$\LL(\alpha)$. Since $\LL(\alpha)$ is $G(\alpha)$-equivariant,
the complex $\LL(\alpha)$ decomposes as $$\LL(\alpha) = \bigoplus_{m\in\KP(\alpha)}
\bigoplus_{\LL}V(\OO_m,\LL)\ox\IC(\OO_m,\LL),$$ where $\LL$
runs over all $G(\alpha)$-equivariant local systems on $\OO_m$, and $\IC(\OO_m,\LL)$
is its corresponding intersection complex, and $V(\LL)$
is a complex of vector spaces with differential $0$.
However, each orbit $\OO_m$ has stablizer $\Aut(M(m))$,
which is connected, thus each $\LL$ must be the constant
sheaf $\underline{\K}$, and $\LL(\alpha)$ decomposes as
\[
    \LL(\alpha) = \bigoplus_{m\in\KP(\alpha)}
    V(\OO_m)\ox\IC(\OO_m,\underline{\K}).
\]    

On the other hand, by \cite[Theorem 2.2]{Re},
for any $m\in\KP(\alpha)$ there exists some $\ii\in\<I\>^\alpha$
such that the image of the map $\pi_{\ii}$ is
the closure of $\OO_m$, thus $\LL_{\ii}$ has support
the closure of $\OO_m$, which shows that each $V(\OO_m)$ must be nonzero.

We now let $\mathbf{Q}(\alpha)$ be the additive subcategory
of $\D_{G(\alpha)}^b(X(\alpha))$ generated by the perverse sheaves
$\IC(\OO_m)$'s and their degree shifts. Its split Grothendieck group
$[\mathbf{Q}(\alpha)]$ is a $\Z[q,q^{-1}]$-module,
where $q$ acts on $\mathbf{Q}(\alpha)$ by upward degree shifting functor.
We set $\mathbf{Q} = \bigoplus_\alpha\mathbf{Q}(\alpha)$.

We now briefly recall the convolution product on $\mathbf{Q}$,
whose construction is originally given in \cite{L}.

Take $\alpha,\beta\in Q^+$. We choose an injective $I$-graded map
$V(\alpha)\to V(\alpha+\beta)$, and identify $V(\alpha)$ as a subspace.
We shall also identify $V(\beta)$ as the corresponding quotient,
so that there exists a surjection $V(\alpha+\beta)\to V(\beta)$
whose kernel is $V(\alpha)$. Let $G(\alpha,\beta)$
be the stablizer of $V(\alpha)$ in $G(\alpha+\beta)$, and $U$ be
the unipotent radical of $G(\alpha,\beta)$. We have $G(\alpha,\beta)/U\cong G(\alpha)\times G(\beta)$.

Let $X(\alpha,\beta)$ be the closed subvariety of $X(\alpha+\beta)$ consisting of 
those $x$ such that $x$ fixes $V(\alpha)$. This subvariety
surjects to $X(\alpha)\times X(\beta)$, and the surjection
is a vector bundle. The surjection is $G(\alpha,\beta)$-equivariant, 
where $G(\alpha,\beta)$ acts on $X(\alpha)\times X(\beta)$
by the quotient $G(\alpha,\beta)/U\cong G(\alpha)\times G(\beta)$.

Let 
\[
    X''(\alpha,\beta) \coloneq  G(\alpha+\beta)\times_{G(\alpha,\beta)}X(\alpha,\beta),
    \hspace{3em}X'(\alpha,\beta) \coloneq  G(\alpha+\beta)\times_{U}X(\alpha,\beta).
\]    
Note that $X''(\alpha,\beta)$ is the variety of pairs $(x,V)$
where $x\in X(\alpha+\beta)$ and $V\subseteq V(\alpha+\beta)$
has dimension vector $\alpha$ such that $x(V)\subseteq V$.

We have a diagram 
\[
    X(\alpha)\times X(\beta)\xleftarrow{p}
    X(\alpha+\beta)\xrightarrow{q}X(\alpha+\beta).
\]
We now take 
\[
    \begin{aligned}
        p_0\:X'(\alpha,\beta)\to X(\alpha)\times X(\beta),&\ (g,x)\mapsto p(x);\\
        \pi\:X'(\alpha,\beta)\to X''(\alpha,\beta),&\ (g,x)\mapsto (g,x);\\
        q_0\:X'(\alpha,\beta)\to X(\alpha+\beta),&\ (g,x)\mapsto gq(x).
    \end{aligned}
\]
The map $p_0$ is smooth, $q_0$ is proper, and $\pi$ is a principal
$G(\alpha)\times G(\beta)$-bundle. Hence for any $\LL_1\in\mathbf{Q}(\alpha)$, we have
$\LL_2\in\mathbf{Q}(\beta)$, we may define their convolution product
\[
    \LL_1\circ\LL_2 \coloneq  (q_0)_*\pi_\flat(p_0)^\dagger(\LL_1\ex\LL_2),
\]
where $(p_0)^\dagger = p_0^*[\dim p_0]$ is the smooth pullback,
$\dim p_0$ is the dimension of the fiber of $p_0$,
and $\pi_\flat$ is the inverse of the equivalence of categories
\[
    \pi^\dagger\:\D^b(X''(\alpha,\beta))\to\D^b_{G(\alpha)\times G(\beta)}
    (X'(\alpha,\beta)).
\]
By \cite[Lemma 9.2.3]{L}, the complex $\LL_1\circ\LL_2$
lies in $\mathbf{Q}(\alpha+\beta)$ and does not depend on the choice of
the maps $V(\alpha)\to V(\alpha+\beta)$ and $V(\alpha+\beta)\to V(\beta)$, 
hence this is a well-defined
convolution product on the category $\mathbf{Q}$, which makes 
$[\mathbf{Q}]$ a $\Z[q,q^{-1}]$-algebra.

Note that the above construction naturally extends to the case where
there are 3 or more components to be multiplicated. In addition,
since all of $(q_0)_*$, $\pi_\flat$ and $(p_0)^\dagger$ commute
with the Verdier duality, so is the convolution product.

By the above construction, we have that for any $\ii,\ii'\in\<I\>$,
$\LL_{\ii}\circ\LL_{\ii'}\cong\LL_{\ii\ii'}$.

The following geometric categorification theorem reveals the relation
 between the category $\mathbf{Q}$
and the algebra $\f$:

\begin{theorem}[{\cite[Theorem 13.2.11]{L}}]\label{f-and-q}
    There exists a $\Z[q,q^{-1}]$-algebra isomorphism
    \[
        \lambda\:\f_{\Z[q,q^{-1}]}\to[\mathbf{Q}],\qquad
        \theta_{\ii}\mapsto[\LL_{\ii}].
    \]     
    The isomorphism
    intertwines the duality $\mathtt{b}$ and the Verdier duality.
\end{theorem}

In fact, this is the first categorification to the Lusztig's algebra $\f$.
By definition, the \term{canonical basis}
of $\f_{\Z[q,q^{-1}]}$ is the inverse image of simple perverse sheaves
in $\mathbf{Q}$. Since $\lambda$ intertwines the duality $\mathtt{b}$ and the Verdier duality,
the canonical basis is invariant under $\mathtt{b}$.

We now link the geometric categorification theorem above,
and the algebraic categorification theorem stated at Theorem \ref{categorification}.

For any $\alpha\in Q^+$ and $\LL$ a complex on $X(\alpha)$,
we define 
\[
    \mu(\LL) \coloneq  \Ext^*_{G(\alpha)}(\LL(\alpha),\LL).
\]
Since $\LL(\alpha)$ is semisimple, for any indecomposable
direct summand $\LL$ of $\LL(\alpha)$, $\mu(\LL)$ is an
indecomposable projective $\Ext^*_{G(\alpha)}(\LL(\alpha),\LL(\alpha))$-module.
By the definition of $\mathbf{Q}(\alpha)$, for any
$\LL\in\mathbf{Q}(\alpha)$, the module $\mu(\LL)$ is a
finitely generated $H(\alpha)$-module. Thus $\mu$
induces an additive functor $\mathbf{Q}(\alpha)
\to\pmod(H(\alpha))$. Summing over all $\alpha$,
we obtain a functor $$\mu\:\mathbf{Q}\to\pmod(H).$$
The functor $\mu$ is additive.

We are now ready to state the following theorem:

\begin{theorem}[{\cite[Theorem 4.4]{VV}}]\label{canonical-basis}
    \begin{enumerate}[1)]
        \item The composition $$\f_{\Z[q,q^{-1}]}
        \xrightarrow{\lambda}[\mathbf{Q}]\xrightarrow{[\mu]}
        [\pmod(H)]$$ is the map $\gamma\:\f_{\Z[q,q^{-1}]}\to[\pmod(H)]$ in (\ref{categorification-of-f}).

        \item The functor $\mu\:\mathbf{Q}\to\pmod(H)$ is an equivalence of additive categories.
        
        \item In the isomorphism (\ref{categorification-of-f}),
        the canonical basis of $\f_{\Z[q,q^{-1}]}$ is mapped
        to $\#$-self-dual indecomposable projective modules;
        similarly in the isomorphism (\ref{categorification-of-f*}),
        the dual canonical basis of $\f_{\Z[q,q^{-1}]}$ is mapped
        to $\*$-self-dual simple modules.
    \end{enumerate}
\end{theorem}

\begin{proof}
    1): To prove this, we only need to show that $\gamma$ and $[\mu]\circ\lambda$ produce
    the same element on each $\theta_{\ii}$. Suppose that
    $\ii\in\<I\>^\alpha$ for some $\alpha\in Q^+$. By 
    Theorem \ref{f-and-q}, $\lambda(\theta_{\ii})
     = [\LL_{\ii}]$. On the other hand, $\mu(\LL_{\ii})
     = \Ext^*_{G(\alpha)}(\LL(\alpha),\LL_{\ii}) = H(\alpha)\bi_{\ii}$,
    thus $[\mu](\lambda(\theta_{\ii})) = [H(\alpha)\bi_{\ii}] = 
    \gamma(\theta_{\ii})$, therefore $\gamma = [\mu]\circ\lambda$.

    2): The maps $\gamma$ and $\lambda$ are isomorphisms,
    thus the map $[\mu]$ is an isomorphism, which shows that 
    the functor $\mu\:\mathbf{Q}\to\pmod(H)$ is an equivalence of additive categories.

    3): The second statement follows immediately from the first one,
    so we will only prove the first one.    

    The map $\lambda$ maps the canonical basis
    to perverse simple sheaves. By the definition of $\mu$,
    it maps simple sheaves to indecomposable projective objects.
    Thus $\gamma$ maps the canonical basis to indecomposable
    projective objects. On the other hand, $\gamma$ maps
    $\mathtt{b}$-invariant elements to $\#$-invariant elements.
    since the canonical basis is $\mathtt{b}$-invariant, $\gamma$
    maps canonical basis to $\#$-self-dual indecomposable
    projective objects.
\end{proof}

\begin{remark}
    Since $\gamma$ and $\lambda$ are $\Z[q,q^{-1}]$-algebra
    isomorphisms, so is $[\mu]$. Therefore for any $\LL\in
    \mathbf{Q}(\alpha)$ and $\LL'\in\mathbf{Q}(\alpha')$,
    we have $$[\Ext_{G(\alpha+\alpha')}^*(\LL(\alpha+\alpha'),\LL\circ\LL')]
     = [\Ext_{G(\alpha)}^*(\LL(\alpha),\LL)\circ\Ext_{G(\alpha')}^*(\LL(\alpha'),\LL')].$$
    Since any object in $\pmod(H)$ is a finite direct sum
    of indecomposable projective modules, and 
    indecomposable projective modules yield a basis
    of $[\pmod(H)]$ over $\Z[q,q^{-1}]$, we must have
    \[
        \Ext_{G(\alpha+\alpha')}^*(\LL(\alpha+\alpha'),\LL\circ\LL')
        \cong\Ext_{G(\alpha)}^*(\LL(\alpha),\LL)\circ\Ext_{G(\alpha')}^*(\LL(\alpha'),\LL').
    \]
    Therefore $\mu$ is a monoidal functor.
\end{remark}

\ 

We now recall the motivation and definition to geometric standard modules.

Take an element $\alpha\in Q^+$.
Recall from the previous subsection that $\LL(\alpha)$
has a decomposition $$\LL(\alpha)\cong\bigoplus_{m\in\KP(\alpha)}
V(\OO_m)\ox\IC(\OO_m),$$ and each $V(\OO_m)$ is a nonzero complex.
By Theorem \ref{canonical-basis},
$\mu(\IC(\OO_m)) = \Ext^*_{G(\alpha)}(\LL(\alpha),
\IC(\OO_m))$ is a $\#$-self-dual indecomposable projective module.
We define it by $\widetilde{P}(m)$, and define
$\widetilde{L}(m)$ to be the head of $\widetilde{P}(m)$.

By the theory of constructible sheaves, for a good stratification
$\mathcal{I}$ of a space $X$, and an open stratum $X_0$,
there exists a diagram of triangulated categories
\[
    \D^b_{\mathcal{I}}(X-X_0)\xrightarrow{i_*}\D^b_{\mathcal{I}}(X)\xrightarrow{j^*}\D^b_{\mathcal{I}}(X_0),
\]
such that $i_*$ and $j^*$ are exact and have natural left and right adjoints,
and the perverse $t$-structure in the middle is formed by gluing
the perverse $t$-structures on the left and on the right.
Here $i\:X-X_0\to X$ and $j\:X_0\to X$ are the inclusions.

Hence for any stratum $X_m$ in $\mathcal{I}$ with the inclusion map $j_m\:X_m\to X$, the functors
$(j_m)_*$ and $(j_m)_!$ are naturally defined, and for any 
local system $\LL$ on $X_m$, the complex $\IC(X_m,\LL)$
is the minimal extension of $\LL[\dim X_m]$. The complex $\IC(X_m,\LL)$
has all its cohomologies supported in the strata that are
contained in the closure of $X_m$.

%In practice, the $*$-pushforwards and $!$-pushforwards are much more 
%easy to characterize than the intersection complexes.
For any $X_m$ and local system $\LL$ on $X_m$, we shall define
\[
    \costd(X_m,\LL) \coloneq  (j_m)_*\underline{\LL}[\dim X_m].
\]
The complex $\IC(X_m,\LL)$ naturally maps to $\costd(X_m,\LL)$.
In addition, the ``decomposition matrix'' between the
intersection complexes and the $*$-pushforwards
is upper-triangular with respect to the partial order of orbit closures.

Following the above general observations, we return to our case.
Again, take an element $\alpha\in Q^+$. For a Kostant partition $m$ of $\alpha$,
we introduce the complex $$\costd(\OO_m) \coloneq  
(j_m)_*\underline{\K}[d_m]$$ on $X(\alpha)$, where
$j_m\:\OO_m\to X(\alpha)$ is the inclusion, and $d_m \coloneq  \dim \OO_m$.
We make the following definition:

\begin{definition}
    The $H(\alpha)$-module $\mu(\costd(\OO_m))$ is called the
    \term{geometric standard module}, denoted by $\widetilde{\std}(m)$.
\end{definition}

The geometric standard module are firstly introduced and studied by Kato 
in \cite{K2}, and also studied by McNamara in \cite{Mc2}.
They are called ``standard modules'' since they have similar
properties to standard modules in highest weight categories.
In fact, we will show in Chapter 5 that $\mod(H(\alpha))$
is a so-called ``polynomial highest weight category''.

We shall record the following proposition, which reflects the
upper-triangular property between the intersection complexes and the $*$-pushforwards:

\begin{proposition}
    Let $\alpha\in Q^+$, and $m,n\in\KP(\alpha)$.
    If $\Hom(\widetilde{P}(n),\widetilde{\std}(m))\ne 0$,
    then $\OO_m\subseteq\overline{\OO_n}$.
\end{proposition}

\begin{proof}
    By assumption, there exists a nonzero map
    $q^k\widetilde{P}(n)\to\widetilde{\std}(m)$. Since the map is nonzero,
    the image of the projection from $\LL(\alpha)$ to the summand $\IC(\OO_n)$
    in $\widetilde{\std}(m)$ must be nonzero. Hence we must have 
    $\Ext^*_{G(\alpha)}(\IC(\OO_n),\costd(\OO_m))\ne 0$.

    On the other hand, if $\OO_m\subseteq\overline{\OO_n}$ does not hold,
    then there exists a $G(\alpha)$-invariant open subset $U$ containing
    $\OO_m$ and not intersecting $\overline{\OO_n}$. Let $j\:U\to
    X(\alpha)$, $j_1\:\OO_m\to U$  be the inclusions. Then $j^*(\IC(\OO_n)) = 0$,
    thus by the $(j^*,j_*)$-adjunction,
    \[
        \Ext^*_{G(\alpha)}(\IC(\OO_n),\costd(\OO_m)) = \Ext^*_{G(\alpha)}(j^*(\IC(\OO_n)),
        (j_1)_*\underline{\K}[d_m]) = 0,
    \] which is a contradiction.
    Thus we must have $\OO_m\subseteq\overline{\OO_n}$.
\end{proof}

This implies the following proposition:

\begin{proposition}\cite[Proposition 4.5]{Mc2}\label{comp-of-geo-std}
    Let $\alpha\in Q^+$, and $m,n\in\KP(\alpha)$.
    If $q^k\widetilde{L}(n)$ is a subquotient of 
    $\widetilde{\std}(m)$ for some $k$, then $\OO_m\subseteq\overline{\OO_n}$.
\end{proposition}

\begin{proof}
    If $q^k\widetilde{L}(n)$ is a subquotient of 
    $\widetilde{\std}(m)$ for some $k$, then there exists a nonzero map
    $q^k\widetilde{P}(n)\to\widetilde{\std}(m)$. By the previous proposition,
    $\OO_m\subseteq\overline{\OO_n}$.
\end{proof}

\section{Algebraic Approach to Standard Modules}

In this section, we recall the algebraic approach to standard modules,
which are originally given by Brundan, Kleshchev and McNamara.
We assume that $\Gamma$ is of finite type throughout this section.

\subsection{Convex Ordering and the PBW Basis}

A \term{convex ordering} on $R^+$ is a total order $\le$ such that
whenever $\beta,\gamma,\beta+\gamma\in R^+$ that satisfies
$\beta<\gamma$, we have $\beta<\beta+\gamma<\gamma$.

According to \cite{P}, there exists a 1-1 correspondence between convex orderings
of $R^+$ and reduced expressions for the longest element $w_0$ of $W$;
given a reduced expression $w_0 = s_{i_1}\cdots s_{i_N}$ where $N = \abs{W}$,
its corresponding convex ordering on $R^+$ is given by
\[
    \alpha_{i_1}<s_{i_1}\alpha_{i_2}<\cdots<s_{i_1}\cdots s_{i_{N-1}}\alpha_{i_N}.
\]    
We also record the following lemma, which is from \cite[Lemma 2.4]{BKM}:

\begin{lemma}\label{sum-in-convex-ordering}
    Suppose $\alpha,\beta_1,\cdots,\beta_k,\gamma_1,\cdots,\gamma_l\in R^+$
    such that $\beta_i<\alpha<\gamma_j$ for all $i,j$. If $\sum\beta_i = \sum\gamma_j$
    then $k = l$ and $\beta_i = \alpha = \gamma_j$ for all $i,j$.
\end{lemma}

From now on until the end of the section, we fix a reduced expression of $w_0$,
which corresponds to a convex ordering on $R^+$. We order the elements in $R^+$
by $\lambda_1<\cdots<\lambda_N$, where 
$\lambda_k = s_{i_1}\cdots s_{i_{k-1}}\alpha_{i_k}$. 

Corresponding to this convex ordering
there exists \term{root vectors} $\{r_\lambda\mid\lambda\in R^+\}$ in $\f$,
given by Lusztig in \cite{L} as follows: fix the positive embedding 
$\f\hookrightarrow U_q(\mathfrak{g})$, $x\mapsto x^+$, and let $E_i = \theta_i^+$,
and consider the braid group generators $T_i \coloneq  T_{i,+}''$ from 
\cite[37.1.3]{L}, then for any $\lambda_k$ the root vector $r_{\lambda_k}$
is the element in $\f$ such that $r_{\lambda_k}^+ = T_{i_1}\cdots T_{i_{k-1}}
E_{i_k}$. We also introduce the \term{dual root vectors} $r^*_\lambda$ By
$r^*_\lambda = (1-q^2)r_\lambda$. By \cite[37.2.4]{L}, the dual root vectors
are invariant under $\ttb^*$ and lie in the dual canonical basis of 
$\f^*_{\Z[q,q^{-1}]}$. 

Recall that for any $\alpha\in Q^+$, a Kostant partition of $\alpha$ is a $N$-tuple
$m = (m_1,\cdots,m_N)$ such that $\sum m_i\lambda_i = \alpha$. A Kostant partition $m$
of $\alpha$ is naturally indentified with a nondecreasing sequence
$\lambda_{i_1},\cdots,\lambda_{i_l}$ of positive roots summing to $\alpha$;
we will write it as $m = (\lambda_{i_1},\cdots,\lambda_{i_l})$.

For any $m,n\in\KP(\alpha)$, we say $n>'m$ if $n_l>m_l$ where 
$l$ is the smallest index such that $n_l\ne m_l$; and we say $n>''m$ if $n_l>m_l$ where 
$l$ is the largest index such that $n_l\ne m_l$. We say $n>m$ if
$n>'m$ and $n>''m$. This gives a partial order on $\KP(\alpha)$.
For any $m\in\KP(\alpha)$, we define the \term{support} of $m$ to be the 
set of positive roots $\lambda_i$ such that $m_i\ne 0$. 

The following lemmas are useful:

\begin{lemma}[{\cite[Lemma 2.5]{BKM}}]\label{minimal-element-in-KP}
    For any $\lambda\in R^+$, $k>1$, the Kostant partition 
    $(\lambda,\cdots,\lambda)$ where $\lambda$ appears $k$ times, is the unique
    smallest element in $\KP(k\lambda)$. This Kostant partition
    will be denoted by $(\lambda^k)$.
\end{lemma}

\begin{lemma}[{\cite[Lemma 2.6]{BKM}}]
    For $\lambda\in R^+$, if $m\in\KP(\lambda)$ is minimal such that
    $m>(\lambda)$, then $m = (\beta,\gamma)$ for positive roots
    $\beta<\lambda<\gamma$. Such Kostant partition is called
    a \term{minimal pair} of $\lambda$.
\end{lemma}

Let $\KP \coloneq  \bigcup \KP(\alpha)$. For any $m\in\KP$, we define
\[
    r_m \coloneq  r_{\lambda_1}^{m_1}\cdots r_{\lambda_N}^{m_N}/[m]!,
    \hspace{3em} r_m^* = q^{s_m}(r_{\lambda_1}^*)^{m_1}\cdots(r_{\lambda_N}^*)^{m_N},
\]    
where 
\[
    [m]! = \prod_{1\le i\le N}[m_i]!,\hspace{3em}
    s_m = \sum_{1\le i\le N}\dfrac{m_i(m_i-1)}2.
\]
The following theorem is from \cite{L}:

\begin{theorem}\label{pbw-basis}
    The elements $\{r_m\mid m\in\KP\}$ and $\{r_m^*\mid m\in\KP\}$
    give a pair of dual bases over $\Q(q)$ for $\f$. They are called the \term{PBW basis} and the 
    \term{dual PBW basis}.
\end{theorem}

\subsection{Proper Standard Modules}

In this subsection, we follow the approach in \cite{Mc} to the proper standard modules. They will
categorify the dual PBW basis.

From now on until the end of this section, we suppose that 3) of
Theorem \ref{canonical-basis} holds,
so that each $\*$-self-dual simple module lies in the dual canonical basis.
This is true in the case that $\K$ is of characteristic zero.

\begin{theorem}[{\cite[Theorems 3.1, 3.2]{Mc}}]\label{proper-standard-modules}
    For each $\lambda\in R^+$, there exists a simple module $L(\lambda)$
    of $H(\lambda)$ such that:
    \begin{enumerate}[1)]
        \item The image of $[L(\lambda)]$ under the isomorphism
        (\ref{categorification-of-f*}) is the dual root vector $r_\lambda^*$;
        \item For every $m\in\KP(\alpha)$ for $\alpha\in Q^+$, the module
        \[
            \pstd(m) \coloneq  q^{s_m}L(\lambda_1)^{\circ m_1}\circ\cdots
            \circ L(\lambda_N)^{\circ m_N}
        \] 
        has a simple head,
        which will be denoted $L(m)$;
        \item The set $\{L(m)\mid m\in\KP(\alpha)\}$
        gives a full set of representatives, each corresponds to an isomorphism class
        up to degree shifts of simple modules of $H(\alpha)$;
        \item For every $m\in\KP(\alpha)$, the module
        \[
            \pcostd(m) \coloneq  q^{-t_m}L(\lambda_N)^{\circ m_N}\circ\cdots
            \circ L(\lambda_1)^{\circ m_1}
        \]    
        has socle $L(m)$,
        where here $t_m = \sum_{i<j}m_im_j\lambda_i\cdot\lambda_j$;
        \item The modules $L(m)$'s are $\*$-self-dual;
        \item For any $m,n\in\KP(\alpha)$ such that $[\pstd(m):L(n)]>0$, 
        we have $n\le m$; $[\pstd(m):L(m)] = 1$; Similarly
        for any $m,n\in\KP(\alpha)$ such that $[\pcostd(m):L(n)]>0$, 
        we have $n\le m$; $[\pcostd(m):L(m)] = 1$;
        \item If $\alpha$ is a root, $\beta+\gamma = \alpha$,
        $\Res_{\beta,\gamma}L(\alpha)\ne 0$, then $\beta$ is a sum of roots
        no less than $\alpha$, and $\gamma$ is a sum of roots no greater than
        $\alpha$.
    \end{enumerate}
    The modules $\pstd(m)$ and $\pcostd(m)$ are called 
    \term{proper standard modules} and \term{proper costandard modules},
    respectively. The modules $L(\lambda)$ for $\lambda\in R^+$
    are called \term{cuspidal modules}.
\end{theorem}

\begin{proof}
    We proceed by simultaneous induction on the height of $\alpha$ and
    $\lambda$. When the height is $1$, $\alpha$ is a simple root.
    In this case $H(\alpha)\cong\K[x]$ where $x$ is in degree $2$,
    and $\KP(\alpha)$ consists of only one element $(\alpha)$. We define
    $L(\alpha) \coloneq  H(\alpha)/\rad H(\alpha)$, then the properties 1)-7)
    are obvious. We now take a positive integer $k>1$ and suppose that
    $L(\lambda)$ is constructed for all $\lambda$ with $\operatorname{ht}(\lambda)<k$,
    and properties 1)-7) are satisfied whenever $\operatorname{ht}(\alpha)<k$.

    We now take $\alpha\in Q^+$ with height $k$. Define $\KP'(\alpha) = \KP(\alpha)$
    when $\alpha$ is not a root, and $\KP'(\alpha) = \KP(\alpha)-\{(\alpha)\}$
    when $\alpha$ is a root. By induction hypothesis, for all $m\in\KP'(\alpha)$,
    $\pstd(m)$ is defined. 

    \begin{lemma}\label{res-of-proper-std}
        For any $m\in\KP'(\alpha)$, $n\in\KP(\alpha)$, $n = (n_1,\cdots,n_N)$,
        define $\Res_n \coloneq  \Res_{n_1\lambda_1,\cdots,n_N\lambda_N}$. Then 
        \[
            \Res_n\pstd(m) = 
            \begin{cases}
                0,&n>'m\text{ or }n>''m;\\
                L(\lambda_1^{m_1})\ex\cdots\ex L(\lambda_N^{m_N}),&n = m.
            \end{cases}
        \]     
    \end{lemma}

    \begin{proof}[Proof of the Lemma]
        Suppose that $m = (m_1,\cdots,m_N) = (\lambda_{i_1},\cdots,\lambda_{i_p})$.
        Suppose that $n\ge''m$ and $\Res_n\pstd(m)\ne 0$.
        Let $l$ be the largest index such that $n_l\ne 0$.
        Then $i_j\le l$ for all $j$.
        By the generalized version of Proposition \ref{mackey},
        $\Res_n\pstd(m)$ has a filtration indexed by all tables of the following
        form:
        \begin{center}
            \begin{tabular}{c|ccc}
                &$\lambda_{i_1}$&$\cdots$&$\lambda_{i_p}$\\
                \hline
                $n_1\lambda_1$&$\nu_{11}$&$\cdots$&$\nu_{1p}$\\
                $\vdots$&$\vdots$&$\ddots$&$\vdots$\\
                $n_l\lambda_l$&$\nu_{l1}$&$\cdots$&$\nu_{lp}$,
            \end{tabular}
        \end{center}
        such that for any $i$, $\sum_j\nu_{ij} = n_i\lambda_i$ and for any $j$,
        $\sum_i\nu_{ij} = \lambda_{i_j}$. Its corresponding subquotient is nonzero
        if and only if $\Res_{\nu_{1j},\cdots,\nu_{lj}}L(\lambda_{i_j})\ne 0$
        for all $j$. In particular $\Res_{\lambda_{i_j}-\nu_{lj},\nu_{lj}}
        L(\lambda_{i_j})\ne 0$, thus by 7), $\nu_{lj}$ is a sum of roots no
        greater than $\lambda_{i_j}$, which is again no greater than $\lambda_l$.
        Now $\sum_j\nu_{lj}$ is a sum of roots no greater than $\lambda_l$,
        but it is equal to $n_l\lambda_i$, by Lemma \ref{sum-in-convex-ordering},
        each $\nu_{lj}$ must be a multiple of $\lambda_l$. Therefore 
        $\nu_{lj} = 0$ whenever $i_j<l$, and $\nu_{lj} = \lambda_l$ or $0$
        whenever $i_j = l$. This shows that $m_l\ge n_l$. But $n\ge''m$,
        thus $m_l\le n_l$. Therefore $n_l = m_l$ and the table has form
        \begin{center}
            \begin{tabular}{c|cccccc}
                &$\lambda_{i_1}$&$\cdots$&$\lambda_{i_{p-m_l}}$
                &$\lambda_l$&$\cdots$&$\lambda_l$\\
                \hline
                $n_1\lambda_1$&$\nu_{11}$&$\cdots$&$\nu_{1,p-m_l}$
                &$0$&$\cdots$&$0$\\
                $\vdots$&$\vdots$&$\ddots$&$\vdots$
                &$\vdots$&$\ddots$&$\vdots$\\
                $n_{l-1}\lambda_{l-1}$&$\nu_{l-1,1}$&$\cdots$&$\nu_{l-1,p-m_l}$
                &$0$&$\cdots$&$0$\\
                $n_l\lambda_l$&$0$&$\cdots$&$0$
                &$\lambda_l$&$\cdots$&$\lambda_l$.
            \end{tabular}
        \end{center}
        Repeat the above argument, we must have $m = n$ and the table must have
        diagonal form. Again by the generalized version of Proposition \ref{mackey},
        we must have 
        \[
            \Res_n\pstd(m) = q^{s_m}
            L(\lambda_1)^{\circ m_1}\ex\cdots\ex L(\lambda_N)^{\circ m_N}.
        \]    
        By induction hypothesis 2), 6) and Lemma \ref{minimal-element-in-KP},
        we have $$q^{m_i(m_i-1)/2}L(\lambda_i)^{\circ m_i}
         = \pstd(\lambda_i^{m_i}) = L(\lambda_i^{m_i}),$$ thus we have 
        \[
            q^{s_m}L(\lambda_1)^{\circ m_1}\ex\cdots\ex L(\lambda_N)^{\circ m_N}
            = L(\lambda_1^{m_1})\ex\cdots\ex L(\lambda_N^{m_N}).
        \]    
        The case where $n\ge'm$ is similar.
    \end{proof}

    We now return to the proof of the original theorem.

    2): Suppose $Q$ is a nonzero quotient of $\pstd(m)$. By adjunction
    there exists a nonzero map $L(\lambda_1^{m_1})\ex\cdots\ex L(\lambda_N^{m_N})
    \to\Res_m(Q)$. Since the source of the map is simple, the map is injective.

    Now suppose that the head of $\pstd(m)$ is reducible, and has form $Q\oplus Q'$
    where $Q$ and $Q'$ nonzero. Since $\Res$ is exact, the surjection 
    $\pstd(m)\to Q\oplus Q'$ induces a surjection 
    \[
        \Res_m(\pstd(m))\to\Res_m(Q)\oplus\Res_m(Q').
    \] 
    By Lemma \ref{res-of-proper-std},
    $\Res_m(\pstd(m))\cong L(\lambda_1^{m_1})\ex\cdots\ex L(\lambda_N^{m_N})$
    is simple, while both $\Res_m(Q)$ and $\Res_m(Q')$ have
    $L(\lambda_1^{m_1})\ex\cdots\ex L(\lambda_N^{m_N})$ as submodules, which
    contradicts to the fact that $\Res_m(\pstd(m))\to
    \Res_m(Q)\oplus\Res_m(Q')$ is surjective. Therefore the head of $\pstd(m)$
    must be irreducible.

    Note that by the proof of 2), we obtain that 
    \begin{equation}\label{res-of-simple}
        \Res_n L(m) = 
        \begin{cases}
            0,&n>'m\text{ or }n>''m;\\
            L(\lambda_1^{m_1})\ex\cdots\ex L(\lambda_N^{m_N}),&n = m.
        \end{cases}
    \end{equation}

    3): The formula (\ref{res-of-simple}) shows that elements of
    $\{L(m)\mid m\in\KP'(\alpha)\}$ are pairwise nonisomorphic simple modules
    up to degree shifts.

    If $\alpha$ is not a root, then $\KP'(\alpha) = \KP(\alpha)$.
    Otherwise, we take $L(\alpha)$ to be the $\*$-self-dual simple module
    that is not isomorphic to any of the $L(m)$'s for $m\in\KP'(\alpha)$
    up to degree shifts.

    In either case, $\{L(m)\mid m\in\KP(\alpha)\}$ 
    consists of pairwise nonisomorphic simple modules up to degree shifts.
    By Theorem \ref{pbw-basis}, $\f_{\alpha}$ has dimension $\abs{\KP(\alpha)}$,
    thus by Theorem \ref{categorification}, $H(\alpha)$ has
    $\abs{\KP(\alpha)}$ simple modules up to degree shifts.
    Therefore $\{L(m)\mid m\in\KP(\alpha)\}$
    gives a set of full representatives, each representing an isomorphism class
    up to degree shifts of simple modules of $H(\alpha)$.

    4): Since $\Res_m(L(m))\cong L(\lambda_1^{m_1})\ex\cdots\ex L(\lambda_N^{m_N}),$
    the coinduction adjunction induces a map $L(m)\to\pcostd(m)$, 
    that is nonzero since $L(m)$ is simple. The remainder of the proof
    is similar to that of 2).

    5): By induction hypothesis and Proposition \ref{dual-of-induction},
    $\pstd(m)^\*\cong\pcostd(m)$. Now the $\*$-dual of 
    the surjection $\pstd(m)\to L(m)$ induces a map $L(m)^\*\to\pcostd(m)$.
    By 4), $\pcostd(m)$ has socle $L(m)$, thus $L(m)^\*$ must be $L(m)$.

    6): Suppose that $[\pstd(m):L(n)]>0$. Since $\Res$ is exact and
    $\Res_nL(n)\ne 0$, we have $\Res_n\pstd(m)\ne 0$. By Lemma \ref{res-of-proper-std},
    $n\le m$. By definition of $L(m)$ we have $[\pstd(m):L(m)]>0$.
    On the other hand, since $$\Res_m(\pstd(m))\cong\Res_m(L(m))
    \cong L(\lambda_1^{m_1})\ex\cdots\ex L(\lambda_N^{m_N}),$$
    we have $[\pstd(m):L(m)]\le 1$. Thus $[\pstd(m):L(m)] = 1$.

    The statement for $\pcostd(m)$ follows by a similar argument.

    7): We only prove the statement for $\gamma$; the statement for $\beta$
    is similar. We shall also ignore the grading. Suppose that
    $L(m)\ex L(m')\subseteq\Res_{\beta,\gamma}L(\alpha)$ for some $m\in\KP(\beta)$
    and $m'\in\KP(\gamma)$. Let $\gamma'$ be the largest element in the
    support of $m'$. Then $\Res_{\alpha-\gamma',\gamma'}L(\alpha)\ne 0$. 
    To show that $\gamma$ is a sum of roots no greater than $\alpha$,
    it suffices to show that $\gamma'\le\alpha$. Thus we may suppose that
    $\gamma$ is a root, that is maximal in all possible choices.
    By the maximality of $\gamma$, we must have $m' = (\gamma)$.

    We now let $\beta'$ be the largest element in the support of $m$.

    If $\beta'\le\gamma$ then $L(m)\circ L(\gamma)$ is a quotient
    of $\pstd(m\amalg\gamma)$, where $m\amalg\gamma$ is the Kostant partition
    obtained by inserting $\gamma$ at the end of the root sequence of
    $m$. Since $L(m)\ex L(\gamma)\hookrightarrow\Res_{\beta,\gamma}L(\alpha)$,
    we obtain a nonzero map $$\pstd(m\amalg\gamma)\to L(m)\circ L(\gamma)
    \to L(\alpha),$$ which is surjective since $L(\alpha)$ is simple.
    Since $\pstd(m\amalg\gamma)$ has simple head $L(m\amalg\gamma)$,
    we must have $m\amalg\gamma = \alpha$, thus $\gamma = \alpha$
    is a sum of roots no greater than $\alpha$.

    This leaves us with the case $\beta'>\gamma$. Take the Kostant partition
    $n$ such that $m = n\amalg\beta'$. Then $\pstd(n)\ex L(\beta')$
    surjects to $\Res_{\beta-\beta',\beta'}L(m)$. Therefore there exists
    a nonzero map 
    \[
        \pstd(n)\ex L(\beta')\ex L(\gamma)\to
        \Res_{\beta-\beta',\beta',\gamma}L(\alpha),
    \] 
    which gives rise to a map
    $\pstd(n)\ex(L(\beta')\circ L(\gamma))\to\Res_{\beta-\beta',\beta'+\gamma}L(\alpha)$.
    Therefore there exists some $n'\in\KP(\beta'+\gamma)$
    and $n_1\in\KP(\beta-\beta')$ such that 
    there exists an inclusion $L(n_1)\ex L(n')\to
    \Res_{\beta-\beta',\beta'+\gamma}L(\alpha)$. By the maximality
    of $\gamma$, all elements in the support of $n'$ are no greater than
    $\gamma$. However the sum of elements in the root sequence of $n'$
    is $\beta'+\gamma$, and $\beta'$ is strictly greater than $\gamma$,
    which contradicts to Lemma \ref{sum-in-convex-ordering}.
    Therefore the case $\beta'>\gamma$ cannot happen,
    which completes the proof.

    1): We take a minimal pair $(\beta,\gamma)$ of $\lambda$.
    By induction hypothesis, we have $[\pstd(\beta,\gamma)] = r_\beta^*r_\gamma^*$,
    and $[\pcostd(\beta,\gamma)] = q^{-\beta\cdot\gamma}r_\gamma^*r_\beta^*$.
    Now consider the composition 
    \[
        \pstd(\beta,\gamma)\to L(\beta,\gamma)
        \to\pcostd(\beta,\gamma).
    \] 
    By 5), the kernel and the cokernel
    of the map has all composition factors of the form $L(n)$
    such that $n<(\beta,\gamma)$. However $(\beta,\gamma)$ is a minimal
    pair, thus $n$ must equal to $(\lambda)$. This shows that
    $[\pstd(\beta,\gamma)]-[\pcostd(\beta,\gamma)]$ is a multiple
    of $[L(\lambda)]$. 
    
    On the other hand, by the Levendorskii-Soibelman formula
    (\cite[Theorem 5.5.2]{LS}), we have $r_\beta r_\gamma
    -q^{-\beta\cdot\gamma}r_\gamma r_\beta$ lies in 
    $\sum_m\Q(q)r_m$, where $m$ runs over all elements in $\KP(\lambda)$
    that are less than $(\beta,\gamma)$. However $(\beta,\gamma)$
    is minimal, thus $m$ can only be $(\lambda)$, 
    which shows that $r_\beta r_\gamma
    -q^{-\beta\cdot\gamma}r_\gamma r_\beta$ is a multiple of $r_\lambda$.
    Therefore $r_\beta^* r_\gamma^*
    -q^{-\beta\cdot\gamma}r_\gamma^* r_\beta^*$ is a multiple of $r_\lambda^*$.

    We know that the specializations of $r_\beta$ and $r_\gamma$ to $q = 1$
    are nonzero root vectors in the weight spaces $\mathfrak{g}_\beta$
    and $\mathfrak{g}_\gamma$. Since $\beta+\gamma$ is a root,
    the root vectors do not commute. Thus $r_\beta^* r_\gamma^*
    -q^{-\beta\cdot\gamma}r_\gamma^* r_\beta^*$ is nonzero.

    The above discussion now shows that $[L(\lambda)]$ is a nonzero
    multiple of $r_\lambda^*$. By our assumption, $[L(\lambda)]$ lies in the dual canonical basis.
    Since $r_\lambda^*$ is known to be in the dual canonical basis,
    we must have $[L(\lambda)] = r_\lambda^*$.
\end{proof}

From this theorem we immediate obtain the following corollary:

\begin{corollary}
    The proper standard modules categorify the dual PBW basis,
    i.e. $[\pstd(m)] = r_m^*$ for all $m\in\KP$.
\end{corollary}

\begin{remark}
    The statement 7) remains true if we assume that $\alpha$ is a 
    multiple of a positive root. The proof is essentially the same.
\end{remark}

\begin{remark}\label{unique-cuspidal}
    Since $\Res_m(L(m))\cong L(\lambda_1^{m_1})\ex\cdots\ex L(\lambda_N^{m_N})
    \ne 0$, we see that if $\alpha$ is a root, and $L$ is a $\*$-self-dual
    simple $H(\alpha)$-module satisfying 7), then $L$ must be isomorphic
    to $L(\alpha)$. This gives a characterization of the cuspidal module.
\end{remark}

For further use, we record the following proposition:

\begin{proposition}\label{res-of-semi-cuspidal}
    If $\alpha\in R^+$, $k\ge 1$ is an integer, then 
    \[
        [\Res_{\alpha,\alpha,\cdots,\alpha}L(\alpha^k)]
        = [k]![L(\alpha)^{\ex k}].
    \]
\end{proposition}

\begin{proof}
    It suffices to prove that $$[\Res_{\alpha,(k-1)\alpha}L(\alpha^k)]
     = [k][L(\alpha)\ex L(\alpha^{k-1})].$$
    We prove this by induction on $k$. The case where $k = 1$ is
    obvious. Suppose the statement is true for $k-1$. We note that
    $L(\alpha^k)\cong q^{k-1}L(\alpha)\circ L(\alpha^{k-1})$.
    We use Proposition \ref{mackey}: nonzero subquotients are 
    identified with $\beta,\gamma,\beta',\gamma'\in Q^+$ such that
    $\Res_{\beta,\gamma}L(\alpha)\ne 0$, $\Res_{\beta',\gamma'}
    L(\alpha^{k-1})\ne 0$, and $\beta+\beta' = \alpha$. By Theorem
    \ref{proper-standard-modules} 7), the elements $\beta$ and $\beta'$ are sums 
    of roots no less than $\alpha$. Since they sum to $\alpha$,
    either $\beta = \alpha$ or $\beta' = \alpha$. Now the subquotient
    corresponding to $\beta = \alpha$ is $q^{k-1}[L(\alpha)\ex 
    L(\alpha^{k-1})]$, where the subquotient
    corresponding to $\beta' = \alpha$ is 
    \[
        q^{k-1} q^{-\alpha\cdot\alpha}
        [(\Ind_{0,\alpha}\ex\Ind_{\alpha,(k-2)\alpha})\iota^*
        (\Res_{0,\alpha}\ex\Res_{\alpha,(k-2)\alpha})(L(\alpha)\ex 
        L(\alpha^{k-1}))],
    \]    
    which is $q^{-1}[k-1][L(\alpha)\ex 
    L(\alpha^{k-1})]$ by induction hypothesis.
    Here we recall that $\iota$ is the natural map 
    \[
        \iota\:H(\beta)\ox H(\gamma)\ox H(\beta')\ox H(\gamma')
        \to H(\beta)\ox H(\beta')\ox H(\gamma)\ox H(\gamma').
    \] 
    Since
    $q^{k-1}+q^{-1}[k-1] = [k]$, the statement is true for $k$,
    completing the proof.
\end{proof}

\begin{corollary}\label{res'}
    For any $n\in\KP$, $n = (\lambda_{i_1},\cdots,\lambda_{i_l})$,
    define $\Res_n' \coloneq  \Res_{\lambda_{i_1},\cdots,\lambda_{i_l}}$. 
    Then 
    \[
        [\Res_n'L(n)] = [n]![L(\lambda_1)^{\ex n_1}\ex\cdots
        \ex L(\lambda_N)^{\ex n_N}],
    \]    
    where $[n]! \coloneq  [n_1]!\cdots[n_N]!$. Moreover $[\Res_m'L(n)] = 0$
    whenever $n\not\le m$.
\end{corollary}

\begin{proof}
    Combine Proposition \ref{res-of-semi-cuspidal} and Lemma 
    \ref{res-of-proper-std}.
\end{proof}

\subsection{Algebraic Standard Modules}

In this subsection, we follow the approach given
in \cite{BKM} to the construction of algebraic standard modules. They will
categorify the PBW basis. 

We first record the following proposition, which is 
\cite[Theorem 3.1]{BKM}:

\begin{proposition}\label{ext-of-cuspidal}
    For any root $\alpha$, we have
    $\Ext^1(L(\alpha),L(\alpha)) = q^{-2}\K$, and
    $\Ext^d(L(\alpha),L(\alpha)) = 0$ if $d\ge 2$.
\end{proposition}

We also record the following result in homological algebra:

\begin{proposition}\label{ext-of-subquo}
    Suppose $U$ and $V$ are finitely generated $H$-modules, and $d>0$ is an integer.
    If $\ext^d(U,L) = 0$ for all irreducible subquotient of $V$,
    then $\ext^d(U,V) = 0$.
\end{proposition}

We proceed the construction of the algebraic standard modules by the following three steps.
We will first construct the \term{root modules} $\std(\alpha)$
when $\alpha$ is a root, and construct the \term{divided power} $\std(\alpha^k)$
when again $\alpha$ is a root, and finally construct
the algebraic standard modules in general cases.

We first construct the root modules.

\begin{lemma}\label{filtr-of-std}
    Suppose $\alpha$ is a positive root.
    There exist $H(\alpha)$-modules $\std_n(\alpha)$,
    $n = 0,1,\cdots$, with $\std_0(\alpha) = 0$, such that 
    there exist exact sequences 
    \begin{equation}\label{exact-1}0\to q^{2(n-1)}L(\alpha)\xrightarrow{i_n}
    \std_n(\alpha)\xrightarrow{p_n}\std_{n-1}(\alpha)\to 0\end{equation}
    and \begin{equation}\label{exact-2}0\to q^2\std_{n-1}(\alpha)\xrightarrow{j_n}
    \std_n(\alpha)\xrightarrow{q_n} L(\alpha)\to 0,\end{equation}
    such that for any $n\ge 1$:
    \begin{enumerate}[1)]
        \item $[\std_n(\alpha)] = \frac{1-q^{2n}}{1-q^2}[L(\alpha)]$;
        \item The module $\std_n(\alpha)$ is cyclic, has head $L(\alpha)$
        and socle $q^{2(n-1)}L(\alpha)$;
        \item If $n\ge 2$, then $\hom(q^2\std_n(\alpha),
        \std_n(\alpha)) = \K$;
        \item The map 
        \[ 
            i_n^*\:\Ext^1(\std_n(\alpha),L(\alpha))
            \to\Ext^1(q^{2(n-1)}L(\alpha),L(\alpha))
        \] 
        is an isomorphism;
        \item We have $\Ext^d(\std_n(\alpha),L(\alpha)) = 0$ for all $d\ge 2$.
    \end{enumerate}
\end{lemma}

\begin{proof}
    We proceed by induction on $n$. When $n = 1$, we set $\std_1(\alpha)
     = L(\alpha)$, the exact sequences and the properties are obvious.
    We now suppose that the statement is true for $n-1$, where
    $n\ge 2$.

    Sequence (\ref{exact-1}): By induction hypothesis 4),
    we have $\Ext^1(\std_{n-1}(\alpha),L(\alpha)) = q^{-2(n-1)}\K$,
    so there exists a unique module $\Delta_n(\alpha)$ up to isomorphism
    that fits into this nonsplit exact sequence, such that 
    the exact sequence nonsplit.

    1): This follows from the induction hypothesis and the 
    sequence (\ref{exact-1}).

    4) and 5): Take $\Hom(-,L(\alpha))$ to the sequence (\ref{exact-1}),
    by induction hypothesis and Proposition \ref{ext-of-cuspidal},
    5) holds. Moreover there exists a long exact sequence:
    \[
        \begin{aligned}
            0&\to\K\xrightarrow{f_1}\Hom(\std(\alpha),L(\alpha))\xrightarrow{f_2}
            q^{-2(n-1)}\K\xrightarrow{f_3}q^{-2(n-1)}\K\\
            &\xrightarrow{f_4}
            \Ext^1(\std_n(\alpha),L(\alpha))\xrightarrow{i_n^*}
            \Ext^1(q^{-2(n-1)}L(\alpha),L(\alpha))\to 0.
        \end{aligned}
    \]
    The map $f_2$ must be zero, or else there exists a nonzero map
    $\std_n(\alpha)\to q^{2(n-1)}L(\alpha)$ which will split
    sequence (\ref{exact-1}). Thus $f_2 = 0$, and  $f_1$ and $f_3$ are isomorphisms,
    then $f_4 = 0$, which shows that $i_n^*$ is an isomorphism.

    2): By 1) and that $\Hom(\std_n(\alpha),L(\alpha)) = \K$,
    $\std_n(\alpha)$ has head $L(\alpha)$ and is generated by 
    any homogeneous element that is not mapped to $0$ in $L(\alpha)$.
    To compute the socle, by the sequence (\ref{exact-1})
    and induction hypothesis 2), it suffices to show that
    $q^{2(n-2)}L(\alpha)$ is not in the socle. To show this
    consider $V \coloneq  p_n^{-1}(\im i_{n-1})$. By induction hypothesis 4),
    the exact sequence $$0\to q^{2(n-1)}L(\alpha)\to V
    \to q^{2(n-2)}L(\alpha)\to 0$$ does not split,
    hence $q^{2(n-2)}L(\alpha)$ is not in the socle.

    Sequence (\ref{exact-2}): In the case $n = 2$ this is just
    sequence (\ref{exact-1}). If $n\ge 3$, take $\hom(q^2\std_{n-1}(\alpha),-)$
    to sequence (\ref{exact-1}), by induction hypothesis 3),
    we obtain that $\hom(q^2\std_{n-1}(\alpha),\std_n(\alpha)) = \K$.
    The nonzero map $q^2\std_{n-1}(\alpha)\to\std_n(\alpha)$ is injective
    since it is injective on socle, and its cokernel must be
    $L(\alpha)$ by 1).

    3): Take $\hom(-,\std_{n-1}(\alpha))$ to sequence (\ref{exact-2}),
    from the long exact sequence, we obtain that $\hom(\std_n(\alpha),\std_{n-1}(\alpha)) = \K$;
    now take $\hom(q^2\std_n(\alpha),-)$ to sequence (\ref{exact-2}),
    from the long exact sequence, we obtain that $\hom(q^2\std_n(\alpha),\std_n(\alpha)) = \K$.
\end{proof}

Now we have a sequence of surjections $$\cdots\to
\std_2(\alpha)\to\std_1(\alpha)\to\std_0(\alpha).$$

\begin{definition}
    We take $\std(\alpha)$ to be the inverse limit of the sequence
    \[
        \cdots\to\std_2(\alpha)\to\std_1(\alpha)\to\std_0(\alpha),
    \]
    and call it the \term{root module}.
\end{definition}

\begin{proposition}\label{root-module}
    Suppose that $\alpha$ is a root.
    There exists an exact sequence \begin{equation}\label{exact-3}
        0\to q^2\std(\alpha)\xrightarrow{j}\std(\alpha)\to L(\alpha)\to 0.
    \end{equation}
    Moreover:\begin{enumerate}[1)]
        \item The module $\std(\alpha)$ is cyclic
        and $[\std(\alpha)] = [L(\alpha)]/(1-q^2)$;
        \item We have $\Dim\Hom(\std(\alpha),L(\alpha)) = 1$, 
        thus $\std(\alpha)$ has irreducible head $L(\alpha)$;
        \item We have $\Ext^d(\std(\alpha),V) = 0$ for all $d\ge 1$ and any finitely
        generated $H$-module $V$ with all irreducible subquotients isomorphic
        to $q^mL(\alpha)$ for some $m$; in other words,
        $\std(\alpha)$ is a projective generator in the
        full subcategory of $\mod(H(\alpha))$ consisting of
        all modules with all irreducible subquotients isomorphic
        to $q^mL(\alpha)$ for some $m$.
        \item We have $\End(\std(\alpha))\cong\K[x]$, where $x$ is in degree $2$.
    \end{enumerate}
\end{proposition}

\begin{proof}
    1): This follows directly from Lemma \ref{filtr-of-std}.

    2): Since the inverse system $$\cdots\to
    \std_2(\alpha)_i\to\std_1(\alpha)_i\to\std_0(\alpha)_i$$
    is stable for every $i$, any homogeneous map $f\:\std(\alpha)
    \to q^mL(\alpha)$ for some $m$ factors through some $\std_n(\alpha)$.
    Since $\Hom(\std_n(\alpha),L(\alpha)) = \K$, we must have
    $\Dim\Hom(\std(\alpha),L(\alpha)) = 1$. The conclusion to the head
    follows by the identity in the Grothendieck group and the dimension
    of the hom-space.

    Sequence (\ref{exact-3}): We need to modify the maps $j_n$ such that the
    diagram commutes:
    \[
      \begin{tikzcd}
        q^2\std_{n-1}(\alpha)\ar[d,"j_n"']&q^2\std_n(\alpha)\ar[l,"p_n"']\ar[d,"j_n"']\\
        \std_n(\alpha)&\std_{n+1}(\alpha)\ar[l,"p_{n+1}"'].
      \end{tikzcd}
    \] 
    In the case $n = 1$ it is automatic (both compositions are $0$);
    in the case $n\ge 2$, by Lemma \ref{filtr-of-std} 3), we have $\hom(q^2\std_n(\alpha),
    \std_n(\alpha)) = \K$. thus $j_{n+1}$
    can be uniquely scaled such that the diagram commutes. Now take the inverse
    limits of the $j_n$'s we obtain a map $j\:q^2\std(\alpha)\to\std(\alpha)$,
    that is injective since each $j_n$ is. By 1), its cokernel
    is isomorphic to $L(\alpha)$.

    3): By Proposition \ref{ext-of-subquo} it suffices to prove this
    for $V = L(\alpha)$. Take $\Ext^d(-,L(\alpha))$ to sequence
    (\ref{exact-3}), noticing that $\Ext^{d+1}(L(\alpha),L(\alpha)) = 0$,
    we have a surjection 
    \[
        \Ext^d(\std(\alpha),L(\alpha))\to
        q^{-2}\Ext^d(\std(\alpha),L(\alpha)).
    \] 
    However $\Ext^d(\std(\alpha),L(\alpha))$
    is bounded below, thus the only way that this could happen is
    $\Ext^d(\std(\alpha),L(\alpha)) = 0$.

    4): The injective map $x \coloneq  j$ generates a free polynomial subalgebra
    $\K[x]$ of $\End(\std(\alpha))$. However by 1),
    we have $\Dim\End(\std(\alpha))\le\Dim\K[x]$. Hence we must have
    $\End(\std(\alpha))\cong\K[x]$.
\end{proof}

By the above discussion, we have the following corollary:

\begin{corollary}
    The root module $\std(\alpha)$ categorify the root vector $r_\alpha$.
\end{corollary}

We next construct the divided powers.

Let $\alpha$ be a root, let $n$ be the height of $\alpha$, and fix a minimal degree
element $v_\alpha\in\std(\alpha)$. By Proposition \ref{root-module},
the element $v_\alpha$ generates $\std_\alpha$.

\begin{lemma}\label{definition-of-tau}
    Let $w\in S_{2n}$ be the element $(1,n+1)(2,n+2)\cdots(n,2n)$.
    Then there exists a unique $H(2\alpha)$-module morphism
    \[
        \tau\:\std(\alpha)\circ\std(\alpha)
        \to\std(\alpha)\circ\std(\alpha),
    \]
    of degree $2$, such that $\tau(\bi_{\alpha,\alpha}\ox
    (v_\alpha\ox v_\alpha)) = \tau_w\bi_{\alpha,\alpha}\ox
    (v_\alpha\ox v_\alpha)$.
\end{lemma}

\begin{proof}
    We apply Proposition \ref{mackey} to 
    $\Res_{\alpha,\alpha}(\std(\alpha)\circ\std(\alpha))$. By similar arguments
    to Proposition \ref{res-of-semi-cuspidal}, there are only two
    nonzero subquotients, corresponding to the double cosets
    $[1]$ and $[w]$. Thus there exists an exact sequence
    \begin{equation}\label{exact-4}
        0\to\std(\alpha)\ex\std(\alpha)\xrightarrow{f}
        \Res_{\alpha,\alpha}(\std(\alpha)\circ\std(\alpha))
        \xrightarrow{g}q^{-2}\std(\alpha)\ex\std(\alpha)\to0,
    \end{equation}
    such that $f(v_\alpha\ox v_\alpha) = \bi_{\alpha,\alpha}
    \ox(v_\alpha\ox v_\alpha)$ and $g(\tau_w\bi_{\alpha,\alpha}
    \ox(v_\alpha\ox v_\alpha)) = v_\alpha\ox v_\alpha$.
    By Proposition \ref{root-module} 3), we have 
    \[
        \Ext^1(q^{-2}\std(\alpha)\ex\std(\alpha),
        \std(\alpha)\ex\std(\alpha)) = 0,
    \] 
    thus the exact
    sequence splits. Let 
    \[
        h\:q^{-2}\std(\alpha)\ex\std(\alpha)
        \to\Res_{\alpha,\alpha}(\std(\alpha)\circ\std(\alpha))
    \]
    be the splitting. Since $\im f$ contains no elements with
    the same degree as $\tau_w\bi_{\alpha,\alpha}
    \ox(v_\alpha\ox v_\alpha)$, we must have $h(v_\alpha\ox v_\alpha)
     = \tau_w\bi_{\alpha,\alpha}\ox(v_\alpha\ox v_\alpha)$.
    Now apply Frobenius reciprocity to $h$, we obtain the
    desired map $\tau$.
\end{proof}

Now from the endomorphism $x\in\End(\std(\alpha))$
we obtain endmorphisms 
\[
    x_1,\cdots,x_k\in\End(\std(\alpha)^{\circ k})
\]
given by $x_i = \bi^{\circ(i-1)}\circ x\circ\bi^{\circ(k-i)}$.
Then the map $g$ in \ref{exact-4} intertwines the endomorphisms
$x_i$ and $x_{i+1}$. The lemma above yields endomorphisms
\[
    \tau_1,\cdots,\tau_{n-1}\in\End(\std(\alpha)^{\circ k})
\]
given by $\tau_i = \bi^{\circ(i-1)}\circ\tau\circ\bi^{\circ(k-i-1)}$.

\begin{lemma}
    The endomorphisms $\tau_i$ square to $0$ and satisfy the type 
    A braid relations, i.e. $\tau_i\tau_j = \tau_j\tau_i$ whenever
    $\abs{i-j}>1$ and $\tau_i\tau_{i+1}\tau_i = \tau_{i+1}\tau_i\tau_{i+1}$.
\end{lemma}

\begin{proof}
    The proof of Lemma \ref{definition-of-tau} shows that 
    as vector spaces, 
    \[
        \bi_{\alpha,\alpha}(\std(\alpha)\circ\std(\alpha))
        \cong\bi_{\alpha,\alpha}\ox(\std(\alpha)\ex\std(\alpha))
        \oplus\tau_w\bi_{\alpha,\alpha}\ox(\std(\alpha)\ex\std(\alpha)).
    \]
    Thus $\tau_w\bi_{\alpha,\alpha}\ox(v_\alpha\ox v_\alpha)$ is of
    minimal degree in $\bi_{\alpha,\alpha}(\std(\alpha)\circ\std(\alpha))$.
    Now the element $\tau_w\bi_{\alpha,\alpha}\ox(v_\alpha\ox v_\alpha)$ is of
    stricly smaller degree, hence must be $0$. This shows that
    $\tau^2 = 0$, hence $\tau_i^2 = 0$ for all $i$.

    The commuting braid relations are obvious. For the noncommuting ones,
    itsuffices to show that $\tau_1\tau_2\tau_1 = \tau_2\tau_1\tau_2$
    in the 3-fold product. Let $w_1,w_2\in S_{3n}$ be the elements
    $(1,n+1)\cdots(n,2n)$ and $(n+1,2n+1)\cdots(2n,3n)$, respectively,
    $w_0 = w_1w_2w_1 = w_2w_1w_2$. By the definition relations of $H(3\alpha)$,
    $(\tau_{w_1}\tau_{w_2}\tau_{w_1}-\tau_{w_2}\tau_{w_1}\tau_{w_2})
    \bi_{\alpha,\alpha,\alpha}\ox(v_\alpha\ox v_\alpha\ox v_\alpha)$
    lies in $$S \coloneq  \sum_{w<w_0}\tau_w\bi_{\alpha,\alpha,\alpha}\ox
    (v_\alpha\ox v_\alpha\ox v_\alpha).$$ By Proposition \ref{mackey},
    we have 
    \[
        S = \bigoplus_{w\in\{1,w_1,w_2,w_1w_2,w_2w_1\}}
        \tau_w\bi_{\alpha,\alpha,\alpha}\ox
        (v_\alpha\ox v_\alpha\ox v_\alpha).
    \] 
    But 
    $(\tau_{w_1}\tau_{w_2}\tau_{w_1}-\tau_{w_2}\tau_{w_1}\tau_{w_2})
    \bi_{\alpha,\alpha,\alpha}\ox(v_\alpha\ox v_\alpha\ox v_\alpha)$
    has degree $3\deg(v_\alpha)-6$ where all elements in $S$
    has degree no less than $3\deg(v_\alpha)-4$, thus
    this element must be $0$. This shows that $\tau_1\tau_2\tau_1
     = \tau_2\tau_1\tau_2$.
\end{proof}

As before, for each $w\in S_n$, we fix once and for all a reduced expression $w = s_{i_1}\cdots s_{i_l}$,
and define $\tau_w \coloneq  \tau_{i_1}\cdots\tau_{i_l}\in\End(\std(\alpha)^{\circ k})$.

\begin{lemma}
    The elements $\{\tau_w x_k^{t_k}\cdots x_1^{t_1}\mid t_1,\cdots,t_k\ge 0,
    w\in S_k\}$ give a basis for $\End(\std(\alpha)^{\circ k})$.
\end{lemma}

\begin{proof}
    They are linearly independent since they produce linearly independent
    vectors while applied to $\bi_{\alpha,\cdots,\alpha}\ox
    (v_\alpha\ox\cdots\ox v_\alpha)$. On the other hand, since
    $\std(\alpha)^{\ex k}$ has head $L(\alpha)^{\ex k}$, we have
    \[
    \begin{aligned}
        \Dim\End(\std(\alpha)^{\circ k})
        & = \Dim\Hom(\std(\alpha)^{\ex k},
        \Res_{\alpha,\cdots,\alpha}\std(\alpha)^{\circ k})\\
        &\le[\Res_{\alpha,\cdots,\alpha}\std(\alpha)^{\circ k}
        :L(\alpha)^{\ex k}]\\
        & = [\Res_{\alpha,\cdots,\alpha}L(\alpha)^{\circ k}
        :L(\alpha)^{\ex k}](1-q^2)^{-k}\\
        & = (1-q^2)^{-k}q^{-k(k-1)/2}[k]!\\
        & = \sum_{w\in S_k}\dfrac{q^{-2\ell(w)}}{(1-q^2)^k},
    \end{aligned}
    \]
    therefore they must span the entire $\End(\std(\alpha)^{\circ k})$,
    completing the proof.
\end{proof}

\begin{lemma}
    There exists a choice for $x\in\End(\std(\alpha))$ such that
    $\tau_i x_j = x_j\tau_i$ if $j\ne i,i+1$, and $\tau_i x_{i+1}
     = x_i\tau_i-1$ and $\tau_i x_i = x_{i+1}\tau_i+1$.
\end{lemma}

\begin{proof}
    The commuting relations are automatic. For the remaining relations,
    it suffices to work in $\End(\std(\alpha)\circ\std(\alpha))$.
    Let $\theta_- \coloneq  \tau x_2-x_1\tau$ and $\theta_+ \coloneq  \tau x_1-x_2\tau$.
    They are of degree zero, and since the map $g$ 
    in \ref{exact-4} intertwines the endomorphisms
    $x_1$ and $x_2$, they map $\bi_{\alpha,\alpha}\ox(v_\alpha\ox v_\alpha)$
    into $\bi_{\alpha,\alpha}\ox(\std(\alpha)\ex\std(\alpha))$.
    Therefore the two maps must be scalars. Suppose $\theta_- = c_-$,
    $\theta_+ = c_+$. Since 
    \[
    \begin{aligned}
        \tau x_1 x_2-x_1 x_2\tau = (x_2\tau+c_+)x_2-x_2(\tau x_2-c_-)
         = (c_++c_-)x_2,\\
        \tau x_1 x_2-x_1 x_2\tau = (x_1\tau+c_-)x_1-x_1(\tau x_1-c_+)
         = (c_++c_-)x_1,\\
    \end{aligned}
    \]
    we must have $c_++c_- = 0$. It suffices to show that 
    $c_+\ne 0$, for then we can replace $x$ with $x/c_+$ and completes 
    the proof. If not, then $c_+ = c_- = 0$, which shows that $\tau$
    fixes $S \coloneq  \im x_1+\im x_2$. However $(\std(\alpha)\circ\std(\alpha))
    /S\cong L(\alpha)\circ L(\alpha)$, thus $\tau$ induces an
    endomorphism $\bar\tau$ on $L(\alpha)\circ L(\alpha)$ whose square is $0$.
    Since $L(\alpha)\circ L(\alpha)$ is simple, $\bar\tau$ is $0$.
    However $\bar\tau$ sends $\bi_{\alpha,\alpha}\ox(\bar{v}_\alpha
    \ox\bar{v}_\alpha)$ to $\tau_w\bi_{\alpha,\alpha}\ox(\bar{v}_\alpha
    \ox\bar{v}_\alpha)$, thus $\bar\tau$ is nonzero, which is a contradiction.
    Therefore $c_+\ne 0$, completing the proof.
\end{proof}

Now the above three lemmas show that $$\End(\std(\alpha)^{\circ k})
\cong\NH(k)^{\mathrm{op}},$$ which shows that $\std(\alpha)^{\circ k}$
is an $(H(k\alpha),\NH(k))$-bimodule. We now define $$\std(\alpha^k)
 \coloneq  \std(\alpha)^{\circ k}\ox_{\NH(k)}P_k.$$

\begin{definition}
    The module $\std(\alpha^k)$ is called the \term{divided power}.
\end{definition}

\begin{proposition}\label{divided-power}
    We have $[k]!\std(\alpha^k)\cong\std(\alpha)^{\circ k}$.
    In particular $[k]![\std(\alpha^k)] = [\std(\alpha)^{\circ k}]$.
\end{proposition}

\begin{proof}
    Recall that $\NH(k)\cong [k]!P_k$, we have
    \[
        \std(\alpha)^{\circ k}\cong \std(\alpha)^{\circ k}\ox_{\NH(k)}\NH(k)
        \cong [k]!\std(\alpha)^{\circ k}\ox_{\NH(k)}P_k
        \cong[k]!\std(\alpha^k),
    \] 
    completing the proof.
\end{proof}

By the above discussion, we have the following corollary:

\begin{corollary}
    The divided power module $\std(\alpha^k)$ categorifies the divided power 
    $r_{(\alpha^k)} = r_\alpha^k/[k]!$.
\end{corollary}

We now construct the algebraic standard module in general cases.

Take $\alpha\in Q^+$. For any $m = (m_1,\cdots, m_N)$,
we define 
\[
    \std(m) \coloneq  \std(\lambda_1^{m_1})\circ\cdots
    \circ\std(\lambda_N^{m_N}).
\]

\begin{definition}
    The module $\std(m)$ is called the \term{algebraic standard module}.
\end{definition}

From the above discussion, the following theorem is immediate:

\begin{theorem}\label{std-modules}
    For any Kostant partition $m = (m_1,\cdots,m_N)$, the algebraic standard module $\std(m)$
    categorifies the element $r_m$ in the PBW basis,
    i.e. $[\std(m)] = r_m$.
\end{theorem}

\begin{remark}
    By the definition of the PBW basis and the dual PBW basis,
    $[\std(m)]$ is a multiple of $[\pstd(m)]$.
    Thus for any $m,n\in\alpha$, if $[\std(m):L(n)]\ne 0$,
    then $[\pstd(m):L(n)]\ne 0$, which shows that $n\le m$.
    Therefore the composition factors of algebraic standard modules
    satisfy the upper-triangular property similar to that of
    geometric standard modules.
\end{remark}

\section{Standard Modules and Polynomial Highest Weight Categories}

In this section, we give the relation between algebraic standard modules
and geometric standard modules. We shall also present that
for any $\alpha\in Q^+$,
$\mod(H(\alpha))$ is a polynomial height weight category.

Throughout this section, we assume that the quiver $\Gamma$ is of finite type
and the field $\K$ is of characteristic zero.

\subsection{The Relation between Algebraic and Geometric Standard Modules}

In this subsection, we give the relation between algebraic standard modules
and geometric standard modules. We will combine the approaches
given in \cite{K2} and \cite{Mc2}.

Recall that in Chapter 3, for each $\alpha\in Q^+$ and $m\in\KP(\alpha)$,
we gave the definition to the geometric modules $$\widetilde{P}(m)
 = \Ext^*_{G(\alpha)}(\LL(\alpha),\IC(\OO_m)),\hspace{3em}
\widetilde{\std}(m) = \Ext^*_{G(\alpha)}(\LL(\alpha),\costd(\OO_m)),$$
and $\widetilde{L}(m)$ is the head of $\widetilde{P}(m)$. On the other hand,
given a convex ordering on $R^+$, the algebraic modules
$L(m)$ and $\std(m)$ are defined in Chapter 4. We set $P(m)$
to be the projective cover of $L(m)$.

The main result of this subsection is the following theorem:

\begin{theorem}\label{alg-geo}
    There exists a convex ordering on $R^+$ such that for any 
    $m\in\KP$,
    \begin{enumerate}[1)]
        \item $\std(m)\cong\widetilde{\std}(m)$;
        \item $L(m)\cong\widetilde{L}(m)$;
        \item $P(m)\cong\widetilde{P}(m)$.
    \end{enumerate}
\end{theorem}

We first construct the convex ordering on $R^+$.

Let $\alpha = \sum c_i\alpha_i$ and $\alpha' = \sum c'_i\alpha_i$
be two elements in $\Q^+$. The \term{Euler form} of the 
quiver $\Gamma$ is defined as $$\<\alpha,\alpha'\>
 = \sum_{i\in I}c_ic'_i-\sum_{e\in E}c_{s(e)}c'_{t(e)}.$$
Here we recall that $I$ is the set of vertices and $E$
is the set of edges.
Since the quiver $\Gamma$ is of finite type, the symmetric
bilinear form 
\[
    (\alpha,\alpha') \coloneq  (\<\alpha,\alpha'\>
    +\<\alpha',\alpha\>)/2
\] 
is positive definite, and 
for any root $\alpha$, $\<\alpha,\alpha\> = 1$.

A reduced expression $s_{i_1}\cdots s_{i_N}$ of the longest
root $w_0$ is said to be \term{adapted} to $\Gamma$, if for all $k$
the vertex $i_k$ is a sink of the quiver of the quiver
$\sigma_{i_{k+1}}\cdots\sigma_{i_N}\Gamma$, where the operation
$\sigma_i$ is the operation on $\Gamma$ which reverses the direction
of all arrows joint to the vertex $i$.

By \cite[Propositions 4.9, 4.12]{L2} and \cite[p59]{Ri},
there always exists a reduced expression of $w_0$
that is adapted to $\Gamma$; and if we define the convex
ordering on positive roots corresponding to the reduced
expression by $\lambda_1<\cdots<\lambda_N$, then for any 
$1\le k,l\le N$ we have 
\[
\begin{aligned}
    \dim\Hom_{\K\Gamma}(M(\lambda_k),M(\lambda_l))& = \begin{cases}
        \<\lambda_k,\lambda_l\>,&k\ge l;\\
        0,&k<l;
    \end{cases}\\
    \dim\Ext^1_{\K\Gamma}(M(\lambda_k),M(\lambda_l))& = \begin{cases}
        -\<\lambda_k,\lambda_l\>,&k<l;\\
        0,&k\ge l.
    \end{cases}
\end{aligned}
\]
Here we recall that $M(\lambda_k)$ is the unique indecomposable $\K\Gamma$-module
with dimension vector $\lambda_k$.
In particular $\dim\End_{\K\Gamma}(M(\lambda_k),M(\lambda_k)) = 1$
for all $k$.
We will fix this convex ordering for the rest of the section.

We now have the following proposition:

\begin{proposition}\label{orbit-and-KP}
    Let $\alpha\in Q^+$, and $m,n\in\KP(\alpha)$.
    If the orbits $\OO_m$ and $\OO_n$ satisfy
    $\OO_m\subseteq\overline{\OO_n}$, then $m\ge n$.
\end{proposition}

\begin{proof}
    For any $\K\Gamma$-module $U$, let $T(\alpha,U)$ be the variety
    of all pairs $(x,y)$, where $x\in X(\alpha)$, and $y\in\Hom(U,V(\alpha))$
    is a map of $\K\Gamma$-modules with respect to the $\K\Gamma$-module
    structure $x$ on $V(\alpha)$. There exists a projection 
    \[
        \pi\:T(\alpha,U)\to X(\alpha)
    \] 
    that is $G(\alpha)$-equivariant,
    and for any orbit $\OO_m$ of $X(\alpha)$, the fiber of the map $\pi$
    at any point in $\OO_m$ is the affine space $\Hom_{\K\Gamma}(U,M(m))$.
    Now the map that assigns a point in $X(\alpha)$ with 
    the dimension of the fiber of $\pi$ at that point is $G(\alpha)$-invariant
    and upper semicontinuous, thus if $\OO_m\subseteq\overline{\OO_n}$
    then the the dimension of the fiber of $\pi$ at points in $\OO_m$
    is greater than that at points in $\OO_n$, which shows that 
    \[
        \dim\Hom_{\K\Gamma}(U,M(m))\ge\dim\Hom_{\K\Gamma}(U,M(n)).
    \]
    This inequality is true for all $\K\Gamma$-modules $U$.
    Similarly, we have the inequality
    \[
        \dim\Hom_{\K\Gamma}(M(m),U)\ge\dim\Hom_{\K\Gamma}(M(n),U)
    \]
    for all $\K\Gamma$-modules $U$.

    Now suppose that $m = (m_1,\cdots,m_N)$, $n = (n_1,\cdots,n_N)$ and 
    that $m\ne n$. Let $l$ be the smallest index such that $m_l\ne n_l$.
    Take $U = M(\lambda_l)$. By the previous upper-triangular property, we have
    \[
    \begin{aligned}
        \dim\Hom_{\K\Gamma}(U,M(m))& = m_l+\sum_{1\le k\le l-1}m_l\<\lambda_l,
        \lambda_k\>,\\
        \dim\Hom_{\K\Gamma}(U,M(n))& = n_l+\sum_{1\le k\le l-1}n_l\<\lambda_l,
        \lambda_k\>.\\
    \end{aligned}
    \] 
    Thus we have 
    \[
        m_l-n_l = \dim\Hom_{\K\Gamma}(U,M(m))-\dim\Hom_{\K\Gamma}(U,M(n))\ge 0.
    \]    
    Since $m_l\ne n_l$, we must have $m_l<n_l$,
    therefore $m>'n$. Similarly, $m>''n$, thus $m>n$,
    completing the proof.
\end{proof}

Combining Proposition \ref{comp-of-geo-std} and the above proposition,
we have the following corollary:

\begin{corollary}\label{comp-of-geo-std-2}
    Let $\alpha\in Q^+$, and $m,n\in\KP(\alpha)$.
    If $q^k\widetilde{L}(n)$ is a subquotient of 
    $\widetilde{\std}(m)$ for some $k$, then $n\le m$.
\end{corollary}

We now give the proof of Theorem \ref{alg-geo}.

\begin{proof}[Proof of Theorem \ref{alg-geo}]
    We proceed by the following cases.

    \textbf{Case 1: }$\alpha$ is a root, and $m$ is the minimal element
    $(\alpha)$.

    2) and 3): By the characterization of cuspidal modules in Remark \ref{unique-cuspidal}, it suffices to show that 
    if $\beta+\gamma = \alpha$ and $\Res_{\beta,\gamma}\widetilde{L}(\alpha)
    \ne 0$, then $\beta$ is a sum of roots
    no less than $\alpha$, and $\gamma$ is a sum of roots no greater than
    $\alpha$.

    Now if $\Res_{\beta,\gamma}\widetilde{L}(\alpha)\ne 0$, then there exists
    $m\in\KP(\beta),\ n\in\KP(\gamma)$ such that $$\Hom(\widetilde{P}(m)
    \ex\widetilde{P}(n),\Res_{\beta,\gamma}\widetilde{L}(\alpha))\ne 0,$$
    thus $\Hom(\widetilde{P}(m)
    \circ\widetilde{P}(n),\widetilde{L}(\alpha))\ne 0$,
    which shows that $$\Ext^*_{G(\alpha)}(\IC(\OO_m)\circ\IC(\OO_n),
    \IC(\OO_{\alpha}))\ne 0.$$ This shows that $\IC(\OO_{\alpha})$
    must be a direct summand of $\IC(\OO_m)\circ\IC(\OO_n)$,
    which shows that $\OO_{\alpha}$ must be in the support of
    $\IC(\OO_m)\circ\IC(\OO_n)$. By the definition of the convolution, 
    there exists $m_0\in\KP(\beta),\ n_0\in\KP(\gamma)$
    such that there exists an exact sequence $$0\to M(m_0)
    \to M(\alpha)\to M(n_0)\to 0.$$ Now by the upper-triangular property
    of the dimension of the hom-spaces between indecomposables modules
    of $\K\Gamma$, all roots in the support of $m_0$ is no less than $\alpha$,
    and all roots in the support of $n_0$ is no greater than $\alpha$.
    Therefore $\beta$ is a sum of roots
    no less than $\alpha$, and $\gamma$ is a sum of roots no greater than
    $\alpha$, completing the proof.

    1): The natural map $\IC(\OO_{\alpha})\to\costd(\OO_{\alpha})$
    induces a map 
    \[
        \widetilde{P}(\alpha) = \mu(\IC(\OO_{\alpha}))
        \to\mu(\costd(\OO_{\alpha})) = \widetilde{\std}(\alpha).\
    \]
    Now we have that 
    \[
    \begin{aligned}
        \widetilde{P}(\alpha)
        & = \Ext^*_{G(\alpha)}(\LL(\alpha),\underline{\K}[d_{\alpha}])\\
        & = H^*_{G(\alpha)}(X(\alpha),\LL(\alpha))[-d_{\alpha}]\\
        & = (V(\OO_\alpha)\ox H^*_{G(\alpha)}(X(\alpha),\underline{\K}))\\
        &\qquad\oplus\bigoplus_{m\in\KP(\alpha),\ m\ne(\alpha)}
        (V(\OO_m)\ox H^*_{G(\alpha)}(X(\alpha),\IC(\OO_m)))[-d_\alpha],
    \end{aligned}
    \] 
    and 
    \[
    \begin{aligned}
        \widetilde{\std}(\alpha)
        & = \Ext^*_{G(\alpha)}(\LL(\alpha),(j_\alpha)_*\underline{\K}[d_{\alpha}])\\
        & = \Ext^*_{G(\alpha)}((j_\alpha)^*\LL(\alpha),\underline{\K}[d_{\alpha}])\\
        & = H^*_{G(\alpha)}(\OO_\alpha,(j_\alpha)^*\LL(\alpha))[-d_{\alpha}]\\
        & = V(\OO_\alpha)\ox H^*_{G(\alpha)}(\OO_\alpha,\underline{\K}).
    \end{aligned}
    \] 
    Under these identifications the map 
    $\widetilde{P}(\alpha)\to\widetilde{\std}(\alpha)$ 
    is identified with the restriction map $H^*_{G(\alpha)}(X(\alpha),\LL(\alpha))
    \to H^*_{G(\alpha)}(\OO_\alpha,(j_\alpha)^*\LL(\alpha))$, which also induces 
    a map 
    \[
    H^*_{G(\alpha)}(X(\alpha),\underline{\K})\to 
    H^*_{G(\alpha)}(\OO_\alpha,\underline{\K}),
    \] The stablizer of
    $\OO_\alpha$ is the diagonal copy of $\mathbb{G}_m$ in $G(\alpha)$,
    and the restriction map is identified with the map
    $H^*_{G(\alpha)}(\pt)\to H^*_{\mathbb{G}_m}(\pt)$, which is surjective.
    Thus the original map $\widetilde{P}(\alpha)\to\widetilde{\std}(\alpha)$ 
    is surjective, therefore $\widetilde{\std}(\alpha)$
    has head $\widetilde{L}(\alpha) = L(\alpha)$.

    In addition, $\widetilde{\std}(\alpha)$ is isomorphic to
    $V(\OO_\alpha)\ox\K[x]$, where $\K[x] = H^*_{\mathbb{G}_m}(\pt)$.

    By Proposition \ref{comp-of-geo-std} and 3), 
    $\widetilde{\std}(\alpha)$ lies in the 
    full subcategory of $\mod(H(\alpha))$ consisting of
    all modules with all irreducible subquotients isomorphic
    to $q^mL(\alpha)$ for some $m$. Since $\std(\alpha)$ 
    is a projective module in this subcategory with head $L(\alpha)$,
    there exists a map $\std(\alpha)\to\widetilde{\std}(\alpha)$,
    which is surjective since it is an isomorphism on the head.
    We extend this map into an exact sequence 
    \[
        0\to K(\alpha)\to \std(\alpha)\to\widetilde{\std}(\alpha)\to 0.
    \]
    Since $\std(\alpha)$ is a projective module in this subcategory,
    we have an exact sequence $$0\to\Hom(\std(\alpha),K(\alpha))
    \to\Hom(\std(\alpha),\std(\alpha))
    \to\Hom(\std(\alpha),\widetilde{\std}(\alpha))\to 0.$$
    By Proposition \ref{root-module}, $\Hom(\std(\alpha),\std(\alpha))
    \cong\K[x]$. On the other hand, since $\widetilde{\std}(\alpha)
    \cong V(\OO_\alpha)\ox\K[x]$, $\Hom(\std(\alpha),\widetilde{\std}(\alpha))$
    is a free $\K[x]$-module. Thus it must be $\K[x]$ and the surjective map
    \[
        \Hom(\std(\alpha),\std(\alpha))
        \to\Hom(\std(\alpha),\widetilde{\std}(\alpha))
    \] 
    must be an isomorphism.
    This shows that $ \Hom(\std(\alpha),K(\alpha)) = 0$.
    Since $\std(\alpha)$ is a projective generator, 
    $K(\alpha) = 0$, which shows that 
    $\std(\alpha)\cong\widetilde{\std}(\alpha)$.

    \textbf{Case 2: }General case.

    1): Suppose $m\in\KP(\alpha)$ has form $(m_1,\cdots,m_N)$
    and $(\lambda_{i_1},\cdots,
    \lambda_{i_l})$. Consider the convolution diagram 
    \[
      \begin{tikzcd}
        X'(\lambda_{i_1},\cdots,\lambda_{i_l})\ar[d, "p_0"']
        \ar[r, "\pi"]&
        X''(\lambda_{i_1},\cdots,\lambda_{i_l})\ar[rd, "q_0"]
        \ar[d, "\pi_0"']\\
        X(\lambda_{i_1})\times\cdots\times X(\lambda_{i_l})&
        X(\lambda_{i_1},\cdots,\lambda_{i_l})\ar[l, "p"']
        \ar[r, "q"]&
        X(\alpha),\\
        \OO_{\lambda_{i_1}}\times\cdots\times\OO_{\lambda_{i_l}}
        \ar[u, hookrightarrow]&
        \OO_{\lambda_{i_1},\cdots,\lambda_{i_l}}
        \ar[l, "p"']\ar[u, hookrightarrow]\ar[ur, "q"']
      \end{tikzcd}
    \] 
    where $\OO_{\lambda_{i_1},\cdots,\lambda_{i_l}}
     = p^{-1}(\OO_{\lambda_{i_1}}\times\cdots\times\OO_{\lambda_{i_l}})$.
    By the lower-triangular property
    of the dimension of the ext-spaces between indecomposables modules
    of $\K\Gamma$, any extension of $M(\lambda_{i_1}),
    \cdots,M(\lambda_{i_l})$ in this order must be the trivial extension.
    Therefore the map $p\:\OO_{\lambda_{i_1},\cdots,\lambda_{i_l}}
    \to\OO_{\lambda_{i_1}}\times\cdots\times\OO_{\lambda_{i_l}}$
    is an isomorphism. On the other hand, the restriction
    of $q_0$ on $\pi_0^{-1}(\OO_{\lambda_{i_1},\cdots,\lambda_{i_l}})$
    has image $\OO_m$, and it is a fiber bundle over $\OO_m$
    with fiber being a product of flag varieties $\mathrm{Fl}(m_1)
    \times\cdots\times\mathrm{Fl}(m_n)$. By the construction of the 
    convolution product, we must have $$\costd(\OO_{\lambda_{i_1}})
    \circ\cdots\circ\costd(\OO_{\lambda_{i_l}}) = [m]!\costd(\OO_m).$$
    Applying $\mu$, we have 
    \[
        \widetilde\std(\lambda_{i_1})
        \circ\cdots\circ\widetilde\std(\lambda_{i_l}) = [m]!\widetilde\std(m).
    \]
    By the previous case, we have $\widetilde\std(\lambda_{i_k})
    \cong\std(\lambda_{i_k})$ for all $k$. By Theorem \ref{std-modules}
    and Proposition \ref{divided-power} which give the construction
    of the algebraic standard modules, we obtain that 
    $\std(m)\cong\widetilde{\std}(m)$.

    2) and 3): By Theorem \ref{proper-standard-modules}
    and Theorem \ref{std-modules}, for any $m\in\KP(\alpha)$, we have
    $[\std(m):L(m)]\ne 0$. Since both $\{L(m)\mid m\in\KP(\alpha)\}$
    and $\{\widetilde{L}(m)\mid m\in\KP(\alpha)\}$ give 
    a set of representatives, each representing an isomorphism class
    up to degree shifts of simple modules of $H(\alpha)$,
    there exists a bijection $\KP(\alpha)\to\KP(\alpha)$,
    $m\mapsto t(m)$ such that $L(m)\cong\widetilde{L}(t(m))$.
    Now $[\widetilde{\std}(m):\widetilde{L}(t(m))]\ne 0$
    for all $m$, by Corollary \ref{comp-of-geo-std-2}, $m\ge t(m)$.
    Since $\KP$ is a finite set, $t$ must be the identity map,
    which shows that $L(m)\cong\widetilde{L}(m)$,
    thus $P(m)\cong\widetilde{P}(m)$.
\end{proof}

\subsection{Polynomial Highest Weight Categories}

In this subsection, we survey on some properties of standard modules.
In particular, we will present that for any $\alpha\in Q^+$,
$\mod(H(\alpha))$ is a polynomial height weight category.
We will mainly follow the arguments given in \cite{BKM}.

We first give the motivation and definition to polynomial height weight categories.

Recall from Chapter 3, for a good stratification
$\mathcal{I}$ of a space $X$, and an open stratum $X_0$,
there exists a gluing diagram of triangulated categories
\[
    \D^b_{\mathcal{I}}(X-X_0)\xrightarrow{i_*}\D^b_{\mathcal{I}}(X)\xrightarrow{j^*}\D^b_{\mathcal{I}}(X_0).
\]
By arguments given in \cite{CPS}, the category $\D^b_{\mathcal{I}}(X)$
has properties similar to that of height weight categories.
However, in our case, for any $\alpha\in Q^+$, the category $\mathbf{Q}(\alpha)$
consists of $G(\alpha)$-equivariant complexes, thus their $G(\alpha)$-equivariant
extension algebras have forms of finite algebras over polynomial rings,
instead of finite dimensional algebras. This leads to the following definition,
which modifies the definition of a highest weight category:

\begin{definition}[{Kleshchev, \cite{Kl15}}]
    Suppose $\mathcal{C}$ is a graded $\K$-linear abelian category
    with a finite set $\{L(\pi)\mid\pi\in\Pi\}$ of representatives
    of all simple objects up to isomorphisms and degree shifts.
    We assume that every simple object $L(\pi)$ has a projective cover
    $P(\pi)$ and for any $M\in\mathcal{C}$ the multiplicity of $L(\pi)$
    in $M$ is a formal Laurent series. Assume that $\Pi$ is a partially
    ordered set. For each $\pi\in\Pi$ let $\mathcal{C}_{\le\pi}$
    be the subcategory of $\mathcal{C}$ consisting of all objects
    whose irreducible subquotients are $q^kL(\sigma)$ for $\sigma\le\pi$
    and $k\in\Z$. Let $\std(\pi)$ be the projective cover of
    $L(\pi)$ in $\mathcal{C}_{\le\pi}$.

    $\mathcal{C}$ is a \term{polynomial highest weight category},
    if for every $\pi\in\Pi$ the following conditions are satisfied:
    \begin{enumerate}[1)]
        \item $P(\pi)$ has a filtration $P(\pi) = P_n\supseteq
        P_{n-1}\supseteq\cdots\supseteq P_0 = 0$ such that 
        $P_n/P_{n-1}\cong\std(\pi)$, and for any $1\le k<n$,
        $P_k/P_{k-1}\cong q^k\std(\sigma)$ for some $k\in\Z$ and $\sigma>\pi$;
        \item $B(\pi) \coloneq  \End(\std(\pi))^{\mathrm{op}}$ is a polynomial ring;
        \item $\std(\pi)$ is a finitely generated free right $B(\pi)$-module.
    \end{enumerate}
\end{definition}

We first compute the head and the endomorphism ring of standard modules.

\begin{lemma}
    Suppose $\alpha$ is a root, $k\ge 1$. Then $\std(\alpha^k)$
    has head $L(\alpha^k)$. In addition $\std(\alpha^k)$ has an exhausive
    filtration $\std(\alpha^k) = V_0\supseteq V_1\supseteq \cdots$
    with $V_0/V_1 = L(\alpha^k)$ and all other subquotients of the form
    $q^{2m}L(\alpha^k)$ with $m\ge 1$.
\end{lemma}

\begin{proof}
    We have 
    \[
    \begin{aligned}
        {[\std(\alpha^k)]}& = [\std(\alpha)^{\circ k}]/[k]!\\
        & = [L(\alpha)^{\circ k}]/([k]!(1-q^2)^n)\\
        & = [L(\alpha^k)]q^{-k(k-1)/2}/([k]!(1-q^2)^n)\\
        & = [L(\alpha^k)]/((1-q^2)(1-q^4)\cdots(1-q^{2k})).
    \end{aligned}
    \]
    To show that $\std(\alpha^k)$ has head $L(\alpha^k)$,
    it suffices to show that $\Dim\Hom(\std(\alpha^k),L(\alpha^k)) = 1$.
    Since $\std(\alpha)^{\ex k}$ has head $L(\alpha)^{\ex k}$,
    by Proposition \ref{res-of-semi-cuspidal}, we have 
    \[
    \begin{aligned}
        \Dim\Hom(\std(\alpha^k),L(\alpha^k))
        & = \Dim\Hom(\std(\alpha)^{\circ k},L(\alpha^k))/[k]!\\
        & = \Dim\Hom(\std(\alpha)^{\ex k},\Res_{\alpha,\cdots,\alpha}
         L(\alpha^k))/[k]!\\
        &\le[\Res_{\alpha,\cdots,\alpha}L(\alpha^k),L(\alpha)^{\ex k}]/[k]!\\
        & = [k]!/[k]! = 1,
    \end{aligned}
    \] 
    On the other hand $\Dim\Hom(\std(\alpha^k),L(\alpha^k))>0$,
    thus $\Dim\Hom(\std(\alpha^k),L(\alpha^k)) = 1$, which shows that
    $\std(\alpha^k)$ has head $L(\alpha^k)$. The exhausive
    filtration follows from the result of the head of $\std(\alpha^k)$
    and the identity in the Grothendieck group.
\end{proof}

\begin{proposition}
    Suppose $\alpha\in Q^+$, $m\in\KP(\alpha)$. Then $\std(m)$
    has head $L(m)$. In addition $\std(m)$ has an exhausive
    filtration $\std(m) = V_0\supseteq V_1\supseteq \cdots$
    with $V_0/V_1 = \pstd(m)$ and all other subquotients of the form
    $q^{2k}\pstd(m)$ with $k\ge 1$.
\end{proposition}

\begin{proof}
    The existence of the filtration is from the filtration
    in the previous lemma and the exactness of the induction.
    From this filtration and Theorem \ref{proper-standard-modules} 2)
    which computes the head of the proper standard modules,
    the only simple module that can appear in the head with nonzero
    multiplicity is $L(m)$. Now 
    \[
    \begin{aligned}
        \Dim\Hom(\std(m),L(m))
        & = \Dim\Hom(\std(\lambda_1^{m_1})\ex\cdots
        \ex\std(\lambda_1N{m_N}),\Res_mL(m))\\
        & = \Dim\Hom(\std(\lambda_1^{m_1})\ex\cdots
        \ex\std(\lambda_1N{m_N}),\\&\hspace{3em}L(\lambda_1^{m_1})\ex\cdots
        \ex L(\lambda_1N{m_N}))\\& = 1,
    \end{aligned}
    \] 
    therefore $\std(m)$ has head $L(m)$.
\end{proof}

\begin{lemma}\label{end-of-divided-power}
    Suppose $\alpha$ is a root, $k\ge 1$. Then 
    $\End\std(\alpha^k)\cong(\K[x_1,\cdots,x_k]^{S_k})^{\mathrm{op}}$.
\end{lemma}

\begin{proof}
    We know that 
    \[
        \End(\std(\alpha)^{\circ k})
        \cong\NH(k)^{\mathrm{op}}.
    \] 
    On the other hand, since 
    $[k]!\std(\alpha^k)\cong\std(\alpha)^{\circ k}$,
    we have $$\End(\std(\alpha)^{\circ k})\cong\operatorname{Mat}
    _{[n]!}(\End(\std(\alpha^k))).$$ Since 
    \[
        \NH(n)\cong\operatorname{Mat}_{[n]!}(\K[x_1,\cdots,x_n]^{S_n}),
    \]
    we conclude that $\End\std(\alpha^k)\cong
    (\K[x_1,\cdots,x_k]^{S_k})^{\mathrm{op}}$.
\end{proof}

\begin{proposition}\label{end-is-polynomial-ring}
    Suppose $\alpha\in Q^+$, $m\in\KP(\alpha)$. Then $\End(\std(m))^{\mathrm{op}}$
    is a polynomial ring.
\end{proposition}

\begin{proof}
    By similar arguments to Lemma \ref{res-of-proper-std},
    $\Res_m(\std(m)) = \std(\lambda_1^{m_1})\ex\cdots\ex\std(\lambda_N^{m_N})$.
    Now 
    \[
    \begin{aligned}
        \End(\std(m))&\cong\Hom(\std(\lambda_1^{m_1})\ex\cdots\ex\std(\lambda_N^{m_N}),
        \Res_m(\std(m)))\\
        &\cong\End(\std(\lambda_1^{m_1})\ex\cdots\ex\std(\lambda_N^{m_N}))\\
        &\cong\bigotimes_{1\le i\le N}\End(\std(\lambda_i^{m_i})).
    \end{aligned}
    \]
    Since each $\End(\std(\lambda_i^{m_i}))^{\mathrm{op}}$ is a polynomial ring,
    so is $\End(\std(m))^{\mathrm{op}}$.
\end{proof}

We next discuss some homological properties of the standard modules.

\begin{proposition}\label{homological-property}
    Suppose $\alpha\in Q^+$.
    \begin{enumerate}[1)]
        \item For any $m\in\KP(\alpha)$ and any $d\ge 1$,
        $\Ext^d(\std(m),V) = 0$ for all $V\in\mod(H(\alpha))$
        with all simple subquotientsof the form $q^kL(n)$
        for $k\in\Z$ and $n\not>m$;
        \item For any $m,n\in\KP(\alpha)$, $d\ge 0$, we have $$
        \Dim\Ext^d(\std(m),\pcostd(n)) = \begin{cases}
            1,&\text{if }d = 0\text{ and }m = n,\\
            0,&\text{otherwise}.
        \end{cases}$$
    \end{enumerate}
\end{proposition}

\begin{proof}
    1): We first show that it is true for $V = \pstd(n)$
    where $n\not>m$. Suppose that $m = (\lambda_{i_1},\cdots,
    \lambda_{i_l})$. Then $$\begin{aligned}
        \Dim\Ext^d(\std(m),\pstd(n))& = \Dim\Ext^d(\std(\lambda_{i_1})
        \circ\cdots\circ\std(\lambda_{i_l}),\pstd(n))/[m]!\\
        & = \Dim\Ext^d(\std(\lambda_{i_1})
        \ex\cdots\ex\std(\lambda_{i_l}),\Res_m'\pstd(n))/[m]!.
    \end{aligned}$$
    By Corollary \ref{res'}, if $n\not\ge m$ then $\Res_m'\pstd(n) = 0$,
    therefore $\Ext^d(\std(m),\pstd(n)) = 0$. If $m = n$ then 
    the above is equal to $$\sum_{d_1+\cdots+d_l = d}\prod_{1\le j\le l}
    \Dim\Ext^{d_j}(\std(\lambda_{i_j}),L(\lambda_{i_j})),$$ 
    which is $0$ by Proposition \ref{root-module} 3).
    This shows that 1) is true in the case that $V = \pstd(n)$.

    By Proposition \ref{ext-of-subquo} and by induction on the 
    ordering of $\KP(\alpha)$, 1) is true for all $V = L(n)$,
    thus is true for all such $V$'s.

    2): $\std(m)$ has head $L(m)$ and all other composition factors
    $q^k L(m')$ where $m'\le m$, where $\pcostd(n)$ has socle $L(n)$
    and all other composition factors $q^k L(n')$ where $n>n'$.
    Therefore $$\Dim\Hom(\std(m),\pcostd(m)) = \Dim\Hom(L(m),L(m)) = 1,$$
    and $\Dim\Hom(\std(m),\pcostd(n)) = 0$ if $m\ne n$. If $d>1$ and $n\not>m$
    then $\Dim\Ext^d(\std(m),\pcostd(n)) = 0$ by 1). If $d>1$ and $n>m$,
    by the coinduction-restriction adjunction, if we set
    $n = (\lambda_{i_1},\cdots,\lambda_{i_l})$, then
    \[
    \begin{aligned}
        \Dim\Ext^d(\std(m),\pcostd(n))& = 
        \Dim\Ext^d(\Res_n'\std(m),L(\lambda_{i_1})\ex\cdots\ex L(\lambda_{i_l})).
    \end{aligned}
    \]
    By Corollary \ref{res'}, $\Res_n'\std(m) = 0$, therefore
    $\Dim\Ext^d(\std(m),\pcostd(n)) = 0$.
\end{proof}

\begin{corollary}
    Suppose that $\alpha\in Q^+$, $m\in\KP(m)$. Then
    $\std(m)$ is the projective cover of $L(m)$ in $\mod(H(\alpha))_{\le m}$.
\end{corollary}

\begin{proof}
    To prove this it suffices to show that:
    \begin{enumerate}[i)]
        \item All composition factors of $\std(m)$ are 
        of the form $q^k L(n)$ where $n\le m$;
        \item If $n\le m$, then $\Dim\Hom(\std(m),L(n)) = \delta_{m,n}$;
        \item If $n\le m$, then $\Dim\Ext^1(\std(m),L(n)) = 0$.
    \end{enumerate}

    For i), by Theorem \ref{proper-standard-modules} 6),
    all composition factors of $\pstd(m)$ are 
    of the form $q^k L(n)$ where $n\le m$; since $[\std(m)]$
    is a multiple of $[\pstd(m)]$, 
    all composition factors of $\std(m)$ must be 
    of the form $q^k L(n)$ where $n\le m$.

    For ii), this follows from the fact that $\std(m)$ has
    head $L(m)$.

    For iii), this follows directly from Proposition
    \ref{homological-property} 1).
\end{proof}

We say a $H(\alpha)$-module $V$ has a \term{standard flag}
if it admits a filtration $V = V_0\supseteq V_1\supseteq\cdots
\supseteq V_n = 0$ such that each $V_{i-1}/V_i$ is of the form
$q^k\std(m)$ for some Kostant partition $m$. By Proposition \ref{homological-property},
there is an equality 
\[
    [V:\std(m)] = \overline{\Dim\Hom(V,\pcostd(m))},
\]
thus the multiplicity $[V:\std(m)]$ is independent of the 
filtration.

Similar to a well-known result in the case of quasi-hereditary algebras (see
\cite[Proposition A2.2]{Do}), we have the following proposition:

\begin{proposition}\cite[Theorem 3.13]{BKM}
    Suppose $V\in\mod(H(\alpha))$ satisfies that $$\Ext^1(V,\pcostd(m)) = 0$$
    for all $m\in\KP(\alpha)$. Then $V$ has a standard flag.
\end{proposition}

As a result, we have the ``BGG-reciprocity'':

\begin{corollary}
    For any $m\in\KP(\alpha)$ the projective module $P(m)$ has a 
    standard flag, and $[P(m):\std(n)] = [\pstd(n):L(m)]$ for all $m,n$.
\end{corollary}

\begin{proof}
    Since $\Ext^1(P(m),\pcostd(n)) = 0$ for all $m,n$, $P(m)$
    has a standard flag. Moreover 
    \[
        [P(m):\std(n)]
         = \overline{\Dim\Hom(P(m),\pcostd(n))} = \overline{[\pcostd(n):L(m)]}
         = [\pstd(n):L(m)],
    \]
    completing the proof.
\end{proof}

We finally state the following theorem:

\begin{theorem}\cite[Theorem 6.11]{Mc2}\label{p-h-w-c}
    For any $\alpha\in Q^+$, the category $\mod(H(\alpha))$ is a 
    polynomial highest weight category.
\end{theorem}

\begin{proof}
    We need to verify the properties 1), 2) and 3).

    1): By the previous corollary, $P(m)$ has a standard flag
    $P(m) = P_l\supseteq P_{l-1}\supseteq P_0 = 0$. By the 
    reciprocity formula, for any $n\not\ge m$, we have 
    $[P(m):\std(n)] = [\pstd(n):L(m)] = 0$, and $[P(m):\std(m)] = 1$.
    Now let $\std_k \coloneq  P_k/P_{k-1}$. Then each $\std_k$ has form
    $q^j\std(n)$ where $n\ge m$. Now $P_l$ has head $L(m)$,
    thus $\std_l = \std(m)$. Therefore all other $\std_k$'s
    must have form $q^j\std(n)$ where $n>m$.

    2): This follows from Proposition \ref{end-is-polynomial-ring}.

    3): We first show that it is true for $m = (\lambda^k)$ where
    $\lambda$ is a root. In this case $m$ is the minimal element
    in $\KP(k\lambda)$, hence $\OO_m$ is the largest orbit in 
    $X(k\lambda)$. Therefore by the proof of Theorem \ref{alg-geo},
    \[
        \std(m)\cong V(\OO_m)\ox H^*_{G(k\lambda)}(\OO_m,\underline{\K}).
    \]
    The stablizer of $\OO_m$ is isomorphic to $\mathrm{GL}(k)$,
    thus $\std(m)\cong V(\OO_m)\ox H^*_{\mathrm{GL}(k)}(\pt)$.
    On the other hand, by Lemma \ref{end-of-divided-power},
    $\End(\std(m))^{\mathrm{op}}\cong\K[x_1,\cdots,x_k]^{S_k}
    \cong H^*_{\mathrm{GL}(k)}(\pt)$. Thus $\std(m)$ is a 
    free right $\End(\std(m))^{\mathrm{op}}$-module,
    of finite rank $\dim V(\OO_m)$.

    We now return to the general case. In this case there exists 
    a finite set $\{\psi_w\}$ of elements in $H(\alpha)$,
    such that 
    \[
        \std(m)\cong\bigoplus\psi_w\ox(\std(\lambda_1^{m_1})
        \ex\cdots\ex\std(\lambda_N^{m_N})).
    \] 
    Moreover for any $\psi_w$,
    every element in $\End(\std(m))\cong\bigotimes_i\End(\std(\lambda_i^{m_i}))$
    fixes the direct summand $\psi_w\ox(\std(\lambda_1^{m_1})\ex\cdots\ex\std(\lambda_N^{m_N}))$.
    Since each $\std(\lambda_i^{m_i})$ is finite free over 
    $\End(\std(\lambda_i^{m_i}))$, each 
    $\psi_w\ox(\std(\lambda_1^{m_1})\ex\cdots\ex\std(\lambda_N^{m_N}))$
    is finite free over $\bigotimes_i\End(\std(\lambda_i^{m_i}))$.
    Summing over $w$, $\std(m)$ is finite free over $\End(\std(m))$.
\end{proof}
