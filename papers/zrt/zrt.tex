% !TeX encoding = UTF-8
% !TeX program = xelatex
% !TeX spellcheck = en_US

\documentclass[degree=bachelor, language=english]{thuthesis}
  % 学位 degree:
  %   doctor | master | bachelor | postdoc
  % 学位类型 degree-type:
  %   academic(默认)| professional
  % 语言 language
  %   chinese(默认)| english
  % 字体库 fontset
  %   windows | mac | fandol | ubuntu
  % 建议终版使用 Windows 平台的字体编译


% 论文基本配置,加载宏包等全局配置
% !TeX root = ./thuthesis-example.tex

% 论文基本信息配置

\thusetup{
  %******************************
  % 注意:
  %   1. 配置里面不要出现空行
  %   2. 不需要的配置信息可以删除
  %   3. 建议先阅读文档中所有关于选项的说明
  %******************************
  %
  % 输出格式
  %   选择打印版(print)或用于提交的电子版(electronic),前者会插入空白页以便直接双面打印
  %
  output = electronic,
  %
  % 标题
  %   可使用“\\”命令手动控制换行
  %
  title  = {一类带椭圆纤维化的K3曲面上的里奇平坦度量},
  title* = {Ricci-Flat Metrics on Some K3\\ Surfaces Admitting Elliptic Fibrations},
  %
  % 学位
  %   1. 学术型
  %      - 中文
  %        需注明所属的学科门类,例如:
  %        哲学、经济学、法学、教育学、文学、历史学、理学、工学、农学、医学、
  %        军事学、管理学、艺术学
  %      - 英文
  %        博士:Doctor of Philosophy
  %        硕士:
  %          哲学、文学、历史学、法学、教育学、艺术学门类,公共管理学科
  %          填写“Master of Arts“,其它填写“Master of Science”
  %   2. 专业型
  %      直接填写专业学位的名称,例如:
  %      教育博士、工程硕士等
  %      Doctor of Education, Master of Engineering
  %   3. 本科生不需要填写
  %
  degree-name  = {工学硕士},
  degree-name* = {Master of Science},
  %
  % 培养单位
  %   填写所属院系的全名
  %
  department = {数学科学系},
  %
  % 学科
  %   1. 学术型学位
  %      获得一级学科授权的学科填写一级学科名称,其他填写二级学科名称
  %   2. 工程硕士
  %      工程领域名称
  %   3. 其他专业型学位
  %      不填写此项
  %   4. 本科生填写专业名称,第二学位论文需标注“(第二学位)”
  %
  discipline  = {数学与应用数学},
  discipline* = {Computer Science and Technology},
  %
  % 姓名
  %
  author  = {张睿桐},
  author* = {Zhang Ruitong},
  %
  % 指导教师
  %   中文姓名和职称之间以英文逗号“,”分开,下同
  %
  supervisor  = {周杰, 助理教授},
  supervisor* = {Assistant Professor Zhou Jie},
  %
  % 副指导教师
  %
  %associate-supervisor  = {李思, 教授},
  %associate-supervisor* = {Professor Li Si},
  %
  % 联合指导教师
  %
   co-supervisor  = {李思, 教授},
   co-supervisor* = {Professor Li Si},
  %
  % 日期
  %   使用 ISO 格式;默认为当前时间
  %
   date = {2021-06-13},
  %
  % 是否在中文封面后的空白页生成书脊(默认 false)
  %
  include-spine = false,
  %
  % 密级和年限
  %   秘密, 机密, 绝密
  %
  % secret-level = {秘密},
  % secret-year  = {10},
  %
  % 博士后专有部分
  %
  % clc                = {分类号},
  % udc                = {UDC},
  % id                 = {编号},
  % discipline-level-1 = {计算机科学与技术},  % 流动站(一级学科)名称
  % discipline-level-2 = {系统结构},          % 专业(二级学科)名称
  % start-date         = {2011-07-01},        % 研究工作起始时间
}

% 载入所需的宏包

% 定理类环境宏包
\usepackage{amsthm}

% 也可以使用 ntheorem
% \usepackage[amsmath,thmmarks,hyperref]{ntheorem}

\thusetup{
  %
  % 数学字体
  % math-style = GB,  % GB | ISO | TeX
  math-font  = xits,  % sitx | xits | libertinus
}

% 可以使用 nomencl 生成符号和缩略语说明
% \usepackage{nomencl}
% \makenomenclature

% 表格加脚注
\usepackage{threeparttable}

% 表格中支持跨行
\usepackage{multirow}

% 固定宽度的表格。
% \usepackage{tabularx}

% 跨页表格
\usepackage{longtable}

% 算法
\usepackage{algorithm}
\usepackage{algorithmic}

% 量和单位
\usepackage{siunitx}

% 参考文献使用 BibTeX + natbib 宏包
% 顺序编码制
\usepackage[sort]{natbib}
\bibliographystyle{thuthesis-numeric}

% 著者-出版年制
% \usepackage{natbib}
% \bibliographystyle{thuthesis-author-year}

% 本科生参考文献的著录格式
% \usepackage[sort]{natbib}
% \bibliographystyle{thuthesis-bachelor}

% 参考文献使用 BibLaTeX 宏包
% \usepackage[backend=biber,style=thuthesis-numeric]{biblatex}
% \usepackage[backend=biber,style=thuthesis-author-year]{biblatex}
% \usepackage[backend=biber,style=apa]{biblatex}
% \usepackage[backend=biber,style=mla-new]{biblatex}
% 声明 BibLaTeX 的数据库
% \addbibresource{ref/refs.bib}

% 定义所有的图片文件在 figures 子目录下
\graphicspath{{figures/}}

% 数学命令
\makeatletter
\newcommand\dif{%  % 微分符号
  \mathop{}\!%
  \ifthu@math@style@TeX
    d%
  \else
    \mathrm{d}%
  \fi
}
\newtheorem{defi}{Definition}[section]
\newtheorem{exa}{Example}[section]
\newtheorem{lemm}{Lemma}[section]
\newtheorem{theo}[lemm]{Theorem}
\newtheorem{coro}[lemm]{Corollary}
\newtheorem{prop}[lemm]{Proposition}
\newtheorem{rema}{Remark}[section]
\def\bo{\mathscr{O}}
\def\bl{\mathcal{L}}
\newcommand{\cp}[1]{\mathbb{C}P^{#1}}
\makeatother

% hyperref 宏包在最后调用
\usepackage{hyperref}



\begin{document}

% 封面
\maketitle

% 学位论文指导小组、公开评阅人和答辩委员会名单

% 使用授权的说明
%\copyrightpage
% 将签字扫描后授权文件 scan-copyright.pdf 替换原始页面
\copyrightpage[file=scan-copyright.pdf]

\frontmatter
% !TeX root = ../thuthesis-example.tex

\begin{abstract}
The explicit construction of Ricci-flat metric on Calabi--Yau manifolds has been an important question long since. On some special Calabi--Yau manifolds, asymptotics using semi-flat metrics have some well results. We'll introduce some results of Hein on constructing and estimating semi-flat metrics on rational elliptic surfaces in this paper. The main reference is Hein's paper \cite{hein2012gravitational}.

  \bigskip
  \noindent
  \textbf{Keywords:} Calabi--Yau manifolds; semi-flat metric; rational elliptic surface
\end{abstract}


% 目录
\tableofcontents

% 插图和附表清单
% 本科生的插图索引和表格索引需要移至正文之后、参考文献前
%\listoffiguresandtables  % 插图和附表清单
% \listoffigures           % 插图清单
% \listoftables            % 附表清单

% 符号对照表
%\input{data/denotation}


% 正文部分
\mainmatter
% !TeX root = ../thuthesis-example.tex

\chapter{Introduction}
Since Yau proved\cite{yau1978ricci} Calabi conjecture, it has always been an important question on constructing the explicit Calabi--Yau metrics. In 2000, Mark Gross and P.M.H.Wilson construct\cite{gross2000large} an asymptotic to Ricci-flat metric on elliptic $K3$ surface, they construct an explicit metric on the smooth part of the fibration, and gluing it with suitable Ouguri--Vafa metric\cite{ooguri1996summing} near the singular fiber. If the elliptic fibers are of area $\epsilon>0$, then the correction term of the approximation to the real Ricci-flat metric is $O(e^{-C/\epsilon})$ for some constant $C>0$.\\ \indent
In various asymptotics towards the goal, ALG and ALH space seems a good attempt. The ALG spaces are asymptotic to the twisted products of flat cones or flat tori, and for ALH spaces, the constructed metrics have an isometrically split cross section $S^1\times T^2$.\\ \indent
Hein gives\cite{hein2012gravitational} some new families of gravitational instantons which helps to the issues of these spaces. In his paper, he describes sets of ALG and ALH spaces, and discuss the volume growth and injective radius decay of the metrics. And this paper is aimed to understand the construction of his metric, in order to better understand how the approximation works.\\ \indent
The main thought is as follows: for a rational elliptic surface, we first classifies its singular fibers with the help of Kodaira's work, and construct the semi-flat metric near the fiber. We require the metric has a proper expression so that we can easily discuss its geometric properties like whether they satisfies the SOB and CYL condition. After that, we try to glue this metric with the original one through complicated Sobolev inequalities, and finally deduce some existence result.\\ \indent
Chapter 2 reviews the basic properties of Weirstrass model, since it is birational to the general elliptic fibration case. Chapter 3 discuss the metric on complex torus bundle, since we want to have a look at the  local behavior on smooth part of the fibration. Chapter 4 then gives the description near singular fibers, and in chapter 5, we construct the semi-flat metric near singular fibers and see the aymptotic behaviors, finally we give an existence result for certain Calabi--Yau metric.



% !TeX root = ../Lie.tex

Now start from a convergent power series $f\in \m \subset \Cc\{\xn\}=\Cc\{x_1,\cdots, x_n\}$, where $\m = \lan x_1,\cdots, x_n\ran$ is the unique maximal ideal of $\Cc\{x_1,\cdots,x_n\}$, the local ring of convergent power series of $n$ variables, which can also be regarded as the coordinate ring of the complex space germ $(\Cc^n,\0)$. Then the origin of $\Cc^n$ lies in the hypersurface $\{ f = 0\} \subset \Cc^n$.

\subsection{Isolated hypersurface singularities} Let $V=V(f)$ denote the germ of $\{f=0\}$ at the origin of $\Cc^n$. Since the singular locus of $V$ is exactly where the gradient of $f$ vanishes, we say $V$ is a germ of isolated hypersurface singularity if the origin $\0\in \Cc^n$  is an isolated point of 
$$\left\{f=0,\pd f{x_1}=0,\cdots, \pd f{x_n}=0\right\}.$$
In this case we also say $f$ defines a isolated hypersurface singularity $V$.

\subsection{Moduli algebra}\label{sec-2.2}
For $f\in \m$, we introduce its Jacobian ideal to be 
\[j(f)\coloneq \lan \pd f{x_1},\cdots, \pd f {x_n}\ran.\]
and its Tjurina algebra to be 
\[A(f)\coloneq  \Cc\{\xn\} / \lan f, j(f)\ran.\]
According to lemma 2.3 of \cite[113]{GL}, $A(f)$ is of finite dimension iff $f$ defines an isolated hypersurface singularity. This algebra is also said to be  moduli algebra of $V=V(f)$. The well-known Mather--Yau theorem states that 

\begin{theorem}[\cite{MY}]\label{MY}
  For  isolated hypersurface singularities $V(f_1), V(f_2)$ defined by $f_1,f_2\in \m\subset \Cc\{\xn\}$, they are isomorphic as complex space germs if and only if their moduli algebras $A(f_1)$ and $A(f_2)$ are isomorphic as $\Cc$-algebras.
\end{theorem}
Consequently, the moduli algebra $A(f)$ is a complete invariant of the isolated hypersurface singularities $V(f)$. 


\subsection{Yau algebra}\label{sec-2.3}
Suppose we have a homomorphism of $\Cc$-algebras  $A\to B$, which is equivalent to say that $B$ is an $A$-algebra.  Denote $\Endo_A B$ the endomorphisms of $B$ as an $A$-module, then we can define the $A$-derivation algebra of $B$ to be the submodule
\[\Der_A B=\{ \delta \in \Endo_A B\,\mid \delta(fg)= \delta(f)g+f\delta(g), \forall f,g \in B\},\]
whose Lie bracket is the commutator of linear endomorphisms. We would consider  the case $A=\Cc$ before section \ref{sec4}, and then the case $A=\Cc\{t\}$ or $\Cc\{\bt\}=\Cc\{t_1,\cdots, t_m\}$ will also be taken into consideration.
\begin{remark}
  To obtain
  \[\Der_\Cc \Cc\{\xn\} = \bigoplus_{i=1}^n\Cc\{\xn\} \pd{}{x_i},\]
  one may need moreover that the derivation over $\Cc\{\xn\}$ is continuous with respect to the $\m$-adic topology. However, this doesn't make any difference when $B$ is also a  finite generated module over $A$. This is the most case we consider. 
\end{remark}

The Tjurina algebra $A(f)$ for $f\in \m$ is a $\Cc$-algebra, so we now can introduce its derivation Lie algebra as
\[L(f)\coloneq  \Der_\Cc A(f).\]
Motivated by Theorem \ref{MY}, we can ask whether  $L(f)$ is also a complete invariant of the isolated hypersurface singularities $V(f)$.	

\subsection{New Lie algebra}\label{sec-2.4}
In \cite{BN}, a new complete invariant of quasi-homogeneous isolated hypersurface singularities $V=V(f)$ was defined as
\[A^*(f)\coloneq \Cc\{\xn\}/\lan f, j(f), H(f)\ran,\]
where
\[H(f)\coloneq \det  \left( \frac{\partial^2 f}{\partial x_i \partial x_j}\right)_{1\leqslant i,j \leqslant n } \]
is the Hessian of $f$. Consequently, another Lie algebra $L^*(f)\coloneq \Der_\Cc A^*(f)$ called new Lie algebras in \cite{BN} arises from the singularity defined by $f$, and a question similar to $L(f)$ can be proposed.

Before the concrete computation in section \ref{sec3}, we present the following results about the grading structure of the above algebras, which can be found in \cite{BN}:	
\begin{proposition}[\cite{BN}]\label{res}
  Let $f\in \Cc\{\xn\}$ be a homogeneous polynomial of degree $d\geqslant 2$ defining an isolated hypersurface singularity in $( \Cc^n, \0)$.
  Note that there are natural grading which can be denoted to be 
  \begin{align*}
    A(f)=\bigoplus_{k=0}^DA(f)_k,&
    A^*(f)=\bigoplus_{k=0}^{D^*}A^*(f)_k.
  \end{align*}
  \item[(1)](Theorem 2.3, Proposition 2.1)  The degree-$D$ part of $A(f)$ is $1$-dimensional with a basis represented by $H(f)$. In particular, $D\coloneq  \deg  H(f) = n(d-2)$ . Moreover, the total dimension of $A(f)$ is  $ (d-1)^n.$
  \item[(2)](Theorem 2.3) The natural projection $A(f) \to A^*(f)$ induces isomorphisms of $\Cc$-linear spaces $A(f)_k\stackrel \sim \to A^*(f)_k, 0 \leqslant k \leqslant D-1$. In particular, $D^*=D-1$. Moreover, the total dimension of $A^*(f)$ is  $ (d-1)^n-1.$
  \item[(3)](Corollary 2.1) $L(f)$  is solvable, while $L^*(f)$ is solvable given $d\geqslant 4$.
  \item[(4)](Remark 6.1) $L(f)$ and $L^*(f)$ are naturally graded with no negative part and share the same total dimension.
  \end{proposition}
  We also denote $\Cc\{\xn\}_k$ to be  the degree $k$-part of $\Cc\{\xn\}$, i.e., the vector space consists of homogeneous polynomials of degree $k$, then we have $\forall K \geqslant 0$, 
  \[\Cc\{\xn\} = \left( \bigoplus_{k=0}^K \Cc\{\xn\}_k \right)\oplus \m^{K+1}\]
  as $\Cc$-vector space.

  %	\[
  %	\Cc\{\xn\}=\overline{\bigoplus_{k\geqslant 0}\Cc\{\xn\}_k},\]
  %	where the closure is taken with respect to the $\m$-adic topology of $\Cc\{\xn\}$. 

% !TeX root = ../thuthesis-example.tex

\chapter{Metric on complex torus bundle}
In this chapter, we discuss some important notions to express the local model of an elliptic fibration explicitly. In this process, we need to choose local trivialization properly and consider the monodromy. We then give a explicit formula of a semi-flat metric using the coordinates obtained by period map.
\section{Local model}
In this section, we discuss the local model, and explain how a metric will be constructed.
\begin{defi}
For an elliptic fibration, we call a metric semi-flat
if it restricts to a flat metric on each fiber.
\end{defi}
Let $f\colon X\rightarrow \mathbb{P}^1$ be a holomorphic submersion such that all fibers $X_p=f^{-1}(p)$ are complex torus, and suppose $f$ admits a holomorphic section $\sigma$ with a 2
-form $\omega$, which restricts to a K{\"a}hler form on each fiber $X_p$ and there exists $c_i\in\mathbb{R}$ and $\xi_i(p)\in H^2(X_p,\mathbb{Z})$ such that
$$[\omega|_{X_p}]=\sum_i c_i\xi_i(p),\qquad  \forall p\in\mathbb{P}^1 \;.$$
\indent For $\epsilon>0$ and $p\in\mathbb{P}^1$, from Yau's result\cite{yau1978ricci}, there exist a unique Ricci-flat K{\"a}hler metric $g_{p,\epsilon}$ with $\textrm{Vol}(X_p,g_{p,\epsilon})=\epsilon$ whose K{\"a}hler class is $[\omega|_{X_p}]$. Restricting $g_{p,\epsilon}$ to $T_{\sigma(p)}X_p$ then induces a Hermitian fiber metric $h_\epsilon$ on the holomorphic vector bundle $E:=\sigma^*T_{X/\mathbb{P}^1}$ over $\mathbb{P}^1$.\\ \indent
Suppose $X$ admits a holomorphic volume form $\Omega$. There exists a unique Riemannian metric $g_{\mathbb{P}^1,\epsilon}$, compatible with the complex structure of $\mathbb{P}^1$, and the faithful pairing 
$$E\otimes T^{1,0}\mathbb{P}^1\rightarrow \mathbb{C}$$
induced by $h_\epsilon$ and $g_{\mathbb{P}^1,\epsilon}$ is isometric to the one induced by $\Omega$.\\ \indent
We can choose a fibre-preserving biholomorphism $X\simeq E/\Lambda$ for a holomorphic lattice bundle $\Lambda\subset E$. Thus $\Lambda$ induces a flat $\mathbb{R}$-linear connection on $E$, thus we can define an integrable horizontal distribution $\mathscr{H}$ on $X$. Then define
$$g_{\textrm{sf},\epsilon}(u,v)=g_{p,\epsilon}(Pu,Pv)+g_{\mathbb{P}^1,\epsilon}(df(u),df(v))$$
for $u,v\in T_xX$, where $P$ is the projection along $\mathscr{H}_x$.\\ \indent
Now we see the metric locally, thus we choose a local trivialization on an open subset $U$ of $\mathbb{P}^1$, thus we have
$$X|_U\simeq E|_U/\Lambda \simeq(U\times \mathbb{C})/\Lambda \; ,$$
denote the coordinate $z$ on $U$ and $w$ on the $\mathbb{C}$.
Fix an oriented basis $\{\tau_1,\tau_2\}$ for $\Lambda$ at a generic point, and extend it to $\tau_i\in \mathscr{O}(U,\mathbb{C})$ that generate the lattice everywhere and let $(\xi^1,\xi^2)$ be $\mathbb{R}$-dual to $(\tau_1,\tau_2)$. The assumption we made before means the existence of a matrix $Q\in M_{2\times2}(\mathbb{R})$, $Q+Q^T=0$, such that
$$\omega=\frac{1}{2}Q_{ij}\xi^i\wedge\xi^j$$
restricts to K{\"a}hler metric on each fiber. Define $T\in \mathscr{O}(U,M_{1\times2}(\mathbb{C}))$ by
$$\tau_1=T_{11}\qquad \tau_2=T_{12}\; .$$
\indent Consider the monodromy, we have a matrix $A\in GL(2,\mathbb{R})$, then $A^TQA=Q$. There exists $S\in GL(2,\mathbb{R})$, unique up to right multiplication by a matrix in $Sp(2,\mathbb{R})$ such that $$S^TQS=\begin{pmatrix}
0 & 1\\
-1 & 0
\end{pmatrix}\; .$$
\\ \indent We thus write $TS=(R,RZ)$ with multivalued holomorphic maps $R:U\rightarrow \mathbb{C}$ and $Z:U\rightarrow \mathbb{C}$. By Griffiths--Harris\cite{griffiths1978principles}, $\omega$ being positive $(1,1)$ is equivalent to the image of $Z$ is contained in upper-half plane. And we have
$$\omega=iH dw\wedge d\bar{w}\; ,$$
and
$$H^{-1}:=2|R|^2\textrm{Im}  Z=i\bar{T}Q^{-1}T\; .$$
\indent Here $Z$ is actually the so called period map, although actually is not well-defined since $S$ is only unique up to multiplication by matrix in $\textrm{Sp}(2,\mathbb{R})$. The flat connection induced before has Christoffel symbol
$$\Gamma(z,w)=\frac{\partial T}{\partial z}\begin{pmatrix}
T \\
\bar{T}
\end{pmatrix}^{-1}\begin{pmatrix}
w\\
\bar{w}
\end{pmatrix}\in\mathbb{C}\; .$$
\\ \indent


\section{Construction of metric}
In this section, we give explicit formula for a semi-flat metric consturcted by Hein\cite{hein2012gravitational}.
\begin{lemm}
For $H$ as defined above and $\epsilon>0$, denote $\displaystyle H(\epsilon)=\frac{\epsilon}{\sqrt{\det Q}}H$, if $\Omega=gdz\wedge dw$ where $g$ is holomorphic, then the K{\"a}hler form of $g_{\textrm{sf},\epsilon}$ is
$$\omega_{\textrm{sf},\epsilon}=i|g|^2H(\epsilon)^{-1}dz\wedge d\bar{z}+iH(\epsilon)(dw-\Gamma dz)\wedge(d\bar{w}-\bar{\Gamma}d\bar{z})\; .$$
It is a closed form whose top power is $2\Omega\wedge\bar{\Omega}$, so is Calabi--Yau.
\end{lemm}
\begin{lemm}
Using the above notations, the metric $g_{\mathbb{P}^1,\epsilon}$ on the base are K{\"a}hler, and has Ricci form 
$$\rho(\omega_{\mathbb{P}^1,\epsilon})=-i\partial\bar{\partial}\log(\textrm{Im} Z)=\frac{i dZ\wedge d\overline{Z}}{4(\textrm{Im} Z)^2}\; .$$
\end{lemm}
\begin{proof}
First notice that $$|R|^2\textrm{Im} Z=\frac{1}{H}\; ,$$and notice that Ricci form is invariant under scalar multiplication on metric. Then denote $$f=\textrm{Im} Z=\frac{1}{2i}(Z-\bar{Z})\; .$$
\indent Then we have
\begin{displaymath}
\begin{split}
\partial\bar{\partial}\log(\textrm{Im} Z)&=\partial\left(\frac{f_{\bar{z}}}{f}d\bar{z}\right)\\
&=\frac{ff_{z\bar{z}}-f_z f_{\bar{z}}}{f^2}dz\wedge d\bar{z}\\
&=\frac{Z_z\bar{Z}_{\bar{z}}}{(Z-\bar{Z})^2}dz\wedge d\bar{z}\\
&=\frac{-|Z_z|^2}{4(\textrm{Im} Z)^2}dz\wedge d\bar{z}\; .
\end{split}
\end{displaymath}
\indent And notice that 
$$dZ\wedge d\bar{Z}=Z_zdz\wedge\overline{Z_z}d\bar{z}=|Z_z|^2dz\wedge d\bar{z}\; .$$
\indent Hence complete the proof.
\end{proof}
We summarize the above conclusions and give an explicit formula as in Gross--Wilson\cite{gross2000large}, but using the notions in Hein\cite{hein2012gravitational}.
\begin{prop}
Let $f\colon X\rightarrow S$ an Weierstrass fibration over a Riemann surface, where $\sigma$ is the given holomorphic section and $f$ has no singular fiber. Suppose $\Omega$ be a holomorphic symplectic form on $X$. Let $\omega_{\textrm{sf},\epsilon}$ be semi-flat K{\"a}hler, constructed from $\sigma,\Omega/\sqrt{2}$, such that the area of the fibers of $f$ are $\epsilon$. In particular, $\omega_{\textrm{sf},\epsilon}^2=\Omega\wedge\bar{\Omega}\; .$\\ \indent
Let $U$ be a domain in $S$, $z$ be the holomorphic coordinate on $U$, and fix a local trivialization $$X|U\simeq (U\times \mathbb{C}_w)/(\mathbb{Z}\tau_1+\mathbb{Z}\tau_2)\; ,$$ 
with multivalued functions $\tau_1,\tau_2$ which the pair $(\tau_1,\tau_2)$ is positively oriented and $\sigma$ corresponds to the zero section. Then $\Omega=gdz\wedge dw$ such that $g:U\rightarrow\mathbb{C}$ is holomorphic and
$$\omega_{\textrm{sf},\epsilon}=i|g|^2\frac{\textrm{Im} (\bar{\tau_1}\tau_2)}{\epsilon}dz\wedge d\bar{z}+\frac{i\epsilon}{2\textrm{Im} (\bar{\tau_1}\tau_2)}(dw-\Gamma dz)\wedge(d\bar{w}-\bar{\Gamma}d\bar{z})\; ,$$
where
$$\Gamma(z,w)=\frac{1}{\textrm{Im} (\bar{\tau_1}\tau_2)}\left(\textrm{Im} (\bar{\tau_1}w)\frac{d\tau_2}{dz}-\textrm{Im} (\bar{\tau_2}w)\frac{d\tau_1}{dz}\right)\; .$$
\indent On $U$, the Ricci form of the induced metric $\omega_{S,\epsilon}$ on the base is given by
$$\rho(\omega_{S,\epsilon})=-i\partial\bar{\partial}\log\textrm{Im} (\tau)=\frac{id\tau\wedge d\bar{\tau}}{4\textrm{Im} (\tau)^2}\; ,$$
the pullback of hyperbolic metric on the upper half-plane $\mathfrak{H}=\{\textrm{Im} (\tau)>0\}$ whose Gaussian curvature is $-2$  under the period map $\tau=\frac{\tau_2}{\tau_1}\colon U\rightarrow \mathfrak{H}$.
\end{prop}
% !TeX root = ../thuthesis-example.tex

\section{Singular fibers in elliptic fibration}
In this section, we mainly focus on the rational elliptic surface case which we have mentioned in section 1. Thanks to Kodaira's work \cite{kodaira1963compact}, there are only finite type of singular fibers on rational elliptic surface.
\subsection{Rational Elliptic surface}
\begin{definition}
A rational elliptic surface is the blowup of $\mathbb{P}^2$ in the base points arise from  a pencil of cubics, i.e.\ a family $sF+tG=0$ where $[s,t]\in\mathbb{P}^1$ and $F,G$ are smooth cubics intersecting in 9 points with multiplicity. Blowing up at these points, we thus have an elliptic fibration $f\colon X\rightarrow\mathbb{P}^1$.
\end{definition}
Since each blow-up will only change the Hodge number $h^{p,q}$ when $p=q=1$ \cite{rao2019dolbeault}, thus the rational elliptic surface's Hodge diamond will look like
\begin{displaymath}
\begin{split}
&\qquad  \qquad 1\\
&\qquad 0 \qquad \qquad 0\\
&0 \qquad \qquad \!\!\! 10 \qquad \qquad \!\! 0\\
& \qquad 0 \qquad \qquad 0\\
& \qquad \qquad 1 \qquad \qquad .
\end{split}
\end{displaymath}
\indent Actually, from \cite{schutt2019elliptic} we know that we can obtain an elliptic $K3$ surfaces $X$ from a given rational elliptic surface $S$ through quadratic base change over $\mathbb{P}^1$. Or see how $K3$ surface are constructed from elliptic surface in Garbagnati and Salgado's paper \cite{garbagnati2019linear}. Here we only need to avoid ramification at non-reduced fibers.\\ \indent
Thus now we turn ourselves to rational elliptic surfaces, especially the local model near singular fiber.\\ \indent
Let $f\colon U\rightarrow \Delta$ be an elliptic fibration over the disk with all fibers regular except possibly for $D=f^{-1}(0)$. Assume that $D$ does not contain $(-1)$-curves, and we also assume that $D$ is reduced or equivalently, there is a holomorphic section correspond to $f$, or equivalently $U|_{\Delta^*}\simeq(\Delta^*\times\mathbb{C})/(\mathbb{Z}\tau_1+\mathbb{Z}\tau_2).$ for some multivalued holomorphic functions $\tau_i\colon\Delta^*\rightarrow\mathbb{C}$.\\ \indent
Suppose $\Omega$ be a meromorphic 2-form on $U$ such that the corresponding divisor is an integer multiple of $D$ up to a scale, and suppose $\Omega=gdz\wedge dw$, $g\colon\Delta^*\rightarrow\mathbb{C}$, here $z$ is the coordinate on $\Delta$ and $w$ the coordinate on the fibers.\\ \indent
In our discussion, we assume $\operatorname{div}(\Omega)=-D$ or 0.\\ \indent
Kodaira classifies \cite{kodaira1963compact} all possible connected indecomposable curves of canonical type. We now give a table of possible singular fibers.\\
\resizebox{\textwidth}{!}{%
\def\arraystretch{1.3}%
\begin{tabular}{lllllll}
\hline
$\mathscr{J}(0)$& $\mathrm{mult}_0\mathscr{J}$ & $A$ & ord$A$ & type & generators $\tau_1,\tau_2$ & N \\
\hline
$\notin\{0,1,\infty\}$ & any & I & 1 & $\uppercase\expandafter{\romannumeral1}_0$ & $1,\tau(z)$ &  0\\
$\notin\{0,1,\infty\}$ & any & -I & 2 & $\uppercase\expandafter{\romannumeral1}_0^*$ & $z^{1/2},z^{1/2}\tau(z)$ &  1\\
$0$& $m\equiv1(3)$ & $B_1$ & 6 & $\uppercase\expandafter{\romannumeral2}$ & $(1-z^{m/3})z^{5/6},\zeta_3(1-\zeta_3z^{m/3})z^{5/6}$ & 1 \\
$0$& $m\equiv1(3)$ & $B_2$ & 3 & $\uppercase\expandafter{\romannumeral4}^*$ & $(1-z^{m/3})z^{1/3},\zeta_3(1-\zeta_3z^{m/3})z^{1/3}$ & 1 \\
$0$& $m\equiv2(3)$ & $B_3$ & 6 & $\uppercase\expandafter{\romannumeral2}^*$ & $(1-z^{m/3})z^{1/6},\zeta_3(1-\zeta_3z^{m/3})z^{1/6}$ & 1 \\
$0$& $m\equiv2(3)$ & $B_4$ & 3 & $\uppercase\expandafter{\romannumeral2}$ & $(1-z^{m/3})z^{2/3},\zeta_3(1-\zeta_3z^{m/3})z^{2/3}$ & 1 \\
$0$& $m\equiv0(3)$ & I & 1 & $\uppercase\expandafter{\romannumeral1}_0$ & $1,\tau(z)$ & 0 \\
$0$& $m\equiv0(3)$ & -I & 2 & $\uppercase\expandafter{\romannumeral1}_0^*$ & $z^{1/2},z^{1/2}\tau(z)$ & 1 \\
$1$& $m\equiv1(2)$ & $B_5$ & 4 & $\uppercase\expandafter{\romannumeral3}$ & $(1-z^{m/2})z^{3/4},i(1+iz^{m/2})z^{3/4}$ & 1 \\
$1$& $m\equiv1(2)$ & $B_6$ & 4 & $\uppercase\expandafter{\romannumeral3}^*$ & $(1-z^{m/2})z^{1/4},i(1+iz^{m/2})z^{1/4}$ & 1 \\
$1$& $m\equiv0(2)$ & I & 1 & $\uppercase\expandafter{\romannumeral1}_0$ & $1,\tau(z)$ & 0\\
$1$& $m\equiv0(2)$ & -I & 2 & $\uppercase\expandafter{\romannumeral1}_0^*$ & $z^{1/2},z^{1/2}\tau(z)$ & 1\\
$\infty$ & $-b$ & $A_b$ & $\infty$ & $\uppercase\expandafter{\romannumeral1}_b$ & $1,\frac{b}{2\pi i}\log z$ & 0 \\
$\infty$ & $-b$ & $-A_b$ & $\infty$ & $\uppercase\expandafter{\romannumeral1}_b^*$ & $z^{1/2},\frac{b}{2\pi i}z^{1/2}\log z$ & 1 \\
\hline
\end{tabular}%
}
where
\begin{align*}
  B_1&=\begin{pmatrix}
    0 & 1 \\
    -1 & 1
  \end{pmatrix}\; ,
 &B_2&=-\begin{pmatrix}
   0 & 1 \\
   -1 & 1
 \end{pmatrix}\; ,
 &B_3&=\begin{pmatrix}
   1 & -1 \\
   1 & 0
 \end{pmatrix}\; , \\
  B_4&=-\begin{pmatrix}
    1 & -1 \\
    1 & 0
  \end{pmatrix}\; ,
 &B_5&=\begin{pmatrix}
   0 & 1 \\
   -1 & 0
 \end{pmatrix}\; ,
 &B_6&=-\begin{pmatrix}
   0 & 1 \\
   -1 & 0
 \end{pmatrix}\; , \\
 &&A_b&=\begin{pmatrix}
   1 & b \\
   0 & 1
 \end{pmatrix}.
\end{align*}


\subsection{Monodromy}
\indent  We now explain the parameters in the above table.\\ \indent
Kodaira's work \cite{kodaira1963compact} says that if $f\colon U\rightarrow \Delta$ is an elliptic fibration over the unit disk with a section $\sigma$, such that all fibers except $D=f^{-1}(0)$ are smooth and $D$ does not contain $(-1)$-curves. Then the pair $(f,\sigma)$ is isomorphic to a canonical version $(\bar{f},\bar{\sigma})$ whose total space $\bar{U}$ is birational equivalent to $X$ quotient by a finite group related to the monodromy of $f$, where $X$ is the total space of an explicit elliptic fibration.\\ \indent
Now we will see how Kodaira's canonical form correspond to a given fibration is explicitly constructed.\\ \indent
Write $U|_{\Delta^*}\simeq(\Delta^*\times\mathbb{C})/(\mathbb{Z}\tau_1+\mathbb{Z}\tau_2).$ as before and transforms as
\[ \tau \mapsto \frac{d\tau+b}{c\tau+a}\; , \]
under the image $[A]\in PSL(2,\mathbb{Z})$ of the monodromy when going around the singularity in counterclockwise orientation. Write this in $\tau_1,\tau_2$ we have $T\mapsto TA$, i.e.
\[ (\tau_1,\tau_2)\mapsto(a\tau_1+c\tau_2,b\tau_1+d\tau_2 )\; . \]\\ \indent
Suppose $j\colon\mathfrak{H}\rightarrow\mathbb{C}$ is the classical elliptic modular function, normalized so that $j(i)=1$, $j(\zeta_3)=0$. Since $j$ is $PSL(2,\mathbb{Z})$-invariant and is only ramified at branch points of order $2,3$ along the orbits of $i,\zeta_3$. And the Kodaira's functional invariant 
\[ \mathscr{J}\\coloneq j\circ \tau \]
is a single-valued meromorphic function on $\Delta^*$.\\ \indent
For the singularity 0, if $\mathscr{J}(0)\in\mathbb{C}\setminus \{0,1\}$, then $\tau$ is single-valued and can be extended to a regular function on $\Delta$ with $\tau(0)$ not in the $PSL(2,\mathbb{Z})$ orbits of $i,\zeta_3$. Thus the stabilizer of $\tau(0)$ in $PSL(2,\mathbb{Z})$ is trivial, and $A=\pm I$.\\ \indent
If $A=I$, then $\tau_1,\tau_2$ can directly extend to single-valued functions on $\Delta$ and \[ U\simeq (\Delta\times\mathbb{C}/\mathbb{Z}\tau_1+\mathbb{Z}\tau_2)\; . \]
\indent If $A=-I$, we first change the coordinate through $z=u^2$ over $\Delta_z^*$. The fibration then first lift to a fibration $U'\rightarrow \Delta_u^*$, then extends to $\Delta_u$ with a smooth central fiber $D'$. The free $\mathbb{Z}_2$-action on $U'$ over $\Delta_u^*$ also can extends but has four fixed points on $D'$. Then the quotient is an elliptic fibration over $\Delta_z$, hence isomorphic to $U$ over $\Delta_z^*$, but with four surface singularites of type $A_1$ over $z=0$. Resolving the singularities we have the normal form $\bar{U}\rightarrow \Delta_z$ with a type $\uppercase\expandafter{\romannumeral1}_0^*$ central fiber. \\ \indent
If $\mathscr{J}(0)=0$, then $\tau$ is possibly multivalued, however it is still regular at 0, with $\tau(0)$ in the orbit of $\zeta_3$. $[A]$ fixes $\tau(0)$, hence have six possibilities for $A$ up to conjugation, including $A=\pm I$. These can be treated as before, also make a base change $z=u^{\operatorname{ord}(A)}$, then extend, and minimally resolving the singularities of the quotient, then blowing down the (-1)-curves in central fiber if it is needed.\\ \indent
Notice that when $\mathscr{J}=1$, the case is the same.\\ \indent
If $\mathscr{J}$ has a pole of order $b\geq 1$, then $A$ is conjugate to $\pm A_b$.\\
\indent Also it is possible to let \[ \tau_1\equiv 1\;, \]
\[ \tau_2=\tau=b\frac{\log z}{2\pi i}\; . \]
\indent If $b=1$, we can fill in the surface 
\[ U|_{\Delta^*}\simeq (\Delta^*\times \mathbb{C})/(\mathbb{Z}+\mathbb{Z}\tau)\; , \]
by using $\wp$-functions to embed it into $\Delta\times \mathbb{P}^2$ and taking the closure. And we have the canonical form $\bar{U}$ and central fiber with a node $\uppercase\expandafter{\romannumeral1}_1$.\\ \indent When $b>1$, with central fiber a cycle of $b$ rational curves $\uppercase\expandafter{\romannumeral1}_b$ and $A=-A_b$ implies $A^2=A_{2b}$, we can process as former cases.\\ \indent
Consider the multiplicity $N$ of $\Phi^*(dz\wedge dw)$ along $D$, so that \[ \Phi^*(dz\wedge dw)=f^Nhdx\wedge dy\; , \]
where $h(p)\neq 0$ in local coordinates $(x,y)$ on $U $ near $p$. So we either have $g(z)=z^{-N-1}k(z)$ or $g(z)=z^{-N}k(z)$, $k(0)=0$, depending on whether div$(\Omega)=-D$ or div$(\Omega)=0$.





% !TeX root = ../thuthesis-example.tex

\section{Metrics near singular fibers}
Combining the two subsections above, we now take a look at the metrics near singular fibers. We will classify all cases through the order of monodromy and have a look at the metric in different cases and proved certain cases in the table we listed in last section.\\ \indent
In the last subsection of this section, we will see that we construct the form in such expression because that it's easier to prove the related geometric properties like SOB and CYL conditions. Finally, we explain briefly how the metric on smooth part and singular part is glued and give an existence proposition of certain Calabi--Yau metric.
\subsection{Trivial monodromy}
\indent If the monodromy is trivial, we can set $\tau_1\equiv 1$ and $\tau_2=\tau$, where $\tau\colon\Delta\rightarrow\mathfrak{H}$ is a regular function that satisfies $D\simeq \mathbb{C}/(\mathbb{Z}+\mathbb{Z}\tau(0))$, but is not constrained otherwise; in particular $N=0$. The div$(\Omega)=0$ case directly follows from the formula in section 2. When div$(\Omega)=-D$, then \[ \Omega=\frac{k(z)}{z}dz\wedge dw\; . \]
thus we have
\[ \omega_{\mathrm{sf},\epsilon}=i|k|^2\frac{\operatorname{Im}(\tau)}{\epsilon|z|^2}dz\wedge dw+\frac{i\epsilon}{2\operatorname{Im}(\tau)}(dw-\Gamma dz)\wedge(d\bar{w}-\bar{\Gamma}d\bar{z})\; , \]
where \[ \Gamma(z,w)=\frac{\operatorname{Im}(w)}{\operatorname{Im}(\tau)}\frac{d\tau}{dz}\; . \]
\indent Notice that $\mu\\coloneq|k(0)|(2\operatorname{Im}\tau(0))^{1/2}$ has an intrinsic interpretation:
\[ \mu^2=i\int_D R\wedge\bar{R} \]
for the Poincare residue $R$ of $\Omega$ along $D$. We change the coordinates on $\Delta\setminus[0,1)$ through
\[ z=\exp\left(\frac{-u\sqrt{\epsilon}}{\mu}\right)\; , \]
here $u$ ranging in the strip $(0,\infty)+i(0,2\pi\frac{\mu}{\epsilon})$, thus we have
\[ dz=-\frac{\sqrt{\epsilon}z}{\mu}du,\qquad d\bar{z}=-\frac{\sqrt{\epsilon}\bar{z}}{\mu}d\bar{u} \]
and
\[ \omega_{\mathrm{sf},\epsilon}=\frac{i}{2}\left(du\wedge d\bar{u}+\frac{\epsilon}{\operatorname{Im}\tau(0)}dw\wedge d\bar{w}\right)\left(1+O_\epsilon\left(\exp\left(-\frac{\sqrt{\epsilon}\:\operatorname{Re}u}{\mu} \right)\right)\right)\; . \]
\indent The error estimate is understood as to hold for all derivatives, and $w$ ranging in fundamental domain spanned by $1,\tau(z(u))$ in $\{u\}\times\mathbb{C}$. So $\omega_{\mathrm{sf},\epsilon}$ decays to a flat cylinder $\mathbb{R}^+\times S^1\times D$, here $D$ carries its flat K{\"a}hler metric with area $\epsilon$ and the length of the circle is $w\pi\frac{\mu}{\sqrt{\epsilon}}$, with exponential rate \[ O_\epsilon\left(\exp\left(\frac{-r\sqrt{\epsilon}}{\mu}\right)\right)\; . \]
\indent Here if we replace $\Omega$ by $|\alpha|\Omega$ then the circle gets changed by scale $|\alpha|$ and the area of the torus is fixed.\\ \indent
\subsection{Finite order monodromy}
If the monodromy is of finite order, the calculations for all types of singular fibers where $A$ is of finite order are similar, thus we only discuss the most difficult one, i.e.\ the Kodaira type $\uppercase\expandafter{\romannumeral2}$ when the central fiber is a cusp.\\ \indent
Follow the table we give above, $\mathscr{J}(0)=0$, $A=E_1$, $m\equiv1~(3)$, $N=1$. Since
\[ j(\tau)=\left(\frac{\tau-\zeta_3}{\tau-\zeta_3^2}\right)^3h(j(\tau))\; , \]
changing coordinates on $\Delta$ we make 
\[ \tau(z)=\zeta_3\frac{1-\zeta_3z^{m/3}}{1-z^{m/3}}\; , \]
hence we have
\[ \frac{\tau-\zeta_3}{\tau-\zeta_3^2}=z^{\frac{m}{3}}\; . \]
\indent Since the monodromy is  described by $E_1$, i.e.
\[ \tau\mapsto A\tau=-\frac{\tau+1}{\tau}\; . \]
\indent Here $\operatorname{ord}A=6$, the pullback $U'$ of $U|_{\Delta^*}$ under $z=u^6$ can be written as
\[ (\Delta_u^*\times\mathbb{C}_v)/(\mathbb{Z}+\mathbb{Z}\tau(u^6))\; , \]
hence can be extended to the whole disc with central fiber $\mathbb{C}/(\mathbb{Z}+\mathbb{Z}\zeta_3)$, and the monodromy group acts on $U'$ by deck transformation is given by
\[ A(u,v)=\left(\zeta_6u,\frac{v}{c\tau(u^6)+a}\right)=\left(\zeta_6u,\zeta_6\frac{1-u^{2m}}{1-\zeta_3u^{2m}}v\right)\; . \]
\indent Notice that the fiberwise linear map
\[ \Phi\colon\Delta^*\times \mathbb{C}\rightarrow \Delta^*\times\mathbb{C},\qquad (u,v)\mapsto (u^6,(1-u^{2m})u^5v)\; , \]
factor through the action generated by monodromy action, inducing an isomorphism of elliptic fibrations
\[ \Phi\colon\Delta^*\times \mathbb{C}\rightarrow (\Delta^*\times\mathbb{C})/(\mathbb{Z}\tau_1+\mathbb{Z}\tau_2) \]
with multivalued generators
\[ \tau_1(z)=\operatorname{pr}_2\Phi(z^{1/6},1)=(1-z^{m/3})z^{5/6}\; , \]
and
\[ \tau_2(z)=\operatorname{pr}_2 \Phi(z^{1/6},\tau(z))=\zeta_3(1-\zeta_3z^{m/3})z^{5/6}=\tau_1(z)\tau(z)\; . \]
\indent Now what remains is to calculate the multiplicity $N$ of $\Phi^*(dz\wedge dw)$ along the central fiber. Resolve the singularities appear on the central fiber of the quotient, then blowing down vertical $(-1)$-curves if necessary. Near the three orbits of fixed points, the monodromy action is conjugate to the $\mathbb{Z}_{6/l}$-action on $\mathbb{C}^2$ generated by $(x,y)\mapsto(\zeta_6^lx,\zeta_6^ly)$ for $l=1,2,3$ respectively, and the germ of $x=0$ corresponds to the central fiber of $U'$.\\ \indent
To resolve the quotients, map
\[ (x,y)\mapsto (x^{\frac{6}{l}},y^{\frac{6}{l}},(x:y))\in \mathbb{C}\times\mathbb{C}\times\mathbb{P}^1,\quad l=1,2,3\; . \]
\indent From the global viewpoint, this produces exceptional curves $E_1,E_2,E_3$ and the strict transform $E$ of the central fiber of $U/\mathbb{Z}_6$. The multiplicities of $z\circ\Phi$ along these curves are $1,2,3,6$, and the multiplicities of $\Phi^*(dz\wedge dw)$ are $1,3,5,10$.\\ \indent
Turns out that $E$ is a (-1)-curve, blowing down $E$ turns $E_3$ into a $(-1)$-curve, repeating the process, blowing down $E_3$ and $E_2$, and the curve $E_1$ becomes a rational cusp curve with self-intersection 0, so $N=1$.\\ \indent
So $\Omega=gdz\wedge dw$ such that $g(z)=z^{-2}k(z)$ or $g(z)=z^{-1}k(z)$ where $k(0)\neq 0$ depending on $\operatorname{div}(\Omega)$. For the first case,
\[ \omega_{\mathrm{sf},\epsilon}=\frac{i|k|^2\sqrt{3}(1-|z|^{2m/3})}{2\epsilon|z|^{7/3}}dz\wedge d\bar{z}+\frac{i\epsilon}{\sqrt{3}|z|^{5/3}(1-|z|^{2m/3})}(dw-\Gamma dz)\wedge(d\bar{w}-\bar{\Gamma}d\bar{z})\; . \]
where
\[ \Gamma(z,w)=\frac{5w}{6z}(1+\textrm{correction})\; , \]
and the correction term is $\frac{1}{w}(\operatorname{Re}(w)P+\operatorname{Im}(w)Q)$ with $P,Q=O(|z|^{m/3})$. When $m$ tends to infinity, the fibration becomes isotrivial and the correction term tends to zero.\\ \indent
To have a better look at it, we again use the coordinate change
\[ z=\left(\frac{u_0}{u}\right)^6,\quad w=\left(\frac{u_0}{u}\right)^5v,\quad \textrm{where }u_0\coloneq 6\sqrt[3]{4}\epsilon^{-\frac{1}{2}}|k(0)|\; . \]
\indent Then $u$ ranges in the domain $\{|u|>u_0,~0<\arg u<\frac{\pi}{3}\}$ and $v$ belongs to a fundamental cell in $\{u\}\times\mathbb{C}$ which converges to the one spanned by $1,\zeta_3$ when $|u|$ tends to infinity. Under the new coordinate, we have
\[ \omega_{\mathrm{sf},\epsilon}=\frac{i}{2}\left(du\wedge d\bar{u}+\frac{\epsilon}{\operatorname{Im}(\zeta_3)}\eta\wedge\bar{\eta}\right)(1+O(|u|^{-4m}))\; , \]
where
\[ \eta=dv-\uppercase\expandafter{\romannumeral2}du,\quad \uppercase\expandafter{\romannumeral2}(u,v)=\frac{1}{u}(Pv+Qu),\quad P,Q=O(|u|^{-2m})\; . \]
\indent In order to estimate the metric, we need a further coordinate change by $(u,v)\mapsto(u,M_uv)$, where $M_u\in GL(2,\mathbb{R})$, $M_u=I+O(|u|^{-2m})$, is defined by $w(M_u(1))=\tau_1(z(u))$ and $w(M_u(\zeta_3))=\tau_2(z(u))$.\\ \indent
\subsection{Infinite monodromy}
Finally we have a look at the infinite monodromy case, i.e.\ $\uppercase\expandafter{\romannumeral1}_b$ and $\uppercase\expandafter{\romannumeral1}_b^*$ type fiber. According to Kodaira's work, for any $\uppercase\expandafter{\romannumeral1}_1$ degeneration of elliptic curves $f\colon U\rightarrow \Delta$, the restriction $U|_{\Delta^*}$ must be abstractly isomorphic to $(\Delta^*\times\mathbb{C})/(\mathbb{Z}\tau_1+\mathbb{Z}\tau_2)$, where $\tau_1=1$, $\tau_2=\frac{1}{2\pi i}\log z$, he also proved that the map
\[ \Psi(z,w)\coloneq \left(z,-\frac{1}{12}-\frac{1}{4\pi^2}\wp_z(w),\frac{i}{8\pi^3}\wp_z'(w)\right)\in \Delta\times\mathbb{C}_{xy}^2 \]
induces an isomorphism from $(\Delta^*\times\mathbb{C})/(\mathbb{Z}\tau_1+\mathbb{Z}\tau_2)$ onto the surface
\[ y^2=4x^3+x^2-g_2(z)x-g_3(z) \]
in $\Delta^*\times\mathbb{P}^2_{xy}$ with the natural elliptic fibration by projection on the first coordinate, where $g_2,g_3$ are regular functions, and $\wp_z$ is the usual Weierstrass function associated with the lattice generated by $\tau_1(z)$ and $\tau_2(z)$. The closure of image of $\Psi$ is a smooth elliptic surface $\bar{U}$ with central fiber the node $y^2=4x^3+x$, representing the Kodaira's canonical form for $\uppercase\expandafter{\romannumeral1}_1$. In our setting, the above $\Phi$ is now $\Psi^{-1}$. The multiplicity $N$ of $\Phi^*(dz\wedge dw)$ along $D$ is obtained by compute $\Psi^*\Theta$, where $\Theta=dz\wedge(dx/y)$. We have
\begin{displaymath}
\begin{split}
\Psi^*\Theta&=dz\wedge\left(\frac{\partial x}{\partial w}\frac{dw}{y}\right)\\
&=\frac{\partial}{\partial w}\left(-\frac{1}{12}-\frac{1}{4\pi^2}\wp_z(w)\right)\frac{8\pi^3}{i\wp_z'(w)}dz\wedge dw\\
&=2\pi idz\wedge dw\; .
\end{split}
\end{displaymath}
Then \[ 2\pi i\Phi^*\Psi^*\Theta=\Phi^*(dz\wedge dw)\; . \]
On the other hand
\[ \Phi^*\Psi^*\Theta=(\Psi\circ\Phi)^*\Theta=\Theta\; . \]
So the multiplicity of $\Phi^*(dz\wedge dw)=2\pi i\Theta$ is 0 along $D$. By Kodaira's work, the $I_b$ case with $b>1$ reduces to $\uppercase\expandafter{\romannumeral1}_1$ by a fiberwise b-fold unramified covering. In this case
\[ \omega_{\mathrm{sf},\epsilon}=i|k|^2\frac{b|\log|z||}{2\pi\epsilon |z|^2}dz\wedge d\bar{z}+\frac{2\pi i\epsilon}{2b|\log|z||}(dw-\Gamma dz)\wedge(d\bar{w}-\bar{\Gamma}d\bar{z})\; , \]
where
\[ \Gamma(z,w)=\frac{\operatorname{Im}(w)}{iz|\log|z||}\; . \]
\subsection{Asymptotics}
The purpose to write the metric in this form is to estimate the form through analytic ways more easily. Before this, we introduce two definitions. We first take a look at the $\uppercase\expandafter{\romannumeral1}_b$ type fier case. \\ \indent
In the following context, we assume $(M,g)$ be a complete Riemannian manifold of dimension $n>2$.
\begin{definition}
$(M,g)$ is called SOB$(\beta)$ if there exists a point $x_0\in M$ and a constant $C\geq 1$ such that the annuli $A(x_0,r,r+s)$ is connected for all $r\geq C$, $s>0$, $|B(x_0,s)|\leq Cs^\beta$ for all $s\geq C$, and \[ \left|B\left(x,\left(1-\frac{1}{C}\right)r(x)\right)\right|\geq\frac{1}{C}r(x)^\beta\; , \]
and
\[ \operatorname{Ric}(x)\geq -\frac{C}{r(x)^2},\quad \textrm{if }r(x)\coloneq d(x_0,x)\geq C\; . \]
\end{definition}
\indent If a manifold satisfies SOB$(\beta)$, then it admits some natural weighted global Sobolev inequalities, for detaill see \cite{hein2011weighted}.
\begin{definition}
A complete Riemannian manifold $M$ with an end is called CYL$(\gamma)$ with $0\leq\gamma<1$, which is `cylindral' asymptotic for short, if there exist a point $x_0\in M$ and a constant $C\geq 1$ such that $|B(x_0,s)|\leq Cs^2$ for all $s\geq C$ and 
\[ A\left(x_0,r(x)-\frac{1}{C}r(x)^\gamma,r(x)+\frac{1}{C}r(x)^\gamma\right)\subset B\left(x,Cr(x)^\gamma(x)\right)\; ,  \]
as well as Ric$(x)\geq -Cr(x)^{-2\gamma}$ for all $x\in M$ with $r(x)\geq C$.
\end{definition}
And we have two related propositions. Both proved in Hein's paper \cite{hein2012gravitational} \cite{hein2011weighted}.\\ \indent
\begin{proposition}
Let $(M^m,\omega_0)$ be a complete K{\"a}hler manifold with
\[ |\mathrm{Rm}|+|\nabla \mathrm{Scal}|+|\nabla^2 \mathrm{Scal} |\leq C \; , \]
where Rm, Scal denote the Riemannian curvature tensor and the scalar curvature. And assmue $M$ satisfies SOB$(\beta)$ condition.\\ \indent
Let $f\in C^{2,\alpha}(M)$ satisfies $|f|\leq Cr^{-\mu}$ for some $\mu>2$. If $\beta\leq 2$ and $\int(e^f-1)\omega^m=0$, or $\beta>2$, then there exists an $0<\bar{\alpha}\leq \alpha$ and $u\in C^{4,\bar{\alpha}}$ such that
\[ \left(\omega_0+i\partial\bar{\partial}u  \right)^m=e^f\omega^m\; . \]
If $\beta\leq 2$, then moreover 
\[ \int_M|\nabla u|^2 d\mathrm{vol}<\infty\; . \]
\end{proposition}
\begin{proposition}
Let $(M^m,\omega_0)$ be a complete K{\"a}hler manifold and $u,f$ are smooth functions on $M $ such that
\[ \sup|\nabla^i u|+\sup|\nabla^i f|<\infty   \]
for all $i\in\mathbb{N}_0$ and satisfies
\[ (\omega_0+i\partial\bar{\partial}u)^m=e^f\omega_0\; . \]

(i) Assume that $M$ is CYL$(\gamma)$ where $0\leq \gamma<1$ and
\[ \exp(\kappa r(x)^{1-\gamma})|B(x,1)|\rightarrow\infty\; , \]
as $r(x)\rightarrow\infty$ for every fixed constant $\kappa>0$. If there exists a $\epsilon>0$ such that
\[ \int_M|\nabla u|^2\omega^m <\infty,\quad |f|\leq C\exp(-\epsilon r^{1-\gamma})\; , \]
then there exists $\delta>0$ such that for all $x\in M$,
\[ \sup_{B(x,1)}|u-u_{B(x,1)}|\leq C\exp(-\delta r(x)^{1-\gamma}) \; , \]
where \[ u_{A}=\frac{1}{m(A)}\int_A u \; , \]
is the average of integral on $A$.\\ \indent

(ii) Assume that $M$ is SOB$(\beta)$ where $0< \beta\leq 2$ and
\[ r(x)^\kappa |B(x,1)|\rightarrow\infty\; , \]
as $r(x)\rightarrow\infty$ for every fixed constant $\kappa>0$. If there exists $\epsilon>0$ such that
\[ \int_M|\nabla u|^2\omega^m<\infty,\quad |f|\leq C\exp(-\epsilon r^{1-\gamma})\; , \]
then there exists a $\delta>0$ such that for all $x\in M$,
\[ \sup_{B(x,1)}|u-u_{B(x,1)}|\leq Cr(x)^{-\delta} \; . \]
\end{proposition}
Now come back to the case of semi-flat metric on rational elliptic surface. We denote $\Delta^*$ be the punctured unit disk with radius $\frac{1}{2}$ and $U=f^{-1}(\Delta^*)$, and also denote $z_x\coloneq f(x)\in\Delta^*$ for $x\in U$ and the metric
\[ g\coloneq \frac{b}{2\pi \epsilon}|k(z)|^2\frac{|\log|z||}{|z|^2}|dz|^2\; . \]
\indent Thus $f\colon(U,g_{\mathrm{sf},\epsilon})\rightarrow(\Delta^*,g)$ is a Riemannian submersion. And we also assume for convenience that $(U,g_{\mathrm{sf},\epsilon})$ embedded into a complete Riemannian manifold $M$ as the only end of it, we fixed a base point $x_0$ on $U$, and denote $r_x\coloneq d_M(x_0,x)$.\\ \indent
For every $z\in\Delta^*$, the diameter of the fiber $f^{-1}(z)$ and the $g$-length of the Euclidean circle of radius $|z|$ in $\Delta^*$ are both $\sim$ $|\log|z||^{\frac{1}{2}  }$.\\ \indent
Notice that
\[ \int \frac{1}{t}\left(\log \frac{1}{t}\right)^{\frac{1}{2}}=-\frac{2}{3}\left(\log \frac{1}{t} \right)^{\frac{3}{2} },\quad \int \frac{1}{t}\left(\log \frac{1}{t}\right)=-\frac{1}{2}\left(\log \frac{1}{t} \right)^{2}\; .  \]
\indent We thus get
\[ |z_x|\ll 1\Longrightarrow r_x\sim d_g(z_x,z_{x_0})\sim |\log|z_x||^{\frac{3}{2}}\; . \]
\indent Also, Vol$(f^{-1}(B),g_{\mathrm{sf},\epsilon})=\epsilon\cdot\operatorname{Vol}(B,g)$ for every $B\subset \Delta^*$. Hence using the above integral again, we get
\[ s\gg 1\Longrightarrow |B(x_0,s)|\sim s^{\frac{4}{3}}\; . \]
The above formulas imply that $M$ satisfies CYL condition for $\gamma=\frac{1}{3}$. Actually we can show that it also satisfies SOB condition for $\beta=\frac{4}{3}$, which suffices to show that
\[ r_x\gg 1\Longrightarrow \{y\in U:|z_x|<|z_y|<|z_x|^{1-\alpha}\}\subset B(x,\frac{r_x}{2}) \]
for some $\alpha<1$. This can also be proved by estimate the diamter of $f^{-1}(z)$ and the $g$-length of the circle of radius $|z|$ in $\Delta^*$.\\ \indent
Actually, using similar methods, we can prove that for $\uppercase\expandafter{\romannumeral1}_b^*$ type fiber, the manifold $M$ can't satisfies the cylindral condition, and satisfies the SOB$(2)$ condition. Hence we can try to prove that $M$ satisfies the conditions in Proposition 5.4.1 and Proposition 5.4.2 and they actually satisfies.\\ \indent
Adding a large number of estimation using various Sobolev inequalities, Hein proved \cite{hein2012gravitational} the following result. Which shows how we try to glue the metric on smooth part and singular part of the fibration together.
\begin{proposition}
There exist a holomorphic section $\widetilde{\sigma}$ over $\Delta^*$, concentric disks $\Delta'\subset \Delta''\subset\Delta'''\subset\Delta$ and a positive $(1,1)$-form $\beta$ on $\mathbb{P}^1$ such that supp$(\beta)\subset \Delta'''\setminus\Delta$, and a constant $\alpha_0>0$, such that for any $\alpha>\alpha_0$, there exist \[ u_\alpha^{\mathrm{int}}\in C^\infty (f^{-1}(\mathbb{P}^1\setminus\Delta'),\mathbb{R})\; , \]
\[ u_\alpha^{\mathrm{ext}}\in C^\infty (f^{-1}(\Delta''\setminus\{0\}),\mathbb{R})\; , \]
with the same complex Hessian on the overlapping area $\Delta''\setminus\Delta'$, and $t_\alpha>0$ such that the closed $(1,1)$-form
\[ \omega_0(\alpha,t)\coloneq \omega+tf^*\beta+i\partial\bar{\partial}u_{\alpha}^{\mathrm{int,ext}}  \]
is positive for $t>t_\alpha$. And it conincides with $\omega$ on $\mathbb{P}^1\setminus \Delta'''$, and coincides with $T^*\omega_{\mathrm{sf}}(\alpha)$ on $\Delta'\setminus\{0\} $ where $T$ is the vertical translation by $\widetilde{\sigma}$ since $\sigma$ gives a Lie group structure on each fiber with unit $\sigma(s)$. In addition, we have 
\[ \int_M\left(\omega_0(\alpha,t)^2-\alpha\Omega\wedge\bar{\Omega} \right)=0 \]
for precisely one $t>t_\alpha$.
\end{proposition}
With the above three proposition, Hein proved an important result of existence. But the statement use a definition called bad cycle, which we write here.
\begin{definition}
A bad 2-cycle is one that arises from the following process and this definition is up to orientation and isotopy. Consider the topological monodromy representation of the fundamental group of $\Delta^*$ in the mapping class group of an arbitrary $F$ over the punctured disk. Choose a simple loop $\gamma\subset F$ such that $[\gamma]\in H_1(F,\mathbb{Z})$ is indivisible and invariant under monodromy. Then, move the loop around the puncture by lifting $\bar{\gamma}\subset \Delta^*$ up to each point in $\gamma$ such that the union of such translates of $\gamma$ is a $T^2$ embedded in $f^{-1}(\gamma)$. 
\end{definition}
Now all preparation has been set up, we now show the final conclusion.
\begin{proposition}
Suppose $X$ is an rational elliptic surface and hence an elliptic fibration over $\mathbb{P}^1$. And $\Omega$ is the rational 2-form that whose corresponding divisor is $-D$. Define $M=X\setminus D$ and fix a small disk $\Delta$ with $z(p)=0$ and all fibers on $\Delta^*$ are soomth.\\ \indent
Let $\omega$ be an arbitrary K{\"a}hler metric on $M$ such that the integral of $\omega^2$ on $M$ is finite, and the integral of $\omega$ for all bad 2-cycles $C$. Here $\omega$ are often taken as the restriction of K{\"a}hler metric on $X$. Then there exists a constant $\alpha_0>0$, depending on $X,\Omega,\omega$ such that for all $\alpha>\alpha_0$, there exists a complete Calabi--Yau metric $\omega_{\mathrm{CY}}$ on $M $ satisfies
\[ \omega_{\mathrm{CY}}^2=\alpha\Omega\wedge \bar{\Omega}\; , \]
and $\omega_{\mathrm{CY}}-\omega$ is $d$-exact on $M$.
\end{proposition}



% 其他部分
\backmatter

% 参考文献
\bibliography{ref/refs}  % 参考文献使用 BibTeX 编译
% \printbibliography       % 参考文献使用 BibLaTeX 编译
% !TeX root = ../thuthesis-example.tex

\begin{acknowledgements}
  特别感谢我的导师周杰老师,以及李思老师在本科期间对我的指导,在上老师们的课,和老师们沟通交流期间,他们看待数学的观点对我产生了潜移默化的影响,也感谢老师们还在各方面热心地提供给我各种帮助。\\ \indent
  除此之外,还要感谢张友金老师,邓邦明老师,邱宇老师,于品老师,姚家燕老师,在本科四年期间在各方面给予了我很多关爱与帮助。也感谢所有数学系每一位老师对我的教育,感谢数学系的优良学风对我的影响。\\ \indent
  感谢我的家人们,你们在我就读清华大学的四年间一直给我提供物质和精神上的支持和关心,让我能心无旁骛地完成学业。\\ \indent
  感谢我的三位室友和朋友们,特别是孔繁浩,卜辰璟和贺宇昕同学,本科四年期间,各个方面都帮助了我很多,合开的讨论班上我也学到了很多东西。\\ \indent
  最后特别感谢我的女朋友冯境,给了我很多的支持和鼓励,陪我走过了大学里最辛苦的一年,也让这一年变的格外幸福美好。这一路,感谢有你,也因为你,让我对未来的每一天都有更大的期待。
  
\end{acknowledgements}

% 附录
\statement[file=scan-statement.pdf]
\appendix
% !TeX root = ../thuthesis-example.tex

\begin{survey}
\label{cha:survey}

\title{Ouguri--Vafa metric and certain metric on rational elliptic surface}
\maketitle


\tableofcontents

\section{Ooguri--Vafa Metric}
This part is mainly the construction of Ooguri and Vafa's construction of the metric near singular fibers\cite{ooguri1996summing}.
Consider the hypermultiplet moduli space for type IIA string on a CY 3-fold. According to Ooguri--Vafa\cite{ooguri1996summing}, the conifold point $z=0$ is what we need to care, so we will see about behavior of the moduli space near the limit $z\rightarrow 0$. We hope to extract a universal deformation of moduli space. The piece we are considering is a  hyperk{\"a}hler manifold of real dimension 4, the complex moduli $z$, and two coordinates $x,t$ which are expectation values of the RR 3-form corresponding to the vanishing cycle and its dual.\\ \indent
The classical metric is given by
$$ds^2=\frac{\lambda^2}{\tau_2}(dt+\tau(z)dx)(dt+\bar{\tau}(\bar{z})dx)+\tau_2dzd\bar{z} \; ,$$
where $\displaystyle \tau(z)=\frac{1}{2\pi i}\log z$ is the moduli space geometry near the conifold, described by the elliptic fibration, and $\lambda$ the coupling constant.\\ \indent
This metric has $U(1)_t\times U(1)_x$ translational invariance in $t$ and $x$. We can rewrite it as
$$ds^2=\lambda^2[V^{-1}(dt-\mathbf{A}\cdot d\mathbf{y})^2+V d\mathbf{y}^2]\; ,$$
where
$$\mathbf{y}=\left(x,\frac{z}{\lambda},\frac{\bar{z}}{\lambda}\right) \; ,$$
$$V=\tau_2=\frac{1}{4\pi}\log\left(\frac{1}{z\bar{z}}\right)\; ,$$
$$A_x=-\tau_1=\frac{i}{4\pi}\log\left(\frac{z}{\bar{z}}\right),\quad A_z=A_{\bar{z}}=0\; .$$
\indent When we discuss how the quantum corrections would modify the metric, considering the math and physics backgrounds, there are various requirements that the potential $V$ has to satisfy:\\ 
(1) The metric is hyperk{\"a}hler if and only if 
$$V^{-1}\Delta V=0,\qquad \nabla V=\nabla \times A\; ,$$
where 
$$\Delta=\partial_x^2+4\lambda^2\partial_z\partial_{\bar{z}}\; .$$
(2) For $|z|\rightarrow \infty$,
$$V\rightarrow \frac{1}{4\pi}\log\left(\frac{1}{z\bar{z}}\right)\; .$$
(3) The metric should be periodic in $x$ with period 1, but not translationally invariant.\\
(4) $V$ is a function of $x$ and $|z|$ only.\\
(5) The singularities of $V$ can be removed by appropriate coordinate transformation.\\ \indent
In fact, there is a unique solution satisfying the above conditions and is given by
$$V=\frac{1}{4\pi}\sum_{n=-\infty}^{\infty}\left(\frac{1}{\sqrt{(x-n)^2+\frac{z\bar{z}}{\lambda^2}}}-\frac{1}{|n|}\right)+\textrm{const}\; .$$
And it can be rewritten as
$$V=\frac{1}{4\pi}\log\left(\frac{\mu^2}{z\bar{z}}\right)+\sum_{m\neq 0}\frac{1}{2\pi}e^{2\pi imx}K_0\left(2\pi \frac{|mz|}{\lambda}\right)\; ,$$
where $\mu$ is some constant and $K_0$ is the modified Bessel function of the second kind.\\ \indent
When $|z|\rightarrow \infty$, using the asymptotic formula of the modified Bessel function
$$K_0(z)\sim \sqrt{\frac{\pi}{2z}}e^{-z}\sum_{n=0}^{\infty}\frac{(-1)^n((2n-1)!!)^2}{n!(8z)^n}\; ,$$
we have
$$V=\frac{1}{4\pi}\log\left(\frac{\mu^2}{z\bar{z}}\right)+\sqrt{\frac{\lambda}{4\pi|mz|}}\sum_{m\neq 0}\exp\left[-2\pi\left(\frac{|mz|}{\lambda}-imx\right)\right]\sum_{n=0}^{\infty}\frac{(-1)^n((2n-1)!!)^2}{n!(8z)^n} \; .$$
Here the correction term is suppressed by the factor $\exp\left[-2\pi\left(\frac{|mz|}{\lambda}-imx\right)\right]$.\\
\\
\section{Rational elliptic surface and certain metric}
This part is the survey when I first read Hein's paper\cite{hein2012gravitational}, introduce the ALG and ALH spaces.
\begin{defi}
A \textbf{rational elliptic surface} is the blowup of $\mathbb{P}^2$ in the base points of a pencil of cubics, i.e.\ a family $sF+tG=0$, $(s:t)\in\mathbb{P}^1$, where $F,G$ are smooth cubics intersecting $9$ points with multiplicity. Blowing up these points, if needed repeatedly, produces an elliptic fibration $f\colon X\rightarrow \mathbb{P}^1$, $X=\mathbb{P}^2 \# 9\bar{\mathbb{P}}^2$.
\end{defi}
\begin{defi}
A \textbf{bad 2-cycle} in $M$ is one that arises from the following process up to orientation and isotopy. Consider the topological monodromy representation of $\pi_1(\Delta^*)=\mathbb{Z}$ in the mapping class group of any fiber $F$ over $\Delta^*$. Take a simple loop $\gamma\subset F$ such that $[\gamma]\in H_1(F,\mathbb{Z})$ is indivisible and invariant under the monodromy, and move $\gamma$ around the puncture by lifting a simple loop $\bar{\gamma}\subset \Delta^*$ up to every point in $\gamma$ such that the union of the translates of $\gamma$ is a $T^2$ embedded in $f^{-1}(\bar{\gamma})$.
\end{defi}
\begin{defi}
Let $\epsilon,\delta,l>0,\theta\in (0,1],\tau\in \mathbb{H}/\textrm{PSL}(2,\mathbb{Z})$, where $\mathbb{H}$ is the upper half-plane. Let $g_{\epsilon,\tau}$ denote the unique flat metric of area $\epsilon$ and modulus $\tau$ on $T^2$. Let $g$ be a complete Riemannian metric on an open $4$-manifold $N$.\\ \indent
(i) The metric $g$ is called ALG$(\delta,[\theta,\epsilon,\tau])$ if there exist  $r_0>0$, a compact subset $K\subset N$ and an embedding $\Phi\colon S(\theta,r_0)\times T^2\hookrightarrow N\setminus K$ with a dense image, where
$$S(\theta,r_0):=\{z\in \mathbb{C}: |z|>r_0,0<\arg z<2\pi\theta   \},$$
such that
$$|\nabla^k_{g_{\textrm{flat}}}(\Phi^*g-g_{\textrm{flat}})|_{g_{\textrm{flat}}}\leq C(k)|z|^{-\delta-k}$$
for all $k\in\mathbb{N}_0$, where $g_{\textrm{flat}}:=g_{\mathbb{C}}\oplus  
g_{\epsilon,\tau}$.\\ \indent
(ii) The metric $g$ is called ALH if there exists $\delta>0$, a compact subset $K\subset N$ and a diffeomorphism $\Phi\colon\mathbb{R}^+\times T^3\rightarrow N\setminus K$ such that
$$|\nabla^k_{g_{\textrm{flat}}}(\Phi^*g-g_{\textrm{flat}})|_{g_{\textrm{flat}}}\leq C(k)e^{-\delta t}$$
for all $k\in\mathbb{N}_0$, where $g_{\textrm{flat}}:=dt^2\otimes h$ for some flat metric $h$ on $T^3$. More specifically, we say that $g$ is ALH$(l,\epsilon,\tau)$ with $h=l^2d\phi^2\otimes g_{\epsilon,\tau}$ with respect to some topological splitting $T^3=S^1\times T^2$, with $\phi\in S^1=\mathbb{R}/2\pi \mathbb{Z}$ and $g_{\epsilon,\tau}$ as above.
\end{defi}



\bibliographystyle{thuthesis-bachelor}
\bibliography{ref/appendix}

\end{survey}
       % 本科生:外文资料的调研阅读报告
% \input{data/appendix-translation}  % 本科生:外文资料的书面翻译
% !TeX root = ../thuthesis-example.tex

\chapter{中文论文概述}
在丘成桐教授证明了卡拉比猜想之后,关于卡拉比--丘度量的具体构造一直是一个重要的问题。在2000年,马克格罗斯和威尔森在椭圆 $K3$ 曲面上构造了一族里奇半平坦的度量,在光滑纤维上表现良好,然后借此做逼近。\\ \indent
海因在2012他的博士论文上给出了关于这一问题的新的进展。和之前的想法类似,在有理椭圆曲面的非奇异纤维的部分构造一个半平坦度量,即限制在任意一根纤维上都是平坦的度量,并用一个与格罗斯和威尔森本质相同但表达不同的形式来更方便的做了分析方面的估计。在这种形式下,我们可以证明,奇点附近的流形在做黎曼嵌入时,嵌入的流形会满足某些几何条件。\\ \indent
例如,对于小平邦彦的分类给出的两类纤维非光滑流形的奇点附近的纤维,嵌入的流形会分别满足某些SOB和CYL条件。其中SOB条件得名于要求流形满足某些整体估计上的索伯列夫不等式,而CYL条件则是将流形与圆柱相比较得到的一些估计。\\ \indent
而在嵌入的流形满足这些几何性质之后,我们就可以利用分析的手段,考虑两种不同的度量在奇点附近如何被“粘”起来,大致的想法是:将一个带孔圆环分为三个同心的带孔圆环,然后构造两个适当的函数,分别在里面两个圆环内有定义和在外面两个圆环内有定义。并且要求他们在第二个圆环内拥有相同的海塞矩阵,然后考虑这两个函数在复几何的两种导数下得到的2-形式,因为他们在第二个圆环中海塞矩阵相同,故这样操作得到的2-形式是相同的,所以可以粘起来得到一个三个圆环上的全局的2-形式。\\ \indent
现在我们来介绍本文的主要内容,后几章中半平坦度量的构造和讨论分析都是依据海因的文章中的结论加以补充。\\ \indent
在第二章中,我们探讨一部分关于魏尔斯特拉斯模型的性质。由于德利涅证明了对于光滑流形为基底上的椭圆纤维丛,都存在一个双有理态射将这个椭圆纤维从和某个魏尔斯特拉斯模型联系起来。首先我们介绍魏尔斯特拉斯基的存在性和性质。然后我们证明对于一个给定的截面,诱导的法向量丛的推出是一个与截面选取无关的不变量,并且将这个向量丛的逆定义为这个魏尔斯特拉斯纤维丛的基本线丛。然后我们发现,魏尔斯特拉斯系数及由此定义的判别式,可以给出基本线丛的高阶张量积的一个全局截面,由此证明了基本线丛的度数一定非负。\\ \indent
同时,我们还会证明魏尔斯特拉斯纤维化的全空间的一些其他性质,比如我们可以将任何一个魏尔斯特拉斯纤维化的全空间嵌入到原来的底空间上的一个复二维射影空间丛上,从而得到这个魏尔斯特拉斯纤维化的一个自然的代数的坐标表示,即写为一个代数多项式的解集。再比如用欧拉序列等短正合列的分析可以证明全空间的典范线丛刚好是基底空间的的典范空间与基本线丛的张量积的拉回。同时我们还可以计算全空间的不规则度。这里我们用到勒雷谱序列,由于基底空间是一个一维的复流形,所以勒雷谱序列在第二页就会退化,由此我们可以计算得到当全空间是基底空间与一个光滑椭圆曲线的乘积时,不规则度为基底空间的亏格加一。而除此之外的情况,不规则度等于基底空间的亏格。而同时我们也可以证明,当且仅当这个魏尔斯特拉斯纤维化的基本线丛是基底空间的结构层对应的线丛时,全空间才能被写成基底空间和一个光滑的椭圆曲线的乘积。\\ \indent
第三章,我们会介绍在一般的环面从上半平坦度量的构造。我们考察每个点附近的同道群,这里同道群指局部的带定向的坐标基在绕每个点这个作用下的改变,例如在非奇异的纤维附近,同道群是平凡的。由小平邦彦的椭圆曲线从的分类可以知道,奇异纤维的同道群可以是有限阶的,即绕奇点旋转有限圈之后平凡,也可以是无限阶的,此时奇异纤维本身不是光滑的,此处的坐标表示也要用到格列菲斯在研究霍奇理论时提出的周期映射概念。\\ \indent
第四章,我们会主要讨论奇异纤维附近的表现。首先我们给出有理椭圆曲面的定义,即一个二维复射影空间中,对满足特定条件的三次曲线束的共9个带重数的交点处做暴涨,得到一个椭圆纤维从,而全空间被称为有理椭圆曲面。因为暴涨只会改变(1,1)阶霍奇数,所以很容易得出有理椭圆曲面的霍奇钻石。由舒特和上松的文章,或者加尔巴尼亚蒂和萨尔加多的文章可以看出 $K3$ 曲面如何由有理椭圆曲面构造出来,事实上,有些 $K3$ 曲面也可以被看做有理椭圆曲面的二重覆叠。\\ \indent
在非奇异纤维的局部平凡化上,因为椭圆曲线可以看做复平面商掉一个格,而格的具体坐标可以用来形容附近的几何表现。在一个带孔圆盘和复平面的乘积上商掉一个与圆盘坐标相关的格,便是研究奇异纤维附近的几何结构的开始。我们列出了一个详细的表格,反应了有理椭圆曲面上的各种奇异纤维附近同道群的结构,以及对应的格的坐标表示,并会解释这个表中各种基本量是如何定义和得出的。除了同道群的表达式和阶数之外,还有一个变量,是经典椭圆模函数和格的坐标函数的复合,这个量在取值0,1,无穷和其他值的时候,也会带来同道群的表示和奇异纤维附近的不同的几何结构。我们注意到,如果将对应的奇异纤维看做全空间的一个除子的话,体积形式在附近的零点重数同样也是我们关心的一个变量。\\ \indent
第五章中,我们首先讨论不同奇异纤维下之前构造的半平坦度量的具体表达式。分为同道群平凡,有限阶和无穷阶的情况来分别讨论。在有限阶同道群的情况下,我们试图先通过坐标变换,将附近的坐标变的更加容易处理,通过分析此处例外除子的性质来推导上一章的表格中提到的内容。\\ \indent
最有趣和最重要的部分当然是同道群为无穷阶的情况,这里奇异纤维本身不再光滑,也是最开始提到的粘度量里最麻烦的一部分。首先,因为这里格的坐标表示依赖于对数函数,所以在取局部坐标的时候,我们也借助了魏尔斯特拉斯椭圆函数来构造。然后在渐进行为一节里,我们首先介绍两个用来描述流形上度量的几何性质,SOB和CYL条件。我们会给出SOB条件和CYL条件的定义,然后讨论在黎曼曲率张量和常数曲率满足某种有界性条件时,满足SOB条件的流形在求解复蒙日--安培方程时会有怎么样的表现。然后满足SOB条件或者CYL条件的流形可以如何把每个点的值用附近一个小球的积分平均值来估计。\\ \indent
然后我们回到有理椭圆曲面的情况,通过一系列估计,我们可以证明对于两种有理椭圆曲面上的奇异纤维,他们在嵌入某个流形时,必须满足相应的SOB条件和CYL条件,从而可以用之前提到的分析结果做逼近。最终,我们会看到如这个概述开头所说的,两种不同的度量是如何被粘起来的。我们最终会看到一个有理椭圆曲面上,存在一个与原来的卡勒度量同一个上同调类的卡拉比--丘度量,使得和自己做外积等于某个大于给定值的倍数的体积形式。\\ \indent
在相关问题的渐进估计中,还有几类重要的流形有突出的表现,即ALE,ALF,ALG,ALH空间。这几类空间是在对引力瞬子的研究中提出的。引力瞬子在末端的部分已经被人们研究了很多,根据体积增长速度的不同,他们被划分为了这四类空间。而海因的文章中,也针对这些情形有相对应的讨论。\\ \indent
对于非奇异纤维,我们构造的卡拉比--丘度量满足某种ALH条件。对于有有限阶同道群的奇异纤维,我们构造的卡拉比--丘度量满足某种ALG条件。而对于无限阶同道群的奇异纤维,我们则只能通过它的切锥来对其进行估计。\\ \indent
对于ALH空间的情况,我们还可以讨论他的唯一性。如果我们有两个相应的微分同胚的流形,且对应的卡勒形式在拉回意义下处于同一个上同调类,且对应的度量在拉回意义下差满足指数衰减条件,则在拉回意义下一定相等。所以我们可以得到,通过我们的方法构造出的度量,由事先给定的卡勒形式的上同调类和三次曲线束的同构类唯一决定。我们可以进一步探讨相应的ALH度量构成的模空间的维数,比如满足指数衰减的截面构成的子空间维数是24。\\ \indent
这个问题仍然处在研究当中,我们还可以问许多相关的问题,比如更进一步的分类问题,再比如我们还可以研究,到底哪种引力瞬子模型对应四维爱因斯坦流形中的奇点?也希望在未来,这个问题上能有更进一步的结果。

% 致谢


% 声明
%\statement
% 将签字扫描后的声明文件 scan-statement.pdf 替换原始页面

% 本科生编译生成的声明页默认不加页脚,插入扫描版时再补上;
% 研究生编译生成时有页眉页脚,插入扫描版时不再重复。
% 也可以手动控制是否加页眉页脚
% \statement[page-style=empty]
% \statement[file=scan-statement.pdf, page-style=plain]

% 个人简历、在学期间完成的相关学术成果
%\input{data/resume}

% 指导教师/指导小组学术评语
%\input{data/comments}

% 答辩委员会决议书
%\input{data/resolution}

% 本科生的综合论文训练记录表(扫描版)
\record{file=scan-record.pdf}

\end{document}
