% !TeX root = ../thuthesis-example.tex

\chapter{中文论文概述}
在丘成桐教授证明了卡拉比猜想之后,关于卡拉比--丘度量的具体构造一直是一个重要的问题。在2000年,马克格罗斯和威尔森在椭圆 $K3$ 曲面上构造了一族里奇半平坦的度量,在光滑纤维上表现良好,然后借此做逼近。\\ \indent
海因在2012他的博士论文上给出了关于这一问题的新的进展。和之前的想法类似,在有理椭圆曲面的非奇异纤维的部分构造一个半平坦度量,即限制在任意一根纤维上都是平坦的度量,并用一个与格罗斯和威尔森本质相同但表达不同的形式来更方便的做了分析方面的估计。在这种形式下,我们可以证明,奇点附近的流形在做黎曼嵌入时,嵌入的流形会满足某些几何条件。\\ \indent
例如,对于小平邦彦的分类给出的两类纤维非光滑流形的奇点附近的纤维,嵌入的流形会分别满足某些SOB和CYL条件。其中SOB条件得名于要求流形满足某些整体估计上的索伯列夫不等式,而CYL条件则是将流形与圆柱相比较得到的一些估计。\\ \indent
而在嵌入的流形满足这些几何性质之后,我们就可以利用分析的手段,考虑两种不同的度量在奇点附近如何被“粘”起来,大致的想法是:将一个带孔圆环分为三个同心的带孔圆环,然后构造两个适当的函数,分别在里面两个圆环内有定义和在外面两个圆环内有定义。并且要求他们在第二个圆环内拥有相同的海塞矩阵,然后考虑这两个函数在复几何的两种导数下得到的2-形式,因为他们在第二个圆环中海塞矩阵相同,故这样操作得到的2-形式是相同的,所以可以粘起来得到一个三个圆环上的全局的2-形式。\\ \indent
现在我们来介绍本文的主要内容,后几章中半平坦度量的构造和讨论分析都是依据海因的文章中的结论加以补充。\\ \indent
在第二章中,我们探讨一部分关于魏尔斯特拉斯模型的性质。由于德利涅证明了对于光滑流形为基底上的椭圆纤维丛,都存在一个双有理态射将这个椭圆纤维从和某个魏尔斯特拉斯模型联系起来。首先我们介绍魏尔斯特拉斯基的存在性和性质。然后我们证明对于一个给定的截面,诱导的法向量丛的推出是一个与截面选取无关的不变量,并且将这个向量丛的逆定义为这个魏尔斯特拉斯纤维丛的基本线丛。然后我们发现,魏尔斯特拉斯系数及由此定义的判别式,可以给出基本线丛的高阶张量积的一个全局截面,由此证明了基本线丛的度数一定非负。\\ \indent
同时,我们还会证明魏尔斯特拉斯纤维化的全空间的一些其他性质,比如我们可以将任何一个魏尔斯特拉斯纤维化的全空间嵌入到原来的底空间上的一个复二维射影空间丛上,从而得到这个魏尔斯特拉斯纤维化的一个自然的代数的坐标表示,即写为一个代数多项式的解集。再比如用欧拉序列等短正合列的分析可以证明全空间的典范线丛刚好是基底空间的的典范空间与基本线丛的张量积的拉回。同时我们还可以计算全空间的不规则度。这里我们用到勒雷谱序列,由于基底空间是一个一维的复流形,所以勒雷谱序列在第二页就会退化,由此我们可以计算得到当全空间是基底空间与一个光滑椭圆曲线的乘积时,不规则度为基底空间的亏格加一。而除此之外的情况,不规则度等于基底空间的亏格。而同时我们也可以证明,当且仅当这个魏尔斯特拉斯纤维化的基本线丛是基底空间的结构层对应的线丛时,全空间才能被写成基底空间和一个光滑的椭圆曲线的乘积。\\ \indent
第三章,我们会介绍在一般的环面从上半平坦度量的构造。我们考察每个点附近的同道群,这里同道群指局部的带定向的坐标基在绕每个点这个作用下的改变,例如在非奇异的纤维附近,同道群是平凡的。由小平邦彦的椭圆曲线从的分类可以知道,奇异纤维的同道群可以是有限阶的,即绕奇点旋转有限圈之后平凡,也可以是无限阶的,此时奇异纤维本身不是光滑的,此处的坐标表示也要用到格列菲斯在研究霍奇理论时提出的周期映射概念。\\ \indent
第四章,我们会主要讨论奇异纤维附近的表现。首先我们给出有理椭圆曲面的定义,即一个二维复射影空间中,对满足特定条件的三次曲线束的共9个带重数的交点处做暴涨,得到一个椭圆纤维从,而全空间被称为有理椭圆曲面。因为暴涨只会改变(1,1)阶霍奇数,所以很容易得出有理椭圆曲面的霍奇钻石。由舒特和上松的文章,或者加尔巴尼亚蒂和萨尔加多的文章可以看出 $K3$ 曲面如何由有理椭圆曲面构造出来,事实上,有些 $K3$ 曲面也可以被看做有理椭圆曲面的二重覆叠。\\ \indent
在非奇异纤维的局部平凡化上,因为椭圆曲线可以看做复平面商掉一个格,而格的具体坐标可以用来形容附近的几何表现。在一个带孔圆盘和复平面的乘积上商掉一个与圆盘坐标相关的格,便是研究奇异纤维附近的几何结构的开始。我们列出了一个详细的表格,反应了有理椭圆曲面上的各种奇异纤维附近同道群的结构,以及对应的格的坐标表示,并会解释这个表中各种基本量是如何定义和得出的。除了同道群的表达式和阶数之外,还有一个变量,是经典椭圆模函数和格的坐标函数的复合,这个量在取值0,1,无穷和其他值的时候,也会带来同道群的表示和奇异纤维附近的不同的几何结构。我们注意到,如果将对应的奇异纤维看做全空间的一个除子的话,体积形式在附近的零点重数同样也是我们关心的一个变量。\\ \indent
第五章中,我们首先讨论不同奇异纤维下之前构造的半平坦度量的具体表达式。分为同道群平凡,有限阶和无穷阶的情况来分别讨论。在有限阶同道群的情况下,我们试图先通过坐标变换,将附近的坐标变的更加容易处理,通过分析此处例外除子的性质来推导上一章的表格中提到的内容。\\ \indent
最有趣和最重要的部分当然是同道群为无穷阶的情况,这里奇异纤维本身不再光滑,也是最开始提到的粘度量里最麻烦的一部分。首先,因为这里格的坐标表示依赖于对数函数,所以在取局部坐标的时候,我们也借助了魏尔斯特拉斯椭圆函数来构造。然后在渐进行为一节里,我们首先介绍两个用来描述流形上度量的几何性质,SOB和CYL条件。我们会给出SOB条件和CYL条件的定义,然后讨论在黎曼曲率张量和常数曲率满足某种有界性条件时,满足SOB条件的流形在求解复蒙日--安培方程时会有怎么样的表现。然后满足SOB条件或者CYL条件的流形可以如何把每个点的值用附近一个小球的积分平均值来估计。\\ \indent
然后我们回到有理椭圆曲面的情况,通过一系列估计,我们可以证明对于两种有理椭圆曲面上的奇异纤维,他们在嵌入某个流形时,必须满足相应的SOB条件和CYL条件,从而可以用之前提到的分析结果做逼近。最终,我们会看到如这个概述开头所说的,两种不同的度量是如何被粘起来的。我们最终会看到一个有理椭圆曲面上,存在一个与原来的卡勒度量同一个上同调类的卡拉比--丘度量,使得和自己做外积等于某个大于给定值的倍数的体积形式。\\ \indent
在相关问题的渐进估计中,还有几类重要的流形有突出的表现,即ALE,ALF,ALG,ALH空间。这几类空间是在对引力瞬子的研究中提出的。引力瞬子在末端的部分已经被人们研究了很多,根据体积增长速度的不同,他们被划分为了这四类空间。而海因的文章中,也针对这些情形有相对应的讨论。\\ \indent
对于非奇异纤维,我们构造的卡拉比--丘度量满足某种ALH条件。对于有有限阶同道群的奇异纤维,我们构造的卡拉比--丘度量满足某种ALG条件。而对于无限阶同道群的奇异纤维,我们则只能通过它的切锥来对其进行估计。\\ \indent
对于ALH空间的情况,我们还可以讨论他的唯一性。如果我们有两个相应的微分同胚的流形,且对应的卡勒形式在拉回意义下处于同一个上同调类,且对应的度量在拉回意义下差满足指数衰减条件,则在拉回意义下一定相等。所以我们可以得到,通过我们的方法构造出的度量,由事先给定的卡勒形式的上同调类和三次曲线束的同构类唯一决定。我们可以进一步探讨相应的ALH度量构成的模空间的维数,比如满足指数衰减的截面构成的子空间维数是24。\\ \indent
这个问题仍然处在研究当中,我们还可以问许多相关的问题,比如更进一步的分类问题,再比如我们还可以研究,到底哪种引力瞬子模型对应四维爱因斯坦流形中的奇点?也希望在未来,这个问题上能有更进一步的结果。