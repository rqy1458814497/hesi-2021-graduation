\usepackage{geometry}
\geometry{
    paperwidth = 155mm,
    paperheight = 235mm,
    outer = 20mm,
    inner = 20mm,
    top = 25mm,
    bottom = 20mm
}

% fonts & unicode
\usepackage[PunctStyle=kaiming]{xeCJK}
\usepackage{amsmath}
\usepackage{unicode-math}

\setCJKmainfont{NotoSerifCJKsc-Regular.otf}[
    Path            = ../../fonts/,
    BoldFont        = NotoSansCJKsc-Medium.otf,
    ItalicFont      = fzktk.ttf,
    Scale           = .97,
    ItalicFeatures  = {Scale = 1}
]

\setCJKsansfont{NotoSansCJKsc-DemiLight.otf}[
    Path            = ../../fonts/,
    BoldFont        = NotoSansCJKsc-Bold.otf,
    Scale           = .97
]

\setCJKmonofont{NotoSansCJKsc-DemiLight.otf}[
    Path            = ../../fonts/,
    BoldFont        = NotoSansCJKsc-Bold.otf,
    Scale           = .9
]

\newCJKfontfamily{\KaiTi}{fzktk.ttf}[
    Path            = ../../fonts/,
    BoldFont        = NotoSansCJKsc-Medium.otf,
    BoldFeatures    = {Scale = .97},
    ItalicFont      = NotoSerifCJKsc-Regular.otf,
    ItalicFeatures  = {Scale = .97}
]

\setmainfont{XITS}[
    Path            = ../../fonts/,
    Extension       = .otf,
    UprightFont     = *-Regular,
    BoldFont        = *-Bold,
    ItalicFont      = *-Italic,
    BoldItalicFont  = *-BoldItalic
]

\setsansfont{Lato}[
    Path            = ../../fonts/,
    Scale           = MatchUppercase,
    Extension       = .ttf,
    UprightFont     = *-Regular,
    BoldFont        = *-Bold,
    ItalicFont      = *-Italic,
    BoldItalicFont  = *-BoldItalic
]

\setmonofont{FiraMono}[
    Path            = ../../fonts/,
    Scale           = .9,
    Extension       = .otf,
    UprightFont     = *-Regular,
    BoldFont        = *-Bold
]

\setmathfont{XITSMath-Regular.otf}[
    Path            = ../../fonts/,
    BoldFont        = XITSMath-Bold.otf
]

\setmathfont{latinmodern-math.otf}[
    Path            = ../../fonts/,
    range           = {frak, bffrak},
    BoldFont        = latinmodern-math.otf
]

\setmathfont{LatoMath.otf}[
    Path            = ../../fonts/,
    Scale           = .95,
    BoldFont        = LatoMath.otf,
    version         = sf
]

\setmathfont{LatoMath.otf}[
    Path            = ../../fonts/,
    Scale           = .95,
    BoldFont        = LatoMath.otf,
    range           = {bb, sfup -> up, sfit -> it, bfsfup -> bfup, bfsfit -> bfit}
]

\setmathfont{STIX2Math.otf}[
    Path            = ../../fonts/,
    BoldFont        = STIX2Math-Bold.otf,
    range           = {\int, \sum, \prod, \coprod, \bigoplus, \bigotimes, \bigcup, \bigcap, \bigvee, \bigwedge}
]

\Umathcode`/  =  "0 "0 "2215    % / -> U+2215 division slash

% headers and footers
\usepackage{fancyhdr}
\fancyhf{}
\fancyhead[CE]{\sf\mathversion{sf}\theheadertitle}
\fancyhead[CO]{\sf\mathversion{sf}\nouppercase{\leftmark}}
\fancyhead[LE,RO]{\textbf{\textsf{\thepage}}}
\headsep=8mm
\headheight=6mm

\AtBeginDocument{
    \renewcommand{\thepage}{\roman{page}}
    \pagestyle{fancy}\thispagestyle{empty}
}

% spacing
\AtBeginDocument{
    \hfuzz=2pt
    \emergencystretch 2em
    \setlength{\belowdisplayshortskip}{\belowdisplayskip}
}

% parts
\usepackage{tikz}

\renewcommand{\titlepage}[2]{%
    \clearpage%
    \thispagestyle{empty}%
    \vspace*{20mm}%
    \centerline{\begin{tikzpicture}
        \node [scale = 3] at (0, 0) {\sffamily 荷\hspace{.5em}思};
        \node [scale = 1.8] at (0, -94.5mm) {\sffamily #1};
        \node [scale = 1.2] at (0, -105mm) {\sffamily 第 #2 期};
        \draw (-16mm, -8mm) -- (16mm, -8mm);
        \draw (-14mm, -100mm) -- (14mm, -100mm);
        \draw (-8mm, -110mm) -- (8mm, -110mm);
    \end{tikzpicture}}%
    \clearpage%
}

\newcommand{\committee}{%
    \clearpage%
    \thispagestyle{empty}%
    \vspace*{120mm}%
}

\newcommand{\committeeitem}[2]{%
    \par%
    {%
        \leftskip=3em%
        \rightskip=8em%
        \parindent=-3em%
        {\bfseries\sffamily#1}\quad%
        {\sffamily#2}%
        \par\vspace{6pt}%
    }%
}

\newcommand{\toctitle}{%
    \clearpage%
    \thispagestyle{empty}%
    \vspace*{15mm}%
    \noindent{\huge\sffamily 目录}%
    \par\vspace{10mm}%
}

\newcommand{\tocsection}[1]{%
    \par\vspace{6mm}%
    \noindent{\large\sffamily #1}%
    \par\vspace{4mm}%
}

\newcommand{\tocitem}[3]{%
    \par%
    {%
        \leftskip=3em%
        \parindent=-3em%
        \makebox[2em][r]{\textbf{\textsf{#3}}}%
        \quad#1%
        \hfill\mbox{}\hfill\phantom{#2}\hfill\makebox[0em][r]{#2}%
        \par\vspace{8pt}%
    }%
}
